\documentclass[10pt]{beamer}
\usefonttheme{professionalfonts}
%\usetheme{CambridgeUS}
%
% Choose how your presentation looks.
%
% For more themes, color themes and font themes, see:
% http://deic.uab.es/~iblanes/beamer_gallery/index_by_theme.html
%
\mode<presentation>
{
  \usetheme{default}      % or try Darmstadt, Madrid, Warsaw, ...
  \usecolortheme{beaver} % or try albatross, beaver, crane, ...
  \usefonttheme{default}  % or try serif, structurebold, ...
  \setbeamertemplate{navigation symbols}{}
  \setbeamertemplate{caption}[numbered]
} 

\usepackage[english]{babel}
\usepackage[utf8x]{inputenc}
\usepackage{tikz}
\usepackage{pgfplots}
\usepackage{array}  % for table column M
\usepackage{makecell} % to break line within a cell
\usepackage{verbatim}
\usepackage{graphicx}
\usepackage{subcaption}
\usepackage{amsfonts}
\captionsetup{compatibility=false}
%\usepackage{dsfont}
\usepackage[absolute,overlay]{textpos}
\usetikzlibrary{calc}
\usetikzlibrary{pgfplots.fillbetween, backgrounds}
\usetikzlibrary{positioning}
\usetikzlibrary{arrows}
\usetikzlibrary{pgfplots.groupplots}
\usetikzlibrary{arrows.meta}
\usetikzlibrary{plotmarks}
\usetikzlibrary{calc}

\usepgfplotslibrary{groupplots}
\pgfplotsset{compat=newest} 
%\pgfplotsset{plot coordinates/math parser=false}

\usepackage{hyperref}
\hypersetup{
    colorlinks=true,
    linkcolor=blue,
    filecolor=magenta,      
    urlcolor=cyan,
}

\pgfmathdeclarefunction{gauss}{2}{%
	\pgfmathparse{1/(#2*sqrt(2*pi))*exp(-((x-#1)^2)/(2*#2^2))}%
}

\pgfmathdeclarefunction{laplacian}{2}{%
	\pgfmathparse{1/(#2*2)*exp(-(abs(x-#1))/(#2))}%
}

\tikzset{
	declare function={
		sign(\x) = (and(\x<0, 1) * -1) +
		(and(\x>0, 1) * 1);
}
}

\DeclareMathOperator{\E}{\mathbb{E}} % expectation

\newcolumntype{M}[1]{>{\centering\arraybackslash}m{#1}}

\definecolor{blue2}{RGB}{51, 105, 232}  
\definecolor{red2}{RGB}{213, 15, 37}  
\definecolor{green2}{RGB}{0, 153, 37}  
\definecolor{green3}{rgb}{0.1922, 0.6392, 0.3294}% 
\definecolor{yellow2}{RGB}{238, 178, 17} 
\definecolor{gray2}{RGB}{102, 102, 102}
\definecolor{orange2}{RGB}{230, 85, 13}

% Qualitative pallete set1 from www.ColorBrewer.org
\definecolor{Qred}{RGB}{228,26,28}
\definecolor{Qblue}{RGB}{55,126,184}
\definecolor{Qgreen}{RGB}{77,175,74}
\definecolor{Qpurple}{RGB}{152,78,163}
\definecolor{Qorange}{RGB}{255,127,0}
\definecolor{Qyellow}{RGB}{255,255,51}
\definecolor{Qbrown}{RGB}{166,86,40}
\definecolor{Qpink}{RGB}{247,129,191}
\definecolor{Qgray}{RGB}{153,153,153}

\title[EE 264]{Discrete-Time Random Signals}
\author{Jose Krause Perin}
\institute{Stanford University}
\date{July 29, 2017}

\begin{document}

\begin{frame}
  \titlepage
\end{frame}

% Uncomment these lines for an automatically generated outline.
%\begin{frame}{Outline}
%  \tableofcontents
%\end{frame}

\begin{frame}{Administrative}

\begin{itemize}
	\item Please \textbf{enroll for 3 units}. EE 264 is not offered for 4 units during the Summer quarter. Deadline: July, 7.
	\item Homework 1 will be released today and it is due next Thursday
\end{itemize}

\end{frame}

\begin{frame}{Last lecture}

\begin{block}{Review of discrete-time signals and systems}
	\begin{itemize}
		\item Systems can be linear, time-invariant, memoryless, causal, and stable \\
		\item LTI systems are completely characterized by their impulse response \\
		\item The output of an LTI system to any signal is given by the convolution sum \\
		\item The complex exponential $e^{j\omega n}$, and more generally $z^n$, are eigenfunctions of LTI systems
		\item Frequency-domain representation of signals
		\begin{itemize}
			\item DTFT 
			\item $z$-transform and ROC
		\end{itemize}
	\end{itemize}
\end{block}

\end{frame}

\section{Introduction}
\begin{frame}{Today's lecture} 

How to analyze discrete-time systems when the input is random?

\begin{block}{Motivation}
	\begin{itemize}
		\item Many signals vary in complicated patterns that
		cannot easily be described by simple equations
		\item It is often convenient and useful to consider
		such signals as being created by some sort of
		random mechanism
	\end{itemize}   	
\end{block}
\end{frame}

%
\begin{frame}{Example: speech signals}

Speech signals are well described by Laplacian distribution

\begin{figure}
	\centering
	\resizebox{\linewidth}{!}{\begin{tikzpicture}
\begin{axis}[%
name=speech,
axis lines*=middle,
xlabel=$t$,
ylabel=$x(t)$,
scale only axis,
axis on top,
%separate axis lines,
% every outer x axis line/.append style={white!15!black},
% every x tick label/.append style={font=\color{white!15!black}},
xmin=3364,
xmax=10100,
ytick=\empty,
xtick=\empty,
axis line style={->,>=stealth},
every axis x label/.style={
	at={(ticklabel* cs:0.97)},
	anchor=north,
},
every axis y label/.style={
	at={(ticklabel* cs:1)},
	anchor=south,
},
every axis y label/.style={at=(current axis.above origin),anchor=south},
%every axis x label/.style={at=(current axis.right of origin),anchor=west},
ymin=-1,
ymax=1,
]
\addplot [line width = 1.5pt, color=blue2!90] table [x={time}, y={sound}] {figs/speech_data.dat};
\end{axis}

\onslide<2|handout:1>{
\begin{axis}[
at=(speech.east), %anchor=east,
anchor=origin,
%at={(1200,380)},
xmin=-5,
xmax=5,
width=3.5in,
height=2in,
no markers,
rotate around={-90:(current axis.origin)}, % Rotate around the origin
axis lines*=center,
every axis y label/.style={at=(current axis.above origin),anchor=south},
every axis x label/.style={at=(current axis.right of origin),anchor=west},
%height=5cm, width=8cm,
xtick=\empty,
ymin=0,
ymax=0.6,
ytick=\empty,
xlabel=$x$,
ylabel=$p_{x(t)}(x)$,
enlargelimits=false, clip=false, axis on top,
grid = major,
axis line style={->,>=stealth},
every axis x label/.style={
	at={(ticklabel* cs:0)},
	xshift=0.3cm,
	anchor=north,
},
every axis y label/.style={
	at={(ticklabel* cs:1)},
	anchor=south,
},
x dir=reverse,
]
\addplot [very thick,black, fill=blue2!70,domain=-5:5, samples=101, clip=true] {laplacian(0,0.9)};
\node at (axis cs: 3, 0.5) {$\displaystyle p_{x(t)}(x) = \frac{1}{2b}\exp\bigg(-\frac{|x-\mu|}{b}\bigg)$};
\node[text width=3.5cm, align=center] at (axis cs: -1.5, 0.5) {Estimated probability density function};
\end{axis}
}
\end{tikzpicture}%}
	\label{fig:speech_and_dist}
\end{figure} 

\end{frame}


%
\begin{frame}{Example: quantization}

Quantization noise is well described by an uniform distribution
\vspace{-0.7cm}
\begin{center}
	\resizebox{\linewidth}{!}{\begin{tikzpicture}
\begin{axis}[
width=4.52in,
height=3.56in,
scale only axis,
separate axis lines,
every outer x axis line/.append style={white!15!black},
every x tick label/.append style={font=\color{white!15!black}},
xmin=0.00,
xmax=50.00,
ymin=-25.00,
ymax=25.00,
xlabel={},
ylabel={},
xmajorgrids,
ymajorgrids,
every outer y axis line/.append style={white!15!black},
every y tick label/.append style={font=\color{white!15!black}},
legend style={draw=white!15!black,fill=white,legend cell align=left}]
\definecolor{matlabColor1}{rgb}{0.850000,0.325000,0.098000}
\addplot [color=matlabColor1, solid, line width=1.5pt, forget plot]
table[row sep=crcr]{
	1 3.9889 \\
	2 11.9668 \\
	3 -3.9889 \\
	4 -3.9889 \\
	5 -3.9889 \\
	6 -3.9889 \\
	7 11.9668 \\
	8 11.9668 \\
	9 3.9889 \\
	10 11.9668 \\
	11 -3.9889 \\
	12 3.9889 \\
	13 3.9889 \\
	14 -19.9446 \\
	15 -3.9889 \\
	16 3.9889 \\
	17 -3.9889 \\
	18 -11.9668 \\
	19 3.9889 \\
	20 3.9889 \\
	21 3.9889 \\
	22 -3.9889 \\
	23 -11.9668 \\
	24 11.9668 \\
	25 -3.9889 \\
	26 -11.9668 \\
	27 -3.9889 \\
	28 -3.9889 \\
	29 3.9889 \\
	30 3.9889 \\
	31 11.9668 \\
	32 11.9668 \\
	33 11.9668 \\
	34 -3.9889 \\
	35 -11.9668 \\
	36 3.9889 \\
	37 3.9889 \\
	38 11.9668 \\
	39 -3.9889 \\
	40 -3.9889 \\
	41 3.9889 \\
	42 3.9889 \\
	43 -3.9889 \\
	44 3.9889 \\
	45 3.9889 \\
	46 -11.9668 \\
	47 11.9668 \\
	48 3.9889 \\
	49 -3.9889 \\
	50 -3.9889 \\
	51 -3.9889 \\
	52 -3.9889 \\
	53 3.9889 \\
	54 3.9889 \\
	55 -3.9889 \\
	56 -3.9889 \\
	57 3.9889 \\
	58 3.9889 \\
	59 -19.9446 \\
	60 -3.9889 \\
	61 3.9889 \\
	62 -11.9668 \\
	63 -3.9889 \\
	64 -3.9889 \\
	65 -3.9889 \\
	66 -3.9889 \\
	67 3.9889 \\
	68 -3.9889 \\
	69 3.9889 \\
	70 -3.9889 \\
	71 3.9889 \\
	72 3.9889 \\
	73 -19.9446 \\
	74 -3.9889 \\
	75 -11.9668 \\
	76 3.9889 \\
	77 11.9668 \\
	78 3.9889 \\
	79 11.9668 \\
	80 11.9668 \\
	81 3.9889 \\
	82 -3.9889 \\
	83 3.9889 \\
	84 3.9889 \\
	85 -3.9889 \\
	86 -3.9889 \\
	87 11.9668 \\
	88 3.9889 \\
	89 -11.9668 \\
	90 -3.9889 \\
	91 3.9889 \\
	92 -3.9889 \\
	93 3.9889 \\
	94 11.9668 \\
	95 3.9889 \\
	96 11.9668 \\
	97 -3.9889 \\
	98 3.9889 \\
	99 3.9889 \\
	100 -11.9668 \\
	101 11.9668 \\
	102 -3.9889 \\
	103 -3.9889 \\
	104 -3.9889 \\
	105 -3.9889 \\
	106 -3.9889 \\
	107 3.9889 \\
	108 3.9889 \\
	109 -3.9889 \\
	110 3.9889 \\
	111 -3.9889 \\
	112 3.9889 \\
	113 -3.9889 \\
	114 3.9889 \\
	115 11.9668 \\
	116 -11.9668 \\
	117 -3.9889 \\
	118 -3.9889 \\
	119 -3.9889 \\
	120 3.9889 \\
	121 3.9889 \\
	122 -3.9889 \\
	123 11.9668 \\
	124 3.9889 \\
	125 -3.9889 \\
	126 3.9889 \\
	127 -11.9668 \\
	128 -3.9889 \\
	129 -3.9889 \\
	130 3.9889 \\
	131 3.9889 \\
	132 3.9889 \\
	133 3.9889 \\
	134 -3.9889 \\
	135 19.9446 \\
	136 3.9889 \\
	137 -3.9889 \\
	138 3.9889 \\
	139 -3.9889 \\
	140 3.9889 \\
	141 -3.9889 \\
	142 11.9668 \\
	143 3.9889 \\
	144 -3.9889 \\
	145 3.9889 \\
	146 -11.9668 \\
	147 11.9668 \\
	148 11.9668 \\
	149 -11.9668 \\
	150 -3.9889 \\
	151 -3.9889 \\
	152 -3.9889 \\
	153 -3.9889 \\
	154 11.9668 \\
	155 3.9889 \\
	156 -3.9889 \\
	157 3.9889 \\
	158 11.9668 \\
	159 11.9668 \\
	160 -3.9889 \\
	161 -3.9889 \\
	162 -19.9446 \\
	163 -3.9889 \\
	164 19.9446 \\
	165 -11.9668 \\
	166 -11.9668 \\
	167 -3.9889 \\
	168 3.9889 \\
	169 3.9889 \\
	170 -3.9889 \\
	171 3.9889 \\
	172 3.9889 \\
	173 3.9889 \\
	174 -3.9889 \\
	175 -11.9668 \\
	176 -3.9889 \\
	177 -3.9889 \\
	178 -3.9889 \\
	179 3.9889 \\
	180 -3.9889 \\
	181 -3.9889 \\
	182 -3.9889 \\
	183 -11.9668 \\
	184 -3.9889 \\
	185 3.9889 \\
	186 3.9889 \\
	187 -3.9889 \\
	188 19.9446 \\
	189 3.9889 \\
	190 -11.9668 \\
	191 -3.9889 \\
	192 -3.9889 \\
	193 11.9668 \\
	194 11.9668 \\
	195 -3.9889 \\
	196 -3.9889 \\
	197 -3.9889 \\
	198 -3.9889 \\
	199 11.9668 \\
	200 -3.9889 \\
	201 -11.9668 \\
	202 3.9889 \\
	203 3.9889 \\
	204 3.9889 \\
	205 -3.9889 \\
	206 -3.9889 \\
	207 11.9668 \\
	208 3.9889 \\
	209 3.9889 \\
	210 11.9668 \\
	211 -3.9889 \\
	212 3.9889 \\
	213 3.9889 \\
	214 3.9889 \\
	215 3.9889 \\
	216 11.9668 \\
	217 3.9889 \\
	218 -3.9889 \\
	219 3.9889 \\
	220 -3.9889 \\
	221 11.9668 \\
	222 3.9889 \\
	223 -3.9889 \\
	224 -3.9889 \\
	225 -19.9446 \\
	226 3.9889 \\
	227 3.9889 \\
	228 -3.9889 \\
	229 3.9889 \\
	230 3.9889 \\
	231 -19.9446 \\
	232 3.9889 \\
	233 -3.9889 \\
	234 -19.9446 \\
	235 11.9668 \\
	236 -3.9889 \\
	237 11.9668 \\
	238 11.9668 \\
	239 -3.9889 \\
	240 -3.9889 \\
	241 3.9889 \\
	242 11.9668 \\
	243 -11.9668 \\
	244 -3.9889 \\
	245 3.9889 \\
	246 3.9889 \\
	247 -3.9889 \\
	248 -3.9889 \\
	249 -3.9889 \\
	250 -3.9889 \\
	251 -11.9668 \\
	252 -11.9668 \\
	253 3.9889 \\
	254 -3.9889 \\
	255 -11.9668 \\
	256 11.9668 \\
	257 3.9889 \\
	258 3.9889 \\
	259 -3.9889 \\
	260 -3.9889 \\
	261 3.9889 \\
	262 -19.9446 \\
	263 -3.9889 \\
	264 11.9668 \\
	265 -3.9889 \\
	266 3.9889 \\
	267 3.9889 \\
	268 -11.9668 \\
	269 -3.9889 \\
	270 -11.9668 \\
	271 3.9889 \\
	272 -3.9889 \\
	273 -3.9889 \\
	274 11.9668 \\
	275 -3.9889 \\
	276 -3.9889 \\
	277 3.9889 \\
	278 3.9889 \\
	279 -3.9889 \\
	280 3.9889 \\
	281 11.9668 \\
	282 3.9889 \\
	283 -3.9889 \\
	284 -3.9889 \\
	285 11.9668 \\
	286 -3.9889 \\
	287 3.9889 \\
	288 3.9889 \\
	289 -3.9889 \\
	290 3.9889 \\
	291 3.9889 \\
	292 3.9889 \\
	293 -3.9889 \\
	294 -3.9889 \\
	295 3.9889 \\
	296 3.9889 \\
	297 -3.9889 \\
	298 -11.9668 \\
	299 -3.9889 \\
	300 3.9889 \\
	301 3.9889 \\
	302 -3.9889 \\
	303 3.9889 \\
	304 3.9889 \\
	305 -3.9889 \\
	306 11.9668 \\
	307 -3.9889 \\
	308 3.9889 \\
	309 11.9668 \\
	310 19.9446 \\
	311 11.9668 \\
	312 -3.9889 \\
	313 -3.9889 \\
	314 -19.9446 \\
	315 -3.9889 \\
	316 -3.9889 \\
	317 -3.9889 \\
	318 -3.9889 \\
	319 -3.9889 \\
	320 -3.9889 \\
	321 3.9889 \\
	322 -3.9889 \\
	323 -3.9889 \\
	324 -11.9668 \\
	325 -3.9889 \\
	326 3.9889 \\
	327 3.9889 \\
	328 3.9889 \\
	329 -3.9889 \\
	330 -11.9668 \\
	331 3.9889 \\
	332 -3.9889 \\
	333 -3.9889 \\
	334 -3.9889 \\
	335 -11.9668 \\
	336 3.9889 \\
	337 3.9889 \\
	338 3.9889 \\
	339 3.9889 \\
	340 11.9668 \\
	341 3.9889 \\
	342 -3.9889 \\
	343 3.9889 \\
	344 3.9889 \\
	345 3.9889 \\
	346 -3.9889 \\
	347 3.9889 \\
	348 3.9889 \\
	349 -11.9668 \\
	350 11.9668 \\
	351 11.9668 \\
	352 -3.9889 \\
	353 3.9889 \\
	354 19.9446 \\
	355 -3.9889 \\
	356 3.9889 \\
	357 11.9668 \\
	358 -11.9668 \\
	359 3.9889 \\
	360 3.9889 \\
	361 3.9889 \\
	362 3.9889 \\
	363 -3.9889 \\
	364 -3.9889 \\
	365 -11.9668 \\
	366 -3.9889 \\
	367 -3.9889 \\
	368 -3.9889 \\
	369 -3.9889 \\
	370 -3.9889 \\
	371 3.9889 \\
	372 11.9668 \\
	373 -11.9668 \\
	374 -3.9889 \\
	375 19.9446 \\
	376 3.9889 \\
	377 3.9889 \\
	378 -11.9668 \\
	379 -3.9889 \\
	380 11.9668 \\
	381 -11.9668 \\
	382 -3.9889 \\
	383 -3.9889 \\
	384 11.9668 \\
	385 -3.9889 \\
	386 3.9889 \\
	387 3.9889 \\
	388 3.9889 \\
	389 3.9889 \\
	390 3.9889 \\
	391 -3.9889 \\
	392 -3.9889 \\
	393 11.9668 \\
	394 3.9889 \\
	395 3.9889 \\
	396 -3.9889 \\
	397 -3.9889 \\
	398 11.9668 \\
	399 -3.9889 \\
	400 3.9889 \\
	401 -3.9889 \\
	402 -3.9889 \\
	403 11.9668 \\
	404 11.9668 \\
	405 3.9889 \\
	406 3.9889 \\
	407 -3.9889 \\
	408 3.9889 \\
	409 3.9889 \\
	410 -11.9668 \\
	411 3.9889 \\
	412 3.9889 \\
	413 -3.9889 \\
	414 3.9889 \\
	415 -3.9889 \\
	416 -3.9889 \\
	417 -3.9889 \\
	418 -3.9889 \\
	419 -11.9668 \\
	420 -11.9668 \\
	421 -11.9668 \\
	422 3.9889 \\
	423 -3.9889 \\
	424 -11.9668 \\
	425 11.9668 \\
	426 11.9668 \\
	427 -11.9668 \\
	428 -3.9889 \\
	429 3.9889 \\
	430 -3.9889 \\
	431 -3.9889 \\
	432 11.9668 \\
	433 3.9889 \\
	434 -3.9889 \\
	435 -11.9668 \\
	436 -3.9889 \\
	437 -3.9889 \\
	438 3.9889 \\
	439 3.9889 \\
	440 3.9889 \\
	441 11.9668 \\
	442 3.9889 \\
	443 -3.9889 \\
	444 -19.9446 \\
	445 -3.9889 \\
	446 3.9889 \\
	447 3.9889 \\
	448 3.9889 \\
	449 19.9446 \\
	450 3.9889 \\
	451 3.9889 \\
	452 3.9889 \\
	453 3.9889 \\
	454 -3.9889 \\
	455 -3.9889 \\
	456 3.9889 \\
	457 11.9668 \\
	458 -3.9889 \\
	459 -3.9889 \\
	460 -3.9889 \\
	461 -11.9668 \\
	462 3.9889 \\
	463 -11.9668 \\
	464 -11.9668 \\
	465 -3.9889 \\
	466 3.9889 \\
	467 11.9668 \\
	468 11.9668 \\
	469 -3.9889 \\
	470 -11.9668 \\
	471 3.9889 \\
	472 3.9889 \\
	473 3.9889 \\
	474 3.9889 \\
	475 11.9668 \\
	476 3.9889 \\
	477 -3.9889 \\
	478 11.9668 \\
	479 -3.9889 \\
	480 3.9889 \\
	481 -3.9889 \\
	482 -11.9668 \\
	483 3.9889 \\
	484 -3.9889 \\
	485 -3.9889 \\
	486 -11.9668 \\
	487 -3.9889 \\
	488 11.9668 \\
	489 3.9889 \\
	490 -3.9889 \\
	491 -3.9889 \\
	492 11.9668 \\
	493 3.9889 \\
	494 -11.9668 \\
	495 -3.9889 \\
	496 -3.9889 \\
	497 -11.9668 \\
	498 3.9889 \\
	499 3.9889 \\
	500 3.9889 \\
	501 11.9668 \\
	502 3.9889 \\
	503 -3.9889 \\
	504 3.9889 \\
	505 3.9889 \\
	506 -3.9889 \\
	507 3.9889 \\
	508 -3.9889 \\
	509 -3.9889 \\
	510 11.9668 \\
	511 -3.9889 \\
	512 3.9889 \\
	513 11.9668 \\
	514 11.9668 \\
	515 11.9668 \\
	516 -27.9225 \\
	517 -11.9668 \\
	518 -3.9889 \\
	519 -3.9889 \\
	520 11.9668 \\
	521 -3.9889 \\
	522 3.9889 \\
	523 3.9889 \\
	524 3.9889 \\
	525 -3.9889 \\
	526 3.9889 \\
	527 3.9889 \\
	528 3.9889 \\
	529 11.9668 \\
	530 -11.9668 \\
	531 -3.9889 \\
	532 3.9889 \\
	533 -3.9889 \\
	534 -3.9889 \\
	535 -3.9889 \\
	536 3.9889 \\
	537 3.9889 \\
	538 -3.9889 \\
	539 11.9668 \\
	540 11.9668 \\
	541 -3.9889 \\
	542 3.9889 \\
	543 -3.9889 \\
	544 3.9889 \\
	545 3.9889 \\
	546 -3.9889 \\
	547 -3.9889 \\
	548 -11.9668 \\
	549 3.9889 \\
	550 -3.9889 \\
	551 -3.9889 \\
	552 3.9889 \\
	553 -11.9668 \\
	554 -3.9889 \\
	555 3.9889 \\
	556 -3.9889 \\
	557 -3.9889 \\
	558 3.9889 \\
	559 -3.9889 \\
	560 -11.9668 \\
	561 -11.9668 \\
	562 -3.9889 \\
	563 3.9889 \\
	564 -11.9668 \\
	565 3.9889 \\
	566 11.9668 \\
	567 11.9668 \\
	568 -3.9889 \\
	569 3.9889 \\
	570 11.9668 \\
	571 -3.9889 \\
	572 3.9889 \\
	573 -3.9889 \\
	574 -3.9889 \\
	575 3.9889 \\
	576 3.9889 \\
	577 -11.9668 \\
	578 -3.9889 \\
	579 3.9889 \\
	580 -3.9889 \\
	581 3.9889 \\
	582 3.9889 \\
	583 -3.9889 \\
	584 3.9889 \\
	585 3.9889 \\
	586 -3.9889 \\
	587 3.9889 \\
	588 3.9889 \\
	589 3.9889 \\
	590 11.9668 \\
	591 11.9668 \\
	592 -11.9668 \\
	593 -11.9668 \\
	594 -3.9889 \\
	595 -11.9668 \\
	596 -3.9889 \\
	597 -3.9889 \\
	598 -3.9889 \\
	599 -3.9889 \\
	600 -3.9889 \\
	601 3.9889 \\
	602 -3.9889 \\
	603 3.9889 \\
	604 3.9889 \\
	605 11.9668 \\
	606 3.9889 \\
	607 3.9889 \\
	608 11.9668 \\
	609 -3.9889 \\
	610 -3.9889 \\
	611 3.9889 \\
	612 3.9889 \\
	613 3.9889 \\
	614 -3.9889 \\
	615 3.9889 \\
	616 11.9668 \\
	617 3.9889 \\
	618 -3.9889 \\
	619 11.9668 \\
	620 -3.9889 \\
	621 -11.9668 \\
	622 3.9889 \\
	623 3.9889 \\
	624 3.9889 \\
	625 -11.9668 \\
	626 -11.9668 \\
	627 -11.9668 \\
	628 -3.9889 \\
	629 3.9889 \\
	630 3.9889 \\
	631 -3.9889 \\
	632 -3.9889 \\
	633 3.9889 \\
	634 -3.9889 \\
	635 11.9668 \\
	636 -3.9889 \\
	637 -11.9668 \\
	638 3.9889 \\
	639 3.9889 \\
	640 3.9889 \\
	641 -3.9889 \\
	642 -3.9889 \\
	643 -3.9889 \\
	644 -3.9889 \\
	645 3.9889 \\
	646 -11.9668 \\
	647 11.9668 \\
	648 11.9668 \\
	649 -3.9889 \\
	650 -3.9889 \\
	651 -3.9889 \\
	652 3.9889 \\
	653 3.9889 \\
	654 11.9668 \\
	655 3.9889 \\
	656 -3.9889 \\
	657 3.9889 \\
	658 -11.9668 \\
	659 -3.9889 \\
	660 -3.9889 \\
	661 -3.9889 \\
	662 -3.9889 \\
	663 -11.9668 \\
	664 -3.9889 \\
	665 3.9889 \\
	666 3.9889 \\
	667 -3.9889 \\
	668 -19.9446 \\
	669 -3.9889 \\
	670 11.9668 \\
	671 -3.9889 \\
	672 3.9889 \\
	673 3.9889 \\
	674 -11.9668 \\
	675 3.9889 \\
	676 3.9889 \\
	677 -11.9668 \\
	678 -3.9889 \\
	679 -3.9889 \\
	680 3.9889 \\
	681 11.9668 \\
	682 -3.9889 \\
	683 11.9668 \\
	684 11.9668 \\
	685 -3.9889 \\
	686 3.9889 \\
	687 11.9668 \\
	688 -3.9889 \\
	689 3.9889 \\
	690 3.9889 \\
	691 -3.9889 \\
	692 11.9668 \\
	693 3.9889 \\
	694 -11.9668 \\
	695 -3.9889 \\
	696 3.9889 \\
	697 3.9889 \\
	698 -11.9668 \\
	699 11.9668 \\
	700 -3.9889 \\
	701 -11.9668 \\
	702 3.9889 \\
	703 3.9889 \\
	704 11.9668 \\
	705 3.9889 \\
	706 3.9889 \\
	707 3.9889 \\
	708 -11.9668 \\
	709 -11.9668 \\
	710 -3.9889 \\
	711 3.9889 \\
	712 11.9668 \\
	713 3.9889 \\
	714 11.9668 \\
	715 3.9889 \\
	716 -3.9889 \\
	717 -3.9889 \\
	718 -3.9889 \\
	719 3.9889 \\
	720 -3.9889 \\
	721 -11.9668 \\
	722 -11.9668 \\
	723 11.9668 \\
	724 11.9668 \\
	725 -3.9889 \\
	726 -11.9668 \\
	727 -3.9889 \\
	728 11.9668 \\
	729 3.9889 \\
	730 -3.9889 \\
	731 3.9889 \\
	732 -3.9889 \\
	733 -11.9668 \\
	734 -3.9889 \\
	735 -11.9668 \\
	736 -3.9889 \\
	737 3.9889 \\
	738 -3.9889 \\
	739 3.9889 \\
	740 -3.9889 \\
	741 -3.9889 \\
	742 -3.9889 \\
	743 11.9668 \\
	744 11.9668 \\
	745 -11.9668 \\
	746 11.9668 \\
	747 -3.9889 \\
	748 -11.9668 \\
	749 -3.9889 \\
	750 -3.9889 \\
	751 3.9889 \\
	752 -3.9889 \\
	753 11.9668 \\
	754 -3.9889 \\
	755 -3.9889 \\
	756 11.9668 \\
	757 11.9668 \\
	758 11.9668 \\
	759 3.9889 \\
	760 3.9889 \\
	761 -11.9668 \\
	762 -3.9889 \\
	763 3.9889 \\
	764 -11.9668 \\
	765 11.9668 \\
	766 -3.9889 \\
	767 -3.9889 \\
	768 11.9668 \\
	769 -11.9668 \\
	770 -3.9889 \\
	771 -11.9668 \\
	772 -11.9668 \\
	773 -3.9889 \\
	774 -3.9889 \\
	775 -3.9889 \\
	776 11.9668 \\
	777 -3.9889 \\
	778 -3.9889 \\
	779 11.9668 \\
	780 -3.9889 \\
	781 11.9668 \\
	782 3.9889 \\
	783 3.9889 \\
	784 11.9668 \\
	785 -3.9889 \\
	786 3.9889 \\
	787 -3.9889 \\
	788 3.9889 \\
	789 11.9668 \\
	790 -3.9889 \\
	791 -3.9889 \\
	792 -3.9889 \\
	793 -3.9889 \\
	794 11.9668 \\
	795 3.9889 \\
	796 -3.9889 \\
	797 3.9889 \\
	798 -3.9889 \\
	799 -3.9889 \\
	800 -3.9889 \\
	801 3.9889 \\
	802 11.9668 \\
	803 -3.9889 \\
	804 -3.9889 \\
	805 -3.9889 \\
	806 11.9668 \\
	807 -3.9889 \\
	808 -3.9889 \\
	809 3.9889 \\
	810 -3.9889 \\
	811 3.9889 \\
	812 -3.9889 \\
	813 -3.9889 \\
	814 3.9889 \\
	815 -11.9668 \\
	816 -11.9668 \\
	817 -3.9889 \\
	818 3.9889 \\
	819 3.9889 \\
	820 3.9889 \\
	821 -3.9889 \\
	822 -3.9889 \\
	823 3.9889 \\
	824 3.9889 \\
	825 -11.9668 \\
	826 -3.9889 \\
	827 3.9889 \\
	828 -3.9889 \\
	829 3.9889 \\
	830 3.9889 \\
	831 11.9668 \\
	832 -3.9889 \\
	833 -11.9668 \\
	834 3.9889 \\
	835 3.9889 \\
	836 3.9889 \\
	837 -11.9668 \\
	838 -3.9889 \\
	839 11.9668 \\
	840 -3.9889 \\
	841 3.9889 \\
	842 3.9889 \\
	843 -3.9889 \\
	844 -3.9889 \\
	845 -11.9668 \\
	846 11.9668 \\
	847 11.9668 \\
	848 3.9889 \\
	849 -3.9889 \\
	850 -3.9889 \\
	851 -3.9889 \\
	852 -3.9889 \\
	853 -3.9889 \\
	854 3.9889 \\
	855 11.9668 \\
	856 -11.9668 \\
	857 -11.9668 \\
	858 -11.9668 \\
	859 -11.9668 \\
	860 3.9889 \\
	861 -3.9889 \\
	862 -11.9668 \\
	863 -3.9889 \\
	864 3.9889 \\
	865 -3.9889 \\
	866 3.9889 \\
	867 -11.9668 \\
	868 -3.9889 \\
	869 11.9668 \\
	870 11.9668 \\
	871 3.9889 \\
	872 -3.9889 \\
	873 11.9668 \\
	874 -3.9889 \\
	875 -3.9889 \\
	876 3.9889 \\
	877 -3.9889 \\
	878 11.9668 \\
	879 3.9889 \\
	880 -3.9889 \\
	881 -3.9889 \\
	882 3.9889 \\
	883 3.9889 \\
	884 -3.9889 \\
	885 11.9668 \\
	886 3.9889 \\
	887 3.9889 \\
	888 3.9889 \\
	889 11.9668 \\
	890 3.9889 \\
	891 3.9889 \\
	892 3.9889 \\
	893 3.9889 \\
	894 -3.9889 \\
	895 -3.9889 \\
	896 -3.9889 \\
	897 -3.9889 \\
	898 3.9889 \\
	899 3.9889 \\
	900 3.9889 \\
	901 -3.9889 \\
	902 3.9889 \\
	903 -3.9889 \\
	904 3.9889 \\
	905 11.9668 \\
	906 -3.9889 \\
	907 3.9889 \\
	908 3.9889 \\
	909 -3.9889 \\
	910 -3.9889 \\
	911 -11.9668 \\
	912 -3.9889 \\
	913 3.9889 \\
	914 -11.9668 \\
	915 -11.9668 \\
	916 3.9889 \\
	917 -3.9889 \\
	918 11.9668 \\
	919 3.9889 \\
	920 -11.9668 \\
	921 3.9889 \\
	922 3.9889 \\
	923 3.9889 \\
	924 3.9889 \\
	925 -3.9889 \\
	926 11.9668 \\
	927 27.9225 \\
	928 -3.9889 \\
	929 -3.9889 \\
	930 3.9889 \\
	931 3.9889 \\
	932 -3.9889 \\
	933 -3.9889 \\
	934 19.9446 \\
	935 3.9889 \\
	936 -11.9668 \\
	937 3.9889 \\
	938 -3.9889 \\
	939 -19.9446 \\
	940 -19.9446 \\
	941 11.9668 \\
	942 3.9889 \\
	943 -3.9889 \\
	944 3.9889 \\
	945 -11.9668 \\
	946 -3.9889 \\
	947 11.9668 \\
	948 3.9889 \\
	949 -3.9889 \\
	950 3.9889 \\
	951 -3.9889 \\
	952 -3.9889 \\
	953 3.9889 \\
	954 3.9889 \\
	955 -11.9668 \\
	956 -11.9668 \\
	957 -3.9889 \\
	958 -11.9668 \\
	959 11.9668 \\
	960 3.9889 \\
	961 -3.9889 \\
	962 3.9889 \\
	963 -3.9889 \\
	964 -3.9889 \\
	965 11.9668 \\
	966 11.9668 \\
	967 3.9889 \\
	968 3.9889 \\
	969 3.9889 \\
	970 11.9668 \\
	971 3.9889 \\
	972 3.9889 \\
	973 11.9668 \\
	974 3.9889 \\
	975 -3.9889 \\
	976 11.9668 \\
	977 11.9668 \\
	978 -3.9889 \\
	979 -3.9889 \\
	980 -11.9668 \\
	981 3.9889 \\
	982 11.9668 \\
	983 -3.9889 \\
	984 -3.9889 \\
	985 -3.9889 \\
	986 3.9889 \\
	987 3.9889 \\
	988 -3.9889 \\
	989 3.9889 \\
	990 3.9889 \\
	991 3.9889 \\
	992 -11.9668 \\
	993 -3.9889 \\
	994 3.9889 \\
	995 -19.9446 \\
	996 -3.9889 \\
	997 -3.9889 \\
	998 -11.9668 \\
	999 -19.9446 \\
	1000 3.9889 \\
	1001 11.9668 \\
	1002 -3.9889 \\
	1003 19.9446 \\
	1004 11.9668 \\
	1005 -11.9668 \\
	1006 3.9889 \\
	1007 3.9889 \\
	1008 -3.9889 \\
	1009 3.9889 \\
	1010 -3.9889 \\
	1011 -3.9889 \\
	1012 -3.9889 \\
	1013 -11.9668 \\
	1014 3.9889 \\
	1015 3.9889 \\
	1016 -11.9668 \\
	1017 -3.9889 \\
	1018 3.9889 \\
	1019 3.9889 \\
	1020 3.9889 \\
	1021 -3.9889 \\
	1022 -3.9889 \\
	1023 -3.9889 \\
	1024 -11.9668 \\
	1025 -3.9889 \\
	1026 19.9446 \\
	1027 -3.9889 \\
	1028 3.9889 \\
	1029 3.9889 \\
	1030 3.9889 \\
	1031 -3.9889 \\
	1032 -11.9668 \\
	1033 11.9668 \\
	1034 3.9889 \\
	1035 -3.9889 \\
	1036 19.9446 \\
	1037 3.9889 \\
	1038 -3.9889 \\
	1039 3.9889 \\
	1040 -3.9889 \\
	1041 11.9668 \\
	1042 3.9889 \\
	1043 -3.9889 \\
	1044 -3.9889 \\
	1045 -11.9668 \\
	1046 -3.9889 \\
	1047 3.9889 \\
	1048 -11.9668 \\
	1049 -3.9889 \\
	1050 3.9889 \\
	1051 -3.9889 \\
	1052 3.9889 \\
	1053 11.9668 \\
	1054 3.9889 \\
	1055 -3.9889 \\
	1056 -3.9889 \\
	1057 -11.9668 \\
	1058 -3.9889 \\
	1059 19.9446 \\
	1060 -3.9889 \\
	1061 -11.9668 \\
	1062 3.9889 \\
	1063 3.9889 \\
	1064 -3.9889 \\
	1065 -3.9889 \\
	1066 -3.9889 \\
	1067 3.9889 \\
	1068 3.9889 \\
	1069 -19.9446 \\
	1070 3.9889 \\
	1071 11.9668 \\
	1072 -3.9889 \\
	1073 3.9889 \\
	1074 -11.9668 \\
	1075 -3.9889 \\
	1076 -3.9889 \\
	1077 11.9668 \\
	1078 11.9668 \\
	1079 -3.9889 \\
	1080 -3.9889 \\
	1081 -3.9889 \\
	1082 3.9889 \\
	1083 -3.9889 \\
	1084 -3.9889 \\
	1085 3.9889 \\
	1086 -3.9889 \\
	1087 11.9668 \\
	1088 -11.9668 \\
	1089 -3.9889 \\
	1090 -11.9668 \\
	1091 3.9889 \\
	1092 11.9668 \\
	1093 -3.9889 \\
	1094 -3.9889 \\
	1095 3.9889 \\
	1096 3.9889 \\
	1097 -3.9889 \\
	1098 -3.9889 \\
	1099 -3.9889 \\
	1100 -3.9889 \\
	1101 3.9889 \\
	1102 -11.9668 \\
	1103 3.9889 \\
	1104 3.9889 \\
	1105 -11.9668 \\
	1106 3.9889 \\
	1107 -3.9889 \\
	1108 -11.9668 \\
	1109 -3.9889 \\
	1110 -3.9889 \\
	1111 3.9889 \\
	1112 -3.9889 \\
	1113 3.9889 \\
	1114 -11.9668 \\
	1115 -3.9889 \\
	1116 3.9889 \\
	1117 -3.9889 \\
	1118 11.9668 \\
	1119 -3.9889 \\
	1120 -3.9889 \\
	1121 -3.9889 \\
	1122 3.9889 \\
	1123 3.9889 \\
	1124 -11.9668 \\
	1125 11.9668 \\
	1126 3.9889 \\
	1127 3.9889 \\
	1128 19.9446 \\
	1129 3.9889 \\
	1130 3.9889 \\
	1131 3.9889 \\
	1132 11.9668 \\
	1133 11.9668 \\
	1134 -3.9889 \\
	1135 -3.9889 \\
	1136 3.9889 \\
	1137 -3.9889 \\
	1138 -3.9889 \\
	1139 -3.9889 \\
	1140 -11.9668 \\
	1141 3.9889 \\
	1142 -3.9889 \\
	1143 3.9889 \\
	1144 3.9889 \\
	1145 -11.9668 \\
	1146 -3.9889 \\
	1147 3.9889 \\
	1148 11.9668 \\
	1149 3.9889 \\
	1150 11.9668 \\
	1151 11.9668 \\
	1152 3.9889 \\
	1153 -11.9668 \\
	1154 -11.9668 \\
	1155 3.9889 \\
	1156 3.9889 \\
	1157 -3.9889 \\
	1158 -3.9889 \\
	1159 -11.9668 \\
	1160 3.9889 \\
	1161 3.9889 \\
	1162 -3.9889 \\
	1163 11.9668 \\
	1164 11.9668 \\
	1165 3.9889 \\
	1166 11.9668 \\
	1167 -3.9889 \\
	1168 3.9889 \\
	1169 -3.9889 \\
	1170 -3.9889 \\
	1171 3.9889 \\
	1172 -3.9889 \\
	1173 3.9889 \\
	1174 -11.9668 \\
	1175 -3.9889 \\
	1176 -3.9889 \\
	1177 -11.9668 \\
	1178 11.9668 \\
	1179 3.9889 \\
	1180 -3.9889 \\
	1181 3.9889 \\
	1182 -3.9889 \\
	1183 11.9668 \\
	1184 -3.9889 \\
	1185 -11.9668 \\
	1186 -3.9889 \\
	1187 -11.9668 \\
	1188 3.9889 \\
	1189 11.9668 \\
	1190 3.9889 \\
	1191 11.9668 \\
	1192 11.9668 \\
	1193 -3.9889 \\
	1194 -3.9889 \\
	1195 3.9889 \\
	1196 3.9889 \\
	1197 3.9889 \\
	1198 3.9889 \\
	1199 -3.9889 \\
	1200 -3.9889 \\
	1201 11.9668 \\
	1202 11.9668 \\
	1203 3.9889 \\
	1204 3.9889 \\
	1205 3.9889 \\
	1206 -11.9668 \\
	1207 -3.9889 \\
	1208 -3.9889 \\
	1209 -19.9446 \\
	1210 -3.9889 \\
	1211 11.9668 \\
	1212 3.9889 \\
	1213 -11.9668 \\
	1214 -11.9668 \\
	1215 -3.9889 \\
	1216 3.9889 \\
	1217 -3.9889 \\
	1218 -3.9889 \\
	1219 -3.9889 \\
	1220 3.9889 \\
	1221 3.9889 \\
	1222 11.9668 \\
	1223 11.9668 \\
	1224 -3.9889 \\
	1225 -11.9668 \\
	1226 -3.9889 \\
	1227 3.9889 \\
	1228 -3.9889 \\
	1229 -3.9889 \\
	1230 -3.9889 \\
	1231 3.9889 \\
	1232 3.9889 \\
	1233 -11.9668 \\
	1234 -3.9889 \\
	1235 3.9889 \\
	1236 -3.9889 \\
	1237 -3.9889 \\
	1238 -3.9889 \\
	1239 11.9668 \\
	1240 -3.9889 \\
	1241 -11.9668 \\
	1242 -3.9889 \\
	1243 -19.9446 \\
	1244 3.9889 \\
	1245 11.9668 \\
	1246 11.9668 \\
	1247 3.9889 \\
	1248 -3.9889 \\
	1249 11.9668 \\
	1250 3.9889 \\
	1251 3.9889 \\
	1252 -3.9889 \\
	1253 -11.9668 \\
	1254 -3.9889 \\
	1255 3.9889 \\
	1256 3.9889 \\
	1257 11.9668 \\
	1258 3.9889 \\
	1259 -11.9668 \\
	1260 3.9889 \\
	1261 -3.9889 \\
	1262 -19.9446 \\
	1263 -3.9889 \\
	1264 3.9889 \\
	1265 -3.9889 \\
	1266 3.9889 \\
	1267 3.9889 \\
	1268 -11.9668 \\
	1269 3.9889 \\
	1270 3.9889 \\
	1271 3.9889 \\
	1272 11.9668 \\
	1273 -3.9889 \\
	1274 11.9668 \\
	1275 -3.9889 \\
	1276 -3.9889 \\
	1277 11.9668 \\
	1278 3.9889 \\
	1279 3.9889 \\
	1280 -3.9889 \\
	1281 -11.9668 \\
	1282 11.9668 \\
	1283 3.9889 \\
	1284 3.9889 \\
	1285 3.9889 \\
	1286 3.9889 \\
	1287 19.9446 \\
	1288 3.9889 \\
	1289 -3.9889 \\
	1290 -3.9889 \\
	1291 3.9889 \\
	1292 -11.9668 \\
	1293 -3.9889 \\
	1294 19.9446 \\
	1295 -11.9668 \\
	1296 -3.9889 \\
	1297 -3.9889 \\
	1298 3.9889 \\
	1299 19.9446 \\
	1300 -3.9889 \\
	1301 -3.9889 \\
	1302 3.9889 \\
	1303 3.9889 \\
	1304 3.9889 \\
	1305 -3.9889 \\
	1306 -11.9668 \\
	1307 3.9889 \\
	1308 3.9889 \\
	1309 3.9889 \\
	1310 -3.9889 \\
	1311 3.9889 \\
	1312 11.9668 \\
	1313 -3.9889 \\
	1314 3.9889 \\
	1315 3.9889 \\
	1316 -3.9889 \\
	1317 3.9889 \\
	1318 -3.9889 \\
	1319 -3.9889 \\
	1320 -3.9889 \\
	1321 -11.9668 \\
	1322 3.9889 \\
	1323 11.9668 \\
	1324 3.9889 \\
	1325 3.9889 \\
	1326 3.9889 \\
	1327 -3.9889 \\
	1328 3.9889 \\
	1329 3.9889 \\
	1330 -3.9889 \\
	1331 -3.9889 \\
	1332 -3.9889 \\
	1333 3.9889 \\
	1334 -3.9889 \\
	1335 -3.9889 \\
	1336 -11.9668 \\
	1337 -19.9446 \\
	1338 -3.9889 \\
	1339 -3.9889 \\
	1340 3.9889 \\
	1341 3.9889 \\
	1342 -11.9668 \\
	1343 -3.9889 \\
	1344 3.9889 \\
	1345 11.9668 \\
	1346 11.9668 \\
	1347 -11.9668 \\
	1348 -3.9889 \\
	1349 -3.9889 \\
	1350 -11.9668 \\
	1351 3.9889 \\
	1352 11.9668 \\
	1353 -3.9889 \\
	1354 -3.9889 \\
	1355 -3.9889 \\
	1356 -3.9889 \\
	1357 -3.9889 \\
	1358 3.9889 \\
	1359 11.9668 \\
	1360 -11.9668 \\
	1361 -11.9668 \\
	1362 3.9889 \\
	1363 -3.9889 \\
	1364 3.9889 \\
	1365 3.9889 \\
	1366 -3.9889 \\
	1367 3.9889 \\
	1368 3.9889 \\
	1369 11.9668 \\
	1370 -3.9889 \\
	1371 3.9889 \\
	1372 -3.9889 \\
	1373 -11.9668 \\
	1374 3.9889 \\
	1375 -3.9889 \\
	1376 11.9668 \\
	1377 -3.9889 \\
	1378 -11.9668 \\
	1379 11.9668 \\
	1380 11.9668 \\
	1381 3.9889 \\
	1382 -11.9668 \\
	1383 3.9889 \\
	1384 3.9889 \\
	1385 -3.9889 \\
	1386 3.9889 \\
	1387 -3.9889 \\
	1388 3.9889 \\
	1389 -11.9668 \\
	1390 -3.9889 \\
	1391 3.9889 \\
	1392 3.9889 \\
	1393 -3.9889 \\
	1394 -11.9668 \\
	1395 3.9889 \\
	1396 -3.9889 \\
	1397 -3.9889 \\
	1398 11.9668 \\
	1399 3.9889 \\
	1400 3.9889 \\
	1401 19.9446 \\
	1402 11.9668 \\
	1403 11.9668 \\
	1404 3.9889 \\
	1405 -11.9668 \\
	1406 3.9889 \\
	1407 3.9889 \\
	1408 -11.9668 \\
	1409 -11.9668 \\
	1410 -3.9889 \\
	1411 -3.9889 \\
	1412 3.9889 \\
	1413 11.9668 \\
	1414 -3.9889 \\
	1415 -11.9668 \\
	1416 3.9889 \\
	1417 11.9668 \\
	1418 11.9668 \\
	1419 3.9889 \\
	1420 -3.9889 \\
	1421 -3.9889 \\
	1422 3.9889 \\
	1423 -3.9889 \\
	1424 -3.9889 \\
	1425 11.9668 \\
	1426 -3.9889 \\
	1427 -3.9889 \\
	1428 -3.9889 \\
	1429 3.9889 \\
	1430 11.9668 \\
	1431 11.9668 \\
	1432 -3.9889 \\
	1433 -3.9889 \\
	1434 3.9889 \\
	1435 -3.9889 \\
	1436 -3.9889 \\
	1437 -11.9668 \\
	1438 3.9889 \\
	1439 3.9889 \\
	1440 3.9889 \\
	1441 19.9446 \\
	1442 3.9889 \\
	1443 -11.9668 \\
	1444 -3.9889 \\
	1445 11.9668 \\
	1446 3.9889 \\
	1447 11.9668 \\
	1448 11.9668 \\
	1449 -3.9889 \\
	1450 11.9668 \\
	1451 3.9889 \\
	1452 -3.9889 \\
	1453 -11.9668 \\
	1454 -19.9446 \\
	1455 3.9889 \\
	1456 3.9889 \\
	1457 -3.9889 \\
	1458 -3.9889 \\
	1459 -11.9668 \\
	1460 -11.9668 \\
	1461 3.9889 \\
	1462 -3.9889 \\
	1463 -3.9889 \\
	1464 -3.9889 \\
	1465 -3.9889 \\
	1466 3.9889 \\
	1467 -11.9668 \\
	1468 11.9668 \\
	1469 19.9446 \\
	1470 -11.9668 \\
	1471 -3.9889 \\
	1472 -3.9889 \\
	1473 -3.9889 \\
	1474 3.9889 \\
	1475 3.9889 \\
	1476 -3.9889 \\
	1477 -3.9889 \\
	1478 3.9889 \\
	1479 -3.9889 \\
	1480 3.9889 \\
	1481 3.9889 \\
	1482 -19.9446 \\
	1483 -3.9889 \\
	1484 -11.9668 \\
	1485 -11.9668 \\
	1486 11.9668 \\
	1487 11.9668 \\
	1488 -11.9668 \\
	1489 -3.9889 \\
	1490 11.9668 \\
	1491 -3.9889 \\
	1492 3.9889 \\
	1493 3.9889 \\
	1494 3.9889 \\
	1495 3.9889 \\
	1496 3.9889 \\
	1497 -3.9889 \\
	1498 -3.9889 \\
	1499 -3.9889 \\
	1500 3.9889 \\
	1501 -3.9889 \\
	1502 -3.9889 \\
	1503 11.9668 \\
	1504 11.9668 \\
	1505 -3.9889 \\
	1506 -11.9668 \\
	1507 -3.9889 \\
	1508 11.9668 \\
	1509 -11.9668 \\
	1510 -11.9668 \\
	1511 11.9668 \\
	1512 3.9889 \\
	1513 11.9668 \\
	1514 3.9889 \\
	1515 -3.9889 \\
	1516 11.9668 \\
	1517 -3.9889 \\
	1518 -3.9889 \\
	1519 3.9889 \\
	1520 -3.9889 \\
	1521 -3.9889 \\
	1522 3.9889 \\
	1523 3.9889 \\
	1524 3.9889 \\
	1525 -11.9668 \\
	1526 -3.9889 \\
	1527 3.9889 \\
	1528 3.9889 \\
	1529 3.9889 \\
	1530 -11.9668 \\
	1531 -3.9889 \\
	1532 -3.9889 \\
	1533 3.9889 \\
	1534 3.9889 \\
	1535 -3.9889 \\
	1536 -3.9889 \\
	1537 11.9668 \\
	1538 3.9889 \\
	1539 -11.9668 \\
	1540 -3.9889 \\
	1541 -3.9889 \\
	1542 3.9889 \\
	1543 -3.9889 \\
	1544 -11.9668 \\
	1545 11.9668 \\
	1546 3.9889 \\
	1547 -3.9889 \\
	1548 -3.9889 \\
	1549 3.9889 \\
	1550 3.9889 \\
	1551 3.9889 \\
	1552 3.9889 \\
	1553 -3.9889 \\
	1554 3.9889 \\
	1555 3.9889 \\
	1556 -11.9668 \\
	1557 3.9889 \\
	1558 11.9668 \\
	1559 11.9668 \\
	1560 11.9668 \\
	1561 -3.9889 \\
	1562 -11.9668 \\
	1563 -3.9889 \\
	1564 -11.9668 \\
	1565 11.9668 \\
	1566 3.9889 \\
	1567 -3.9889 \\
	1568 11.9668 \\
	1569 11.9668 \\
	1570 -3.9889 \\
	1571 -11.9668 \\
	1572 3.9889 \\
	1573 -3.9889 \\
	1574 -11.9668 \\
	1575 -11.9668 \\
	1576 3.9889 \\
	1577 3.9889 \\
	1578 -11.9668 \\
	1579 -3.9889 \\
	1580 -3.9889 \\
	1581 -3.9889 \\
	1582 -3.9889 \\
	1583 -3.9889 \\
	1584 11.9668 \\
	1585 -3.9889 \\
	1586 -11.9668 \\
	1587 3.9889 \\
	1588 11.9668 \\
	1589 -3.9889 \\
	1590 3.9889 \\
	1591 -3.9889 \\
	1592 -11.9668 \\
	1593 -3.9889 \\
	1594 -3.9889 \\
	1595 3.9889 \\
	1596 3.9889 \\
	1597 3.9889 \\
	1598 -3.9889 \\
	1599 -3.9889 \\
	1600 3.9889 \\
	1601 -3.9889 \\
	1602 3.9889 \\
	1603 3.9889 \\
	1604 -11.9668 \\
	1605 -3.9889 \\
	1606 3.9889 \\
	1607 3.9889 \\
	1608 3.9889 \\
	1609 11.9668 \\
	1610 -11.9668 \\
	1611 -3.9889 \\
	1612 11.9668 \\
	1613 -3.9889 \\
	1614 3.9889 \\
	1615 11.9668 \\
	1616 3.9889 \\
	1617 -3.9889 \\
	1618 3.9889 \\
	1619 -11.9668 \\
	1620 -3.9889 \\
	1621 3.9889 \\
	1622 -3.9889 \\
	1623 -11.9668 \\
	1624 3.9889 \\
	1625 11.9668 \\
	1626 3.9889 \\
	1627 3.9889 \\
	1628 -3.9889 \\
	1629 3.9889 \\
	1630 3.9889 \\
	1631 -3.9889 \\
	1632 -3.9889 \\
	1633 3.9889 \\
	1634 19.9446 \\
	1635 -3.9889 \\
	1636 -3.9889 \\
	1637 -3.9889 \\
	1638 -3.9889 \\
	1639 -3.9889 \\
	1640 -3.9889 \\
	1641 -3.9889 \\
	1642 -3.9889 \\
	1643 3.9889 \\
	1644 3.9889 \\
	1645 -3.9889 \\
	1646 -3.9889 \\
	1647 -3.9889 \\
	1648 -3.9889 \\
	1649 -3.9889 \\
	1650 3.9889 \\
	1651 3.9889 \\
	1652 -3.9889 \\
	1653 -19.9446 \\
	1654 3.9889 \\
	1655 3.9889 \\
	1656 -3.9889 \\
	1657 3.9889 \\
	1658 -3.9889 \\
	1659 19.9446 \\
	1660 3.9889 \\
	1661 -3.9889 \\
	1662 11.9668 \\
	1663 -3.9889 \\
	1664 -3.9889 \\
	1665 3.9889 \\
	1666 3.9889 \\
	1667 -3.9889 \\
	1668 3.9889 \\
	1669 3.9889 \\
	1670 -3.9889 \\
	1671 3.9889 \\
	1672 3.9889 \\
	1673 3.9889 \\
	1674 3.9889 \\
	1675 -11.9668 \\
	1676 -3.9889 \\
	1677 11.9668 \\
	1678 11.9668 \\
	1679 -3.9889 \\
	1680 -3.9889 \\
	1681 -3.9889 \\
	1682 -3.9889 \\
	1683 11.9668 \\
	1684 3.9889 \\
	1685 3.9889 \\
	1686 -3.9889 \\
	1687 3.9889 \\
	1688 3.9889 \\
	1689 -3.9889 \\
	1690 3.9889 \\
	1691 3.9889 \\
	1692 -3.9889 \\
	1693 -11.9668 \\
	1694 -3.9889 \\
	1695 3.9889 \\
	1696 -11.9668 \\
	1697 -3.9889 \\
	1698 11.9668 \\
	1699 -11.9668 \\
	1700 -3.9889 \\
	1701 -3.9889 \\
	1702 -19.9446 \\
	1703 11.9668 \\
	1704 3.9889 \\
	1705 -3.9889 \\
	1706 -3.9889 \\
	1707 -3.9889 \\
	1708 11.9668 \\
	1709 -11.9668 \\
	1710 3.9889 \\
	1711 11.9668 \\
	1712 -3.9889 \\
	1713 3.9889 \\
	1714 3.9889 \\
	1715 11.9668 \\
	1716 -3.9889 \\
	1717 -3.9889 \\
	1718 -3.9889 \\
	1719 -3.9889 \\
	1720 3.9889 \\
	1721 -11.9668 \\
	1722 -3.9889 \\
	1723 -3.9889 \\
	1724 -11.9668 \\
	1725 -3.9889 \\
	1726 3.9889 \\
	1727 3.9889 \\
	1728 -3.9889 \\
	1729 11.9668 \\
	1730 -3.9889 \\
	1731 11.9668 \\
	1732 -3.9889 \\
	1733 -3.9889 \\
	1734 3.9889 \\
	1735 -11.9668 \\
	1736 3.9889 \\
	1737 -3.9889 \\
	1738 -11.9668 \\
	1739 -3.9889 \\
	1740 3.9889 \\
	1741 19.9446 \\
	1742 -3.9889 \\
	1743 -11.9668 \\
	1744 -3.9889 \\
	1745 3.9889 \\
	1746 -3.9889 \\
	1747 -3.9889 \\
	1748 3.9889 \\
	1749 11.9668 \\
	1750 3.9889 \\
	1751 3.9889 \\
	1752 3.9889 \\
	1753 -3.9889 \\
	1754 11.9668 \\
	1755 -3.9889 \\
	1756 -3.9889 \\
	1757 3.9889 \\
	1758 -11.9668 \\
	1759 -3.9889 \\
	1760 -3.9889 \\
	1761 -3.9889 \\
	1762 3.9889 \\
	1763 3.9889 \\
	1764 -11.9668 \\
	1765 -3.9889 \\
	1766 11.9668 \\
	1767 3.9889 \\
	1768 3.9889 \\
	1769 3.9889 \\
	1770 -3.9889 \\
	1771 -11.9668 \\
	1772 -11.9668 \\
	1773 -3.9889 \\
	1774 19.9446 \\
	1775 3.9889 \\
	1776 -3.9889 \\
	1777 11.9668 \\
	1778 11.9668 \\
	1779 -3.9889 \\
	1780 3.9889 \\
	1781 3.9889 \\
	1782 -3.9889 \\
	1783 11.9668 \\
	1784 3.9889 \\
	1785 -3.9889 \\
	1786 3.9889 \\
	1787 -3.9889 \\
	1788 -3.9889 \\
	1789 -11.9668 \\
	1790 11.9668 \\
	1791 3.9889 \\
	1792 -3.9889 \\
	1793 -19.9446 \\
	1794 3.9889 \\
	1795 3.9889 \\
	1796 -3.9889 \\
	1797 -3.9889 \\
	1798 3.9889 \\
	1799 -3.9889 \\
	1800 -3.9889 \\
	1801 -3.9889 \\
	1802 3.9889 \\
	1803 3.9889 \\
	1804 -11.9668 \\
	1805 11.9668 \\
	1806 -3.9889 \\
	1807 -19.9446 \\
	1808 3.9889 \\
	1809 -3.9889 \\
	1810 -3.9889 \\
	1811 3.9889 \\
	1812 -3.9889 \\
	1813 -11.9668 \\
	1814 3.9889 \\
	1815 -3.9889 \\
	1816 11.9668 \\
	1817 -3.9889 \\
	1818 -11.9668 \\
	1819 11.9668 \\
	1820 3.9889 \\
	1821 11.9668 \\
	1822 -11.9668 \\
	1823 -11.9668 \\
	1824 -3.9889 \\
	1825 -11.9668 \\
	1826 3.9889 \\
	1827 -3.9889 \\
	1828 19.9446 \\
	1829 3.9889 \\
	1830 3.9889 \\
	1831 19.9446 \\
	1832 3.9889 \\
	1833 19.9446 \\
	1834 3.9889 \\
	1835 -3.9889 \\
	1836 3.9889 \\
	1837 3.9889 \\
	1838 3.9889 \\
	1839 3.9889 \\
	1840 3.9889 \\
	1841 -3.9889 \\
	1842 -3.9889 \\
	1843 3.9889 \\
	1844 -3.9889 \\
	1845 3.9889 \\
	1846 3.9889 \\
	1847 -11.9668 \\
	1848 -3.9889 \\
	1849 -3.9889 \\
	1850 -11.9668 \\
	1851 3.9889 \\
	1852 3.9889 \\
	1853 3.9889 \\
	1854 3.9889 \\
	1855 11.9668 \\
	1856 -11.9668 \\
	1857 11.9668 \\
	1858 -3.9889 \\
	1859 -3.9889 \\
	1860 -3.9889 \\
	1861 -11.9668 \\
	1862 -3.9889 \\
	1863 3.9889 \\
	1864 3.9889 \\
	1865 3.9889 \\
	1866 11.9668 \\
	1867 3.9889 \\
	1868 3.9889 \\
	1869 3.9889 \\
	1870 11.9668 \\
	1871 3.9889 \\
	1872 -11.9668 \\
	1873 -3.9889 \\
	1874 -11.9668 \\
	1875 3.9889 \\
	1876 3.9889 \\
	1877 -11.9668 \\
	1878 -11.9668 \\
	1879 -3.9889 \\
	1880 3.9889 \\
	1881 -3.9889 \\
	1882 3.9889 \\
	1883 -3.9889 \\
	1884 -11.9668 \\
	1885 3.9889 \\
	1886 -19.9446 \\
	1887 -11.9668 \\
	1888 11.9668 \\
	1889 -3.9889 \\
	1890 -3.9889 \\
	1891 11.9668 \\
	1892 -3.9889 \\
	1893 3.9889 \\
	1894 3.9889 \\
	1895 -3.9889 \\
	1896 11.9668 \\
	1897 -11.9668 \\
	1898 3.9889 \\
	1899 3.9889 \\
	1900 3.9889 \\
	1901 11.9668 \\
	1902 3.9889 \\
	1903 3.9889 \\
	1904 -11.9668 \\
	1905 -3.9889 \\
	1906 3.9889 \\
	1907 -3.9889 \\
	1908 3.9889 \\
	1909 -11.9668 \\
	1910 3.9889 \\
	1911 11.9668 \\
	1912 3.9889 \\
	1913 -3.9889 \\
	1914 3.9889 \\
	1915 3.9889 \\
	1916 3.9889 \\
	1917 3.9889 \\
	1918 3.9889 \\
	1919 3.9889 \\
	1920 11.9668 \\
	1921 -11.9668 \\
	1922 3.9889 \\
	1923 -3.9889 \\
	1924 -3.9889 \\
	1925 3.9889 \\
	1926 -3.9889 \\
	1927 -3.9889 \\
	1928 11.9668 \\
	1929 3.9889 \\
	1930 -11.9668 \\
	1931 -3.9889 \\
	1932 -3.9889 \\
	1933 3.9889 \\
	1934 -3.9889 \\
	1935 -3.9889 \\
	1936 3.9889 \\
	1937 3.9889 \\
	1938 3.9889 \\
	1939 -3.9889 \\
	1940 3.9889 \\
	1941 11.9668 \\
	1942 -3.9889 \\
	1943 -11.9668 \\
	1944 3.9889 \\
	1945 11.9668 \\
	1946 -11.9668 \\
	1947 -3.9889 \\
	1948 3.9889 \\
	1949 3.9889 \\
	1950 3.9889 \\
	1951 3.9889 \\
	1952 3.9889 \\
	1953 3.9889 \\
	1954 -11.9668 \\
	1955 -3.9889 \\
	1956 3.9889 \\
	1957 3.9889 \\
	1958 11.9668 \\
	1959 -3.9889 \\
	1960 -11.9668 \\
	1961 -19.9446 \\
	1962 -19.9446 \\
	1963 3.9889 \\
	1964 -11.9668 \\
	1965 -11.9668 \\
	1966 11.9668 \\
	1967 3.9889 \\
	1968 -3.9889 \\
	1969 11.9668 \\
	1970 -3.9889 \\
	1971 -3.9889 \\
	1972 11.9668 \\
	1973 3.9889 \\
	1974 3.9889 \\
	1975 -11.9668 \\
	1976 -3.9889 \\
	1977 3.9889 \\
	1978 11.9668 \\
	1979 11.9668 \\
	1980 -3.9889 \\
	1981 -3.9889 \\
	1982 -11.9668 \\
	1983 3.9889 \\
	1984 11.9668 \\
	1985 -3.9889 \\
	1986 19.9446 \\
	1987 19.9446 \\
	1988 -3.9889 \\
	1989 3.9889 \\
	1990 3.9889 \\
	1991 3.9889 \\
	1992 -11.9668 \\
	1993 -3.9889 \\
	1994 3.9889 \\
	1995 3.9889 \\
	1996 3.9889 \\
	1997 -3.9889 \\
	1998 -3.9889 \\
	1999 3.9889 \\
	2000 11.9668 \\
	2001 3.9889 \\
	2002 -11.9668 \\
	2003 -11.9668 \\
	2004 -3.9889 \\
	2005 -11.9668 \\
	2006 -11.9668 \\
	2007 -3.9889 \\
	2008 -3.9889 \\
	2009 -3.9889 \\
	2010 3.9889 \\
	2011 3.9889 \\
	2012 -3.9889 \\
	2013 -3.9889 \\
	2014 3.9889 \\
	2015 -3.9889 \\
	2016 -3.9889 \\
	2017 3.9889 \\
	2018 -3.9889 \\
	2019 3.9889 \\
	2020 -3.9889 \\
	2021 -3.9889 \\
	2022 11.9668 \\
	2023 -11.9668 \\
	2024 -3.9889 \\
	2025 11.9668 \\
	2026 3.9889 \\
	2027 11.9668 \\
	2028 -3.9889 \\
	2029 -3.9889 \\
	2030 3.9889 \\
	2031 3.9889 \\
	2032 -3.9889 \\
	2033 3.9889 \\
	2034 3.9889 \\
	2035 11.9668 \\
	2036 3.9889 \\
	2037 -11.9668 \\
	2038 3.9889 \\
	2039 3.9889 \\
	2040 -3.9889 \\
	2041 3.9889 \\
	2042 -3.9889 \\
	2043 3.9889 \\
	2044 3.9889 \\
	2045 3.9889 \\
	2046 -3.9889 \\
	2047 -3.9889 \\
	2048 -3.9889 \\
	2049 3.9889 \\
	2050 -11.9668 \\
	2051 -3.9889 \\
	2052 -11.9668 \\
	2053 -3.9889 \\
	2054 11.9668 \\
	2055 -11.9668 \\
	2056 -3.9889 \\
	2057 3.9889 \\
	2058 11.9668 \\
	2059 19.9446 \\
	2060 3.9889 \\
	2061 3.9889 \\
	2062 3.9889 \\
	2063 3.9889 \\
	2064 -11.9668 \\
	2065 -3.9889 \\
	2066 11.9668 \\
	2067 11.9668 \\
	2068 -3.9889 \\
	2069 3.9889 \\
	2070 3.9889 \\
	2071 -11.9668 \\
	2072 -3.9889 \\
	2073 -3.9889 \\
	2074 -11.9668 \\
	2075 -3.9889 \\
	2076 3.9889 \\
	2077 3.9889 \\
	2078 -3.9889 \\
	2079 -3.9889 \\
	2080 3.9889 \\
	2081 -11.9668 \\
	2082 -3.9889 \\
	2083 3.9889 \\
	2084 -3.9889 \\
	2085 11.9668 \\
	2086 3.9889 \\
	2087 3.9889 \\
	2088 3.9889 \\
	2089 3.9889 \\
	2090 11.9668 \\
	2091 -3.9889 \\
	2092 -3.9889 \\
	2093 3.9889 \\
	2094 3.9889 \\
	2095 -3.9889 \\
	2096 -3.9889 \\
	2097 11.9668 \\
	2098 -3.9889 \\
	2099 -3.9889 \\
	2100 -3.9889 \\
	2101 3.9889 \\
	2102 11.9668 \\
	2103 3.9889 \\
	2104 3.9889 \\
	2105 -11.9668 \\
	2106 -3.9889 \\
	2107 3.9889 \\
	2108 11.9668 \\
	2109 11.9668 \\
	2110 -3.9889 \\
	2111 -11.9668 \\
	2112 -3.9889 \\
	2113 11.9668 \\
	2114 -3.9889 \\
	2115 -11.9668 \\
	2116 3.9889 \\
	2117 -3.9889 \\
	2118 -3.9889 \\
	2119 -3.9889 \\
	2120 3.9889 \\
	2121 -3.9889 \\
	2122 -3.9889 \\
	2123 3.9889 \\
	2124 -3.9889 \\
	2125 11.9668 \\
	2126 3.9889 \\
	2127 -11.9668 \\
	2128 -19.9446 \\
	2129 -11.9668 \\
	2130 -3.9889 \\
	2131 -3.9889 \\
	2132 3.9889 \\
	2133 -3.9889 \\
	2134 11.9668 \\
	2135 3.9889 \\
	2136 -11.9668 \\
	2137 3.9889 \\
	2138 11.9668 \\
	2139 11.9668 \\
	2140 3.9889 \\
	2141 11.9668 \\
	2142 3.9889 \\
	2143 11.9668 \\
	2144 -3.9889 \\
	2145 -3.9889 \\
	2146 11.9668 \\
	2147 -11.9668 \\
	2148 -3.9889 \\
	2149 3.9889 \\
	2150 3.9889 \\
	2151 -3.9889 \\
	2152 -11.9668 \\
	2153 11.9668 \\
	2154 -3.9889 \\
	2155 -3.9889 \\
	2156 11.9668 \\
	2157 -3.9889 \\
	2158 3.9889 \\
	2159 3.9889 \\
	2160 -3.9889 \\
	2161 -3.9889 \\
	2162 -3.9889 \\
	2163 3.9889 \\
	2164 3.9889 \\
	2165 -11.9668 \\
	2166 -3.9889 \\
	2167 -3.9889 \\
	2168 3.9889 \\
	2169 3.9889 \\
	2170 3.9889 \\
	2171 -3.9889 \\
	2172 -3.9889 \\
	2173 11.9668 \\
	2174 3.9889 \\
	2175 -3.9889 \\
	2176 3.9889 \\
	2177 -3.9889 \\
	2178 11.9668 \\
	2179 11.9668 \\
	2180 -3.9889 \\
	2181 3.9889 \\
	2182 -3.9889 \\
	2183 -3.9889 \\
	2184 -3.9889 \\
	2185 3.9889 \\
	2186 -3.9889 \\
	2187 3.9889 \\
	2188 3.9889 \\
	2189 3.9889 \\
	2190 -3.9889 \\
	2191 -11.9668 \\
	2192 3.9889 \\
	2193 11.9668 \\
	2194 11.9668 \\
	2195 -11.9668 \\
	2196 -11.9668 \\
	2197 3.9889 \\
	2198 -11.9668 \\
	2199 3.9889 \\
	2200 3.9889 \\
	2201 -3.9889 \\
	2202 -11.9668 \\
	2203 -3.9889 \\
	2204 3.9889 \\
	2205 3.9889 \\
	2206 -3.9889 \\
	2207 -3.9889 \\
	2208 -3.9889 \\
	2209 -11.9668 \\
	2210 3.9889 \\
	2211 -3.9889 \\
	2212 11.9668 \\
	2213 11.9668 \\
	2214 -3.9889 \\
	2215 11.9668 \\
	2216 -11.9668 \\
	2217 -3.9889 \\
	2218 3.9889 \\
	2219 -3.9889 \\
	2220 11.9668 \\
	2221 -3.9889 \\
	2222 3.9889 \\
	2223 3.9889 \\
	2224 -3.9889 \\
	2225 -3.9889 \\
	2226 -3.9889 \\
	2227 -11.9668 \\
	2228 -3.9889 \\
	2229 3.9889 \\
	2230 3.9889 \\
	2231 -3.9889 \\
	2232 -11.9668 \\
	2233 -11.9668 \\
	2234 -3.9889 \\
	2235 -3.9889 \\
	2236 -3.9889 \\
	2237 3.9889 \\
	2238 -3.9889 \\
	2239 11.9668 \\
	2240 19.9446 \\
	2241 3.9889 \\
	2242 3.9889 \\
	2243 3.9889 \\
	2244 3.9889 \\
	2245 3.9889 \\
	2246 -3.9889 \\
	2247 3.9889 \\
	2248 3.9889 \\
	2249 -3.9889 \\
	2250 -3.9889 \\
	2251 3.9889 \\
	2252 3.9889 \\
	2253 11.9668 \\
	2254 11.9668 \\
	2255 11.9668 \\
	2256 -3.9889 \\
	2257 -3.9889 \\
	2258 3.9889 \\
	2259 -11.9668 \\
	2260 11.9668 \\
	2261 -3.9889 \\
	2262 -11.9668 \\
	2263 -3.9889 \\
	2264 -11.9668 \\
	2265 19.9446 \\
	2266 -3.9889 \\
	2267 -11.9668 \\
	2268 3.9889 \\
	2269 -3.9889 \\
	2270 -3.9889 \\
	2271 3.9889 \\
	2272 11.9668 \\
	2273 -3.9889 \\
	2274 19.9446 \\
	2275 3.9889 \\
	2276 -3.9889 \\
	2277 11.9668 \\
	2278 -11.9668 \\
	2279 -3.9889 \\
	2280 -3.9889 \\
	2281 -3.9889 \\
	2282 -11.9668 \\
	2283 -3.9889 \\
	2284 19.9446 \\
	2285 -3.9889 \\
	2286 -3.9889 \\
	2287 3.9889 \\
	2288 -3.9889 \\
	2289 -3.9889 \\
	2290 -3.9889 \\
	2291 -3.9889 \\
	2292 -3.9889 \\
	2293 3.9889 \\
	2294 11.9668 \\
	2295 -3.9889 \\
	2296 -3.9889 \\
	2297 -3.9889 \\
	2298 -11.9668 \\
	2299 3.9889 \\
	2300 3.9889 \\
	2301 -3.9889 \\
	2302 -3.9889 \\
	2303 -3.9889 \\
	2304 3.9889 \\
	2305 -3.9889 \\
	2306 3.9889 \\
	2307 11.9668 \\
	2308 3.9889 \\
	2309 -3.9889 \\
	2310 -3.9889 \\
	2311 -3.9889 \\
	2312 3.9889 \\
	2313 -3.9889 \\
	2314 3.9889 \\
	2315 3.9889 \\
	2316 -3.9889 \\
	2317 3.9889 \\
	2318 -3.9889 \\
	2319 3.9889 \\
	2320 3.9889 \\
	2321 -3.9889 \\
	2322 11.9668 \\
	2323 3.9889 \\
	2324 -11.9668 \\
	2325 -3.9889 \\
	2326 -11.9668 \\
	2327 -3.9889 \\
	2328 3.9889 \\
	2329 3.9889 \\
	2330 -11.9668 \\
	2331 -27.9225 \\
	2332 11.9668 \\
	2333 -3.9889 \\
	2334 -19.9446 \\
	2335 11.9668 \\
	2336 11.9668 \\
	2337 11.9668 \\
	2338 3.9889 \\
	2339 -11.9668 \\
	2340 3.9889 \\
	2341 11.9668 \\
	2342 -3.9889 \\
	2343 -3.9889 \\
	2344 3.9889 \\
	2345 3.9889 \\
	2346 -3.9889 \\
	2347 3.9889 \\
	2348 11.9668 \\
	2349 3.9889 \\
	2350 11.9668 \\
	2351 3.9889 \\
	2352 3.9889 \\
	2353 -11.9668 \\
	2354 -11.9668 \\
	2355 3.9889 \\
	2356 -19.9446 \\
	2357 -3.9889 \\
	2358 -11.9668 \\
	2359 -3.9889 \\
	2360 11.9668 \\
	2361 -3.9889 \\
	2362 3.9889 \\
	2363 -11.9668 \\
	2364 -3.9889 \\
	2365 11.9668 \\
	2366 11.9668 \\
	2367 11.9668 \\
	2368 3.9889 \\
	2369 -3.9889 \\
	2370 3.9889 \\
	2371 -3.9889 \\
	2372 3.9889 \\
	2373 3.9889 \\
	2374 -11.9668 \\
	2375 3.9889 \\
	2376 11.9668 \\
	2377 -19.9446 \\
	2378 -3.9889 \\
	2379 11.9668 \\
	2380 3.9889 \\
	2381 19.9446 \\
	2382 11.9668 \\
	2383 3.9889 \\
	2384 11.9668 \\
	2385 3.9889 \\
	2386 -3.9889 \\
	2387 3.9889 \\
	2388 -3.9889 \\
	2389 -3.9889 \\
	2390 -3.9889 \\
	2391 -3.9889 \\
	2392 3.9889 \\
	2393 3.9889 \\
	2394 3.9889 \\
	2395 3.9889 \\
	2396 -3.9889 \\
	2397 3.9889 \\
	2398 3.9889 \\
	2399 -3.9889 \\
	2400 -3.9889 \\
	2401 -3.9889 \\
	2402 11.9668 \\
	2403 -3.9889 \\
	2404 3.9889 \\
	2405 3.9889 \\
	2406 -3.9889 \\
	2407 -3.9889 \\
	2408 3.9889 \\
	2409 11.9668 \\
	2410 3.9889 \\
	2411 11.9668 \\
	2412 3.9889 \\
	2413 -11.9668 \\
	2414 3.9889 \\
	2415 -11.9668 \\
	2416 -19.9446 \\
	2417 3.9889 \\
	2418 3.9889 \\
	2419 -11.9668 \\
	2420 -11.9668 \\
	2421 3.9889 \\
	2422 3.9889 \\
	2423 -11.9668 \\
	2424 -3.9889 \\
	2425 11.9668 \\
	2426 11.9668 \\
	2427 -3.9889 \\
	2428 -11.9668 \\
	2429 3.9889 \\
	2430 3.9889 \\
	2431 -3.9889 \\
	2432 -3.9889 \\
	2433 11.9668 \\
	2434 -3.9889 \\
	2435 3.9889 \\
	2436 11.9668 \\
	2437 3.9889 \\
	2438 3.9889 \\
	2439 -3.9889 \\
	2440 -3.9889 \\
	2441 -11.9668 \\
	2442 3.9889 \\
	2443 -3.9889 \\
	2444 -11.9668 \\
	2445 11.9668 \\
	2446 3.9889 \\
	2447 -3.9889 \\
	2448 -3.9889 \\
	2449 -3.9889 \\
	2450 11.9668 \\
	2451 -3.9889 \\
	2452 -3.9889 \\
	2453 3.9889 \\
	2454 -11.9668 \\
	2455 -3.9889 \\
	2456 11.9668 \\
	2457 -3.9889 \\
	2458 -11.9668 \\
	2459 -3.9889 \\
	2460 3.9889 \\
	2461 3.9889 \\
	2462 -11.9668 \\
	2463 -3.9889 \\
	2464 3.9889 \\
	2465 19.9446 \\
	2466 11.9668 \\
	2467 -3.9889 \\
	2468 -11.9668 \\
	2469 3.9889 \\
	2470 3.9889 \\
	2471 3.9889 \\
	2472 3.9889 \\
	2473 -11.9668 \\
	2474 3.9889 \\
	2475 3.9889 \\
	2476 3.9889 \\
	2477 11.9668 \\
	2478 -3.9889 \\
	2479 -3.9889 \\
	2480 -11.9668 \\
	2481 -3.9889 \\
	2482 3.9889 \\
	2483 3.9889 \\
	2484 3.9889 \\
	2485 -11.9668 \\
	2486 11.9668 \\
	2487 -3.9889 \\
	2488 -11.9668 \\
	2489 3.9889 \\
	2490 -3.9889 \\
	2491 3.9889 \\
	2492 -3.9889 \\
	2493 -3.9889 \\
	2494 3.9889 \\
	2495 -3.9889 \\
	2496 3.9889 \\
	2497 -3.9889 \\
	2498 3.9889 \\
	2499 3.9889 \\
	2500 3.9889 \\
	2501 -3.9889 \\
	2502 3.9889 \\
	2503 3.9889 \\
	2504 -3.9889 \\
	2505 11.9668 \\
	2506 -3.9889 \\
	2507 -3.9889 \\
	2508 3.9889 \\
	2509 3.9889 \\
	2510 3.9889 \\
	2511 -19.9446 \\
	2512 3.9889 \\
	2513 3.9889 \\
	2514 3.9889 \\
	2515 3.9889 \\
	2516 -11.9668 \\
	2517 3.9889 \\
	2518 11.9668 \\
	2519 -3.9889 \\
	2520 -3.9889 \\
	2521 3.9889 \\
	2522 -3.9889 \\
	2523 -3.9889 \\
	2524 -3.9889 \\
	2525 -3.9889 \\
	2526 -11.9668 \\
	2527 -3.9889 \\
	2528 27.9225 \\
	2529 3.9889 \\
	2530 -3.9889 \\
	2531 -3.9889 \\
	2532 -3.9889 \\
	2533 -11.9668 \\
	2534 -3.9889 \\
	2535 -3.9889 \\
	2536 -11.9668 \\
	2537 3.9889 \\
	2538 3.9889 \\
	2539 -3.9889 \\
	2540 -3.9889 \\
	2541 -11.9668 \\
	2542 3.9889 \\
	2543 -3.9889 \\
	2544 3.9889 \\
	2545 3.9889 \\
	2546 3.9889 \\
	2547 -3.9889 \\
	2548 -3.9889 \\
	2549 3.9889 \\
	2550 3.9889 \\
	2551 3.9889 \\
	2552 3.9889 \\
	2553 3.9889 \\
	2554 3.9889 \\
	2555 3.9889 \\
	2556 -11.9668 \\
	2557 -3.9889 \\
	2558 3.9889 \\
	2559 3.9889 \\
	2560 3.9889 \\
	2561 3.9889 \\
	2562 3.9889 \\
	2563 -3.9889 \\
	2564 -3.9889 \\
	2565 -3.9889 \\
	2566 -3.9889 \\
	2567 -11.9668 \\
	2568 -3.9889 \\
	2569 3.9889 \\
	2570 -3.9889 \\
	2571 -3.9889 \\
	2572 -3.9889 \\
	2573 -3.9889 \\
	2574 -11.9668 \\
	2575 -3.9889 \\
	2576 3.9889 \\
	2577 3.9889 \\
	2578 -3.9889 \\
	2579 3.9889 \\
	2580 -3.9889 \\
	2581 -11.9668 \\
	2582 11.9668 \\
	2583 3.9889 \\
	2584 3.9889 \\
	2585 -3.9889 \\
	2586 3.9889 \\
	2587 11.9668 \\
	2588 -11.9668 \\
	2589 3.9889 \\
	2590 -3.9889 \\
	2591 -11.9668 \\
	2592 3.9889 \\
	2593 11.9668 \\
	2594 3.9889 \\
	2595 -3.9889 \\
	2596 11.9668 \\
	2597 3.9889 \\
	2598 11.9668 \\
	2599 11.9668 \\
	2600 3.9889 \\
	2601 -3.9889 \\
	2602 -11.9668 \\
	2603 3.9889 \\
	2604 3.9889 \\
	2605 3.9889 \\
	2606 11.9668 \\
	2607 11.9668 \\
	2608 -3.9889 \\
	2609 -3.9889 \\
	2610 -3.9889 \\
	2611 -3.9889 \\
	2612 11.9668 \\
	2613 -11.9668 \\
	2614 -19.9446 \\
	2615 3.9889 \\
	2616 3.9889 \\
	2617 3.9889 \\
	2618 3.9889 \\
	2619 11.9668 \\
	2620 3.9889 \\
	2621 -3.9889 \\
	2622 -3.9889 \\
	2623 3.9889 \\
	2624 3.9889 \\
	2625 11.9668 \\
	2626 -3.9889 \\
	2627 3.9889 \\
	2628 -3.9889 \\
	2629 -11.9668 \\
	2630 -3.9889 \\
	2631 -3.9889 \\
	2632 3.9889 \\
	2633 -3.9889 \\
	2634 3.9889 \\
	2635 3.9889 \\
	2636 -3.9889 \\
	2637 -11.9668 \\
	2638 3.9889 \\
	2639 -3.9889 \\
	2640 -3.9889 \\
	2641 11.9668 \\
	2642 3.9889 \\
	2643 -3.9889 \\
	2644 -3.9889 \\
	2645 3.9889 \\
	2646 -3.9889 \\
	2647 3.9889 \\
	2648 19.9446 \\
	2649 11.9668 \\
	2650 -3.9889 \\
	2651 -3.9889 \\
	2652 3.9889 \\
	2653 -3.9889 \\
	2654 3.9889 \\
	2655 -3.9889 \\
	2656 -11.9668 \\
	2657 3.9889 \\
	2658 -11.9668 \\
	2659 -3.9889 \\
	2660 3.9889 \\
	2661 -3.9889 \\
	2662 -3.9889 \\
	2663 3.9889 \\
	2664 -3.9889 \\
	2665 3.9889 \\
	2666 3.9889 \\
	2667 -3.9889 \\
	2668 11.9668 \\
	2669 19.9446 \\
	2670 11.9668 \\
	2671 -3.9889 \\
	2672 -3.9889 \\
	2673 3.9889 \\
	2674 -3.9889 \\
	2675 -3.9889 \\
	2676 3.9889 \\
	2677 3.9889 \\
	2678 -11.9668 \\
	2679 11.9668 \\
	2680 3.9889 \\
	2681 3.9889 \\
	2682 11.9668 \\
	2683 -3.9889 \\
	2684 3.9889 \\
	2685 -11.9668 \\
	2686 -11.9668 \\
	2687 -3.9889 \\
	2688 3.9889 \\
	2689 -11.9668 \\
	2690 3.9889 \\
	2691 19.9446 \\
	2692 11.9668 \\
	2693 3.9889 \\
	2694 3.9889 \\
	2695 3.9889 \\
	2696 3.9889 \\
	2697 11.9668 \\
	2698 -3.9889 \\
	2699 3.9889 \\
	2700 -3.9889 \\
	2701 3.9889 \\
	2702 3.9889 \\
	2703 -11.9668 \\
	2704 19.9446 \\
	2705 -3.9889 \\
	2706 -3.9889 \\
	2707 3.9889 \\
	2708 -11.9668 \\
	2709 11.9668 \\
	2710 -3.9889 \\
	2711 -11.9668 \\
	2712 -3.9889 \\
	2713 3.9889 \\
	2714 3.9889 \\
	2715 3.9889 \\
	2716 11.9668 \\
	2717 -3.9889 \\
	2718 -3.9889 \\
	2719 -3.9889 \\
	2720 3.9889 \\
	2721 -3.9889 \\
	2722 -11.9668 \\
	2723 3.9889 \\
	2724 3.9889 \\
	2725 -11.9668 \\
	2726 -3.9889 \\
	2727 -3.9889 \\
	2728 -3.9889 \\
	2729 3.9889 \\
	2730 11.9668 \\
	2731 11.9668 \\
	2732 -3.9889 \\
	2733 -3.9889 \\
	2734 -3.9889 \\
	2735 -3.9889 \\
	2736 -3.9889 \\
	2737 -3.9889 \\
	2738 -3.9889 \\
	2739 -3.9889 \\
	2740 3.9889 \\
	2741 -3.9889 \\
	2742 -3.9889 \\
	2743 -3.9889 \\
	2744 -11.9668 \\
	2745 3.9889 \\
	2746 -11.9668 \\
	2747 -3.9889 \\
	2748 -3.9889 \\
	2749 3.9889 \\
	2750 11.9668 \\
	2751 -3.9889 \\
	2752 -11.9668 \\
	2753 -11.9668 \\
	2754 11.9668 \\
	2755 3.9889 \\
	2756 -3.9889 \\
	2757 -3.9889 \\
	2758 -11.9668 \\
	2759 3.9889 \\
	2760 3.9889 \\
	2761 3.9889 \\
	2762 3.9889 \\
	2763 -3.9889 \\
	2764 -3.9889 \\
	2765 -3.9889 \\
	2766 3.9889 \\
	2767 11.9668 \\
	2768 19.9446 \\
	2769 -3.9889 \\
	2770 -27.9225 \\
	2771 -3.9889 \\
	2772 3.9889 \\
	2773 -3.9889 \\
	2774 -3.9889 \\
	2775 -3.9889 \\
	2776 19.9446 \\
	2777 -3.9889 \\
	2778 3.9889 \\
	2779 11.9668 \\
	2780 -11.9668 \\
	2781 -11.9668 \\
	2782 -19.9446 \\
	2783 3.9889 \\
	2784 -3.9889 \\
	2785 -11.9668 \\
	2786 -3.9889 \\
	2787 -3.9889 \\
	2788 3.9889 \\
	2789 -3.9889 \\
	2790 3.9889 \\
	2791 -3.9889 \\
	2792 -3.9889 \\
	2793 11.9668 \\
	2794 -11.9668 \\
	2795 -11.9668 \\
	2796 3.9889 \\
	2797 -3.9889 \\
	2798 -3.9889 \\
	2799 11.9668 \\
	2800 -11.9668 \\
	2801 3.9889 \\
	2802 11.9668 \\
	2803 -3.9889 \\
	2804 11.9668 \\
	2805 3.9889 \\
	2806 11.9668 \\
	2807 3.9889 \\
	2808 3.9889 \\
	2809 3.9889 \\
	2810 -3.9889 \\
	2811 11.9668 \\
	2812 11.9668 \\
	2813 3.9889 \\
	2814 3.9889 \\
	2815 3.9889 \\
	2816 -11.9668 \\
	2817 -3.9889 \\
	2818 11.9668 \\
	2819 -3.9889 \\
	2820 -11.9668 \\
	2821 -3.9889 \\
	2822 -3.9889 \\
	2823 -3.9889 \\
	2824 -3.9889 \\
	2825 3.9889 \\
	2826 3.9889 \\
	2827 -3.9889 \\
	2828 3.9889 \\
	2829 -3.9889 \\
	2830 -3.9889 \\
	2831 3.9889 \\
	2832 3.9889 \\
	2833 3.9889 \\
	2834 -11.9668 \\
	2835 3.9889 \\
	2836 3.9889 \\
	2837 3.9889 \\
	2838 3.9889 \\
	2839 -27.9225 \\
	2840 -3.9889 \\
	2841 3.9889 \\
	2842 11.9668 \\
	2843 3.9889 \\
	2844 11.9668 \\
	2845 3.9889 \\
	2846 -11.9668 \\
	2847 3.9889 \\
	2848 3.9889 \\
	2849 11.9668 \\
	2850 11.9668 \\
	2851 11.9668 \\
	2852 11.9668 \\
	2853 -3.9889 \\
	2854 -11.9668 \\
	2855 -11.9668 \\
	2856 11.9668 \\
	2857 -11.9668 \\
	2858 -3.9889 \\
	2859 -3.9889 \\
	2860 -3.9889 \\
	2861 3.9889 \\
	2862 -11.9668 \\
	2863 -3.9889 \\
	2864 3.9889 \\
	2865 11.9668 \\
	2866 -3.9889 \\
	2867 -11.9668 \\
	2868 11.9668 \\
	2869 -3.9889 \\
	2870 -3.9889 \\
	2871 -3.9889 \\
	2872 -3.9889 \\
	2873 -3.9889 \\
	2874 3.9889 \\
	2875 -3.9889 \\
	2876 3.9889 \\
	2877 19.9446 \\
	2878 3.9889 \\
	2879 3.9889 \\
	2880 -3.9889 \\
	2881 3.9889 \\
	2882 3.9889 \\
	2883 -11.9668 \\
	2884 3.9889 \\
	2885 -3.9889 \\
	2886 -3.9889 \\
	2887 -3.9889 \\
	2888 -3.9889 \\
	2889 3.9889 \\
	2890 -3.9889 \\
	2891 -3.9889 \\
	2892 -3.9889 \\
	2893 11.9668 \\
	2894 19.9446 \\
	2895 -3.9889 \\
	2896 -3.9889 \\
	2897 3.9889 \\
	2898 -3.9889 \\
	2899 3.9889 \\
	2900 3.9889 \\
	2901 -3.9889 \\
	2902 3.9889 \\
	2903 3.9889 \\
	2904 -3.9889 \\
	2905 -11.9668 \\
	2906 3.9889 \\
	2907 11.9668 \\
	2908 3.9889 \\
	2909 -3.9889 \\
	2910 -3.9889 \\
	2911 -3.9889 \\
	2912 3.9889 \\
	2913 -3.9889 \\
	2914 -3.9889 \\
	2915 11.9668 \\
	2916 -3.9889 \\
	2917 -3.9889 \\
	2918 3.9889 \\
	2919 -3.9889 \\
	2920 11.9668 \\
	2921 3.9889 \\
	2922 -11.9668 \\
	2923 -3.9889 \\
	2924 -3.9889 \\
	2925 -3.9889 \\
	2926 -3.9889 \\
	2927 11.9668 \\
	2928 19.9446 \\
	2929 11.9668 \\
	2930 -3.9889 \\
	2931 3.9889 \\
	2932 -3.9889 \\
	2933 -3.9889 \\
	2934 19.9446 \\
	2935 -3.9889 \\
	2936 -11.9668 \\
	2937 -11.9668 \\
	2938 11.9668 \\
	2939 3.9889 \\
	2940 -19.9446 \\
	2941 3.9889 \\
	2942 -3.9889 \\
	2943 3.9889 \\
	2944 3.9889 \\
	2945 -3.9889 \\
	2946 3.9889 \\
	2947 -11.9668 \\
	2948 3.9889 \\
	2949 -11.9668 \\
	2950 -11.9668 \\
	2951 -11.9668 \\
	2952 3.9889 \\
	2953 19.9446 \\
	2954 -3.9889 \\
	2955 -3.9889 \\
	2956 3.9889 \\
	2957 3.9889 \\
	2958 11.9668 \\
	2959 19.9446 \\
	2960 -3.9889 \\
	2961 -11.9668 \\
	2962 3.9889 \\
	2963 -11.9668 \\
	2964 -11.9668 \\
	2965 3.9889 \\
	2966 3.9889 \\
	2967 3.9889 \\
	2968 -11.9668 \\
	2969 -11.9668 \\
	2970 3.9889 \\
	2971 -11.9668 \\
	2972 -3.9889 \\
	2973 -3.9889 \\
	2974 -3.9889 \\
	2975 11.9668 \\
	2976 -3.9889 \\
	2977 11.9668 \\
	2978 -3.9889 \\
	2979 -3.9889 \\
	2980 -3.9889 \\
	2981 -3.9889 \\
	2982 11.9668 \\
	2983 3.9889 \\
	2984 11.9668 \\
	2985 3.9889 \\
	2986 11.9668 \\
	2987 -3.9889 \\
	2988 -11.9668 \\
	2989 11.9668 \\
	2990 -3.9889 \\
	2991 3.9889 \\
	2992 -3.9889 \\
	2993 -11.9668 \\
	2994 3.9889 \\
	2995 3.9889 \\
	2996 -3.9889 \\
	2997 11.9668 \\
	2998 3.9889 \\
	2999 -11.9668 \\
	3000 11.9668 \\
	3001 -3.9889 \\
	3002 -3.9889 \\
	3003 -3.9889 \\
	3004 3.9889 \\
	3005 3.9889 \\
	3006 -3.9889 \\
	3007 11.9668 \\
	3008 -11.9668 \\
	3009 11.9668 \\
	3010 11.9668 \\
	3011 -3.9889 \\
	3012 -3.9889 \\
	3013 3.9889 \\
	3014 3.9889 \\
	3015 -3.9889 \\
	3016 -11.9668 \\
	3017 3.9889 \\
	3018 3.9889 \\
	3019 -3.9889 \\
	3020 -11.9668 \\
	3021 -3.9889 \\
	3022 -3.9889 \\
	3023 11.9668 \\
	3024 11.9668 \\
	3025 -3.9889 \\
	3026 11.9668 \\
	3027 -3.9889 \\
	3028 -11.9668 \\
	3029 -3.9889 \\
	3030 11.9668 \\
	3031 3.9889 \\
	3032 -11.9668 \\
	3033 3.9889 \\
	3034 -3.9889 \\
	3035 11.9668 \\
	3036 3.9889 \\
	3037 -11.9668 \\
	3038 11.9668 \\
	3039 -3.9889 \\
	3040 3.9889 \\
	3041 3.9889 \\
	3042 -3.9889 \\
	3043 -11.9668 \\
	3044 -3.9889 \\
	3045 3.9889 \\
	3046 -11.9668 \\
	3047 3.9889 \\
	3048 -3.9889 \\
	3049 3.9889 \\
	3050 -3.9889 \\
	3051 -3.9889 \\
	3052 3.9889 \\
	3053 3.9889 \\
	3054 19.9446 \\
	3055 -3.9889 \\
	3056 -3.9889 \\
	3057 -3.9889 \\
	3058 -3.9889 \\
	3059 -11.9668 \\
	3060 -3.9889 \\
	3061 11.9668 \\
	3062 -11.9668 \\
	3063 -3.9889 \\
	3064 3.9889 \\
	3065 3.9889 \\
	3066 3.9889 \\
	3067 3.9889 \\
	3068 3.9889 \\
	3069 -3.9889 \\
	3070 -3.9889 \\
	3071 3.9889 \\
	3072 3.9889 \\
	3073 -11.9668 \\
	3074 -3.9889 \\
	3075 3.9889 \\
	3076 -11.9668 \\
	3077 3.9889 \\
	3078 -3.9889 \\
	3079 -11.9668 \\
	3080 3.9889 \\
	3081 -3.9889 \\
	3082 3.9889 \\
	3083 3.9889 \\
	3084 3.9889 \\
	3085 19.9446 \\
	3086 3.9889 \\
	3087 -11.9668 \\
	3088 3.9889 \\
	3089 3.9889 \\
	3090 11.9668 \\
	3091 3.9889 \\
	3092 -3.9889 \\
	3093 -3.9889 \\
	3094 -3.9889 \\
	3095 3.9889 \\
	3096 -3.9889 \\
	3097 -19.9446 \\
	3098 -3.9889 \\
	3099 -3.9889 \\
	3100 3.9889 \\
	3101 3.9889 \\
	3102 -3.9889 \\
	3103 -3.9889 \\
	3104 3.9889 \\
	3105 -3.9889 \\
	3106 -11.9668 \\
	3107 11.9668 \\
	3108 -3.9889 \\
	3109 -3.9889 \\
	3110 11.9668 \\
	3111 3.9889 \\
	3112 -3.9889 \\
	3113 -11.9668 \\
	3114 -11.9668 \\
	3115 -3.9889 \\
	3116 -3.9889 \\
	3117 11.9668 \\
	3118 11.9668 \\
	3119 3.9889 \\
	3120 -3.9889 \\
	3121 3.9889 \\
	3122 11.9668 \\
	3123 -3.9889 \\
	3124 -3.9889 \\
	3125 -3.9889 \\
	3126 -3.9889 \\
	3127 -3.9889 \\
	3128 -19.9446 \\
	3129 11.9668 \\
	3130 3.9889 \\
	3131 11.9668 \\
	3132 11.9668 \\
	3133 3.9889 \\
	3134 -3.9889 \\
	3135 -11.9668 \\
	3136 3.9889 \\
	3137 3.9889 \\
	3138 3.9889 \\
	3139 -3.9889 \\
	3140 -3.9889 \\
	3141 3.9889 \\
	3142 11.9668 \\
	3143 3.9889 \\
	3144 -3.9889 \\
	3145 3.9889 \\
	3146 -11.9668 \\
	3147 -3.9889 \\
	3148 3.9889 \\
	3149 -3.9889 \\
	3150 11.9668 \\
	3151 3.9889 \\
	3152 -3.9889 \\
	3153 -3.9889 \\
	3154 -3.9889 \\
	3155 -11.9668 \\
	3156 -11.9668 \\
	3157 3.9889 \\
	3158 -11.9668 \\
	3159 -11.9668 \\
	3160 -3.9889 \\
	3161 -3.9889 \\
	3162 3.9889 \\
	3163 -3.9889 \\
	3164 -11.9668 \\
	3165 3.9889 \\
	3166 -3.9889 \\
	3167 3.9889 \\
	3168 11.9668 \\
	3169 -11.9668 \\
	3170 -3.9889 \\
	3171 -3.9889 \\
	3172 3.9889 \\
	3173 3.9889 \\
	3174 -11.9668 \\
	3175 3.9889 \\
	3176 11.9668 \\
	3177 19.9446 \\
	3178 3.9889 \\
	3179 -3.9889 \\
	3180 3.9889 \\
	3181 -3.9889 \\
	3182 3.9889 \\
	3183 -3.9889 \\
	3184 3.9889 \\
	3185 11.9668 \\
	3186 -3.9889 \\
	3187 3.9889 \\
	3188 11.9668 \\
	3189 -3.9889 \\
	3190 -3.9889 \\
	3191 -11.9668 \\
	3192 -3.9889 \\
	3193 11.9668 \\
	3194 3.9889 \\
	3195 3.9889 \\
	3196 -3.9889 \\
	3197 -11.9668 \\
	3198 -3.9889 \\
	3199 11.9668 \\
	3200 11.9668 \\
	3201 19.9446 \\
	3202 -3.9889 \\
	3203 3.9889 \\
	3204 3.9889 \\
	3205 -11.9668 \\
	3206 11.9668 \\
	3207 -3.9889 \\
	3208 11.9668 \\
	3209 3.9889 \\
	3210 -3.9889 \\
	3211 3.9889 \\
	3212 3.9889 \\
	3213 -3.9889 \\
	3214 -11.9668 \\
	3215 3.9889 \\
	3216 3.9889 \\
	3217 3.9889 \\
	3218 -11.9668 \\
	3219 -19.9446 \\
	3220 3.9889 \\
	3221 3.9889 \\
	3222 3.9889 \\
	3223 3.9889 \\
	3224 3.9889 \\
	3225 3.9889 \\
	3226 11.9668 \\
	3227 3.9889 \\
	3228 -3.9889 \\
	3229 -3.9889 \\
	3230 3.9889 \\
	3231 -3.9889 \\
	3232 -11.9668 \\
	3233 3.9889 \\
	3234 -3.9889 \\
	3235 3.9889 \\
	3236 3.9889 \\
	3237 -3.9889 \\
	3238 -3.9889 \\
	3239 -11.9668 \\
	3240 -3.9889 \\
	3241 3.9889 \\
	3242 11.9668 \\
	3243 3.9889 \\
	3244 3.9889 \\
	3245 -3.9889 \\
	3246 -11.9668 \\
	3247 -3.9889 \\
	3248 11.9668 \\
	3249 -3.9889 \\
	3250 -11.9668 \\
	3251 11.9668 \\
	3252 -11.9668 \\
	3253 -11.9668 \\
	3254 11.9668 \\
	3255 -3.9889 \\
	3256 -3.9889 \\
	3257 -11.9668 \\
	3258 -3.9889 \\
	3259 -3.9889 \\
	3260 -3.9889 \\
	3261 11.9668 \\
	3262 11.9668 \\
	3263 3.9889 \\
	3264 3.9889 \\
	3265 3.9889 \\
	3266 -3.9889 \\
	3267 -3.9889 \\
	3268 11.9668 \\
	3269 3.9889 \\
	3270 3.9889 \\
	3271 3.9889 \\
	3272 -3.9889 \\
	3273 11.9668 \\
	3274 3.9889 \\
	3275 11.9668 \\
	3276 3.9889 \\
	3277 -11.9668 \\
	3278 3.9889 \\
	3279 -3.9889 \\
	3280 3.9889 \\
	3281 3.9889 \\
	3282 3.9889 \\
	3283 3.9889 \\
	3284 -11.9668 \\
	3285 3.9889 \\
	3286 -11.9668 \\
	3287 3.9889 \\
	3288 -3.9889 \\
	3289 -11.9668 \\
	3290 11.9668 \\
	3291 11.9668 \\
	3292 3.9889 \\
	3293 3.9889 \\
	3294 -3.9889 \\
	3295 -3.9889 \\
	3296 -3.9889 \\
	3297 -3.9889 \\
	3298 3.9889 \\
	3299 3.9889 \\
	3300 -3.9889 \\
	3301 -11.9668 \\
	3302 -11.9668 \\
	3303 3.9889 \\
	3304 11.9668 \\
	3305 3.9889 \\
	3306 -3.9889 \\
	3307 11.9668 \\
	3308 3.9889 \\
	3309 -3.9889 \\
	3310 11.9668 \\
	3311 11.9668 \\
	3312 -3.9889 \\
	3313 -3.9889 \\
	3314 11.9668 \\
	3315 -11.9668 \\
	3316 -19.9446 \\
	3317 -3.9889 \\
	3318 -3.9889 \\
	3319 3.9889 \\
	3320 11.9668 \\
	3321 3.9889 \\
	3322 -11.9668 \\
	3323 -3.9889 \\
	3324 -3.9889 \\
	3325 -3.9889 \\
	3326 3.9889 \\
	3327 -11.9668 \\
	3328 -3.9889 \\
	3329 -3.9889 \\
	3330 3.9889 \\
	3331 3.9889 \\
	3332 -3.9889 \\
	3333 -3.9889 \\
	3334 -3.9889 \\
	3335 -3.9889 \\
	3336 3.9889 \\
	3337 11.9668 \\
	3338 3.9889 \\
	3339 -3.9889 \\
	3340 -3.9889 \\
	3341 -3.9889 \\
	3342 11.9668 \\
	3343 3.9889 \\
	3344 -11.9668 \\
	3345 11.9668 \\
	3346 11.9668 \\
	3347 11.9668 \\
	3348 11.9668 \\
	3349 19.9446 \\
	3350 3.9889 \\
	3351 -11.9668 \\
	3352 3.9889 \\
	3353 -3.9889 \\
	3354 -11.9668 \\
	3355 -3.9889 \\
	3356 3.9889 \\
	3357 -3.9889 \\
	3358 -3.9889 \\
	3359 3.9889 \\
	3360 -3.9889 \\
	3361 3.9889 \\
	3362 -11.9668 \\
	3363 -3.9889 \\
	3364 11.9668 \\
	3365 3.9889 \\
	3366 3.9889 \\
	3367 -3.9889 \\
	3368 3.9889 \\
	3369 -3.9889 \\
	3370 -11.9668 \\
	3371 3.9889 \\
	3372 11.9668 \\
	3373 3.9889 \\
	3374 -3.9889 \\
	3375 3.9889 \\
	3376 3.9889 \\
	3377 3.9889 \\
	3378 -3.9889 \\
	3379 11.9668 \\
	3380 -3.9889 \\
	3381 -11.9668 \\
	3382 3.9889 \\
	3383 -27.9225 \\
	3384 3.9889 \\
	3385 3.9889 \\
	3386 -11.9668 \\
	3387 3.9889 \\
	3388 3.9889 \\
	3389 3.9889 \\
	3390 -3.9889 \\
	3391 -3.9889 \\
	3392 3.9889 \\
	3393 -3.9889 \\
	3394 3.9889 \\
	3395 -3.9889 \\
	3396 -19.9446 \\
	3397 -3.9889 \\
	3398 -3.9889 \\
	3399 -3.9889 \\
	3400 -3.9889 \\
	3401 -11.9668 \\
	3402 11.9668 \\
	3403 11.9668 \\
	3404 -11.9668 \\
	3405 -3.9889 \\
	3406 -3.9889 \\
	3407 3.9889 \\
	3408 11.9668 \\
	3409 -3.9889 \\
	3410 3.9889 \\
	3411 3.9889 \\
	3412 -3.9889 \\
	3413 3.9889 \\
	3414 3.9889 \\
	3415 3.9889 \\
	3416 11.9668 \\
	3417 3.9889 \\
	3418 -3.9889 \\
	3419 -3.9889 \\
	3420 -3.9889 \\
	3421 3.9889 \\
	3422 3.9889 \\
	3423 3.9889 \\
	3424 -3.9889 \\
	3425 3.9889 \\
	3426 11.9668 \\
	3427 11.9668 \\
	3428 3.9889 \\
	3429 -3.9889 \\
	3430 3.9889 \\
	3431 -11.9668 \\
	3432 -3.9889 \\
	3433 3.9889 \\
	3434 -3.9889 \\
	3435 11.9668 \\
	3436 -11.9668 \\
	3437 -11.9668 \\
	3438 3.9889 \\
	3439 -3.9889 \\
	3440 -3.9889 \\
	3441 3.9889 \\
	3442 3.9889 \\
	3443 -3.9889 \\
	3444 11.9668 \\
	3445 3.9889 \\
	3446 -3.9889 \\
	3447 11.9668 \\
	3448 3.9889 \\
	3449 -11.9668 \\
	3450 -3.9889 \\
	3451 11.9668 \\
	3452 -11.9668 \\
	3453 -3.9889 \\
	3454 3.9889 \\
	3455 -11.9668 \\
	3456 -3.9889 \\
	3457 -3.9889 \\
	3458 11.9668 \\
	3459 -11.9668 \\
	3460 3.9889 \\
	3461 3.9889 \\
	3462 -3.9889 \\
	3463 3.9889 \\
	3464 -11.9668 \\
	3465 3.9889 \\
	3466 -3.9889 \\
	3467 -11.9668 \\
	3468 11.9668 \\
	3469 3.9889 \\
	3470 -3.9889 \\
	3471 -3.9889 \\
	3472 -3.9889 \\
	3473 -3.9889 \\
	3474 19.9446 \\
	3475 3.9889 \\
	3476 -11.9668 \\
	3477 19.9446 \\
	3478 -3.9889 \\
	3479 -3.9889 \\
	3480 11.9668 \\
	3481 11.9668 \\
	3482 3.9889 \\
	3483 -3.9889 \\
	3484 3.9889 \\
	3485 -3.9889 \\
	3486 -3.9889 \\
	3487 3.9889 \\
	3488 3.9889 \\
	3489 -3.9889 \\
	3490 11.9668 \\
	3491 11.9668 \\
	3492 3.9889 \\
	3493 11.9668 \\
	3494 -11.9668 \\
	3495 3.9889 \\
	3496 -3.9889 \\
	3497 -3.9889 \\
	3498 11.9668 \\
	3499 -3.9889 \\
	3500 -3.9889 \\
	3501 -3.9889 \\
	3502 -3.9889 \\
	3503 3.9889 \\
	3504 -11.9668 \\
	3505 -11.9668 \\
	3506 -3.9889 \\
	3507 3.9889 \\
	3508 -3.9889 \\
	3509 -3.9889 \\
	3510 3.9889 \\
	3511 -3.9889 \\
	3512 3.9889 \\
	3513 11.9668 \\
	3514 11.9668 \\
	3515 3.9889 \\
	3516 -3.9889 \\
	3517 3.9889 \\
	3518 3.9889 \\
	3519 -11.9668 \\
	3520 -19.9446 \\
	3521 -3.9889 \\
	3522 -3.9889 \\
	3523 -19.9446 \\
	3524 -3.9889 \\
	3525 3.9889 \\
	3526 -3.9889 \\
	3527 3.9889 \\
	3528 3.9889 \\
	3529 -3.9889 \\
	3530 -3.9889 \\
	3531 -11.9668 \\
	3532 -3.9889 \\
	3533 -3.9889 \\
	3534 3.9889 \\
	3535 3.9889 \\
	3536 3.9889 \\
	3537 3.9889 \\
	3538 -11.9668 \\
	3539 3.9889 \\
	3540 3.9889 \\
	3541 3.9889 \\
	3542 11.9668 \\
	3543 3.9889 \\
	3544 3.9889 \\
	3545 -3.9889 \\
	3546 -11.9668 \\
	3547 -11.9668 \\
	3548 3.9889 \\
	3549 3.9889 \\
	3550 3.9889 \\
	3551 -11.9668 \\
	3552 -3.9889 \\
	3553 3.9889 \\
	3554 -19.9446 \\
	3555 -11.9668 \\
	3556 -3.9889 \\
	3557 -3.9889 \\
	3558 -3.9889 \\
	3559 3.9889 \\
	3560 3.9889 \\
	3561 -3.9889 \\
	3562 11.9668 \\
	3563 3.9889 \\
	3564 3.9889 \\
	3565 -3.9889 \\
	3566 -3.9889 \\
	3567 3.9889 \\
	3568 -3.9889 \\
	3569 -3.9889 \\
	3570 3.9889 \\
	3571 -3.9889 \\
	3572 3.9889 \\
	3573 3.9889 \\
	3574 3.9889 \\
	3575 3.9889 \\
	3576 -11.9668 \\
	3577 3.9889 \\
	3578 19.9446 \\
	3579 3.9889 \\
	3580 11.9668 \\
	3581 -3.9889 \\
	3582 11.9668 \\
	3583 11.9668 \\
	3584 -11.9668 \\
	3585 11.9668 \\
	3586 -3.9889 \\
	3587 11.9668 \\
	3588 19.9446 \\
	3589 11.9668 \\
	3590 -3.9889 \\
	3591 -3.9889 \\
	3592 19.9446 \\
	3593 -3.9889 \\
	3594 -11.9668 \\
	3595 3.9889 \\
	3596 11.9668 \\
	3597 -3.9889 \\
	3598 3.9889 \\
	3599 -3.9889 \\
	3600 -11.9668 \\
	3601 3.9889 \\
	3602 -3.9889 \\
	3603 3.9889 \\
	3604 11.9668 \\
	3605 -3.9889 \\
	3606 -3.9889 \\
	3607 -11.9668 \\
	3608 -3.9889 \\
	3609 3.9889 \\
	3610 -3.9889 \\
	3611 -3.9889 \\
	3612 3.9889 \\
	3613 -3.9889 \\
	3614 -11.9668 \\
	3615 -3.9889 \\
	3616 -11.9668 \\
	3617 3.9889 \\
	3618 3.9889 \\
	3619 -3.9889 \\
	3620 3.9889 \\
	3621 3.9889 \\
	3622 3.9889 \\
	3623 3.9889 \\
	3624 -3.9889 \\
	3625 -3.9889 \\
	3626 -3.9889 \\
	3627 -11.9668 \\
	3628 11.9668 \\
	3629 3.9889 \\
	3630 -11.9668 \\
	3631 -3.9889 \\
	3632 -11.9668 \\
	3633 3.9889 \\
	3634 11.9668 \\
	3635 -3.9889 \\
	3636 -3.9889 \\
	3637 -3.9889 \\
	3638 -3.9889 \\
	3639 -3.9889 \\
	3640 -11.9668 \\
	3641 3.9889 \\
	3642 -3.9889 \\
	3643 -11.9668 \\
	3644 3.9889 \\
	3645 -3.9889 \\
	3646 3.9889 \\
	3647 -3.9889 \\
	3648 -3.9889 \\
	3649 3.9889 \\
	3650 -3.9889 \\
	3651 3.9889 \\
	3652 3.9889 \\
	3653 11.9668 \\
	3654 11.9668 \\
	3655 -11.9668 \\
	3656 -3.9889 \\
	3657 -3.9889 \\
	3658 3.9889 \\
	3659 3.9889 \\
	3660 -3.9889 \\
	3661 -19.9446 \\
	3662 -3.9889 \\
	3663 -3.9889 \\
	3664 3.9889 \\
	3665 11.9668 \\
	3666 -11.9668 \\
	3667 -3.9889 \\
	3668 3.9889 \\
	3669 -3.9889 \\
	3670 -11.9668 \\
	3671 -11.9668 \\
	3672 3.9889 \\
	3673 3.9889 \\
	3674 -3.9889 \\
	3675 -11.9668 \\
	3676 -3.9889 \\
	3677 3.9889 \\
	3678 3.9889 \\
	3679 -3.9889 \\
	3680 3.9889 \\
	3681 3.9889 \\
	3682 -3.9889 \\
	3683 11.9668 \\
	3684 -11.9668 \\
	3685 -11.9668 \\
	3686 11.9668 \\
	3687 11.9668 \\
	3688 3.9889 \\
	3689 -3.9889 \\
	3690 3.9889 \\
	3691 3.9889 \\
	3692 -19.9446 \\
	3693 3.9889 \\
	3694 3.9889 \\
	3695 -3.9889 \\
	3696 3.9889 \\
	3697 -3.9889 \\
	3698 3.9889 \\
	3699 -3.9889 \\
	3700 3.9889 \\
	3701 -3.9889 \\
	3702 3.9889 \\
	3703 19.9446 \\
	3704 -3.9889 \\
	3705 -3.9889 \\
	3706 11.9668 \\
	3707 3.9889 \\
	3708 3.9889 \\
	3709 3.9889 \\
	3710 11.9668 \\
	3711 -3.9889 \\
	3712 -3.9889 \\
	3713 3.9889 \\
	3714 11.9668 \\
	3715 3.9889 \\
	3716 -11.9668 \\
	3717 3.9889 \\
	3718 -3.9889 \\
	3719 -11.9668 \\
	3720 -3.9889 \\
	3721 11.9668 \\
	3722 3.9889 \\
	3723 11.9668 \\
	3724 11.9668 \\
	3725 -11.9668 \\
	3726 -11.9668 \\
	3727 -3.9889 \\
	3728 3.9889 \\
	3729 3.9889 \\
	3730 -11.9668 \\
	3731 -3.9889 \\
	3732 11.9668 \\
	3733 3.9889 \\
	3734 -3.9889 \\
	3735 -11.9668 \\
	3736 3.9889 \\
	3737 -11.9668 \\
	3738 -3.9889 \\
	3739 3.9889 \\
	3740 -3.9889 \\
	3741 -3.9889 \\
	3742 -11.9668 \\
	3743 -3.9889 \\
	3744 3.9889 \\
	3745 3.9889 \\
	3746 3.9889 \\
	3747 -3.9889 \\
	3748 11.9668 \\
	3749 11.9668 \\
	3750 -11.9668 \\
	3751 -3.9889 \\
	3752 3.9889 \\
	3753 -3.9889 \\
	3754 -3.9889 \\
	3755 3.9889 \\
	3756 -3.9889 \\
	3757 11.9668 \\
	3758 19.9446 \\
	3759 -3.9889 \\
	3760 -3.9889 \\
	3761 -3.9889 \\
	3762 -3.9889 \\
	3763 -3.9889 \\
	3764 3.9889 \\
	3765 3.9889 \\
	3766 -11.9668 \\
	3767 3.9889 \\
	3768 -3.9889 \\
	3769 -3.9889 \\
	3770 3.9889 \\
	3771 3.9889 \\
	3772 3.9889 \\
	3773 3.9889 \\
	3774 11.9668 \\
	3775 11.9668 \\
	3776 -3.9889 \\
	3777 3.9889 \\
	3778 -3.9889 \\
	3779 3.9889 \\
	3780 3.9889 \\
	3781 3.9889 \\
	3782 -11.9668 \\
	3783 -3.9889 \\
	3784 3.9889 \\
	3785 -3.9889 \\
	3786 3.9889 \\
	3787 3.9889 \\
	3788 -3.9889 \\
	3789 11.9668 \\
	3790 3.9889 \\
	3791 -11.9668 \\
	3792 3.9889 \\
	3793 -3.9889 \\
	3794 3.9889 \\
	3795 3.9889 \\
	3796 -19.9446 \\
	3797 3.9889 \\
	3798 -3.9889 \\
	3799 -11.9668 \\
	3800 -3.9889 \\
	3801 -11.9668 \\
	3802 -3.9889 \\
	3803 -3.9889 \\
	3804 -19.9446 \\
	3805 3.9889 \\
	3806 3.9889 \\
	3807 11.9668 \\
	3808 11.9668 \\
	3809 -3.9889 \\
	3810 -3.9889 \\
	3811 3.9889 \\
	3812 11.9668 \\
	3813 -11.9668 \\
	3814 -11.9668 \\
	3815 3.9889 \\
	3816 -3.9889 \\
	3817 3.9889 \\
	3818 3.9889 \\
	3819 11.9668 \\
	3820 11.9668 \\
	3821 -11.9668 \\
	3822 3.9889 \\
	3823 -3.9889 \\
	3824 -3.9889 \\
	3825 -11.9668 \\
	3826 3.9889 \\
	3827 11.9668 \\
	3828 3.9889 \\
	3829 3.9889 \\
	3830 -3.9889 \\
	3831 3.9889 \\
	3832 3.9889 \\
	3833 3.9889 \\
	3834 -3.9889 \\
	3835 3.9889 \\
	3836 11.9668 \\
	3837 -3.9889 \\
	3838 11.9668 \\
	3839 -3.9889 \\
	3840 -3.9889 \\
	3841 -3.9889 \\
	3842 3.9889 \\
	3843 3.9889 \\
	3844 -11.9668 \\
	3845 11.9668 \\
	3846 11.9668 \\
	3847 -3.9889 \\
	3848 -3.9889 \\
	3849 -3.9889 \\
	3850 11.9668 \\
	3851 3.9889 \\
	3852 -3.9889 \\
	3853 -3.9889 \\
	3854 -3.9889 \\
	3855 -3.9889 \\
	3856 -11.9668 \\
	3857 11.9668 \\
	3858 3.9889 \\
	3859 -19.9446 \\
	3860 -3.9889 \\
	3861 3.9889 \\
	3862 3.9889 \\
	3863 -3.9889 \\
	3864 -3.9889 \\
	3865 3.9889 \\
	3866 3.9889 \\
	3867 -3.9889 \\
	3868 -11.9668 \\
	3869 -3.9889 \\
	3870 -11.9668 \\
	3871 -11.9668 \\
	3872 -3.9889 \\
	3873 -11.9668 \\
	3874 11.9668 \\
	3875 3.9889 \\
	3876 -11.9668 \\
	3877 -3.9889 \\
	3878 3.9889 \\
	3879 3.9889 \\
	3880 3.9889 \\
	3881 19.9446 \\
	3882 -3.9889 \\
	3883 11.9668 \\
	3884 11.9668 \\
	3885 -19.9446 \\
	3886 -3.9889 \\
	3887 11.9668 \\
	3888 3.9889 \\
	3889 -3.9889 \\
	3890 -3.9889 \\
	3891 3.9889 \\
	3892 3.9889 \\
	3893 11.9668 \\
	3894 3.9889 \\
	3895 -3.9889 \\
	3896 -3.9889 \\
	3897 -3.9889 \\
	3898 -11.9668 \\
	3899 3.9889 \\
	3900 11.9668 \\
	3901 3.9889 \\
	3902 3.9889 \\
	3903 -3.9889 \\
	3904 3.9889 \\
	3905 -11.9668 \\
	3906 -19.9446 \\
	3907 3.9889 \\
	3908 3.9889 \\
	3909 -3.9889 \\
	3910 11.9668 \\
	3911 11.9668 \\
	3912 11.9668 \\
	3913 -11.9668 \\
	3914 3.9889 \\
	3915 -3.9889 \\
	3916 -19.9446 \\
	3917 3.9889 \\
	3918 -11.9668 \\
	3919 3.9889 \\
	3920 3.9889 \\
	3921 3.9889 \\
	3922 3.9889 \\
	3923 -3.9889 \\
	3924 3.9889 \\
	3925 -11.9668 \\
	3926 -3.9889 \\
	3927 11.9668 \\
	3928 11.9668 \\
	3929 11.9668 \\
	3930 -3.9889 \\
	3931 -3.9889 \\
	3932 19.9446 \\
	3933 3.9889 \\
	3934 -3.9889 \\
	3935 3.9889 \\
	3936 -3.9889 \\
	3937 -3.9889 \\
	3938 3.9889 \\
	3939 3.9889 \\
	3940 -3.9889 \\
	3941 -11.9668 \\
	3942 -11.9668 \\
	3943 -3.9889 \\
	3944 11.9668 \\
	3945 -3.9889 \\
	3946 -3.9889 \\
	3947 3.9889 \\
	3948 -3.9889 \\
	3949 3.9889 \\
	3950 -3.9889 \\
	3951 -3.9889 \\
	3952 11.9668 \\
	3953 3.9889 \\
	3954 -11.9668 \\
	3955 3.9889 \\
	3956 3.9889 \\
	3957 -3.9889 \\
	3958 3.9889 \\
	3959 3.9889 \\
	3960 3.9889 \\
	3961 -3.9889 \\
	3962 3.9889 \\
	3963 3.9889 \\
	3964 -11.9668 \\
	3965 -3.9889 \\
	3966 3.9889 \\
	3967 -3.9889 \\
	3968 -11.9668 \\
	3969 11.9668 \\
	3970 -11.9668 \\
	3971 11.9668 \\
	3972 11.9668 \\
	3973 -3.9889 \\
	3974 -3.9889 \\
	3975 3.9889 \\
	3976 3.9889 \\
	3977 11.9668 \\
	3978 3.9889 \\
	3979 -3.9889 \\
	3980 3.9889 \\
	3981 3.9889 \\
	3982 -3.9889 \\
	3983 3.9889 \\
	3984 3.9889 \\
	3985 -3.9889 \\
	3986 -3.9889 \\
	3987 -3.9889 \\
	3988 -3.9889 \\
	3989 3.9889 \\
	3990 3.9889 \\
	3991 -11.9668 \\
	3992 -3.9889 \\
	3993 3.9889 \\
	3994 3.9889 \\
	3995 3.9889 \\
	3996 3.9889 \\
	3997 -3.9889 \\
	3998 -3.9889 \\
	3999 -3.9889 \\
	4000 -3.9889 \\
	4001 -3.9889 \\
	4002 -3.9889 \\
	4003 -11.9668 \\
	4004 -3.9889 \\
	4005 -3.9889 \\
	4006 -3.9889 \\
	4007 -3.9889 \\
	4008 -3.9889 \\
	4009 11.9668 \\
	4010 11.9668 \\
	4011 3.9889 \\
	4012 3.9889 \\
	4013 11.9668 \\
	4014 3.9889 \\
	4015 3.9889 \\
	4016 -11.9668 \\
	4017 -3.9889 \\
	4018 3.9889 \\
	4019 -11.9668 \\
	4020 -3.9889 \\
	4021 3.9889 \\
	4022 -11.9668 \\
	4023 19.9446 \\
	4024 11.9668 \\
	4025 -3.9889 \\
	4026 -11.9668 \\
	4027 -11.9668 \\
	4028 11.9668 \\
	4029 3.9889 \\
	4030 3.9889 \\
	4031 -3.9889 \\
	4032 3.9889 \\
	4033 -3.9889 \\
	4034 -3.9889 \\
	4035 -11.9668 \\
	4036 3.9889 \\
	4037 3.9889 \\
	4038 -3.9889 \\
	4039 -11.9668 \\
	4040 -11.9668 \\
	4041 3.9889 \\
	4042 -3.9889 \\
	4043 -3.9889 \\
	4044 11.9668 \\
	4045 3.9889 \\
	4046 -11.9668 \\
	4047 -3.9889 \\
	4048 11.9668 \\
	4049 3.9889 \\
	4050 3.9889 \\
	4051 3.9889 \\
	4052 3.9889 \\
	4053 3.9889 \\
	4054 3.9889 \\
	4055 3.9889 \\
	4056 -3.9889 \\
	4057 3.9889 \\
	4058 3.9889 \\
	4059 3.9889 \\
	4060 3.9889 \\
	4061 3.9889 \\
	4062 19.9446 \\
	4063 -11.9668 \\
	4064 -3.9889 \\
	4065 11.9668 \\
	4066 -3.9889 \\
	4067 11.9668 \\
	4068 3.9889 \\
	4069 3.9889 \\
	4070 3.9889 \\
	4071 -3.9889 \\
	4072 -3.9889 \\
	4073 -3.9889 \\
	4074 -3.9889 \\
	4075 -3.9889 \\
	4076 -11.9668 \\
	4077 3.9889 \\
	4078 11.9668 \\
	4079 -19.9446 \\
	4080 -3.9889 \\
	4081 3.9889 \\
	4082 -3.9889 \\
	4083 3.9889 \\
	4084 -19.9446 \\
	4085 -3.9889 \\
	4086 3.9889 \\
	4087 3.9889 \\
	4088 -3.9889 \\
	4089 -11.9668 \\
	4090 3.9889 \\
	4091 -3.9889 \\
	4092 3.9889 \\
	4093 3.9889 \\
	4094 3.9889 \\
	4095 3.9889 \\
	4096 -3.9889 \\
	4097 -11.9668 \\
	4098 -3.9889 \\
	4099 -11.9668 \\
	4100 -3.9889 \\
	4101 -3.9889 \\
	4102 -3.9889 \\
	4103 -11.9668 \\
	4104 3.9889 \\
	4105 3.9889 \\
	4106 -3.9889 \\
	4107 -3.9889 \\
	4108 -3.9889 \\
	4109 11.9668 \\
	4110 3.9889 \\
	4111 -3.9889 \\
	4112 3.9889 \\
	4113 -3.9889 \\
	4114 -11.9668 \\
	4115 -11.9668 \\
	4116 -3.9889 \\
	4117 3.9889 \\
	4118 3.9889 \\
	4119 -3.9889 \\
	4120 -3.9889 \\
	4121 -11.9668 \\
	4122 -3.9889 \\
	4123 -3.9889 \\
	4124 3.9889 \\
	4125 3.9889 \\
	4126 -3.9889 \\
	4127 3.9889 \\
	4128 11.9668 \\
	4129 3.9889 \\
	4130 3.9889 \\
	4131 3.9889 \\
	4132 3.9889 \\
	4133 11.9668 \\
	4134 3.9889 \\
	4135 -11.9668 \\
	4136 3.9889 \\
	4137 3.9889 \\
	4138 -3.9889 \\
	4139 11.9668 \\
	4140 11.9668 \\
	4141 -3.9889 \\
	4142 -3.9889 \\
	4143 3.9889 \\
	4144 -3.9889 \\
	4145 11.9668 \\
	4146 3.9889 \\
	4147 3.9889 \\
	4148 -3.9889 \\
	4149 -11.9668 \\
	4150 19.9446 \\
	4151 -11.9668 \\
	4152 -3.9889 \\
	4153 3.9889 \\
	4154 -3.9889 \\
	4155 3.9889 \\
	4156 3.9889 \\
	4157 11.9668 \\
	4158 -3.9889 \\
	4159 -3.9889 \\
	4160 -3.9889 \\
	4161 3.9889 \\
	4162 11.9668 \\
	4163 3.9889 \\
	4164 -11.9668 \\
	4165 11.9668 \\
	4166 3.9889 \\
	4167 -11.9668 \\
	4168 -3.9889 \\
	4169 -3.9889 \\
	4170 3.9889 \\
	4171 3.9889 \\
	4172 11.9668 \\
	4173 3.9889 \\
	4174 11.9668 \\
	4175 19.9446 \\
	4176 -11.9668 \\
	4177 -3.9889 \\
	4178 3.9889 \\
	4179 11.9668 \\
	4180 11.9668 \\
	4181 -3.9889 \\
	4182 3.9889 \\
	4183 -3.9889 \\
	4184 -3.9889 \\
	4185 -3.9889 \\
	4186 -3.9889 \\
	4187 -3.9889 \\
	4188 3.9889 \\
	4189 3.9889 \\
	4190 -3.9889 \\
	4191 -3.9889 \\
	4192 3.9889 \\
	4193 3.9889 \\
	4194 3.9889 \\
	4195 -3.9889 \\
	4196 -11.9668 \\
	4197 -3.9889 \\
	4198 -11.9668 \\
	4199 -11.9668 \\
	4200 -11.9668 \\
	4201 -11.9668 \\
	4202 -11.9668 \\
	4203 3.9889 \\
	4204 11.9668 \\
	4205 -3.9889 \\
	4206 3.9889 \\
	4207 3.9889 \\
	4208 3.9889 \\
	4209 3.9889 \\
	4210 -11.9668 \\
	4211 3.9889 \\
	4212 3.9889 \\
	4213 -11.9668 \\
	4214 -3.9889 \\
	4215 -3.9889 \\
	4216 3.9889 \\
	4217 11.9668 \\
	4218 3.9889 \\
	4219 3.9889 \\
	4220 -3.9889 \\
	4221 -3.9889 \\
	4222 3.9889 \\
	4223 -11.9668 \\
	4224 -3.9889 \\
	4225 3.9889 \\
	4226 11.9668 \\
	4227 3.9889 \\
	4228 3.9889 \\
	4229 -11.9668 \\
	4230 -3.9889 \\
	4231 3.9889 \\
	4232 3.9889 \\
	4233 11.9668 \\
	4234 -3.9889 \\
	4235 11.9668 \\
	4236 3.9889 \\
	4237 -3.9889 \\
	4238 -3.9889 \\
	4239 -19.9446 \\
	4240 11.9668 \\
	4241 -11.9668 \\
	4242 -11.9668 \\
	4243 3.9889 \\
	4244 -11.9668 \\
	4245 -11.9668 \\
	4246 11.9668 \\
	4247 3.9889 \\
	4248 3.9889 \\
	4249 3.9889 \\
	4250 3.9889 \\
	4251 -3.9889 \\
	4252 3.9889 \\
	4253 11.9668 \\
	4254 -3.9889 \\
	4255 3.9889 \\
	4256 -3.9889 \\
	4257 -11.9668 \\
	4258 -3.9889 \\
	4259 -3.9889 \\
	4260 -3.9889 \\
	4261 -11.9668 \\
	4262 3.9889 \\
	4263 11.9668 \\
	4264 -3.9889 \\
	4265 3.9889 \\
	4266 3.9889 \\
	4267 -11.9668 \\
	4268 -3.9889 \\
	4269 3.9889 \\
	4270 -3.9889 \\
	4271 3.9889 \\
	4272 11.9668 \\
	4273 3.9889 \\
	4274 -3.9889 \\
	4275 -3.9889 \\
	4276 -3.9889 \\
	4277 -3.9889 \\
	4278 3.9889 \\
	4279 -19.9446 \\
	4280 3.9889 \\
	4281 11.9668 \\
	4282 -3.9889 \\
	4283 11.9668 \\
	4284 -3.9889 \\
	4285 3.9889 \\
	4286 11.9668 \\
	4287 3.9889 \\
	4288 3.9889 \\
	4289 -3.9889 \\
	4290 3.9889 \\
	4291 11.9668 \\
	4292 -19.9446 \\
	4293 -3.9889 \\
	4294 3.9889 \\
	4295 -3.9889 \\
	4296 11.9668 \\
	4297 3.9889 \\
	4298 3.9889 \\
	4299 11.9668 \\
	4300 3.9889 \\
	4301 -3.9889 \\
	4302 3.9889 \\
	4303 -3.9889 \\
	4304 3.9889 \\
	4305 11.9668 \\
	4306 -11.9668 \\
	4307 3.9889 \\
	4308 11.9668 \\
	4309 3.9889 \\
	4310 -3.9889 \\
	4311 3.9889 \\
	4312 3.9889 \\
	4313 -3.9889 \\
	4314 -3.9889 \\
	4315 -11.9668 \\
	4316 -3.9889 \\
	4317 11.9668 \\
	4318 3.9889 \\
	4319 11.9668 \\
	4320 3.9889 \\
	4321 3.9889 \\
	4322 -3.9889 \\
	4323 -3.9889 \\
	4324 3.9889 \\
	4325 11.9668 \\
	4326 3.9889 \\
	4327 -11.9668 \\
	4328 -3.9889 \\
	4329 3.9889 \\
	4330 3.9889 \\
	4331 -11.9668 \\
	4332 -3.9889 \\
	4333 3.9889 \\
	4334 -3.9889 \\
	4335 11.9668 \\
	4336 11.9668 \\
	4337 11.9668 \\
	4338 11.9668 \\
	4339 -3.9889 \\
	4340 3.9889 \\
	4341 -11.9668 \\
	4342 -3.9889 \\
	4343 3.9889 \\
	4344 -3.9889 \\
	4345 3.9889 \\
	4346 -11.9668 \\
	4347 -3.9889 \\
	4348 -3.9889 \\
	4349 -11.9668 \\
	4350 -3.9889 \\
	4351 -19.9446 \\
	4352 -3.9889 \\
	4353 3.9889 \\
	4354 3.9889 \\
	4355 3.9889 \\
	4356 -3.9889 \\
	4357 -3.9889 \\
	4358 3.9889 \\
	4359 -3.9889 \\
	4360 -3.9889 \\
	4361 3.9889 \\
	4362 -3.9889 \\
	4363 3.9889 \\
	4364 3.9889 \\
	4365 -3.9889 \\
	4366 -3.9889 \\
	4367 -3.9889 \\
	4368 3.9889 \\
	4369 3.9889 \\
	4370 11.9668 \\
	4371 -3.9889 \\
	4372 -3.9889 \\
	4373 3.9889 \\
	4374 -11.9668 \\
	4375 -3.9889 \\
	4376 -3.9889 \\
	4377 3.9889 \\
	4378 11.9668 \\
	4379 3.9889 \\
	4380 3.9889 \\
	4381 -3.9889 \\
	4382 19.9446 \\
	4383 27.9225 \\
	4384 -11.9668 \\
	4385 3.9889 \\
	4386 11.9668 \\
	4387 -11.9668 \\
	4388 -3.9889 \\
	4389 3.9889 \\
	4390 3.9889 \\
	4391 -3.9889 \\
	4392 -3.9889 \\
	4393 -11.9668 \\
	4394 3.9889 \\
	4395 -3.9889 \\
	4396 -3.9889 \\
	4397 -3.9889 \\
	4398 -11.9668 \\
	4399 3.9889 \\
	4400 -3.9889 \\
	4401 3.9889 \\
	4402 -3.9889 \\
	4403 -3.9889 \\
	4404 -3.9889 \\
	4405 3.9889 \\
	4406 3.9889 \\
	4407 -11.9668 \\
	4408 -11.9668 \\
	4409 3.9889 \\
	4410 -3.9889 \\
	4411 -3.9889 \\
	4412 3.9889 \\
	4413 3.9889 \\
	4414 -3.9889 \\
	4415 3.9889 \\
	4416 3.9889 \\
	4417 3.9889 \\
	4418 -3.9889 \\
	4419 -3.9889 \\
	4420 3.9889 \\
	4421 3.9889 \\
	4422 11.9668 \\
	4423 11.9668 \\
	4424 -3.9889 \\
	4425 3.9889 \\
	4426 -3.9889 \\
	4427 -11.9668 \\
	4428 -3.9889 \\
	4429 3.9889 \\
	4430 19.9446 \\
	4431 -19.9446 \\
	4432 -11.9668 \\
	4433 3.9889 \\
	4434 11.9668 \\
	4435 3.9889 \\
	4436 -3.9889 \\
	4437 3.9889 \\
	4438 -3.9889 \\
	4439 3.9889 \\
	4440 3.9889 \\
	4441 3.9889 \\
	4442 3.9889 \\
	4443 -3.9889 \\
	4444 11.9668 \\
	4445 -3.9889 \\
	4446 -11.9668 \\
	4447 3.9889 \\
	4448 -3.9889 \\
	4449 -3.9889 \\
	4450 3.9889 \\
	4451 -3.9889 \\
	4452 -19.9446 \\
	4453 3.9889 \\
	4454 -3.9889 \\
	4455 -19.9446 \\
	4456 -11.9668 \\
	4457 -3.9889 \\
	4458 3.9889 \\
	4459 11.9668 \\
	4460 11.9668 \\
	4461 -3.9889 \\
	4462 3.9889 \\
	4463 -3.9889 \\
	4464 -3.9889 \\
	4465 11.9668 \\
	4466 -3.9889 \\
	4467 -3.9889 \\
	4468 -3.9889 \\
	4469 -3.9889 \\
	4470 11.9668 \\
	4471 3.9889 \\
	4472 -3.9889 \\
	4473 -3.9889 \\
	4474 -3.9889 \\
	4475 3.9889 \\
	4476 3.9889 \\
	4477 3.9889 \\
	4478 3.9889 \\
	4479 -3.9889 \\
	4480 3.9889 \\
	4481 11.9668 \\
	4482 -3.9889 \\
	4483 -11.9668 \\
	4484 3.9889 \\
	4485 -3.9889 \\
	4486 3.9889 \\
	4487 3.9889 \\
	4488 -3.9889 \\
	4489 3.9889 \\
	4490 11.9668 \\
	4491 3.9889 \\
	4492 -11.9668 \\
	4493 11.9668 \\
	4494 -3.9889 \\
	4495 -11.9668 \\
	4496 -3.9889 \\
	4497 -11.9668 \\
	4498 -3.9889 \\
	4499 3.9889 \\
	4500 3.9889 \\
	4501 11.9668 \\
	4502 -11.9668 \\
	4503 -3.9889 \\
	4504 11.9668 \\
	4505 -3.9889 \\
	4506 -3.9889 \\
	4507 -11.9668 \\
	4508 -3.9889 \\
	4509 -3.9889 \\
	4510 3.9889 \\
	4511 3.9889 \\
	4512 -11.9668 \\
	4513 -11.9668 \\
	4514 -3.9889 \\
	4515 -3.9889 \\
	4516 -3.9889 \\
	4517 3.9889 \\
	4518 11.9668 \\
	4519 -3.9889 \\
	4520 3.9889 \\
	4521 -3.9889 \\
	4522 3.9889 \\
	4523 3.9889 \\
	4524 3.9889 \\
	4525 -3.9889 \\
	4526 -11.9668 \\
	4527 3.9889 \\
	4528 -3.9889 \\
	4529 3.9889 \\
	4530 3.9889 \\
	4531 11.9668 \\
	4532 11.9668 \\
	4533 -3.9889 \\
	4534 -3.9889 \\
	4535 -3.9889 \\
	4536 3.9889 \\
	4537 -3.9889 \\
	4538 11.9668 \\
	4539 19.9446 \\
	4540 -3.9889 \\
	4541 11.9668 \\
	4542 3.9889 \\
	4543 -11.9668 \\
	4544 -3.9889 \\
	4545 -3.9889 \\
	4546 11.9668 \\
	4547 3.9889 \\
	4548 11.9668 \\
	4549 3.9889 \\
	4550 -3.9889 \\
	4551 -3.9889 \\
	4552 -11.9668 \\
	4553 -3.9889 \\
	4554 -3.9889 \\
	4555 3.9889 \\
	4556 -3.9889 \\
	4557 -3.9889 \\
	4558 -3.9889 \\
	4559 11.9668 \\
	4560 19.9446 \\
	4561 3.9889 \\
	4562 3.9889 \\
	4563 -3.9889 \\
	4564 3.9889 \\
	4565 3.9889 \\
	4566 -11.9668 \\
	4567 3.9889 \\
	4568 3.9889 \\
	4569 3.9889 \\
	4570 -3.9889 \\
	4571 3.9889 \\
	4572 -3.9889 \\
	4573 3.9889 \\
	4574 11.9668 \\
	4575 3.9889 \\
	4576 11.9668 \\
	4577 -11.9668 \\
	4578 -3.9889 \\
	4579 11.9668 \\
	4580 11.9668 \\
	4581 3.9889 \\
	4582 -3.9889 \\
	4583 11.9668 \\
	4584 3.9889 \\
	4585 -3.9889 \\
	4586 3.9889 \\
	4587 -3.9889 \\
	4588 -3.9889 \\
	4589 -3.9889 \\
	4590 -11.9668 \\
	4591 -3.9889 \\
	4592 3.9889 \\
	4593 -11.9668 \\
	4594 -3.9889 \\
	4595 3.9889 \\
	4596 -3.9889 \\
	4597 -3.9889 \\
	4598 -3.9889 \\
	4599 3.9889 \\
	4600 -11.9668 \\
	4601 -3.9889 \\
	4602 11.9668 \\
	4603 -3.9889 \\
	4604 -11.9668 \\
	4605 -3.9889 \\
	4606 -3.9889 \\
	4607 -3.9889 \\
	4608 -3.9889 \\
	4609 -3.9889 \\
	4610 3.9889 \\
	4611 -3.9889 \\
	4612 3.9889 \\
	4613 3.9889 \\
	4614 -3.9889 \\
	4615 -3.9889 \\
	4616 3.9889 \\
	4617 3.9889 \\
	4618 -3.9889 \\
	4619 -11.9668 \\
	4620 3.9889 \\
	4621 -11.9668 \\
	4622 -3.9889 \\
	4623 3.9889 \\
	4624 -11.9668 \\
	4625 3.9889 \\
	4626 11.9668 \\
	4627 3.9889 \\
	4628 -3.9889 \\
	4629 3.9889 \\
	4630 3.9889 \\
	4631 -19.9446 \\
	4632 11.9668 \\
	4633 3.9889 \\
	4634 -3.9889 \\
	4635 3.9889 \\
	4636 3.9889 \\
	4637 3.9889 \\
	4638 11.9668 \\
	4639 27.9225 \\
	4640 11.9668 \\
	4641 3.9889 \\
	4642 -19.9446 \\
	4643 -3.9889 \\
	4644 19.9446 \\
	4645 3.9889 \\
	4646 3.9889 \\
	4647 -11.9668 \\
	4648 -3.9889 \\
	4649 -3.9889 \\
	4650 -11.9668 \\
	4651 -3.9889 \\
	4652 -3.9889 \\
	4653 3.9889 \\
	4654 3.9889 \\
	4655 -3.9889 \\
	4656 -3.9889 \\
	4657 3.9889 \\
	4658 -3.9889 \\
	4659 -3.9889 \\
	4660 -3.9889 \\
	4661 -3.9889 \\
	4662 -3.9889 \\
	4663 -3.9889 \\
	4664 3.9889 \\
	4665 -3.9889 \\
	4666 -3.9889 \\
	4667 -3.9889 \\
	4668 -3.9889 \\
	4669 11.9668 \\
	4670 -3.9889 \\
	4671 -3.9889 \\
	4672 3.9889 \\
	4673 -3.9889 \\
	4674 -3.9889 \\
	4675 -3.9889 \\
	4676 11.9668 \\
	4677 -11.9668 \\
	4678 -19.9446 \\
	4679 -3.9889 \\
	4680 -11.9668 \\
	4681 -3.9889 \\
	4682 3.9889 \\
	4683 -3.9889 \\
	4684 -3.9889 \\
	4685 -3.9889 \\
	4686 3.9889 \\
	4687 -3.9889 \\
	4688 -3.9889 \\
	4689 -3.9889 \\
	4690 -11.9668 \\
	4691 -3.9889 \\
	4692 -3.9889 \\
	4693 -11.9668 \\
	4694 3.9889 \\
	4695 11.9668 \\
	4696 -3.9889 \\
	4697 3.9889 \\
	4698 19.9446 \\
	4699 11.9668 \\
	4700 -3.9889 \\
	4701 -3.9889 \\
	4702 -3.9889 \\
	4703 -3.9889 \\
	4704 11.9668 \\
	4705 11.9668 \\
	4706 11.9668 \\
	4707 -3.9889 \\
	4708 -3.9889 \\
	4709 3.9889 \\
	4710 -11.9668 \\
	4711 -3.9889 \\
	4712 -3.9889 \\
	4713 -3.9889 \\
	4714 3.9889 \\
	4715 11.9668 \\
	4716 3.9889 \\
	4717 -3.9889 \\
	4718 -3.9889 \\
	4719 3.9889 \\
	4720 11.9668 \\
	4721 3.9889 \\
	4722 3.9889 \\
	4723 3.9889 \\
	4724 3.9889 \\
	4725 3.9889 \\
	4726 -3.9889 \\
	4727 3.9889 \\
	4728 -3.9889 \\
	4729 -11.9668 \\
	4730 3.9889 \\
	4731 -3.9889 \\
	4732 3.9889 \\
	4733 3.9889 \\
	4734 -3.9889 \\
	4735 11.9668 \\
	4736 3.9889 \\
	4737 11.9668 \\
	4738 -3.9889 \\
	4739 11.9668 \\
	4740 11.9668 \\
	4741 -3.9889 \\
	4742 3.9889 \\
	4743 -3.9889 \\
	4744 3.9889 \\
	4745 -3.9889 \\
	4746 -3.9889 \\
	4747 11.9668 \\
	4748 11.9668 \\
	4749 3.9889 \\
	4750 -11.9668 \\
	4751 3.9889 \\
	4752 -3.9889 \\
	4753 -3.9889 \\
	4754 19.9446 \\
	4755 -3.9889 \\
	4756 3.9889 \\
	4757 11.9668 \\
	4758 -3.9889 \\
	4759 11.9668 \\
	4760 11.9668 \\
	4761 11.9668 \\
	4762 -11.9668 \\
	4763 -11.9668 \\
	4764 3.9889 \\
	4765 3.9889 \\
	4766 3.9889 \\
	4767 -11.9668 \\
	4768 -19.9446 \\
	4769 3.9889 \\
	4770 -3.9889 \\
	4771 -11.9668 \\
	4772 3.9889 \\
	4773 -3.9889 \\
	4774 3.9889 \\
	4775 3.9889 \\
	4776 3.9889 \\
	4777 3.9889 \\
	4778 3.9889 \\
	4779 3.9889 \\
	4780 3.9889 \\
	4781 -3.9889 \\
	4782 -3.9889 \\
	4783 -3.9889 \\
	4784 -3.9889 \\
	4785 11.9668 \\
	4786 -3.9889 \\
	4787 -11.9668 \\
	4788 -3.9889 \\
	4789 3.9889 \\
	4790 3.9889 \\
	4791 3.9889 \\
	4792 3.9889 \\
	4793 -3.9889 \\
	4794 -3.9889 \\
	4795 -11.9668 \\
	4796 -3.9889 \\
	4797 -3.9889 \\
	4798 -19.9446 \\
	4799 -3.9889 \\
	4800 -3.9889 \\
	4801 3.9889 \\
	4802 -3.9889 \\
	4803 3.9889 \\
	4804 19.9446 \\
	4805 -3.9889 \\
	4806 -3.9889 \\
	4807 -11.9668 \\
	4808 11.9668 \\
	4809 11.9668 \\
	4810 3.9889 \\
	4811 11.9668 \\
	4812 -3.9889 \\
	4813 -3.9889 \\
	4814 3.9889 \\
	4815 11.9668 \\
	4816 -3.9889 \\
	4817 -3.9889 \\
	4818 -3.9889 \\
	4819 -3.9889 \\
	4820 -3.9889 \\
	4821 -11.9668 \\
	4822 -11.9668 \\
	4823 -11.9668 \\
	4824 3.9889 \\
	4825 11.9668 \\
	4826 3.9889 \\
	4827 -3.9889 \\
	4828 -3.9889 \\
	4829 -3.9889 \\
	4830 -3.9889 \\
	4831 3.9889 \\
	4832 3.9889 \\
	4833 19.9446 \\
	4834 11.9668 \\
	4835 -3.9889 \\
	4836 3.9889 \\
	4837 -3.9889 \\
	4838 3.9889 \\
	4839 -3.9889 \\
	4840 3.9889 \\
	4841 -3.9889 \\
	4842 3.9889 \\
	4843 19.9446 \\
	4844 -3.9889 \\
	4845 3.9889 \\
	4846 3.9889 \\
	4847 -19.9446 \\
	4848 -3.9889 \\
	4849 -3.9889 \\
	4850 -3.9889 \\
	4851 -3.9889 \\
	4852 -11.9668 \\
	4853 11.9668 \\
	4854 11.9668 \\
	4855 -11.9668 \\
	4856 3.9889 \\
	4857 3.9889 \\
	4858 3.9889 \\
	4859 -3.9889 \\
	4860 -11.9668 \\
	4861 -3.9889 \\
	4862 3.9889 \\
	4863 3.9889 \\
	4864 -3.9889 \\
	4865 -3.9889 \\
	4866 -11.9668 \\
	4867 -3.9889 \\
	4868 -19.9446 \\
	4869 -3.9889 \\
	4870 3.9889 \\
	4871 11.9668 \\
	4872 -3.9889 \\
	4873 -11.9668 \\
	4874 3.9889 \\
	4875 -3.9889 \\
	4876 -11.9668 \\
	4877 3.9889 \\
	4878 -3.9889 \\
	4879 3.9889 \\
	4880 3.9889 \\
	4881 -3.9889 \\
	4882 11.9668 \\
	4883 11.9668 \\
	4884 11.9668 \\
	4885 11.9668 \\
	4886 3.9889 \\
	4887 3.9889 \\
	4888 -11.9668 \\
	4889 11.9668 \\
	4890 19.9446 \\
	4891 3.9889 \\
	4892 3.9889 \\
	4893 -3.9889 \\
	4894 3.9889 \\
	4895 3.9889 \\
	4896 -11.9668 \\
	4897 -3.9889 \\
	4898 -3.9889 \\
	4899 -11.9668 \\
	4900 3.9889 \\
	4901 3.9889 \\
	4902 3.9889 \\
	4903 11.9668 \\
	4904 11.9668 \\
	4905 -11.9668 \\
	4906 -3.9889 \\
	4907 3.9889 \\
	4908 -3.9889 \\
	4909 -3.9889 \\
	4910 -11.9668 \\
	4911 -11.9668 \\
	4912 3.9889 \\
	4913 -3.9889 \\
	4914 -3.9889 \\
	4915 -3.9889 \\
	4916 -3.9889 \\
	4917 3.9889 \\
	4918 11.9668 \\
	4919 11.9668 \\
	4920 -3.9889 \\
	4921 -3.9889 \\
	4922 -3.9889 \\
	4923 3.9889 \\
	4924 -3.9889 \\
	4925 -3.9889 \\
	4926 3.9889 \\
	4927 -3.9889 \\
	4928 11.9668 \\
	4929 11.9668 \\
	4930 3.9889 \\
	4931 3.9889 \\
	4932 3.9889 \\
	4933 -3.9889 \\
	4934 -11.9668 \\
	4935 3.9889 \\
	4936 3.9889 \\
	4937 -3.9889 \\
	4938 -3.9889 \\
	4939 11.9668 \\
	4940 -3.9889 \\
	4941 -19.9446 \\
	4942 11.9668 \\
	4943 3.9889 \\
	4944 11.9668 \\
	4945 11.9668 \\
	4946 -3.9889 \\
	4947 3.9889 \\
	4948 -3.9889 \\
	4949 3.9889 \\
	4950 3.9889 \\
	4951 -3.9889 \\
	4952 3.9889 \\
	4953 -3.9889 \\
	4954 -3.9889 \\
	4955 3.9889 \\
	4956 3.9889 \\
	4957 3.9889 \\
	4958 3.9889 \\
	4959 -11.9668 \\
	4960 3.9889 \\
	4961 3.9889 \\
	4962 -3.9889 \\
	4963 -3.9889 \\
	4964 -19.9446 \\
	4965 3.9889 \\
	4966 3.9889 \\
	4967 -3.9889 \\
	4968 19.9446 \\
	4969 -3.9889 \\
	4970 -19.9446 \\
	4971 -11.9668 \\
	4972 -3.9889 \\
	4973 -3.9889 \\
	4974 -3.9889 \\
	4975 -3.9889 \\
	4976 -3.9889 \\
	4977 11.9668 \\
	4978 -3.9889 \\
	4979 -11.9668 \\
	4980 3.9889 \\
	4981 3.9889 \\
	4982 3.9889 \\
	4983 3.9889 \\
	4984 -3.9889 \\
	4985 11.9668 \\
	4986 -3.9889 \\
	4987 3.9889 \\
	4988 11.9668 \\
	4989 -3.9889 \\
	4990 -3.9889 \\
	4991 3.9889 \\
	4992 3.9889 \\
	4993 -11.9668 \\
	4994 3.9889 \\
	4995 -3.9889 \\
	4996 11.9668 \\
	4997 3.9889 \\
	4998 3.9889 \\
	4999 3.9889 \\
	5000 -19.9446 \\
	5001 3.9889 \\
	5002 -3.9889 \\
	5003 -3.9889 \\
	5004 -3.9889 \\
	5005 3.9889 \\
	5006 3.9889 \\
	5007 3.9889 \\
	5008 3.9889 \\
	5009 -19.9446 \\
	5010 -3.9889 \\
	5011 3.9889 \\
	5012 -11.9668 \\
	5013 -3.9889 \\
	5014 -3.9889 \\
	5015 -11.9668 \\
	5016 -3.9889 \\
	5017 3.9889 \\
	5018 -3.9889 \\
	5019 -3.9889 \\
	5020 -3.9889 \\
	5021 11.9668 \\
	5022 -3.9889 \\
	5023 -3.9889 \\
	5024 11.9668 \\
	5025 3.9889 \\
	5026 3.9889 \\
	5027 11.9668 \\
	5028 3.9889 \\
	5029 -3.9889 \\
	5030 11.9668 \\
	5031 11.9668 \\
	5032 3.9889 \\
	5033 3.9889 \\
	5034 3.9889 \\
	5035 -3.9889 \\
	5036 3.9889 \\
	5037 11.9668 \\
	5038 3.9889 \\
	5039 11.9668 \\
	5040 -3.9889 \\
	5041 -11.9668 \\
	5042 -3.9889 \\
	5043 -11.9668 \\
	5044 -3.9889 \\
	5045 3.9889 \\
	5046 3.9889 \\
	5047 3.9889 \\
	5048 3.9889 \\
	5049 -3.9889 \\
	5050 11.9668 \\
	5051 11.9668 \\
	5052 -19.9446 \\
	5053 3.9889 \\
	5054 3.9889 \\
	5055 3.9889 \\
	5056 -3.9889 \\
	5057 -3.9889 \\
	5058 -11.9668 \\
	5059 3.9889 \\
	5060 -11.9668 \\
	5061 -11.9668 \\
	5062 3.9889 \\
	5063 3.9889 \\
	5064 -3.9889 \\
	5065 -3.9889 \\
	5066 11.9668 \\
	5067 -11.9668 \\
	5068 -3.9889 \\
	5069 11.9668 \\
	5070 3.9889 \\
	5071 -19.9446 \\
	5072 -3.9889 \\
	5073 3.9889 \\
	5074 3.9889 \\
	5075 19.9446 \\
	5076 -3.9889 \\
	5077 3.9889 \\
	5078 -11.9668 \\
	5079 -19.9446 \\
	5080 3.9889 \\
	5081 -11.9668 \\
	5082 3.9889 \\
	5083 11.9668 \\
	5084 -3.9889 \\
	5085 3.9889 \\
	5086 3.9889 \\
	5087 3.9889 \\
	5088 3.9889 \\
	5089 -3.9889 \\
	5090 3.9889 \\
	5091 11.9668 \\
	5092 3.9889 \\
	5093 -3.9889 \\
	5094 -3.9889 \\
	5095 3.9889 \\
	5096 3.9889 \\
	5097 3.9889 \\
	5098 3.9889 \\
	5099 -3.9889 \\
	5100 3.9889 \\
	5101 -3.9889 \\
	5102 11.9668 \\
	5103 3.9889 \\
	5104 -3.9889 \\
	5105 -3.9889 \\
	5106 -11.9668 \\
	5107 -3.9889 \\
	5108 -3.9889 \\
	5109 3.9889 \\
	5110 -3.9889 \\
	5111 -3.9889 \\
	5112 -3.9889 \\
	5113 3.9889 \\
	5114 19.9446 \\
	5115 -3.9889 \\
	5116 -3.9889 \\
	5117 3.9889 \\
	5118 3.9889 \\
	5119 -3.9889 \\
	5120 -3.9889 \\
	5121 -3.9889 \\
	5122 -3.9889 \\
	5123 -3.9889 \\
	5124 -3.9889 \\
	5125 -3.9889 \\
	5126 -3.9889 \\
	5127 -3.9889 \\
	5128 -3.9889 \\
	5129 -3.9889 \\
	5130 3.9889 \\
	5131 11.9668 \\
	5132 11.9668 \\
	5133 -3.9889 \\
	5134 11.9668 \\
	5135 3.9889 \\
	5136 -3.9889 \\
	5137 3.9889 \\
	5138 -19.9446 \\
	5139 3.9889 \\
	5140 11.9668 \\
	5141 -3.9889 \\
	5142 11.9668 \\
	5143 3.9889 \\
	5144 3.9889 \\
	5145 -3.9889 \\
	5146 3.9889 \\
	5147 3.9889 \\
	5148 3.9889 \\
	5149 3.9889 \\
	5150 -11.9668 \\
	5151 3.9889 \\
	5152 -11.9668 \\
	5153 -11.9668 \\
	5154 3.9889 \\
	5155 -3.9889 \\
	5156 -3.9889 \\
	5157 -3.9889 \\
	5158 11.9668 \\
	5159 -19.9446 \\
	5160 -3.9889 \\
	5161 3.9889 \\
	5162 3.9889 \\
	5163 11.9668 \\
	5164 3.9889 \\
	5165 -3.9889 \\
	5166 -19.9446 \\
	5167 -11.9668 \\
	5168 11.9668 \\
	5169 3.9889 \\
	5170 -3.9889 \\
	5171 11.9668 \\
	5172 3.9889 \\
	5173 3.9889 \\
	5174 -3.9889 \\
	5175 -11.9668 \\
	5176 11.9668 \\
	5177 -3.9889 \\
	5178 3.9889 \\
	5179 3.9889 \\
	5180 -3.9889 \\
	5181 11.9668 \\
	5182 -3.9889 \\
	5183 3.9889 \\
	5184 3.9889 \\
	5185 3.9889 \\
	5186 -11.9668 \\
	5187 3.9889 \\
	5188 3.9889 \\
	5189 3.9889 \\
	5190 -3.9889 \\
	5191 -3.9889 \\
	5192 19.9446 \\
	5193 3.9889 \\
	5194 -3.9889 \\
	5195 3.9889 \\
	5196 3.9889 \\
	5197 11.9668 \\
	5198 -3.9889 \\
	5199 -3.9889 \\
	5200 3.9889 \\
	5201 -3.9889 \\
	5202 -3.9889 \\
	5203 3.9889 \\
	5204 -11.9668 \\
	5205 3.9889 \\
	5206 3.9889 \\
	5207 -3.9889 \\
	5208 -3.9889 \\
	5209 -19.9446 \\
	5210 -3.9889 \\
	5211 -3.9889 \\
	5212 3.9889 \\
	5213 11.9668 \\
	5214 -3.9889 \\
	5215 19.9446 \\
	5216 3.9889 \\
	5217 -11.9668 \\
	5218 3.9889 \\
	5219 11.9668 \\
	5220 -3.9889 \\
	5221 -19.9446 \\
	5222 -3.9889 \\
	5223 -3.9889 \\
	5224 3.9889 \\
	5225 3.9889 \\
	5226 3.9889 \\
	5227 -11.9668 \\
	5228 -11.9668 \\
	5229 11.9668 \\
	5230 3.9889 \\
	5231 11.9668 \\
	5232 11.9668 \\
	5233 -11.9668 \\
	5234 -3.9889 \\
	5235 3.9889 \\
	5236 -3.9889 \\
	5237 -3.9889 \\
	5238 -3.9889 \\
	5239 -3.9889 \\
	5240 3.9889 \\
	5241 -3.9889 \\
	5242 -11.9668 \\
	5243 3.9889 \\
	5244 3.9889 \\
	5245 -11.9668 \\
	5246 11.9668 \\
	5247 3.9889 \\
	5248 -3.9889 \\
	5249 3.9889 \\
	5250 3.9889 \\
	5251 -3.9889 \\
	5252 -3.9889 \\
	5253 -3.9889 \\
	5254 -11.9668 \\
	5255 11.9668 \\
	5256 -11.9668 \\
	5257 -11.9668 \\
	5258 -3.9889 \\
	5259 -3.9889 \\
	5260 3.9889 \\
	5261 -3.9889 \\
	5262 3.9889 \\
	5263 -11.9668 \\
	5264 -3.9889 \\
	5265 -3.9889 \\
	5266 3.9889 \\
	5267 3.9889 \\
	5268 -11.9668 \\
	5269 -3.9889 \\
	5270 -3.9889 \\
	5271 3.9889 \\
	5272 3.9889 \\
	5273 -3.9889 \\
	5274 19.9446 \\
	5275 -3.9889 \\
	5276 -3.9889 \\
	5277 3.9889 \\
	5278 -3.9889 \\
	5279 3.9889 \\
	5280 3.9889 \\
	5281 -3.9889 \\
	5282 3.9889 \\
	5283 11.9668 \\
	5284 3.9889 \\
	5285 -3.9889 \\
	5286 3.9889 \\
	5287 19.9446 \\
	5288 11.9668 \\
	5289 3.9889 \\
	5290 -3.9889 \\
	5291 -3.9889 \\
	5292 -11.9668 \\
	5293 -3.9889 \\
	5294 3.9889 \\
	5295 3.9889 \\
	5296 -3.9889 \\
	5297 -3.9889 \\
	5298 3.9889 \\
	5299 -3.9889 \\
	5300 3.9889 \\
	5301 -3.9889 \\
	5302 -3.9889 \\
	5303 3.9889 \\
	5304 -3.9889 \\
	5305 11.9668 \\
	5306 3.9889 \\
	5307 11.9668 \\
	5308 19.9446 \\
	5309 -11.9668 \\
	5310 -3.9889 \\
	5311 -3.9889 \\
	5312 -3.9889 \\
	5313 -3.9889 \\
	5314 -3.9889 \\
	5315 11.9668 \\
	5316 -3.9889 \\
	5317 3.9889 \\
	5318 -3.9889 \\
	5319 3.9889 \\
	5320 -11.9668 \\
	5321 -11.9668 \\
	5322 -3.9889 \\
	5323 -3.9889 \\
	5324 -3.9889 \\
	5325 -3.9889 \\
	5326 19.9446 \\
	5327 27.9225 \\
	5328 3.9889 \\
	5329 -3.9889 \\
	5330 3.9889 \\
	5331 3.9889 \\
	5332 -3.9889 \\
	5333 -11.9668 \\
	5334 3.9889 \\
	5335 -3.9889 \\
	5336 -3.9889 \\
	5337 3.9889 \\
	5338 -3.9889 \\
	5339 11.9668 \\
	5340 11.9668 \\
	5341 3.9889 \\
	5342 -3.9889 \\
	5343 -3.9889 \\
	5344 11.9668 \\
	5345 -3.9889 \\
	5346 3.9889 \\
	5347 11.9668 \\
	5348 3.9889 \\
	5349 -3.9889 \\
	5350 -3.9889 \\
	5351 3.9889 \\
	5352 -3.9889 \\
	5353 3.9889 \\
	5354 3.9889 \\
	5355 -3.9889 \\
	5356 3.9889 \\
	5357 3.9889 \\
	5358 -3.9889 \\
	5359 -19.9446 \\
	5360 -3.9889 \\
	5361 -3.9889 \\
	5362 -3.9889 \\
	5363 3.9889 \\
	5364 -11.9668 \\
	5365 3.9889 \\
	5366 -3.9889 \\
	5367 -3.9889 \\
	5368 11.9668 \\
	5369 -11.9668 \\
	5370 -11.9668 \\
	5371 11.9668 \\
	5372 3.9889 \\
	5373 3.9889 \\
	5374 3.9889 \\
	5375 -3.9889 \\
	5376 11.9668 \\
	5377 -3.9889 \\
	5378 -3.9889 \\
	5379 -3.9889 \\
	5380 3.9889 \\
	5381 -3.9889 \\
	5382 -3.9889 \\
	5383 -3.9889 \\
	5384 11.9668 \\
	5385 -3.9889 \\
	5386 -3.9889 \\
	5387 3.9889 \\
	5388 -3.9889 \\
	5389 11.9668 \\
	5390 -3.9889 \\
	5391 11.9668 \\
	5392 3.9889 \\
	5393 -11.9668 \\
	5394 -3.9889 \\
	5395 -3.9889 \\
	5396 -3.9889 \\
	5397 3.9889 \\
	5398 3.9889 \\
	5399 3.9889 \\
	5400 3.9889 \\
	5401 -3.9889 \\
	5402 -3.9889 \\
	5403 -3.9889 \\
	5404 -3.9889 \\
	5405 -3.9889 \\
	5406 3.9889 \\
	5407 19.9446 \\
	5408 -3.9889 \\
	5409 -3.9889 \\
	5410 3.9889 \\
	5411 -3.9889 \\
	5412 -3.9889 \\
	5413 3.9889 \\
	5414 3.9889 \\
	5415 -19.9446 \\
	5416 -3.9889 \\
	5417 11.9668 \\
	5418 -3.9889 \\
	5419 -3.9889 \\
	5420 -3.9889 \\
	5421 -11.9668 \\
	5422 -3.9889 \\
	5423 11.9668 \\
	5424 3.9889 \\
	5425 3.9889 \\
	5426 11.9668 \\
	5427 -3.9889 \\
	5428 3.9889 \\
	5429 11.9668 \\
	5430 3.9889 \\
	5431 -3.9889 \\
	5432 3.9889 \\
	5433 -3.9889 \\
	5434 11.9668 \\
	5435 -3.9889 \\
	5436 -3.9889 \\
	5437 11.9668 \\
	5438 -3.9889 \\
	5439 3.9889 \\
	5440 3.9889 \\
	5441 3.9889 \\
	5442 3.9889 \\
	5443 3.9889 \\
	5444 -3.9889 \\
	5445 -11.9668 \\
	5446 -3.9889 \\
	5447 3.9889 \\
	5448 3.9889 \\
	5449 3.9889 \\
	5450 -11.9668 \\
	5451 -3.9889 \\
	5452 3.9889 \\
	5453 -11.9668 \\
	5454 3.9889 \\
	5455 11.9668 \\
	5456 -3.9889 \\
	5457 3.9889 \\
	5458 -11.9668 \\
	5459 11.9668 \\
	5460 11.9668 \\
	5461 -3.9889 \\
	5462 3.9889 \\
	5463 11.9668 \\
	5464 11.9668 \\
	5465 -3.9889 \\
	5466 3.9889 \\
	5467 3.9889 \\
	5468 3.9889 \\
	5469 3.9889 \\
	5470 -11.9668 \\
	5471 3.9889 \\
	5472 11.9668 \\
	5473 -3.9889 \\
	5474 -3.9889 \\
	5475 -3.9889 \\
	5476 3.9889 \\
	5477 3.9889 \\
	5478 3.9889 \\
	5479 -3.9889 \\
	5480 -11.9668 \\
	5481 -3.9889 \\
	5482 -3.9889 \\
	5483 -11.9668 \\
	5484 11.9668 \\
	5485 11.9668 \\
	5486 3.9889 \\
	5487 3.9889 \\
	5488 -3.9889 \\
	5489 -3.9889 \\
	5490 3.9889 \\
	5491 19.9446 \\
	5492 3.9889 \\
	5493 -3.9889 \\
	5494 3.9889 \\
	5495 3.9889 \\
	5496 -11.9668 \\
	5497 -11.9668 \\
	5498 -3.9889 \\
	5499 -19.9446 \\
	5500 3.9889 \\
	5501 3.9889 \\
	5502 -3.9889 \\
	5503 -3.9889 \\
	5504 -11.9668 \\
	5505 -3.9889 \\
	5506 11.9668 \\
	5507 11.9668 \\
	5508 3.9889 \\
	5509 3.9889 \\
	5510 -11.9668 \\
	5511 3.9889 \\
	5512 3.9889 \\
	5513 -3.9889 \\
	5514 -3.9889 \\
	5515 -3.9889 \\
	5516 -3.9889 \\
	5517 -3.9889 \\
	5518 -3.9889 \\
	5519 -3.9889 \\
	5520 3.9889 \\
	5521 -3.9889 \\
	5522 11.9668 \\
	5523 3.9889 \\
	5524 3.9889 \\
	5525 -3.9889 \\
	5526 -11.9668 \\
	5527 11.9668 \\
	5528 3.9889 \\
	5529 3.9889 \\
	5530 3.9889 \\
	5531 -11.9668 \\
	5532 -19.9446 \\
	5533 -3.9889 \\
	5534 3.9889 \\
	5535 11.9668 \\
	5536 3.9889 \\
	5537 -3.9889 \\
	5538 3.9889 \\
	5539 11.9668 \\
	5540 -3.9889 \\
	5541 -3.9889 \\
	5542 3.9889 \\
	5543 -3.9889 \\
	5544 -3.9889 \\
	5545 11.9668 \\
	5546 3.9889 \\
	5547 -3.9889 \\
	5548 -3.9889 \\
	5549 -3.9889 \\
	5550 -3.9889 \\
	5551 3.9889 \\
	5552 -3.9889 \\
	5553 3.9889 \\
	5554 3.9889 \\
	5555 -11.9668 \\
	5556 19.9446 \\
	5557 -3.9889 \\
	5558 -11.9668 \\
	5559 11.9668 \\
	5560 -3.9889 \\
	5561 3.9889 \\
	5562 -11.9668 \\
	5563 3.9889 \\
	5564 3.9889 \\
	5565 -11.9668 \\
	5566 11.9668 \\
	5567 -3.9889 \\
	5568 -3.9889 \\
	5569 -3.9889 \\
	5570 -3.9889 \\
	5571 3.9889 \\
	5572 -3.9889 \\
	5573 -3.9889 \\
	5574 3.9889 \\
	5575 3.9889 \\
	5576 19.9446 \\
	5577 -3.9889 \\
	5578 3.9889 \\
	5579 11.9668 \\
	5580 -3.9889 \\
	5581 3.9889 \\
	5582 -11.9668 \\
	5583 -3.9889 \\
	5584 -3.9889 \\
	5585 3.9889 \\
	5586 -3.9889 \\
	5587 -11.9668 \\
	5588 11.9668 \\
	5589 3.9889 \\
	5590 3.9889 \\
	5591 -3.9889 \\
	5592 -11.9668 \\
	5593 -3.9889 \\
	5594 -3.9889 \\
	5595 11.9668 \\
	5596 -3.9889 \\
	5597 -19.9446 \\
	5598 3.9889 \\
	5599 -3.9889 \\
	5600 -11.9668 \\
	5601 -11.9668 \\
	5602 -11.9668 \\
	5603 -3.9889 \\
	5604 3.9889 \\
	5605 19.9446 \\
	5606 3.9889 \\
	5607 -11.9668 \\
	5608 3.9889 \\
	5609 -11.9668 \\
	5610 3.9889 \\
	5611 3.9889 \\
	5612 -3.9889 \\
	5613 11.9668 \\
	5614 -3.9889 \\
	5615 3.9889 \\
	5616 11.9668 \\
	5617 -3.9889 \\
	5618 3.9889 \\
	5619 11.9668 \\
	5620 3.9889 \\
	5621 11.9668 \\
	5622 3.9889 \\
	5623 -3.9889 \\
	5624 3.9889 \\
	5625 -19.9446 \\
	5626 -3.9889 \\
	5627 11.9668 \\
	5628 11.9668 \\
	5629 -3.9889 \\
	5630 -19.9446 \\
	5631 -3.9889 \\
	5632 -3.9889 \\
	5633 3.9889 \\
	5634 3.9889 \\
	5635 3.9889 \\
	5636 3.9889 \\
	5637 -3.9889 \\
	5638 -3.9889 \\
	5639 -3.9889 \\
	5640 3.9889 \\
	5641 11.9668 \\
	5642 3.9889 \\
	5643 3.9889 \\
	5644 -3.9889 \\
	5645 3.9889 \\
	5646 11.9668 \\
	5647 11.9668 \\
	5648 -3.9889 \\
	5649 3.9889 \\
	5650 11.9668 \\
	5651 -3.9889 \\
	5652 3.9889 \\
	5653 -3.9889 \\
	5654 -3.9889 \\
	5655 3.9889 \\
	5656 -19.9446 \\
	5657 -3.9889 \\
	5658 3.9889 \\
	5659 11.9668 \\
	5660 3.9889 \\
	5661 3.9889 \\
	5662 3.9889 \\
	5663 -3.9889 \\
	5664 -3.9889 \\
	5665 3.9889 \\
	5666 -11.9668 \\
	5667 -3.9889 \\
	5668 -3.9889 \\
	5669 -3.9889 \\
	5670 3.9889 \\
	5671 -3.9889 \\
	5672 -3.9889 \\
	5673 -11.9668 \\
	5674 -11.9668 \\
	5675 3.9889 \\
	5676 -11.9668 \\
	5677 3.9889 \\
	5678 3.9889 \\
	5679 11.9668 \\
	5680 11.9668 \\
	5681 -19.9446 \\
	5682 -11.9668 \\
	5683 3.9889 \\
	5684 11.9668 \\
	5685 -3.9889 \\
	5686 3.9889 \\
	5687 11.9668 \\
	5688 -11.9668 \\
	5689 3.9889 \\
	5690 -11.9668 \\
	5691 -3.9889 \\
	5692 3.9889 \\
	5693 3.9889 \\
	5694 11.9668 \\
	5695 3.9889 \\
	5696 -3.9889 \\
	5697 -3.9889 \\
	5698 3.9889 \\
	5699 3.9889 \\
	5700 -3.9889 \\
	5701 11.9668 \\
	5702 -3.9889 \\
	5703 -11.9668 \\
	5704 -11.9668 \\
	5705 3.9889 \\
	5706 -11.9668 \\
	5707 -11.9668 \\
	5708 -3.9889 \\
	5709 -19.9446 \\
	5710 -3.9889 \\
	5711 3.9889 \\
	5712 3.9889 \\
	5713 3.9889 \\
	5714 -19.9446 \\
	5715 3.9889 \\
	5716 3.9889 \\
	5717 3.9889 \\
	5718 -3.9889 \\
	5719 -3.9889 \\
	5720 11.9668 \\
	5721 3.9889 \\
	5722 -3.9889 \\
	5723 3.9889 \\
	5724 -3.9889 \\
	5725 -3.9889 \\
	5726 3.9889 \\
	5727 3.9889 \\
	5728 11.9668 \\
	5729 -3.9889 \\
	5730 -11.9668 \\
	5731 3.9889 \\
	5732 3.9889 \\
	5733 3.9889 \\
	5734 11.9668 \\
	5735 3.9889 \\
	5736 3.9889 \\
	5737 3.9889 \\
	5738 11.9668 \\
	5739 3.9889 \\
	5740 -3.9889 \\
	5741 -11.9668 \\
	5742 3.9889 \\
	5743 -3.9889 \\
	5744 -3.9889 \\
	5745 3.9889 \\
	5746 11.9668 \\
	5747 11.9668 \\
	5748 -3.9889 \\
	5749 -11.9668 \\
	5750 -11.9668 \\
	5751 3.9889 \\
	5752 11.9668 \\
	5753 -3.9889 \\
	5754 3.9889 \\
	5755 11.9668 \\
	5756 -3.9889 \\
	5757 -3.9889 \\
	5758 3.9889 \\
	5759 3.9889 \\
	5760 3.9889 \\
	5761 11.9668 \\
	5762 -3.9889 \\
	5763 11.9668 \\
	5764 -3.9889 \\
	5765 -11.9668 \\
	5766 3.9889 \\
	5767 3.9889 \\
	5768 11.9668 \\
	5769 -3.9889 \\
	5770 3.9889 \\
	5771 -3.9889 \\
	5772 -3.9889 \\
	5773 3.9889 \\
	5774 3.9889 \\
	5775 11.9668 \\
	5776 -3.9889 \\
	5777 3.9889 \\
	5778 -3.9889 \\
	5779 -11.9668 \\
	5780 -3.9889 \\
	5781 3.9889 \\
	5782 3.9889 \\
	5783 -3.9889 \\
	5784 3.9889 \\
	5785 3.9889 \\
	5786 -3.9889 \\
	5787 -3.9889 \\
	5788 3.9889 \\
	5789 3.9889 \\
	5790 -3.9889 \\
	5791 3.9889 \\
	5792 3.9889 \\
	5793 11.9668 \\
	5794 11.9668 \\
	5795 -11.9668 \\
	5796 -11.9668 \\
	5797 3.9889 \\
	5798 11.9668 \\
	5799 3.9889 \\
	5800 -11.9668 \\
	5801 -3.9889 \\
	5802 3.9889 \\
	5803 -11.9668 \\
	5804 -3.9889 \\
	5805 3.9889 \\
	5806 3.9889 \\
	5807 -3.9889 \\
	5808 -11.9668 \\
	5809 3.9889 \\
	5810 11.9668 \\
	5811 -3.9889 \\
	5812 11.9668 \\
	5813 3.9889 \\
	5814 -19.9446 \\
	5815 -3.9889 \\
	5816 -3.9889 \\
	5817 -11.9668 \\
	5818 -3.9889 \\
	5819 -3.9889 \\
	5820 3.9889 \\
	5821 3.9889 \\
	5822 19.9446 \\
	5823 -11.9668 \\
	5824 -11.9668 \\
	5825 3.9889 \\
	5826 11.9668 \\
	5827 -3.9889 \\
	5828 -3.9889 \\
	5829 -3.9889 \\
	5830 -11.9668 \\
	5831 -11.9668 \\
	5832 11.9668 \\
	5833 3.9889 \\
	5834 -3.9889 \\
	5835 -3.9889 \\
	5836 -3.9889 \\
	5837 11.9668 \\
	5838 -3.9889 \\
	5839 3.9889 \\
	5840 11.9668 \\
	5841 -3.9889 \\
	5842 -3.9889 \\
	5843 -19.9446 \\
	5844 11.9668 \\
	5845 11.9668 \\
	5846 -11.9668 \\
	5847 3.9889 \\
	5848 3.9889 \\
	5849 -3.9889 \\
	5850 -3.9889 \\
	5851 3.9889 \\
	5852 11.9668 \\
	5853 -3.9889 \\
	5854 3.9889 \\
	5855 -3.9889 \\
	5856 3.9889 \\
	5857 11.9668 \\
	5858 -11.9668 \\
	5859 3.9889 \\
	5860 3.9889 \\
	5861 3.9889 \\
	5862 3.9889 \\
	5863 -11.9668 \\
	5864 -3.9889 \\
	5865 3.9889 \\
	5866 -11.9668 \\
	5867 -11.9668 \\
	5868 3.9889 \\
	5869 -3.9889 \\
	5870 -3.9889 \\
	5871 -3.9889 \\
	5872 -11.9668 \\
	5873 3.9889 \\
	5874 3.9889 \\
	5875 -11.9668 \\
	5876 -3.9889 \\
	5877 -3.9889 \\
	5878 3.9889 \\
	5879 3.9889 \\
	5880 3.9889 \\
	5881 3.9889 \\
	5882 -3.9889 \\
	5883 3.9889 \\
	5884 3.9889 \\
	5885 3.9889 \\
	5886 11.9668 \\
	5887 3.9889 \\
	5888 -3.9889 \\
	5889 -3.9889 \\
	5890 3.9889 \\
	5891 19.9446 \\
	5892 3.9889 \\
	5893 -3.9889 \\
	5894 3.9889 \\
	5895 -3.9889 \\
	5896 -3.9889 \\
	5897 3.9889 \\
	5898 11.9668 \\
	5899 11.9668 \\
	5900 -11.9668 \\
	5901 3.9889 \\
	5902 -3.9889 \\
	5903 3.9889 \\
	5904 3.9889 \\
	5905 -3.9889 \\
	5906 3.9889 \\
	5907 -3.9889 \\
	5908 11.9668 \\
	5909 -3.9889 \\
	5910 -3.9889 \\
	5911 3.9889 \\
	5912 3.9889 \\
	5913 -11.9668 \\
	5914 11.9668 \\
	5915 11.9668 \\
	5916 -19.9446 \\
	5917 11.9668 \\
	5918 -3.9889 \\
	5919 -3.9889 \\
	5920 3.9889 \\
	5921 -11.9668 \\
	5922 3.9889 \\
	5923 3.9889 \\
	5924 -11.9668 \\
	5925 -3.9889 \\
	5926 3.9889 \\
	5927 -3.9889 \\
	5928 -3.9889 \\
	5929 3.9889 \\
	5930 -3.9889 \\
	5931 3.9889 \\
	5932 -3.9889 \\
	5933 3.9889 \\
	5934 -11.9668 \\
	5935 -11.9668 \\
	5936 3.9889 \\
	5937 -3.9889 \\
	5938 3.9889 \\
	5939 -3.9889 \\
	5940 3.9889 \\
	5941 -3.9889 \\
	5942 -3.9889 \\
	5943 11.9668 \\
	5944 -3.9889 \\
	5945 3.9889 \\
	5946 -3.9889 \\
	5947 -11.9668 \\
	5948 -3.9889 \\
	5949 -11.9668 \\
	5950 3.9889 \\
	5951 11.9668 \\
	5952 11.9668 \\
	5953 11.9668 \\
	5954 -3.9889 \\
	5955 -3.9889 \\
	5956 3.9889 \\
	5957 -3.9889 \\
	5958 3.9889 \\
	5959 3.9889 \\
	5960 -3.9889 \\
	5961 3.9889 \\
	5962 -3.9889 \\
	5963 11.9668 \\
	5964 11.9668 \\
	5965 -3.9889 \\
	5966 3.9889 \\
	5967 3.9889 \\
	5968 3.9889 \\
	5969 3.9889 \\
	5970 -3.9889 \\
	5971 3.9889 \\
	5972 -11.9668 \\
	5973 -3.9889 \\
	5974 3.9889 \\
	5975 3.9889 \\
	5976 3.9889 \\
	5977 11.9668 \\
	5978 -3.9889 \\
	5979 -3.9889 \\
	5980 3.9889 \\
	5981 -3.9889 \\
	5982 3.9889 \\
	5983 -3.9889 \\
	5984 -3.9889 \\
	5985 -3.9889 \\
	5986 3.9889 \\
	5987 11.9668 \\
	5988 -3.9889 \\
	5989 3.9889 \\
	5990 -11.9668 \\
	5991 3.9889 \\
	5992 -3.9889 \\
	5993 -19.9446 \\
	5994 3.9889 \\
	5995 11.9668 \\
	5996 -11.9668 \\
	5997 -11.9668 \\
	5998 11.9668 \\
	5999 -3.9889 \\
	6000 -3.9889 \\
	6001 3.9889 \\
	6002 3.9889 \\
	6003 11.9668 \\
	6004 -3.9889 \\
	6005 3.9889 \\
	6006 11.9668 \\
	6007 3.9889 \\
	6008 11.9668 \\
	6009 -11.9668 \\
	6010 -19.9446 \\
	6011 -11.9668 \\
	6012 -3.9889 \\
	6013 -3.9889 \\
	6014 -3.9889 \\
	6015 -3.9889 \\
	6016 -11.9668 \\
	6017 3.9889 \\
	6018 -3.9889 \\
	6019 -3.9889 \\
	6020 3.9889 \\
	6021 3.9889 \\
	6022 3.9889 \\
	6023 -3.9889 \\
	6024 -11.9668 \\
	6025 11.9668 \\
	6026 19.9446 \\
	6027 19.9446 \\
	6028 3.9889 \\
	6029 -3.9889 \\
	6030 -3.9889 \\
	6031 3.9889 \\
	6032 3.9889 \\
	6033 -3.9889 \\
	6034 3.9889 \\
	6035 -3.9889 \\
	6036 3.9889 \\
	6037 11.9668 \\
	6038 -11.9668 \\
	6039 3.9889 \\
	6040 3.9889 \\
	6041 3.9889 \\
	6042 -3.9889 \\
	6043 -3.9889 \\
	6044 3.9889 \\
	6045 -11.9668 \\
	6046 3.9889 \\
	6047 3.9889 \\
	6048 -19.9446 \\
	6049 -11.9668 \\
	6050 -3.9889 \\
	6051 3.9889 \\
	6052 -3.9889 \\
	6053 -3.9889 \\
	6054 3.9889 \\
	6055 3.9889 \\
	6056 11.9668 \\
	6057 3.9889 \\
	6058 -19.9446 \\
	6059 -3.9889 \\
	6060 3.9889 \\
	6061 -3.9889 \\
	6062 -3.9889 \\
	6063 3.9889 \\
	6064 3.9889 \\
	6065 -11.9668 \\
	6066 -11.9668 \\
	6067 -3.9889 \\
	6068 3.9889 \\
	6069 -3.9889 \\
	6070 -3.9889 \\
	6071 3.9889 \\
	6072 3.9889 \\
	6073 -11.9668 \\
	6074 -3.9889 \\
	6075 3.9889 \\
	6076 -11.9668 \\
	6077 3.9889 \\
	6078 11.9668 \\
	6079 -3.9889 \\
	6080 3.9889 \\
	6081 -19.9446 \\
	6082 -11.9668 \\
	6083 3.9889 \\
	6084 3.9889 \\
	6085 -11.9668 \\
	6086 -3.9889 \\
	6087 3.9889 \\
	6088 -3.9889 \\
	6089 11.9668 \\
	6090 -3.9889 \\
	6091 -3.9889 \\
	6092 3.9889 \\
	6093 -3.9889 \\
	6094 11.9668 \\
	6095 -3.9889 \\
	6096 -3.9889 \\
	6097 3.9889 \\
	6098 3.9889 \\
	6099 3.9889 \\
	6100 -3.9889 \\
	6101 -11.9668 \\
	6102 -3.9889 \\
	6103 3.9889 \\
	6104 -11.9668 \\
	6105 3.9889 \\
	6106 3.9889 \\
	6107 3.9889 \\
	6108 11.9668 \\
	6109 3.9889 \\
	6110 3.9889 \\
	6111 -3.9889 \\
	6112 3.9889 \\
	6113 -3.9889 \\
	6114 -3.9889 \\
	6115 11.9668 \\
	6116 11.9668 \\
	6117 11.9668 \\
	6118 -3.9889 \\
	6119 3.9889 \\
	6120 -3.9889 \\
	6121 3.9889 \\
	6122 11.9668 \\
	6123 -11.9668 \\
	6124 3.9889 \\
	6125 -3.9889 \\
	6126 -3.9889 \\
	6127 -3.9889 \\
	6128 -3.9889 \\
	6129 3.9889 \\
	6130 -11.9668 \\
	6131 -11.9668 \\
	6132 -3.9889 \\
	6133 -3.9889 \\
	6134 -3.9889 \\
	6135 -3.9889 \\
	6136 3.9889 \\
	6137 11.9668 \\
	6138 3.9889 \\
	6139 -3.9889 \\
	6140 3.9889 \\
	6141 -3.9889 \\
	6142 -3.9889 \\
	6143 3.9889 \\
	6144 3.9889 \\
	6145 -3.9889 \\
	6146 3.9889 \\
	6147 3.9889 \\
	6148 3.9889 \\
	6149 3.9889 \\
	6150 -3.9889 \\
	6151 -3.9889 \\
	6152 -11.9668 \\
	6153 3.9889 \\
	6154 3.9889 \\
	6155 3.9889 \\
	6156 3.9889 \\
	6157 3.9889 \\
	6158 -3.9889 \\
	6159 3.9889 \\
	6160 11.9668 \\
	6161 -3.9889 \\
	6162 -3.9889 \\
	6163 3.9889 \\
	6164 -3.9889 \\
	6165 3.9889 \\
	6166 3.9889 \\
	6167 -3.9889 \\
	6168 -3.9889 \\
	6169 -3.9889 \\
	6170 3.9889 \\
	6171 11.9668 \\
	6172 -3.9889 \\
	6173 -19.9446 \\
	6174 -3.9889 \\
	6175 -3.9889 \\
	6176 -3.9889 \\
	6177 3.9889 \\
	6178 -11.9668 \\
	6179 -3.9889 \\
	6180 11.9668 \\
	6181 -11.9668 \\
	6182 -11.9668 \\
	6183 -3.9889 \\
	6184 11.9668 \\
	6185 3.9889 \\
	6186 -11.9668 \\
	6187 -3.9889 \\
	6188 -3.9889 \\
	6189 3.9889 \\
	6190 -3.9889 \\
	6191 11.9668 \\
	6192 3.9889 \\
	6193 -11.9668 \\
	6194 3.9889 \\
	6195 3.9889 \\
	6196 11.9668 \\
	6197 -3.9889 \\
	6198 3.9889 \\
	6199 11.9668 \\
	6200 -11.9668 \\
	6201 11.9668 \\
	6202 3.9889 \\
	6203 3.9889 \\
	6204 3.9889 \\
	6205 -11.9668 \\
	6206 -3.9889 \\
	6207 3.9889 \\
	6208 3.9889 \\
	6209 3.9889 \\
	6210 -3.9889 \\
	6211 -3.9889 \\
	6212 3.9889 \\
	6213 11.9668 \\
	6214 11.9668 \\
	6215 -3.9889 \\
	6216 -3.9889 \\
	6217 11.9668 \\
	6218 -3.9889 \\
	6219 -3.9889 \\
	6220 3.9889 \\
	6221 3.9889 \\
	6222 -3.9889 \\
	6223 -11.9668 \\
	6224 -3.9889 \\
	6225 -19.9446 \\
	6226 3.9889 \\
	6227 3.9889 \\
	6228 3.9889 \\
	6229 3.9889 \\
	6230 -11.9668 \\
	6231 3.9889 \\
	6232 3.9889 \\
	6233 3.9889 \\
	6234 -3.9889 \\
	6235 -3.9889 \\
	6236 3.9889 \\
	6237 11.9668 \\
	6238 3.9889 \\
	6239 -3.9889 \\
	6240 3.9889 \\
	6241 -3.9889 \\
	6242 -3.9889 \\
	6243 -3.9889 \\
	6244 -11.9668 \\
	6245 3.9889 \\
	6246 3.9889 \\
	6247 3.9889 \\
	6248 -3.9889 \\
	6249 -3.9889 \\
	6250 3.9889 \\
	6251 3.9889 \\
	6252 11.9668 \\
	6253 -3.9889 \\
	6254 -3.9889 \\
	6255 3.9889 \\
	6256 3.9889 \\
	6257 3.9889 \\
	6258 -11.9668 \\
	6259 -11.9668 \\
	6260 3.9889 \\
	6261 -11.9668 \\
	6262 -11.9668 \\
	6263 3.9889 \\
	6264 -11.9668 \\
	6265 3.9889 \\
	6266 11.9668 \\
	6267 -3.9889 \\
	6268 -3.9889 \\
	6269 11.9668 \\
	6270 11.9668 \\
	6271 -3.9889 \\
	6272 -3.9889 \\
	6273 3.9889 \\
	6274 3.9889 \\
	6275 11.9668 \\
	6276 3.9889 \\
	6277 -3.9889 \\
	6278 11.9668 \\
	6279 11.9668 \\
	6280 3.9889 \\
	6281 -3.9889 \\
	6282 -3.9889 \\
	6283 -3.9889 \\
	6284 3.9889 \\
	6285 11.9668 \\
	6286 11.9668 \\
	6287 -3.9889 \\
	6288 -3.9889 \\
	6289 -3.9889 \\
	6290 -11.9668 \\
	6291 3.9889 \\
	6292 -3.9889 \\
	6293 -3.9889 \\
	6294 3.9889 \\
	6295 3.9889 \\
	6296 11.9668 \\
	6297 -3.9889 \\
	6298 -11.9668 \\
	6299 -3.9889 \\
	6300 3.9889 \\
	6301 -3.9889 \\
	6302 -11.9668 \\
	6303 -3.9889 \\
	6304 3.9889 \\
	6305 3.9889 \\
	6306 -3.9889 \\
	6307 -3.9889 \\
	6308 -3.9889 \\
	6309 3.9889 \\
	6310 -11.9668 \\
	6311 3.9889 \\
	6312 3.9889 \\
	6313 3.9889 \\
	6314 11.9668 \\
	6315 -3.9889 \\
	6316 3.9889 \\
	6317 3.9889 \\
	6318 11.9668 \\
	6319 11.9668 \\
	6320 -11.9668 \\
	6321 -3.9889 \\
	6322 3.9889 \\
	6323 11.9668 \\
	6324 -11.9668 \\
	6325 -3.9889 \\
	6326 3.9889 \\
	6327 -3.9889 \\
	6328 -3.9889 \\
	6329 11.9668 \\
	6330 3.9889 \\
	6331 -19.9446 \\
	6332 -3.9889 \\
	6333 -11.9668 \\
	6334 -3.9889 \\
	6335 11.9668 \\
	6336 -3.9889 \\
	6337 3.9889 \\
	6338 -3.9889 \\
	6339 -3.9889 \\
	6340 3.9889 \\
	6341 3.9889 \\
	6342 3.9889 \\
	6343 -3.9889 \\
	6344 3.9889 \\
	6345 -3.9889 \\
	6346 -11.9668 \\
	6347 3.9889 \\
	6348 -3.9889 \\
	6349 3.9889 \\
	6350 -3.9889 \\
	6351 -11.9668 \\
	6352 -3.9889 \\
	6353 -11.9668 \\
	6354 -3.9889 \\
	6355 3.9889 \\
	6356 -3.9889 \\
	6357 -3.9889 \\
	6358 -11.9668 \\
	6359 11.9668 \\
	6360 11.9668 \\
	6361 -3.9889 \\
	6362 3.9889 \\
	6363 -11.9668 \\
	6364 -3.9889 \\
	6365 3.9889 \\
	6366 -11.9668 \\
	6367 11.9668 \\
	6368 11.9668 \\
	6369 19.9446 \\
	6370 3.9889 \\
	6371 3.9889 \\
	6372 3.9889 \\
	6373 -19.9446 \\
	6374 -3.9889 \\
	6375 3.9889 \\
	6376 3.9889 \\
	6377 -3.9889 \\
	6378 -11.9668 \\
	6379 3.9889 \\
	6380 -3.9889 \\
	6381 -3.9889 \\
	6382 -3.9889 \\
	6383 3.9889 \\
	6384 3.9889 \\
	6385 -3.9889 \\
	6386 3.9889 \\
	6387 -3.9889 \\
	6388 3.9889 \\
	6389 11.9668 \\
	6390 -3.9889 \\
	6391 -3.9889 \\
	6392 11.9668 \\
	6393 3.9889 \\
	6394 3.9889 \\
	6395 19.9446 \\
	6396 -11.9668 \\
	6397 -3.9889 \\
	6398 -3.9889 \\
	6399 -3.9889 \\
	6400 3.9889 \\
};

\definecolor{matlabColor2}{rgb}{0.000000,0.447000,0.741000}
\addplot [color=matlabColor2, solid, line width=1.5pt, forget plot]
table[row sep=crcr]{
	1 5.0596 \\
	2 14.2641 \\
	3 -0.2257 \\
	4 -4.7159 \\
	5 -1.9147 \\
	6 -1.8173 \\
	7 14.0427 \\
	8 11.0529 \\
	9 0.37048 \\
	10 10.5232 \\
	11 -6.0546 \\
	12 5.8402 \\
	13 5.6921 \\
	14 -20.0238 \\
	15 -4.6902 \\
	16 2.8504 \\
	17 -2.5298 \\
	18 -10.8225 \\
	19 4.5342 \\
	20 5.7125 \\
	21 0.84232 \\
	22 -3.1055 \\
	23 -7.9955 \\
	24 11.8632 \\
	25 -7.002 \\
	26 -8.6958 \\
	27 -5.6284 \\
	28 -4.1437 \\
	29 5.6921 \\
	30 1.3989 \\
	31 15.3735 \\
	32 12.438 \\
	33 12.6491 \\
	34 -6.7998 \\
	35 -10.1829 \\
	36 5.9705 \\
	37 1.3907 \\
	38 8.8232 \\
	39 -4.0457 \\
	40 -5.3537 \\
	41 2.1593 \\
	42 1.4114 \\
	43 -4.77 \\
	44 2.2291 \\
	45 5.6921 \\
	46 -8.8313 \\
	47 9.0595 \\
	48 4.6332 \\
	49 -7.5895 \\
	50 -3.3543 \\
	51 -6.1609 \\
	52 -4.5968 \\
	53 2.2114 \\
	54 5.8615 \\
	55 -5.1451 \\
	56 -7.492 \\
	57 1.9423 \\
	58 2.6057 \\
	59 -17.0486 \\
	60 -5.4278 \\
	61 5.6921 \\
	62 -11.7954 \\
	63 -1.4202 \\
	64 -0.5023 \\
	65 -1.2649 \\
	66 -1.0408 \\
	67 3.4175 \\
	68 -3.9917 \\
	69 5.2408 \\
	70 -5.1268 \\
	71 5.9294 \\
	72 0.60608 \\
	73 -19.434 \\
	74 -6.1922 \\
	75 -10.5206 \\
	76 1.4482 \\
	77 15.1209 \\
	78 6.7869 \\
	79 8.2455 \\
	80 9.0227 \\
	81 1.2649 \\
	82 -5.3378 \\
	83 0.95686 \\
	84 6.4045 \\
	85 -6.3131 \\
	86 -3.0306 \\
	87 9.4721 \\
	88 0.040095 \\
	89 -11.3206 \\
	90 -4.8932 \\
	91 6.2396 \\
	92 -6.1757 \\
	93 5.7943 \\
	94 12.0507 \\
	95 0.9067 \\
	96 8.6561 \\
	97 -6.3246 \\
	98 5.7401 \\
	99 0.53988 \\
	100 -9.6843 \\
	101 8.5461 \\
	102 -6.2749 \\
	103 -5.0954 \\
	104 -2.0542 \\
	105 -3.3344 \\
	106 -5.1769 \\
	107 1.7173 \\
	108 6.7884 \\
	109 -1.6409 \\
	110 5.1492 \\
	111 -0.10496 \\
	112 0.095873 \\
	113 -6.3246 \\
	114 3.8425 \\
	115 11.3126 \\
	116 -11.3776 \\
	117 -4.944 \\
	118 -2.7107 \\
	119 -4.1985 \\
	120 5.2846 \\
	121 1.2013 \\
	122 -6.3122 \\
	123 11.6351 \\
	124 7.3369 \\
	125 -6.6252 \\
	126 2.4073 \\
	127 -10.0951 \\
	128 -2.2807 \\
	129 -1.2649 \\
	130 5.1433 \\
	131 0.83129 \\
	132 7.2471 \\
	133 1.8263 \\
	134 -5.2736 \\
	135 16.8999 \\
	136 3.7425 \\
	137 -3.9482 \\
	138 3.8811 \\
	139 -6.5631 \\
	140 5.8671 \\
	141 -1.5122 \\
	142 8.631 \\
	143 4.4547 \\
	144 -6.42 \\
	145 6.3246 \\
	146 -7.9933 \\
	147 10.7178 \\
	148 7.9906 \\
	149 -8.9105 \\
	150 -2.0377 \\
	151 -1.068 \\
	152 -4.9352 \\
	153 -7.002 \\
	154 10.5869 \\
	155 4.6707 \\
	156 -2.1671 \\
	157 1.226 \\
	158 14.5415 \\
	159 10.9824 \\
	160 -7.3409 \\
	161 -1.2649 \\
	162 -21.3658 \\
	163 -4.2521 \\
	164 18.7871 \\
	165 -8.8918 \\
	166 -10.6675 \\
	167 -2.1299 \\
	168 0.27538 \\
	169 1.4184 \\
	170 -1.4019 \\
	171 0.82744 \\
	172 5.9732 \\
	173 7.8368 \\
	174 -2.2191 \\
	175 -12.6562 \\
	176 -0.69926 \\
	177 -1.2649 \\
	178 -7.0783 \\
	179 2.3882 \\
	180 -2.2308 \\
	181 -1.7327 \\
	182 -7.1746 \\
	183 -11.1721 \\
	184 -6.3487 \\
	185 1.9423 \\
	186 3.0493 \\
	187 -3.5607 \\
	188 16.6443 \\
	189 5.0985 \\
	190 -10.979 \\
	191 -0.25101 \\
	192 -6.0274 \\
	193 11.3842 \\
	194 9.2257 \\
	195 -6.0942 \\
	196 -7.6541 \\
	197 -7.4929 \\
	198 -6.1321 \\
	199 9.1672 \\
	200 -1.908 \\
	201 -13.697 \\
	202 1.1948 \\
	203 6.6078 \\
	204 3.6645 \\
	205 -0.57075 \\
	206 -4.4772 \\
	207 8.7794 \\
	208 0.73521 \\
	209 5.0596 \\
	210 9.8664 \\
	211 -0.22513 \\
	212 5.2201 \\
	213 0.093417 \\
	214 6.4998 \\
	215 3.7904 \\
	216 10.9185 \\
	217 4.8426 \\
	218 -5.9808 \\
	219 4.1292 \\
	220 -6.4932 \\
	221 8.366 \\
	222 4.3493 \\
	223 -0.73867 \\
	224 -5.0987 \\
	225 -16.4438 \\
	226 7.7868 \\
	227 4.3597 \\
	228 -7.9215 \\
	229 3.3913 \\
	230 3.3303 \\
	231 -16.5508 \\
	232 1.8671 \\
	233 -6.5416 \\
	234 -18.8281 \\
	235 15.1514 \\
	236 -2.7157 \\
	237 8.3772 \\
	238 15.2167 \\
	239 -7.2151 \\
	240 -2.9107 \\
	241 2.5298 \\
	242 9.0524 \\
	243 -9.8214 \\
	244 -7.9671 \\
	245 6.5381 \\
	246 5.2007 \\
	247 -3.9962 \\
	248 -7.7339 \\
	249 -2.3128 \\
	250 -4.2936 \\
	251 -9.0478 \\
	252 -12.7782 \\
	253 6.5959 \\
	254 -1.6532 \\
	255 -8.4151 \\
	256 10.4178 \\
	257 7.5895 \\
	258 3.5699 \\
	259 -6.2421 \\
	260 -0.7818 \\
	261 0.26913 \\
	262 -19.4621 \\
	263 -5.3183 \\
	264 12.0434 \\
	265 -4.6892 \\
	266 6.2642 \\
	267 0.49622 \\
	268 -9.4392 \\
	269 -4.1345 \\
	270 -11.8156 \\
	271 2.6647 \\
	272 -7.0232 \\
	273 -2.5298 \\
	274 9.2382 \\
	275 -5.4872 \\
	276 -0.43519 \\
	277 5.0873 \\
	278 6.0597 \\
	279 -3.4175 \\
	280 3.1857 \\
	281 10.5797 \\
	282 7.6514 \\
	283 -3.7267 \\
	284 -3.9986 \\
	285 10.8759 \\
	286 -5.4225 \\
	287 7.9183 \\
	288 5.3547 \\
	289 -7.5895 \\
	290 7.1894 \\
	291 2.1186 \\
	292 6.7465 \\
	293 -1.1 \\
	294 -5.1096 \\
	295 1.398 \\
	296 1.8494 \\
	297 -2.9003 \\
	298 -11.046 \\
	299 -5.3155 \\
	300 5.7623 \\
	301 3.6105 \\
	302 -7.796 \\
	303 7.3098 \\
	304 1.4164 \\
	305 6.6613e-16 \\
	306 9.9743 \\
	307 -6.6161 \\
	308 0.8354 \\
	309 8.3927 \\
	310 18.1383 \\
	311 11.3496 \\
	312 -2.424 \\
	313 -5.52 \\
	314 -16.4347 \\
	315 -0.52547 \\
	316 -0.78513 \\
	317 -7.8221 \\
	318 -1.1181 \\
	319 -6.7257 \\
	320 -2.1874 \\
	321 2.5298 \\
	322 -6.8731 \\
	323 -5.7624 \\
	324 -12.1039 \\
	325 -4.1933 \\
	326 3.7278 \\
	327 1.5086 \\
	328 4.022 \\
	329 -2.3128 \\
	330 -12.1204 \\
	331 6.3234 \\
	332 -1.2844 \\
	333 -7.2459 \\
	334 -0.70368 \\
	335 -8.8801 \\
	336 2.1037 \\
	337 1.2649 \\
	338 5.0805 \\
	339 5.3392 \\
	340 9.1511 \\
	341 4.7472 \\
	342 -7.5604 \\
	343 3.415 \\
	344 1.8659 \\
	345 2.5298 \\
	346 -6.8447 \\
	347 4.2179 \\
	348 6.6866 \\
	349 -9.2996 \\
	350 9.3208 \\
	351 10.712 \\
	352 -1.5037 \\
	353 5.0596 \\
	354 18.4994 \\
	355 -2.6052 \\
	356 0.49636 \\
	357 8.295 \\
	358 -10.7163 \\
	359 1.5368 \\
	360 0.66107 \\
	361 4.8426 \\
	362 0.88241 \\
	363 -5.1759 \\
	364 -6.0357 \\
	365 -14.2576 \\
	366 -1.1775 \\
	367 -3.1186 \\
	368 -2.3142 \\
	369 -1.2649 \\
	370 -5.5983 \\
	371 5.5583 \\
	372 10.0459 \\
	373 -11.3787 \\
	374 -0.78555 \\
	375 20.4995 \\
	376 1.0405 \\
	377 2.5298 \\
	378 -8.2049 \\
	379 -2.8355 \\
	380 8.2229 \\
	381 -12.2039 \\
	382 -7.2841 \\
	383 -0.37506 \\
	384 9.3038 \\
	385 -2.5298 \\
	386 5.7643 \\
	387 4.6923 \\
	388 6.7402 \\
	389 1.5234 \\
	390 3.4933 \\
	391 -4.3626 \\
	392 -6.1311 \\
	393 9.9658 \\
	394 2.8313 \\
	395 7.3272 \\
	396 -5.2284 \\
	397 -1.2545 \\
	398 8.594 \\
	399 -7.6234 \\
	400 2.6198 \\
	401 -8.8818e-16 \\
	402 -3.1268 \\
	403 13.7652 \\
	404 14.0011 \\
	405 3.8383 \\
	406 4.6745 \\
	407 -1.1853 \\
	408 5.6405 \\
	409 6.1711 \\
	410 -12.6811 \\
	411 1.4252 \\
	412 1.2097 \\
	413 -1.9466 \\
	414 2.0737 \\
	415 -7.6629 \\
	416 -4.2298 \\
	417 -8.8818e-16 \\
	418 -0.99873 \\
	419 -8.1583 \\
	420 -15.422 \\
	421 -15.0034 \\
	422 0.76799 \\
	423 -5.5321 \\
	424 -10.6008 \\
	425 15.3324 \\
	426 9.518 \\
	427 -10.0974 \\
	428 -7.0876 \\
	429 5.3562 \\
	430 -2.1337 \\
	431 -4.0789 \\
	432 8.9911 \\
	433 5.0596 \\
	434 -5.8793 \\
	435 -9.2513 \\
	436 -4.9065 \\
	437 -3.0074 \\
	438 7.7453 \\
	439 2.0086 \\
	440 0.50632 \\
	441 11.5377 \\
	442 5.4404 \\
	443 -4.7625 \\
	444 -18.7963 \\
	445 -4.6849 \\
	446 4.2747 \\
	447 3.1384 \\
	448 2.3358 \\
	449 17.7088 \\
	450 2.2102 \\
	451 1.0728 \\
	452 4.1748 \\
	453 5.708 \\
	454 -2.5581 \\
	455 -5.5093 \\
	456 5.9519 \\
	457 10.2727 \\
	458 -7.6914 \\
	459 -6.6849 \\
	460 -6.503 \\
	461 -11.3765 \\
	462 0.70514 \\
	463 -10.9299 \\
	464 -12.2271 \\
	465 -5.0596 \\
	466 1.3403 \\
	467 10.6215 \\
	468 8.1239 \\
	469 -6.9464 \\
	470 -11.9993 \\
	471 1.5599 \\
	472 2.0389 \\
	473 6.478 \\
	474 2.2849 \\
	475 11.3463 \\
	476 7.1741 \\
	477 -6.993 \\
	478 10.3046 \\
	479 -2.1558 \\
	480 1.2573 \\
	481 0 \\
	482 -8.012 \\
	483 3.7053 \\
	484 -3.2376 \\
	485 -4.4431 \\
	486 -13.6173 \\
	487 -5.1264 \\
	488 10.9197 \\
	489 4.9062 \\
	490 -6.5192 \\
	491 -6.2445 \\
	492 12.5978 \\
	493 7.8887 \\
	494 -8.8065 \\
	495 -5.0911 \\
	496 -3.5419 \\
	497 -10.1193 \\
	498 1.5091 \\
	499 4.4049 \\
	500 0.97182 \\
	501 8.2113 \\
	502 4.3306 \\
	503 -1.4775 \\
	504 1.061 \\
	505 1.1115 \\
	506 -7.4562 \\
	507 7.0767 \\
	508 -0.96745 \\
	509 -7.2279 \\
	510 13.6172 \\
	511 -6.6874 \\
	512 2.5637 \\
	513 10.1193 \\
	514 8.533 \\
	515 8.7883 \\
	516 -24.9226 \\
	517 -14.1827 \\
	518 -4.8338 \\
	519 -4.2736 \\
	520 10.7665 \\
	521 -1.3285 \\
	522 3.5231 \\
	523 5.5747 \\
	524 4.2645 \\
	525 -5.6184 \\
	526 0.14303 \\
	527 5.3113 \\
	528 2.6427 \\
	529 12.6491 \\
	530 -8.5426 \\
	531 -7.2678 \\
	532 1.9306 \\
	533 -7.1335 \\
	534 -3.3619 \\
	535 -1.0183 \\
	536 2.9294 \\
	537 0.37048 \\
	538 -4.5918 \\
	539 13.2894 \\
	540 8.3994 \\
	541 -7.1349 \\
	542 7.715 \\
	543 -1.0222 \\
	544 3.8577 \\
	545 7.5895 \\
	546 -4.8805 \\
	547 -3.4207 \\
	548 -8.3929 \\
	549 4.2804 \\
	550 -3.7082 \\
	551 -1.7568 \\
	552 5.854 \\
	553 -13.8505 \\
	554 -5.3718 \\
	555 7.3494 \\
	556 -4.0001 \\
	557 -7.5546 \\
	558 7.8936 \\
	559 -2.9608 \\
	560 -13.026 \\
	561 -12.6491 \\
	562 -0.03033 \\
	563 7.3938 \\
	564 -8.321 \\
	565 4.3867 \\
	566 8.6625 \\
	567 9.5785 \\
	568 -0.39358 \\
	569 2.1593 \\
	570 9.2651 \\
	571 -6.409 \\
	572 4.5163 \\
	573 -2.4604 \\
	574 -0.29507 \\
	575 1.2015 \\
	576 3.7759 \\
	577 -12.6491 \\
	578 -7.7756 \\
	579 1.7002 \\
	580 -6.4412 \\
	581 2.6218 \\
	582 2.1137 \\
	583 -7.9214 \\
	584 0.64562 \\
	585 1.2649 \\
	586 -4.7185 \\
	587 3.3052 \\
	588 3.2473 \\
	589 6.1906 \\
	590 15.4842 \\
	591 9.9392 \\
	592 -14.1022 \\
	593 -8.8544 \\
	594 -0.068388 \\
	595 -11.1042 \\
	596 -0.059024 \\
	597 -3.9777 \\
	598 -1.1052 \\
	599 -2.6873 \\
	600 -4.4817 \\
	601 6.8485 \\
	602 -1.4862 \\
	603 2.7956 \\
	604 6.1512 \\
	605 13.6707 \\
	606 4.6388 \\
	607 1.4574 \\
	608 10.2294 \\
	609 -7.5895 \\
	610 -1.8547 \\
	611 6.6366 \\
	612 6.0449 \\
	613 2.2208 \\
	614 -0.74754 \\
	615 5.2453 \\
	616 10.2811 \\
	617 1.2649 \\
	618 -2.765 \\
	619 11.7411 \\
	620 -6.6381 \\
	621 -11.7742 \\
	622 1.5989 \\
	623 2.6637 \\
	624 2.3986 \\
	625 -8.8544 \\
	626 -14.5922 \\
	627 -11.1094 \\
	628 -0.59263 \\
	629 1.6649 \\
	630 5.7487 \\
	631 -1.358 \\
	632 -0.33748 \\
	633 3.2708 \\
	634 -7.5762 \\
	635 11.2138 \\
	636 -3.8082 \\
	637 -15.6766 \\
	638 2.9859 \\
	639 7.84 \\
	640 7.5818 \\
	641 -3.7947 \\
	642 -1.9212 \\
	643 -7.3849 \\
	644 -6.7265 \\
	645 4.0101 \\
	646 -9.2048 \\
	647 13.3206 \\
	648 9.5918 \\
	649 -5.52 \\
	650 -7.8088 \\
	651 -7.6113 \\
	652 3.6271 \\
	653 1.9625 \\
	654 14.9712 \\
	655 4.0234 \\
	656 -2.0862 \\
	657 3.7947 \\
	658 -11.686 \\
	659 -1.4253 \\
	660 -0.023949 \\
	661 -4.6656 \\
	662 -1.8673 \\
	663 -8.9228 \\
	664 -3.6258 \\
	665 2.1593 \\
	666 3 \\
	667 -3.7122 \\
	668 -17.785 \\
	669 -2.837 \\
	670 11.9059 \\
	671 -4.213 \\
	672 0.10385 \\
	673 1.2649 \\
	674 -13.8616 \\
	675 2.5987 \\
	676 1.1334 \\
	677 -11.3825 \\
	678 -1.2422 \\
	679 -7.3887 \\
	680 5.9756 \\
	681 10.5797 \\
	682 -6.9845 \\
	683 15.8955 \\
	684 14.6485 \\
	685 -5.4503 \\
	686 6.4521 \\
	687 8.1332 \\
	688 -0.79412 \\
	689 6.3246 \\
	690 5.2363 \\
	691 -1.9918 \\
	692 10.7256 \\
	693 4.4485 \\
	694 -8.2093 \\
	695 -5.2124 \\
	696 4.7395 \\
	697 0.37048 \\
	698 -9.7748 \\
	699 8.691 \\
	700 -4.7311 \\
	701 -11.384 \\
	702 0.63712 \\
	703 5.3194 \\
	704 15.5851 \\
	705 3.7947 \\
	706 7.4117 \\
	707 4.5059 \\
	708 -11.8988 \\
	709 -10.7486 \\
	710 -0.98552 \\
	711 6.6882 \\
	712 8.4285 \\
	713 6.7849 \\
	714 13.2552 \\
	715 7.5626 \\
	716 -0.95776 \\
	717 -4.8587 \\
	718 -2.2931 \\
	719 1.1949 \\
	720 -3.6 \\
	721 -11.3842 \\
	722 -11.2223 \\
	723 11.2864 \\
	724 9.2001 \\
	725 -6.4602 \\
	726 -10.4183 \\
	727 -5.1601 \\
	728 12.7764 \\
	729 5.52 \\
	730 -6.0236 \\
	731 3.5757 \\
	732 -2.3247 \\
	733 -13.5531 \\
	734 -1.5753 \\
	735 -9.454 \\
	736 -6.091 \\
	737 1.2649 \\
	738 -7.1174 \\
	739 7.5975 \\
	740 -5.1279 \\
	741 -6.309 \\
	742 -0.22207 \\
	743 8.0406 \\
	744 9.5356 \\
	745 -9.3148 \\
	746 10.758 \\
	747 -3.3105 \\
	748 -11.5396 \\
	749 -3.2547 \\
	750 -0.99692 \\
	751 4.9031 \\
	752 -6.2049 \\
	753 8.8544 \\
	754 -5.6253 \\
	755 -7.1629 \\
	756 8.9349 \\
	757 10.8687 \\
	758 14.9048 \\
	759 3.6942 \\
	760 7.8686 \\
	761 -10.5797 \\
	762 -6.4959 \\
	763 1.2436 \\
	764 -9.8483 \\
	765 9.0174 \\
	766 -3.4732 \\
	767 -4.8473 \\
	768 10.968 \\
	769 -8.8544 \\
	770 -0.14015 \\
	771 -13.8923 \\
	772 -13.1166 \\
	773 -2.4917 \\
	774 -1.7436 \\
	775 -6.3793 \\
	776 12.357 \\
	777 -2.3128 \\
	778 -3.8937 \\
	779 15.5253 \\
	780 -1.1194 \\
	781 10.3613 \\
	782 1.1519 \\
	783 2.115 \\
	784 10.0967 \\
	785 -7.5895 \\
	786 2.1418 \\
	787 -0.026464 \\
	788 5.8174 \\
	789 13.1808 \\
	790 -1.805 \\
	791 -4.9603 \\
	792 -3.4025 \\
	793 -2.2229 \\
	794 10.4849 \\
	795 5.8864 \\
	796 -1.9939 \\
	797 1.2959 \\
	798 -7.7383 \\
	799 -2.5704 \\
	800 -1.4043 \\
	801 6.3246 \\
	802 10.4774 \\
	803 -0.74242 \\
	804 -3.3786 \\
	805 -2.6578 \\
	806 12.1919 \\
	807 -4.2234 \\
	808 -7.173 \\
	809 4.8426 \\
	810 -4.9791 \\
	811 3.5596 \\
	812 -4.4256 \\
	813 -6.5666 \\
	814 0.95914 \\
	815 -9.0337 \\
	816 -11.6319 \\
	817 -2.5298 \\
	818 0.23923 \\
	819 0.78439 \\
	820 7.3085 \\
	821 -0.44175 \\
	822 -6.9528 \\
	823 5.8777 \\
	824 5.877 \\
	825 -12.956 \\
	826 -2.1154 \\
	827 4.0845 \\
	828 -1.3069 \\
	829 2.4989 \\
	830 1.8412 \\
	831 14.1148 \\
	832 -2.6234 \\
	833 -8.8544 \\
	834 2.8925 \\
	835 4.563 \\
	836 4.9202 \\
	837 -8.1417 \\
	838 -1.3132 \\
	839 14.7064 \\
	840 -5.1761 \\
	841 3.6413 \\
	842 2.8199 \\
	843 -1.0936 \\
	844 -6.344 \\
	845 -11.9211 \\
	846 10.3229 \\
	847 9.3541 \\
	848 2.7135 \\
	849 -1.2649 \\
	850 -0.37991 \\
	851 -5.1265 \\
	852 -5.9555 \\
	853 -4.7989 \\
	854 2.0801 \\
	855 11.9565 \\
	856 -8.4188 \\
	857 -9.8387 \\
	858 -14.4385 \\
	859 -12.1192 \\
	860 3.4426 \\
	861 -2.5759 \\
	862 -9.6704 \\
	863 -5.2589 \\
	864 1.6028 \\
	865 -1.2649 \\
	866 7.0749 \\
	867 -7.9944 \\
	868 -7.5789 \\
	869 10.8885 \\
	870 10.3794 \\
	871 4.7262 \\
	872 -6.0232 \\
	873 9.0078 \\
	874 -4.8928 \\
	875 -7.156 \\
	876 1.7615 \\
	877 -5.8775 \\
	878 9.232 \\
	879 0.6083 \\
	880 -7.314 \\
	881 -6.3246 \\
	882 7.6458 \\
	883 3.0643 \\
	884 -4.7466 \\
	885 14.7011 \\
	886 5.4974 \\
	887 0.6305 \\
	888 4.5785 \\
	889 9.8387 \\
	890 7.3017 \\
	891 0.56427 \\
	892 6.4772 \\
	893 2.6658 \\
	894 -4.1939 \\
	895 -1.3056 \\
	896 -4.2969 \\
	897 -6.3246 \\
	898 6.1097 \\
	899 0.38809 \\
	900 7.1355 \\
	901 -0.63246 \\
	902 1.5723 \\
	903 -3.7652 \\
	904 0.54467 \\
	905 9.5318 \\
	906 -5.3042 \\
	907 6.6459 \\
	908 3.1505 \\
	909 -4.2764 \\
	910 -2.5351 \\
	911 -9.223 \\
	912 -1.2044 \\
	913 3.7947 \\
	914 -10.3485 \\
	915 -8.418 \\
	916 2.8037 \\
	917 -0.63246 \\
	918 10.2814 \\
	919 4.0096 \\
	920 -8.0579 \\
	921 0.37048 \\
	922 1.5111 \\
	923 0.94153 \\
	924 0.023598 \\
	925 -4.5674 \\
	926 9.7138 \\
	927 24.9965 \\
	928 -0.35142 \\
	929 -6.3246 \\
	930 4.6134 \\
	931 1.8683 \\
	932 -1.6847 \\
	933 -0.63246 \\
	934 16.2095 \\
	935 2.4346 \\
	936 -8.2689 \\
	937 0.58751 \\
	938 -0.088024 \\
	939 -17.4599 \\
	940 -22.5587 \\
	941 8.378 \\
	942 2.9353 \\
	943 -3.309 \\
	944 3.0649 \\
	945 -8.8544 \\
	946 -3.4809 \\
	947 9.3053 \\
	948 5.9245 \\
	949 -0.63246 \\
	950 4.5777 \\
	951 -0.76305 \\
	952 -4.6966 \\
	953 2.1593 \\
	954 7.8555 \\
	955 -8.4501 \\
	956 -9.9733 \\
	957 -2.0641 \\
	958 -13.2651 \\
	959 10.9178 \\
	960 3.7907 \\
	961 -3.7947 \\
	962 5.7386 \\
	963 -4.2408 \\
	964 -1.4182 \\
	965 12.7797 \\
	966 10.7626 \\
	967 7.9206 \\
	968 0.10423 \\
	969 0.43405 \\
	970 13.4914 \\
	971 5.5103 \\
	972 0.64202 \\
	973 9.0644 \\
	974 0.72287 \\
	975 -2.2627 \\
	976 10.2734 \\
	977 10.1193 \\
	978 -2.8797 \\
	979 -7.335 \\
	980 -8.3958 \\
	981 2.9848 \\
	982 12.6041 \\
	983 -1.2807 \\
	984 -7.5718 \\
	985 -6.3246 \\
	986 4.9245 \\
	987 3.312 \\
	988 -1.9625 \\
	989 3.1757 \\
	990 0.62888 \\
	991 6.4054 \\
	992 -13.7259 \\
	993 -6.3246 \\
	994 1.9474 \\
	995 -22.1893 \\
	996 -0.81905 \\
	997 -0.040737 \\
	998 -14.7043 \\
	999 -17.3016 \\
	1000 2.4877 \\
	1001 14.7449 \\
	1002 -6.2403 \\
	1003 17.1869 \\
	1004 8.7094 \\
	1005 -13.9071 \\
	1006 6.4146 \\
	1007 1.535 \\
	1008 -2.4005 \\
	1009 2.5298 \\
	1010 -1.8913 \\
	1011 -0.96415 \\
	1012 -1.0863 \\
	1013 -8.1343 \\
	1014 0.86334 \\
	1015 4.5541 \\
	1016 -8.963 \\
	1017 -6.3246 \\
	1018 0.088326 \\
	1019 3.6604 \\
	1020 3.4623 \\
	1021 -0.86289 \\
	1022 -2.1131 \\
	1023 -4.6298 \\
	1024 -9.6939 \\
	1025 -1.2649 \\
	1026 17.1665 \\
	1027 -0.40149 \\
	1028 7.6103 \\
	1029 6.4569 \\
	1030 3.4429 \\
	1031 -1.7324 \\
	1032 -12.4991 \\
	1033 10.9502 \\
	1034 4.85 \\
	1035 -0.52693 \\
	1036 19.7519 \\
	1037 2.3593 \\
	1038 -1.1348 \\
	1039 0.24142 \\
	1040 -6.7429 \\
	1041 10.1193 \\
	1042 4.6098 \\
	1043 -7.8861 \\
	1044 -3.2302 \\
	1045 -8.947 \\
	1046 -3.99 \\
	1047 2.2812 \\
	1048 -11.1764 \\
	1049 -5.0596 \\
	1050 1.786 \\
	1051 -4.3284 \\
	1052 2.2029 \\
	1053 8.7202 \\
	1054 1.5225 \\
	1055 -5.6826 \\
	1056 -2.312 \\
	1057 -13.914 \\
	1058 -3.206 \\
	1059 16.7677 \\
	1060 -3.0352 \\
	1061 -8.7697 \\
	1062 3.6626 \\
	1063 4.1505 \\
	1064 -4.2059 \\
	1065 -3.3607 \\
	1066 -4.7079 \\
	1067 5.9715 \\
	1068 4.3203 \\
	1069 -18.676 \\
	1070 7.8782 \\
	1071 9.2034 \\
	1072 -2.0132 \\
	1073 2.5298 \\
	1074 -12.6223 \\
	1075 -7.4322 \\
	1076 -5.0693 \\
	1077 13.7896 \\
	1078 12.7262 \\
	1079 -7.8429 \\
	1080 -3.4053 \\
	1081 -5.0596 \\
	1082 0.14743 \\
	1083 -7.2237 \\
	1084 -1.9525 \\
	1085 0.0070792 \\
	1086 -1.7733 \\
	1087 14.5603 \\
	1088 -8.6012 \\
	1089 -3.7947 \\
	1090 -10.1087 \\
	1091 7.4597 \\
	1092 8.1031 \\
	1093 -6.9238 \\
	1094 -6.8052 \\
	1095 2.0497 \\
	1096 2.8824 \\
	1097 -3.5777 \\
	1098 -2.6037 \\
	1099 -3.8106 \\
	1100 -1.5682 \\
	1101 1.341 \\
	1102 -11.3042 \\
	1103 2.6436 \\
	1104 6.836 \\
	1105 -10.1193 \\
	1106 4.9751 \\
	1107 -2.7663 \\
	1108 -9.609 \\
	1109 -5.9258 \\
	1110 -1.3851 \\
	1111 0.22568 \\
	1112 -6.8434 \\
	1113 6.9384 \\
	1114 -13.4985 \\
	1115 -7.114 \\
	1116 4.3495 \\
	1117 -2.1192 \\
	1118 14.3429 \\
	1119 -1.0405 \\
	1120 -1.2799 \\
	1121 -6.3246 \\
	1122 1.8624 \\
	1123 1.3277 \\
	1124 -14.057 \\
	1125 12.5074 \\
	1126 4.2966 \\
	1127 7.1228 \\
	1128 17.4715 \\
	1129 3.5777 \\
	1130 0.26576 \\
	1131 4.1797 \\
	1132 14.6234 \\
	1133 11.091 \\
	1134 -5.9815 \\
	1135 -5.9968 \\
	1136 7.4998 \\
	1137 -7.5895 \\
	1138 -4.061 \\
	1139 -4.9731 \\
	1140 -11.2481 \\
	1141 7.9316 \\
	1142 -5.0874 \\
	1143 1.769 \\
	1144 0.9022 \\
	1145 -14.5279 \\
	1146 -6.3212 \\
	1147 0.63741 \\
	1148 10.2743 \\
	1149 7.3959 \\
	1150 11.0558 \\
	1151 8.4055 \\
	1152 2.0212 \\
	1153 -11.3842 \\
	1154 -14.5679 \\
	1155 7.2567 \\
	1156 0.43029 \\
	1157 -6.7445 \\
	1158 -6.7392 \\
	1159 -9.987 \\
	1160 2.9998 \\
	1161 2.8367 \\
	1162 -2.5997 \\
	1163 11.394 \\
	1164 8.6556 \\
	1165 4.1271 \\
	1166 8.1042 \\
	1167 -3.6049 \\
	1168 0.4935 \\
	1169 -5.0596 \\
	1170 -5.1095 \\
	1171 4.4177 \\
	1172 -3.7664 \\
	1173 2.6345 \\
	1174 -8.0309 \\
	1175 -2.8271 \\
	1176 -0.9771 \\
	1177 -8.5475 \\
	1178 15.5204 \\
	1179 1.7852 \\
	1180 -2.3807 \\
	1181 3.866 \\
	1182 -0.34332 \\
	1183 10.8681 \\
	1184 -6.4645 \\
	1185 -8.8544 \\
	1186 -0.49603 \\
	1187 -13.6738 \\
	1188 0.86059 \\
	1189 14.8579 \\
	1190 3.0873 \\
	1191 9.153 \\
	1192 10.7721 \\
	1193 -7.8964 \\
	1194 -5.9597 \\
	1195 7.6563 \\
	1196 6.6693 \\
	1197 4.2932 \\
	1198 4.7548 \\
	1199 -4.5356 \\
	1200 -1.125 \\
	1201 12.6491 \\
	1202 10.3529 \\
	1203 1.8196 \\
	1204 5.0957 \\
	1205 1.9012 \\
	1206 -13.5079 \\
	1207 -2.4464 \\
	1208 -1.3704 \\
	1209 -19.2806 \\
	1210 -5.164 \\
	1211 14.7618 \\
	1212 6.7699 \\
	1213 -9.7565 \\
	1214 -9.6594 \\
	1215 -1.6797 \\
	1216 3.6951 \\
	1217 -1.2649 \\
	1218 -3.7091 \\
	1219 -3.9091 \\
	1220 4.7602 \\
	1221 1.0526 \\
	1222 8.1249 \\
	1223 10.8309 \\
	1224 -2.845 \\
	1225 -13.2629 \\
	1226 -0.90523 \\
	1227 7.0563 \\
	1228 -1.196 \\
	1229 -3.5828 \\
	1230 -3.3858 \\
	1231 4.1181 \\
	1232 3.7331 \\
	1233 -15.1789 \\
	1234 -2.5481 \\
	1235 5.3461 \\
	1236 -5.2726 \\
	1237 -2.3926 \\
	1238 -5.3287 \\
	1239 13.4756 \\
	1240 -1.5739 \\
	1241 -14.2209 \\
	1242 -1.8711 \\
	1243 -17.7758 \\
	1244 1.9362 \\
	1245 10.5919 \\
	1246 10.9391 \\
	1247 5.5568 \\
	1248 -5.3616 \\
	1249 8.8544 \\
	1250 4.6805 \\
	1251 5.6358 \\
	1252 -4.4992 \\
	1253 -12.0028 \\
	1254 -1.6107 \\
	1255 3.8661 \\
	1256 3.1643 \\
	1257 8.2033 \\
	1258 7.1211 \\
	1259 -11.5319 \\
	1260 3.1696 \\
	1261 -2.6147 \\
	1262 -18.8192 \\
	1263 -3.0585 \\
	1264 0.89182 \\
	1265 -5.0596 \\
	1266 0.27152 \\
	1267 3.2262 \\
	1268 -9.6428 \\
	1269 5.7533 \\
	1270 6.3136 \\
	1271 5.3288 \\
	1272 9.7153 \\
	1273 -3.4878 \\
	1274 9.1755 \\
	1275 -3.2266 \\
	1276 -1.4704 \\
	1277 8.2546 \\
	1278 1.671 \\
	1279 5.4189 \\
	1280 -5.6282 \\
	1281 -10.1193 \\
	1282 8.5556 \\
	1283 7.0043 \\
	1284 2.5944 \\
	1285 0.95184 \\
	1286 5.1538 \\
	1287 17.6884 \\
	1288 0.99619 \\
	1289 -1.4819 \\
	1290 -0.48684 \\
	1291 4.6594 \\
	1292 -13.0491 \\
	1293 -0.53617 \\
	1294 21.3059 \\
	1295 -12.5388 \\
	1296 -0.77778 \\
	1297 -3.7947 \\
	1298 1.4662 \\
	1299 17.609 \\
	1300 -7.8825 \\
	1301 -1.9226 \\
	1302 6.457 \\
	1303 3.9271 \\
	1304 2.0938 \\
	1305 -6.0176 \\
	1306 -9.3635 \\
	1307 2.634 \\
	1308 1.7371 \\
	1309 0.1484 \\
	1310 -0.33023 \\
	1311 0.76685 \\
	1312 15.1208 \\
	1313 0 \\
	1314 3.9835 \\
	1315 2.5533 \\
	1316 -2.7466 \\
	1317 7.1616 \\
	1318 -7.119 \\
	1319 -4.4839 \\
	1320 -5.4245 \\
	1321 -8.6374 \\
	1322 0.77531 \\
	1323 9.7014 \\
	1324 2.8198 \\
	1325 1.8011 \\
	1326 3.7544 \\
	1327 -7.7039 \\
	1328 2.0167 \\
	1329 1.2649 \\
	1330 -1.0296 \\
	1331 -5.8802 \\
	1332 -1.6131 \\
	1333 6.4583 \\
	1334 -4.193 \\
	1335 -1.7729 \\
	1336 -14.5136 \\
	1337 -16.7508 \\
	1338 -5.9964 \\
	1339 -2.8638 \\
	1340 5.925 \\
	1341 1.1165 \\
	1342 -12.8138 \\
	1343 -0.94246 \\
	1344 2.5841 \\
	1345 11.3842 \\
	1346 9.3308 \\
	1347 -9.6345 \\
	1348 -2.8425 \\
	1349 -4.4125 \\
	1350 -14.9234 \\
	1351 2.7657 \\
	1352 10.34 \\
	1353 -0.37048 \\
	1354 -7.7894 \\
	1355 -6.9757 \\
	1356 -0.99719 \\
	1357 -5.5213 \\
	1358 1.3464 \\
	1359 11.4763 \\
	1360 -9.6855 \\
	1361 -13.914 \\
	1362 1.9375 \\
	1363 -5.6844 \\
	1364 2.8211 \\
	1365 0.40936 \\
	1366 -3.3246 \\
	1367 3.794 \\
	1368 1.5352 \\
	1369 8.177 \\
	1370 -7.7357 \\
	1371 4.2542 \\
	1372 -0.56148 \\
	1373 -15.014 \\
	1374 0.60807 \\
	1375 -5.856 \\
	1376 11.1203 \\
	1377 -3.7947 \\
	1378 -12.6874 \\
	1379 8.2599 \\
	1380 11.3897 \\
	1381 4.9365 \\
	1382 -11.3916 \\
	1383 6.4333 \\
	1384 2.4838 \\
	1385 -2.1593 \\
	1386 2.678 \\
	1387 -1.9403 \\
	1388 0.85319 \\
	1389 -8.8267 \\
	1390 -0.51622 \\
	1391 0.35474 \\
	1392 6.466 \\
	1393 -1.2649 \\
	1394 -11.979 \\
	1395 6.0112 \\
	1396 -2.9769 \\
	1397 -3.4631 \\
	1398 8.0267 \\
	1399 3.2338 \\
	1400 4.5158 \\
	1401 17.1212 \\
	1402 11.0663 \\
	1403 10.7694 \\
	1404 7.4929 \\
	1405 -13.6449 \\
	1406 4.9958 \\
	1407 3.0964 \\
	1408 -11.5966 \\
	1409 -11.3842 \\
	1410 -4.402 \\
	1411 -1.3986 \\
	1412 4.7572 \\
	1413 12.8076 \\
	1414 -5.1597 \\
	1415 -8.2084 \\
	1416 6.5914 \\
	1417 8.4203 \\
	1418 9.2262 \\
	1419 0.59912 \\
	1420 -0.97021 \\
	1421 -2.1172 \\
	1422 1.0099 \\
	1423 -2.2064 \\
	1424 -2.5169 \\
	1425 10.1193 \\
	1426 -0.75085 \\
	1427 -7.2254 \\
	1428 -2.2262 \\
	1429 2.7591 \\
	1430 8.2806 \\
	1431 9.8574 \\
	1432 -4.1752 \\
	1433 -5.0596 \\
	1434 4.6217 \\
	1435 -1.5342 \\
	1436 -3.996 \\
	1437 -9.9833 \\
	1438 4.7711 \\
	1439 3.187 \\
	1440 2.7465 \\
	1441 21.5035 \\
	1442 3.4325 \\
	1443 -8.2653 \\
	1444 -1.8314 \\
	1445 11.8395 \\
	1446 5.554 \\
	1447 8.6414 \\
	1448 9.2923 \\
	1449 -5.8905 \\
	1450 12.3925 \\
	1451 3.6094 \\
	1452 -3.5088 \\
	1453 -8.092 \\
	1454 -16.1392 \\
	1455 4.7043 \\
	1456 0.67567 \\
	1457 -7.5895 \\
	1458 -1.7784 \\
	1459 -11.5526 \\
	1460 -15.3541 \\
	1461 0.42182 \\
	1462 -4.4587 \\
	1463 -4.1828 \\
	1464 -3.2479 \\
	1465 -5.0596 \\
	1466 3.7844 \\
	1467 -9.6499 \\
	1468 10.9144 \\
	1469 17.6627 \\
	1470 -10.2647 \\
	1471 -6.7327 \\
	1472 -7.2702 \\
	1473 0 \\
	1474 2.2317 \\
	1475 1.4695 \\
	1476 -2.4631 \\
	1477 -3.722 \\
	1478 3.6481 \\
	1479 -4.8988 \\
	1480 6.4267 \\
	1481 0.43405 \\
	1482 -19.0412 \\
	1483 -1.25 \\
	1484 -8.1234 \\
	1485 -10.4065 \\
	1486 12.5681 \\
	1487 9.4712 \\
	1488 -13.2079 \\
	1489 -3.7947 \\
	1490 10.9745 \\
	1491 -2.4221 \\
	1492 6.4541 \\
	1493 6.4365 \\
	1494 0.025025 \\
	1495 7.5922 \\
	1496 0.72482 \\
	1497 -3.0538 \\
	1498 -6.8377 \\
	1499 -6.0561 \\
	1500 2.1452 \\
	1501 -5.7432 \\
	1502 -5.8868 \\
	1503 15.8495 \\
	1504 14.5842 \\
	1505 0 \\
	1506 -11.1107 \\
	1507 -6.0775 \\
	1508 10.7968 \\
	1509 -10.4989 \\
	1510 -8.4315 \\
	1511 11.0526 \\
	1512 6.0604 \\
	1513 14.7449 \\
	1514 4.2657 \\
	1515 -0.19628 \\
	1516 12.4821 \\
	1517 -2.0256 \\
	1518 -3.6237 \\
	1519 6.1752 \\
	1520 -4.3712 \\
	1521 -3.7947 \\
	1522 5.4353 \\
	1523 6.596 \\
	1524 1.8559 \\
	1525 -9.9243 \\
	1526 -6.9983 \\
	1527 1.8669 \\
	1528 4.0214 \\
	1529 0.52394 \\
	1530 -8.2519 \\
	1531 -7.2425 \\
	1532 -0.81318 \\
	1533 0.46653 \\
	1534 0.67561 \\
	1535 -1.5721 \\
	1536 -6.2149 \\
	1537 10.1193 \\
	1538 4.1677 \\
	1539 -8.0353 \\
	1540 -5.3542 \\
	1541 -3.5388 \\
	1542 5.6189 \\
	1543 -4.2216 \\
	1544 -11.317 \\
	1545 15.1154 \\
	1546 2.7172 \\
	1547 -5.8295 \\
	1548 -0.061292 \\
	1549 3.0811 \\
	1550 4.1001 \\
	1551 2.7709 \\
	1552 7.4631 \\
	1553 -2.5298 \\
	1554 2.858 \\
	1555 1.1312 \\
	1556 -8.1609 \\
	1557 6.3548 \\
	1558 10.6477 \\
	1559 10.828 \\
	1560 11.9972 \\
	1561 -7.7429 \\
	1562 -8.9306 \\
	1563 -5.9175 \\
	1564 -10.4138 \\
	1565 13.5606 \\
	1566 7.4923 \\
	1567 -1.2223 \\
	1568 11.1706 \\
	1569 10.1193 \\
	1570 -1.6512 \\
	1571 -11.4544 \\
	1572 0.33616 \\
	1573 -7.4113 \\
	1574 -12.3815 \\
	1575 -8.385 \\
	1576 0.39457 \\
	1577 2.5934 \\
	1578 -11.438 \\
	1579 -0.00018968 \\
	1580 -7.4467 \\
	1581 -5.7008 \\
	1582 -0.70964 \\
	1583 -5.8998 \\
	1584 11.7326 \\
	1585 8.8818e-16 \\
	1586 -13.963 \\
	1587 7.1913 \\
	1588 14.6569 \\
	1589 -2.9942 \\
	1590 2.3819 \\
	1591 -2.2332 \\
	1592 -11.1844 \\
	1593 -2.3764 \\
	1594 -0.286 \\
	1595 7.7355 \\
	1596 0.037273 \\
	1597 1.7082 \\
	1598 -0.7432 \\
	1599 -6.816 \\
	1600 6.2691 \\
	1601 -2.5298 \\
	1602 7.7068 \\
	1603 4.5806 \\
	1604 -11.96 \\
	1605 -7.3393 \\
	1606 7.2146 \\
	1607 3.5478 \\
	1608 4.0225 \\
	1609 14.3744 \\
	1610 -9.1674 \\
	1611 -7.2157 \\
	1612 8.7103 \\
	1613 -1.5588 \\
	1614 2.0734 \\
	1615 11.5666 \\
	1616 1.8725 \\
	1617 -2.5298 \\
	1618 0.52645 \\
	1619 -9.0422 \\
	1620 -3.584 \\
	1621 5.8448 \\
	1622 -0.47433 \\
	1623 -9.5336 \\
	1624 3.2965 \\
	1625 10.2727 \\
	1626 1.0774 \\
	1627 3.6205 \\
	1628 -0.749 \\
	1629 3.4067 \\
	1630 3.0796 \\
	1631 -1.0411 \\
	1632 -6.4103 \\
	1633 2.5298 \\
	1634 18.9337 \\
	1635 -1.3604 \\
	1636 -3.7287 \\
	1637 -2.997 \\
	1638 -4.2668 \\
	1639 -1.2425 \\
	1640 -6.9523 \\
	1641 -1.7253 \\
	1642 -1.8923 \\
	1643 4.8082 \\
	1644 2.0841 \\
	1645 -7.6025 \\
	1646 -7.8238 \\
	1647 -7.4537 \\
	1648 -5.5722 \\
	1649 2.2204e-16 \\
	1650 1.0568 \\
	1651 2.6784 \\
	1652 -6.2127 \\
	1653 -23.3365 \\
	1654 2.0455 \\
	1655 1.1208 \\
	1656 -4.4159 \\
	1657 4.9062 \\
	1658 -3.0625 \\
	1659 17.1096 \\
	1660 0.26108 \\
	1661 -1.8349 \\
	1662 13.3306 \\
	1663 -2.0238 \\
	1664 -1.0197 \\
	1665 2.5298 \\
	1666 3.3574 \\
	1667 -2.2576 \\
	1668 4.1702 \\
	1669 2.8429 \\
	1670 -2.0857 \\
	1671 5.4634 \\
	1672 1.2588 \\
	1673 3.0538 \\
	1674 3.4306 \\
	1675 -8.0615 \\
	1676 -7.3006 \\
	1677 9.2671 \\
	1678 14.5858 \\
	1679 -5.003 \\
	1680 -5.6345 \\
	1681 -6.3246 \\
	1682 -0.72481 \\
	1683 8.2207 \\
	1684 4.7769 \\
	1685 5.7173 \\
	1686 -3.729 \\
	1687 5.9796 \\
	1688 4.0858 \\
	1689 -2.2229 \\
	1690 5.3629 \\
	1691 1.1186 \\
	1692 -2.3546 \\
	1693 -9.291 \\
	1694 -0.56951 \\
	1695 3.1766 \\
	1696 -10.9189 \\
	1697 -5.0596 \\
	1698 10.155 \\
	1699 -8.5149 \\
	1700 -7.8025 \\
	1701 -3.3668 \\
	1702 -18.2103 \\
	1703 10.1379 \\
	1704 6.1013 \\
	1705 -0.52394 \\
	1706 -4.1171 \\
	1707 -4.731 \\
	1708 15.8969 \\
	1709 -10.9661 \\
	1710 1.6532 \\
	1711 10.0148 \\
	1712 -1.842 \\
	1713 6.3246 \\
	1714 3.1762 \\
	1715 10.7551 \\
	1716 -0.13415 \\
	1717 -2.6636 \\
	1718 -5.1073 \\
	1719 -0.29448 \\
	1720 3.9203 \\
	1721 -12.956 \\
	1722 -1.2697 \\
	1723 -1.589 \\
	1724 -12.3119 \\
	1725 -6.7188 \\
	1726 4.2116 \\
	1727 5.9426 \\
	1728 -2.0305 \\
	1729 15.1789 \\
	1730 -2.2285 \\
	1731 9.0516 \\
	1732 -7.9601 \\
	1733 -1.3968 \\
	1734 6.0776 \\
	1735 -10.044 \\
	1736 6.9963 \\
	1737 -4.4721 \\
	1738 -8.4922 \\
	1739 -5.9798 \\
	1740 7.9179 \\
	1741 16.3467 \\
	1742 -4.7273 \\
	1743 -13.0394 \\
	1744 -4.4261 \\
	1745 5.0596 \\
	1746 -6.8756 \\
	1747 -6.283 \\
	1748 4.4538 \\
	1749 9.465 \\
	1750 3.1658 \\
	1751 0.83285 \\
	1752 1.1281 \\
	1753 -4.6892 \\
	1754 9.6547 \\
	1755 -0.38156 \\
	1756 -5.4505 \\
	1757 0.28441 \\
	1758 -8.9533 \\
	1759 -3.1169 \\
	1760 -2.9805 \\
	1761 -2.5298 \\
	1762 3.6827 \\
	1763 0.023986 \\
	1764 -9.4373 \\
	1765 -1.9639 \\
	1766 8.902 \\
	1767 4.1121 \\
	1768 4.6587 \\
	1769 4.4721 \\
	1770 -5.6423 \\
	1771 -11.2365 \\
	1772 -15.0034 \\
	1773 -5.0896 \\
	1774 17.2764 \\
	1775 0.88277 \\
	1776 -7.5839 \\
	1777 15.1789 \\
	1778 10.5299 \\
	1779 -4.7086 \\
	1780 1.739 \\
	1781 1.4851 \\
	1782 -5.3368 \\
	1783 13.3024 \\
	1784 4.6789 \\
	1785 -2.9003 \\
	1786 0.23932 \\
	1787 -5.7844 \\
	1788 -5.7492 \\
	1789 -14.0713 \\
	1790 13.0853 \\
	1791 2.0106 \\
	1792 -3.3396 \\
	1793 -16.4438 \\
	1794 3.3418 \\
	1795 6.492 \\
	1796 -0.27964 \\
	1797 -5.8342 \\
	1798 4.4214 \\
	1799 -5.2425 \\
	1800 -3.3649 \\
	1801 -0.089894 \\
	1802 0.11284 \\
	1803 5.9711 \\
	1804 -10.6054 \\
	1805 8.9881 \\
	1806 -4.9066 \\
	1807 -21.0302 \\
	1808 3.6998 \\
	1809 -2.5298 \\
	1810 -3.6681 \\
	1811 4.7794 \\
	1812 -5.9905 \\
	1813 -10.8985 \\
	1814 3.7194 \\
	1815 -0.040784 \\
	1816 9.8769 \\
	1817 -0.74097 \\
	1818 -11.5569 \\
	1819 10.186 \\
	1820 4.5398 \\
	1821 9.6914 \\
	1822 -9.5007 \\
	1823 -9.6697 \\
	1824 -4.2334 \\
	1825 -13.914 \\
	1826 3.0311 \\
	1827 -1.4192 \\
	1828 16.0736 \\
	1829 5.6172 \\
	1830 1.4657 \\
	1831 17.5624 \\
	1832 5.2252 \\
	1833 17.7986 \\
	1834 4.1826 \\
	1835 -5.9737 \\
	1836 5.3807 \\
	1837 0.60723 \\
	1838 0.15602 \\
	1839 6.5285 \\
	1840 5.1983 \\
	1841 -5.0596 \\
	1842 -1.9068 \\
	1843 2.7969 \\
	1844 -4.5354 \\
	1845 3.526 \\
	1846 7.1063 \\
	1847 -12.7132 \\
	1848 -2.5966 \\
	1849 -4.3187 \\
	1850 -12.907 \\
	1851 2.4657 \\
	1852 0.47647 \\
	1853 3.4817 \\
	1854 6.7897 \\
	1855 9.4265 \\
	1856 -8.7456 \\
	1857 10.1193 \\
	1858 -6.8618 \\
	1859 -3.7109 \\
	1860 -0.52582 \\
	1861 -8.7078 \\
	1862 -6.9261 \\
	1863 3.8483 \\
	1864 3.1686 \\
	1865 2.4663 \\
	1866 8.835 \\
	1867 4.0627 \\
	1868 4.6544 \\
	1869 1.5415 \\
	1870 12.172 \\
	1871 4.0633 \\
	1872 -14.8399 \\
	1873 -7.5895 \\
	1874 -10.4516 \\
	1875 6.0949 \\
	1876 2.5991 \\
	1877 -14.7478 \\
	1878 -13.5214 \\
	1879 -5.1943 \\
	1880 3.1237 \\
	1881 -3.4242 \\
	1882 6.2416 \\
	1883 -4.4328 \\
	1884 -8.0116 \\
	1885 0.70209 \\
	1886 -18.6595 \\
	1887 -12.4054 \\
	1888 10.5606 \\
	1889 -2.5298 \\
	1890 -6.792 \\
	1891 13.6244 \\
	1892 -4.8929 \\
	1893 1.2083 \\
	1894 4.9782 \\
	1895 -1.2255 \\
	1896 12.2194 \\
	1897 -10.0557 \\
	1898 3.4295 \\
	1899 7.6209 \\
	1900 1.9532 \\
	1901 10.8905 \\
	1902 2.1229 \\
	1903 2.2884 \\
	1904 -14.663 \\
	1905 -2.5298 \\
	1906 5.321 \\
	1907 -3.3592 \\
	1908 0.63756 \\
	1909 -10.6404 \\
	1910 1.147 \\
	1911 12.2567 \\
	1912 3.5555 \\
	1913 -1.6354 \\
	1914 2.374 \\
	1915 5.3983 \\
	1916 1.4903 \\
	1917 4.5746 \\
	1918 6.4718 \\
	1919 1.4282 \\
	1920 9.0901 \\
	1921 -8.8544 \\
	1922 1.938 \\
	1923 -2.0918 \\
	1924 -0.61399 \\
	1925 4.8714 \\
	1926 -0.24325 \\
	1927 -3.5059 \\
	1928 9.2553 \\
	1929 5.303 \\
	1930 -9.3043 \\
	1931 -4.5188 \\
	1932 -2.3995 \\
	1933 5.6455 \\
	1934 -1.0619 \\
	1935 -4.7228 \\
	1936 0.49378 \\
	1937 1.2649 \\
	1938 1.8434 \\
	1939 -1.2224 \\
	1940 7.414 \\
	1941 9.8182 \\
	1942 -4.671 \\
	1943 -15.7278 \\
	1944 6.4001 \\
	1945 9.5318 \\
	1946 -12.3047 \\
	1947 -6.3095 \\
	1948 2.8635 \\
	1949 7.8728 \\
	1950 6.3319 \\
	1951 3.8299 \\
	1952 6.1646 \\
	1953 3.7947 \\
	1954 -9.4214 \\
	1955 -2.486 \\
	1956 7.5422 \\
	1957 2.9351 \\
	1958 11.6683 \\
	1959 -2.5313 \\
	1960 -9.8919 \\
	1961 -17.9521 \\
	1962 -20.9127 \\
	1963 3.6624 \\
	1964 -8.6265 \\
	1965 -9.7471 \\
	1966 12.8148 \\
	1967 1.5824 \\
	1968 -4.799 \\
	1969 11.3842 \\
	1970 -4.9751 \\
	1971 -7.4628 \\
	1972 11.1432 \\
	1973 5.1437 \\
	1974 0.76934 \\
	1975 -8.7726 \\
	1976 -1.7142 \\
	1977 0.58751 \\
	1978 9.3361 \\
	1979 15.3692 \\
	1980 -2.144 \\
	1981 -1.2413 \\
	1982 -12.1655 \\
	1983 4.5499 \\
	1984 9.2704 \\
	1985 -5.0596 \\
	1986 21.1145 \\
	1987 16.2642 \\
	1988 -2.653 \\
	1989 2.029 \\
	1990 5.2767 \\
	1991 1.1476 \\
	1992 -10.4121 \\
	1993 -0.6774 \\
	1994 5.8713 \\
	1995 3.4122 \\
	1996 1.9665 \\
	1997 -5.7719 \\
	1998 -3.4675 \\
	1999 3.5421 \\
	2000 14.328 \\
	2001 7.5895 \\
	2002 -15.2101 \\
	2003 -8.6721 \\
	2004 -3.0358 \\
	2005 -14.7665 \\
	2006 -9.0684 \\
	2007 -4.8516 \\
	2008 -7.3971 \\
	2009 -4.4721 \\
	2010 0.96481 \\
	2011 3.2684 \\
	2012 -5.2068 \\
	2013 -1.9218 \\
	2014 4.6818 \\
	2015 -7.6962 \\
	2016 -4.2256 \\
	2017 2.5298 \\
	2018 -4.7683 \\
	2019 6.6888 \\
	2020 -3.5613 \\
	2021 -5.9509 \\
	2022 12.1171 \\
	2023 -10.4675 \\
	2024 -4.5349 \\
	2025 8.2669 \\
	2026 5.5515 \\
	2027 12.764 \\
	2028 -5.8541 \\
	2029 -7.7081 \\
	2030 4.4887 \\
	2031 2.6968 \\
	2032 -0.67764 \\
	2033 2.5298 \\
	2034 3.7151 \\
	2035 8.4875 \\
	2036 2.7948 \\
	2037 -14.1994 \\
	2038 5.8748 \\
	2039 6.5821 \\
	2040 -1.205 \\
	2041 4.4721 \\
	2042 -2.0598 \\
	2043 3.3238 \\
	2044 0.37059 \\
	2045 2.7527 \\
	2046 -4.7243 \\
	2047 -6.1321 \\
	2048 -1.0545 \\
	2049 2.5298 \\
	2050 -10.48 \\
	2051 -3.2234 \\
	2052 -11.5725 \\
	2053 -5.6721 \\
	2054 14.1577 \\
	2055 -8.8394 \\
	2056 -0.23322 \\
	2057 2.5298 \\
	2058 8.7393 \\
	2059 18.2621 \\
	2060 4.1702 \\
	2061 1.7533 \\
	2062 3.1527 \\
	2063 3.6998 \\
	2064 -11.7607 \\
	2065 -6.3246 \\
	2066 14.1473 \\
	2067 8.8718 \\
	2068 -5.6522 \\
	2069 0.40143 \\
	2070 3.1493 \\
	2071 -12.3744 \\
	2072 -3.5663 \\
	2073 -4.5357 \\
	2074 -12.4109 \\
	2075 -3.1475 \\
	2076 1.0648 \\
	2077 1.742 \\
	2078 -5.4768 \\
	2079 -1.6956 \\
	2080 2.1026 \\
	2081 -12.6491 \\
	2082 -3.3221 \\
	2083 0.83838 \\
	2084 -5.6164 \\
	2085 11.2557 \\
	2086 2.213 \\
	2087 1.8116 \\
	2088 6.5898 \\
	2089 2.5298 \\
	2090 8.36 \\
	2091 -6.7523 \\
	2092 -1.2277 \\
	2093 3.5234 \\
	2094 4.6941 \\
	2095 -0.94022 \\
	2096 -5.2583 \\
	2097 8.8544 \\
	2098 -3.4514 \\
	2099 -2.475 \\
	2100 -6.5168 \\
	2101 1.6045 \\
	2102 13.1208 \\
	2103 3.1754 \\
	2104 2.5095 \\
	2105 -8.1134 \\
	2106 -0.71404 \\
	2107 2.8049 \\
	2108 10.1718 \\
	2109 10.69 \\
	2110 -5.5212 \\
	2111 -10.1354 \\
	2112 -5.5624 \\
	2113 8.8544 \\
	2114 -5.9387 \\
	2115 -11.8777 \\
	2116 0.96391 \\
	2117 -2.6064 \\
	2118 -1.1405 \\
	2119 -0.94128 \\
	2120 6.8978 \\
	2121 -5.52 \\
	2122 -7.5471 \\
	2123 1.5976 \\
	2124 -6.5056 \\
	2125 9.9164 \\
	2126 4.5666 \\
	2127 -13.649 \\
	2128 -21.83 \\
	2129 -8.8544 \\
	2130 -2.9832 \\
	2131 -5.1601 \\
	2132 5.512 \\
	2133 -4.1192 \\
	2134 11.0319 \\
	2135 3.7253 \\
	2136 -12.1659 \\
	2137 7.219 \\
	2138 13.0129 \\
	2139 8.1059 \\
	2140 4.2495 \\
	2141 9.2393 \\
	2142 7.36 \\
	2143 9.7923 \\
	2144 -3.0454 \\
	2145 -1.2649 \\
	2146 9.5547 \\
	2147 -9.6731 \\
	2148 -2.2124 \\
	2149 3.4372 \\
	2150 5.474 \\
	2151 -6.2763 \\
	2152 -11.5434 \\
	2153 10.5797 \\
	2154 -2.1842 \\
	2155 -0.69163 \\
	2156 8.819 \\
	2157 -1.3689 \\
	2158 4.9497 \\
	2159 1.5908 \\
	2160 -4.8175 \\
	2161 -1.2649 \\
	2162 -6.8096 \\
	2163 4.8106 \\
	2164 5.1961 \\
	2165 -9.3608 \\
	2166 -3.3588 \\
	2167 -4.0972 \\
	2168 1.5219 \\
	2169 5.4301 \\
	2170 3.7633 \\
	2171 -2.2905 \\
	2172 -2.5798 \\
	2173 10.0413 \\
	2174 0.60689 \\
	2175 -5.3236 \\
	2176 1.1818 \\
	2177 -2.5298 \\
	2178 14.4386 \\
	2179 10.2297 \\
	2180 -3.4881 \\
	2181 1.742 \\
	2182 -3.103 \\
	2183 -2.2401 \\
	2184 -0.89688 \\
	2185 3.5777 \\
	2186 -4.061 \\
	2187 0.42805 \\
	2188 4.87 \\
	2189 5.4769 \\
	2190 -3.371 \\
	2191 -12.6828 \\
	2192 6.5935 \\
	2193 11.3842 \\
	2194 11.9292 \\
	2195 -11.8339 \\
	2196 -12.6861 \\
	2197 1.7533 \\
	2198 -8.2368 \\
	2199 5.4228 \\
	2200 5.3718 \\
	2201 -2.7468 \\
	2202 -10.9763 \\
	2203 -1.7718 \\
	2204 6.6034 \\
	2205 1.9067 \\
	2206 -0.94022 \\
	2207 -3.1833 \\
	2208 -1.0329 \\
	2209 -10.1193 \\
	2210 3.411 \\
	2211 -1.2791 \\
	2212 8.6825 \\
	2213 10.69 \\
	2214 -6.7166 \\
	2215 14.1696 \\
	2216 -8.4474 \\
	2217 -3.5777 \\
	2218 3.3108 \\
	2219 -6.2341 \\
	2220 9.3986 \\
	2221 -3.4711 \\
	2222 4.4444 \\
	2223 5.2035 \\
	2224 -3.6403 \\
	2225 -1.2649 \\
	2226 -6.4553 \\
	2227 -9.7659 \\
	2228 -4.804 \\
	2229 3.5234 \\
	2230 0.62619 \\
	2231 -2.6074 \\
	2232 -10.5916 \\
	2233 -9.9023 \\
	2234 -2.346 \\
	2235 -5.0712 \\
	2236 -3.5167 \\
	2237 3.6769 \\
	2238 -2.0735 \\
	2239 11.0966 \\
	2240 17.7034 \\
	2241 1.2649 \\
	2242 0.99782 \\
	2243 4.6013 \\
	2244 4.2408 \\
	2245 0.12924 \\
	2246 -1.7675 \\
	2247 5.4809 \\
	2248 5.0507 \\
	2249 -2.3764 \\
	2250 -1.3701 \\
	2251 7.0329 \\
	2252 6.7531 \\
	2253 9.6429 \\
	2254 11.2952 \\
	2255 9.6216 \\
	2256 -5.4139 \\
	2257 -1.2649 \\
	2258 3.981 \\
	2259 -12.47 \\
	2260 9.6117 \\
	2261 -3.1537 \\
	2262 -10.0655 \\
	2263 -3.9801 \\
	2264 -11.4179 \\
	2265 17.9521 \\
	2266 -0.90098 \\
	2267 -10.8077 \\
	2268 4.3731 \\
	2269 -0.79915 \\
	2270 -7.8704 \\
	2271 0.89976 \\
	2272 8.0074 \\
	2273 -3.7947 \\
	2274 17.3949 \\
	2275 0.83554 \\
	2276 -1.5897 \\
	2277 13.3507 \\
	2278 -10.1446 \\
	2279 -1.0655 \\
	2280 -4.6158 \\
	2281 -7.7429 \\
	2282 -15.6856 \\
	2283 -5.0676 \\
	2284 19.3541 \\
	2285 -3.0115 \\
	2286 -1.9078 \\
	2287 2.1153 \\
	2288 -5.72 \\
	2289 -3.7947 \\
	2290 -4.6544 \\
	2291 -0.55628 \\
	2292 -2.4499 \\
	2293 2.3228 \\
	2294 9.3814 \\
	2295 -5.0609 \\
	2296 -6.2216 \\
	2297 -5.303 \\
	2298 -14.9416 \\
	2299 1.2529 \\
	2300 3.5072 \\
	2301 -3.3025 \\
	2302 -4.0997 \\
	2303 -2.9514 \\
	2304 6.8884 \\
	2305 0 \\
	2306 2.9909 \\
	2307 15.7908 \\
	2308 1.7643 \\
	2309 -3.0679 \\
	2310 -1.6844 \\
	2311 -4.9168 \\
	2312 7.6497 \\
	2313 -3.9482 \\
	2314 4.6346 \\
	2315 2.998 \\
	2316 -7.9368 \\
	2317 0.70757 \\
	2318 -5.2218 \\
	2319 0.64733 \\
	2320 2.896 \\
	2321 0 \\
	2322 11.314 \\
	2323 4.1869 \\
	2324 -8.7786 \\
	2325 -6.7705 \\
	2326 -10.5649 \\
	2327 -4.3767 \\
	2328 3.4507 \\
	2329 7.002 \\
	2330 -10.5531 \\
	2331 -24.0371 \\
	2332 8.3681 \\
	2333 -7.4384 \\
	2334 -18.9957 \\
	2335 9.456 \\
	2336 8.0805 \\
	2337 12.6491 \\
	2338 0.92903 \\
	2339 -14.6389 \\
	2340 3.7728 \\
	2341 8.2174 \\
	2342 -7.2493 \\
	2343 -4.4763 \\
	2344 2.6963 \\
	2345 1.4184 \\
	2346 -3.2435 \\
	2347 1.8302 \\
	2348 8.8 \\
	2349 5.617 \\
	2350 11.5067 \\
	2351 3.2798 \\
	2352 1.0949 \\
	2353 -10.1193 \\
	2354 -10.1915 \\
	2355 3.4784 \\
	2356 -16.7013 \\
	2357 -5.9685 \\
	2358 -8.4678 \\
	2359 -0.10704 \\
	2360 13.5801 \\
	2361 -1.9423 \\
	2362 3.2515 \\
	2363 -14.9065 \\
	2364 -2.7312 \\
	2365 13.763 \\
	2366 11.1873 \\
	2367 15.6726 \\
	2368 4.3524 \\
	2369 -7.5895 \\
	2370 3.7395 \\
	2371 -5.9678 \\
	2372 2.3808 \\
	2373 4.5713 \\
	2374 -13.7454 \\
	2375 7.3096 \\
	2376 9.4215 \\
	2377 -17.3646 \\
	2378 -0.05408 \\
	2379 12.4387 \\
	2380 2.8794 \\
	2381 16.6712 \\
	2382 13.8205 \\
	2383 5.6325 \\
	2384 12.3699 \\
	2385 1.2649 \\
	2386 -4.6594 \\
	2387 0.17649 \\
	2388 -6.4825 \\
	2389 -4.3655 \\
	2390 -3.7007 \\
	2391 -6.346 \\
	2392 5.6902 \\
	2393 0.43405 \\
	2394 3.4351 \\
	2395 2.1516 \\
	2396 -7.7478 \\
	2397 6.6563 \\
	2398 2.3224 \\
	2399 -0.027266 \\
	2400 -7.4806 \\
	2401 -2.5298 \\
	2402 10.6436 \\
	2403 -5.7279 \\
	2404 4.2399 \\
	2405 2.8012 \\
	2406 -6.1756 \\
	2407 -3.4532 \\
	2408 2.5123 \\
	2409 14.8348 \\
	2410 3.3928 \\
	2411 8.1786 \\
	2412 2.1082 \\
	2413 -9.6057 \\
	2414 4.5814 \\
	2415 -9.3175 \\
	2416 -15.9604 \\
	2417 1.2649 \\
	2418 1.9507 \\
	2419 -8.8991 \\
	2420 -11.3428 \\
	2421 4.5825 \\
	2422 3.4215 \\
	2423 -13.7373 \\
	2424 -0.16206 \\
	2425 14.7449 \\
	2426 11.0415 \\
	2427 -7.4102 \\
	2428 -15.525 \\
	2429 3.9869 \\
	2430 0.34401 \\
	2431 -5.3591 \\
	2432 -7.259 \\
	2433 13.914 \\
	2434 -1.5622 \\
	2435 3.9191 \\
	2436 13.9904 \\
	2437 4.157 \\
	2438 0.27575 \\
	2439 -5.6523 \\
	2440 -5.9533 \\
	2441 -13.914 \\
	2442 0.37564 \\
	2443 -2.6589 \\
	2444 -12.6523 \\
	2445 14.6834 \\
	2446 2.9506 \\
	2447 -6.256 \\
	2448 -5.1639 \\
	2449 -2.5298 \\
	2450 9.1935 \\
	2451 -4.261 \\
	2452 -2.8743 \\
	2453 6.3394 \\
	2454 -9.1372 \\
	2455 -7.344 \\
	2456 10.0859 \\
	2457 -3.5777 \\
	2458 -9.2939 \\
	2459 -2 \\
	2460 3.9122 \\
	2461 4.7822 \\
	2462 -9.6669 \\
	2463 -4.1729 \\
	2464 7.7691 \\
	2465 16.4438 \\
	2466 13.7068 \\
	2467 -3.3388 \\
	2468 -8.3017 \\
	2469 0.68557 \\
	2470 1.7309 \\
	2471 7.6709 \\
	2472 5.3282 \\
	2473 -13.914 \\
	2474 7.815 \\
	2475 5.5765 \\
	2476 0.62613 \\
	2477 14.2824 \\
	2478 -5.4239 \\
	2479 -7.2617 \\
	2480 -10.9707 \\
	2481 -5.0596 \\
	2482 1.5687 \\
	2483 6.2105 \\
	2484 4.0047 \\
	2485 -8.6522 \\
	2486 7.9815 \\
	2487 -5.228 \\
	2488 -8.1181 \\
	2489 3.5777 \\
	2490 -2.433 \\
	2491 1.6123 \\
	2492 -3.7649 \\
	2493 -0.86038 \\
	2494 2.0381 \\
	2495 -7.1736 \\
	2496 1.9632 \\
	2497 -5.0596 \\
	2498 0.43111 \\
	2499 4.9038 \\
	2500 7.6348 \\
	2501 -1.4287 \\
	2502 5.7529 \\
	2503 3.3235 \\
	2504 -7.0765 \\
	2505 14.5914 \\
	2506 -0.10074 \\
	2507 -4.5654 \\
	2508 1.9995 \\
	2509 6.2924 \\
	2510 0.31311 \\
	2511 -18.931 \\
	2512 4.1098 \\
	2513 5.0596 \\
	2514 1.5587 \\
	2515 1.3244 \\
	2516 -8.3327 \\
	2517 7.3681 \\
	2518 9.1831 \\
	2519 -5.2868 \\
	2520 -5.2685 \\
	2521 2.1593 \\
	2522 -1.1569 \\
	2523 -2.5488 \\
	2524 -6.1934 \\
	2525 -3.8774 \\
	2526 -8.4207 \\
	2527 -1.7542 \\
	2528 24.1113 \\
	2529 7.5895 \\
	2530 -6.2928 \\
	2531 -0.80969 \\
	2532 -4.2187 \\
	2533 -8.0395 \\
	2534 -2.5538 \\
	2535 -7.5569 \\
	2536 -11.3119 \\
	2537 5.6472 \\
	2538 0.44563 \\
	2539 -0.28846 \\
	2540 -4.7879 \\
	2541 -8.2982 \\
	2542 1.8877 \\
	2543 -3.9092 \\
	2544 0.59353 \\
	2545 5.0596 \\
	2546 0.51685 \\
	2547 -7.9483 \\
	2548 -3.3396 \\
	2549 4.6299 \\
	2550 0.336 \\
	2551 4.0266 \\
	2552 6.5515 \\
	2553 0.37048 \\
	2554 2.5024 \\
	2555 4.8728 \\
	2556 -9.2879 \\
	2557 -1.7062 \\
	2558 5.7166 \\
	2559 4.7899 \\
	2560 4.6974 \\
	2561 6.3246 \\
	2562 3.2404 \\
	2563 -3.8819 \\
	2564 -7.1067 \\
	2565 -2.4529 \\
	2566 -4.4861 \\
	2567 -13.8148 \\
	2568 -2.7067 \\
	2569 0.89443 \\
	2570 -3.0455 \\
	2571 -4.2873 \\
	2572 -7.9346 \\
	2573 -7.6629 \\
	2574 -8.8863 \\
	2575 -2.6642 \\
	2576 1.1142 \\
	2577 3.7947 \\
	2578 -3.6035 \\
	2579 0.05053 \\
	2580 -3.6143 \\
	2581 -9.3323 \\
	2582 8.2667 \\
	2583 0.86851 \\
	2584 3.2787 \\
	2585 -3.9482 \\
	2586 5.6986 \\
	2587 11.8739 \\
	2588 -14.9351 \\
	2589 7.2 \\
	2590 -1.7352 \\
	2591 -15.8023 \\
	2592 7.359 \\
	2593 13.914 \\
	2594 1.4605 \\
	2595 -2.8593 \\
	2596 8.2853 \\
	2597 3.5907 \\
	2598 12.1262 \\
	2599 9.5546 \\
	2600 3.1097 \\
	2601 -0.89443 \\
	2602 -9.8152 \\
	2603 0.73781 \\
	2604 1.7255 \\
	2605 2.8203 \\
	2606 8.1166 \\
	2607 13.9627 \\
	2608 -0.63314 \\
	2609 -6.3246 \\
	2610 -3.6272 \\
	2611 -5.0904 \\
	2612 11.0315 \\
	2613 -14.5739 \\
	2614 -18.4367 \\
	2615 5.9216 \\
	2616 0.52704 \\
	2617 1.4184 \\
	2618 4.6323 \\
	2619 8.5162 \\
	2620 3.2971 \\
	2621 -4.8872 \\
	2622 -0.024863 \\
	2623 7.0337 \\
	2624 7.3217 \\
	2625 11.3842 \\
	2626 -7.7107 \\
	2627 5.0663 \\
	2628 -1.1895 \\
	2629 -11.9384 \\
	2630 -6.2365 \\
	2631 -5.8303 \\
	2632 0.5609 \\
	2633 -1.4184 \\
	2634 1.2751 \\
	2635 7.238 \\
	2636 -2.2087 \\
	2637 -11.3313 \\
	2638 5.4089 \\
	2639 -5.087 \\
	2640 -6.3933 \\
	2641 11.3842 \\
	2642 4.7928 \\
	2643 -0.43351 \\
	2644 -5.8725 \\
	2645 5.4054 \\
	2646 -2.2004 \\
	2647 0.4624 \\
	2648 20.1225 \\
	2649 8.4839 \\
	2650 -7.6194 \\
	2651 -6.3128 \\
	2652 5.4587 \\
	2653 -2.5457 \\
	2654 1.423 \\
	2655 -6.8889 \\
	2656 -10.7962 \\
	2657 3.7947 \\
	2658 -9.1468 \\
	2659 -2.5151 \\
	2660 6.4206 \\
	2661 -2.5894 \\
	2662 -6.5173 \\
	2663 0.10678 \\
	2664 -0.54093 \\
	2665 3.9482 \\
	2666 4.869 \\
	2667 -5.1253 \\
	2668 10.6333 \\
	2669 16.4808 \\
	2670 8.8465 \\
	2671 -2.9887 \\
	2672 -3.6485 \\
	2673 1.2649 \\
	2674 -4.3665 \\
	2675 -6.1294 \\
	2676 2.1274 \\
	2677 1.533 \\
	2678 -12.9748 \\
	2679 8.4047 \\
	2680 7.2417 \\
	2681 6.695 \\
	2682 9.883 \\
	2683 -6.9671 \\
	2684 6.9651 \\
	2685 -10.1933 \\
	2686 -10.0838 \\
	2687 -3.3579 \\
	2688 1.4773 \\
	2689 -10.1193 \\
	2690 1.5056 \\
	2691 17.4792 \\
	2692 15.6548 \\
	2693 2.9583 \\
	2694 4.3943 \\
	2695 1.3522 \\
	2696 1.6832 \\
	2697 9.3783 \\
	2698 -0.46318 \\
	2699 3.8958 \\
	2700 -7.9581 \\
	2701 5.6868 \\
	2702 4.131 \\
	2703 -11.2374 \\
	2704 16.5183 \\
	2705 -6.3246 \\
	2706 -1.2156 \\
	2707 5.4631 \\
	2708 -9.6593 \\
	2709 10.0838 \\
	2710 -2.0618 \\
	2711 -10.8953 \\
	2712 -0.80247 \\
	2713 7.3724 \\
	2714 0.83289 \\
	2715 3.3711 \\
	2716 13.8716 \\
	2717 -0.64039 \\
	2718 -7.7801 \\
	2719 -3.2674 \\
	2720 6.5633 \\
	2721 -8.8818e-16 \\
	2722 -13.0561 \\
	2723 0.83387 \\
	2724 3.5791 \\
	2725 -13.1675 \\
	2726 -4.5014 \\
	2727 -0.15434 \\
	2728 -7.9264 \\
	2729 5.8006 \\
	2730 14.4431 \\
	2731 8.7372 \\
	2732 -1.3933 \\
	2733 -1.8921 \\
	2734 -2.7788 \\
	2735 -2.8287 \\
	2736 -0.87515 \\
	2737 -1.2649 \\
	2738 -1.8884 \\
	2739 -7.5494 \\
	2740 2.3417 \\
	2741 -2.4045 \\
	2742 -4.1959 \\
	2743 -4.4336 \\
	2744 -8.0257 \\
	2745 0.21702 \\
	2746 -12.3734 \\
	2747 -6.9327 \\
	2748 -4.2214 \\
	2749 4.4351 \\
	2750 14.8886 \\
	2751 -3.953 \\
	2752 -9.2308 \\
	2753 -10.1193 \\
	2754 8.1442 \\
	2755 2.4 \\
	2756 -1.3171 \\
	2757 -6.6531 \\
	2758 -9.6508 \\
	2759 0.36097 \\
	2760 2.1127 \\
	2761 0.58751 \\
	2762 1.0079 \\
	2763 -1.1381 \\
	2764 -3.5958 \\
	2765 -4.8282 \\
	2766 0.55082 \\
	2767 8.0635 \\
	2768 17.869 \\
	2769 -2.5298 \\
	2770 -26.7459 \\
	2771 -1.0489 \\
	2772 0.54053 \\
	2773 -1.4152 \\
	2774 -4.7273 \\
	2775 -1.6478 \\
	2776 20.4454 \\
	2777 -4.9961 \\
	2778 7.1816 \\
	2779 8.9806 \\
	2780 -10.5281 \\
	2781 -10.0698 \\
	2782 -17.0177 \\
	2783 4.8786 \\
	2784 -7.3605 \\
	2785 -10.1193 \\
	2786 -3.8908 \\
	2787 -4.1943 \\
	2788 5.787 \\
	2789 -6.0859 \\
	2790 5.3307 \\
	2791 -1.4754 \\
	2792 -2.6427 \\
	2793 9.5318 \\
	2794 -8.4203 \\
	2795 -9.2944 \\
	2796 4.4243 \\
	2797 -4.0262 \\
	2798 -0.97468 \\
	2799 9.4064 \\
	2800 -12.3235 \\
	2801 5.0596 \\
	2802 12.1648 \\
	2803 -1.7824 \\
	2804 10.7938 \\
	2805 6.5647 \\
	2806 8.0687 \\
	2807 1.1005 \\
	2808 7.5063 \\
	2809 7.5259 \\
	2810 -4.6204 \\
	2811 11.1372 \\
	2812 9.0743 \\
	2813 1.2154 \\
	2814 3.2414 \\
	2815 4.6115 \\
	2816 -10.4277 \\
	2817 -6.3246 \\
	2818 8.0714 \\
	2819 -3.4038 \\
	2820 -12.6401 \\
	2821 -2.5082 \\
	2822 -4.5496 \\
	2823 -3.6061 \\
	2824 -0.31771 \\
	2825 2.4663 \\
	2826 2.7321 \\
	2827 -6.4255 \\
	2828 4.9477 \\
	2829 -2.9611 \\
	2830 -1.6776 \\
	2831 4.8829 \\
	2832 0.58336 \\
	2833 6.3246 \\
	2834 -15.0517 \\
	2835 2.8202 \\
	2836 7.2884 \\
	2837 2.987 \\
	2838 4.2919 \\
	2839 -25.3417 \\
	2840 -6.1083 \\
	2841 7.002 \\
	2842 9.1373 \\
	2843 7.5943 \\
	2844 8.3187 \\
	2845 1.6909 \\
	2846 -13.5464 \\
	2847 7.5327 \\
	2848 7.9284 \\
	2849 8.8544 \\
	2850 9.8384 \\
	2851 9.6878 \\
	2852 8.4995 \\
	2853 -3.0753 \\
	2854 -12.9208 \\
	2855 -8.3462 \\
	2856 9.7238 \\
	2857 -10.0557 \\
	2858 -5.3384 \\
	2859 -5.8598 \\
	2860 -7.6178 \\
	2861 7.1899 \\
	2862 -12.3911 \\
	2863 -5.4136 \\
	2864 3.3386 \\
	2865 8.8544 \\
	2866 -4.3889 \\
	2867 -8.0563 \\
	2868 9.9556 \\
	2869 -4.9929 \\
	2870 -3.2079 \\
	2871 -0.3991 \\
	2872 -3.9605 \\
	2873 -1.9423 \\
	2874 1.2872 \\
	2875 -1.4165 \\
	2876 3.5824 \\
	2877 16.8487 \\
	2878 7.3562 \\
	2879 5.3929 \\
	2880 -3.1641 \\
	2881 5.0596 \\
	2882 5.5129 \\
	2883 -14.7964 \\
	2884 0.016774 \\
	2885 -6.1762 \\
	2886 -4.5623 \\
	2887 -3.2149 \\
	2888 -5.7771 \\
	2889 7.3724 \\
	2890 -4.4381 \\
	2891 -3.5195 \\
	2892 -0.34892 \\
	2893 8.0977 \\
	2894 19.7378 \\
	2895 -1.7537 \\
	2896 -5.6177 \\
	2897 3.7947 \\
	2898 -2.7698 \\
	2899 7.7026 \\
	2900 3.7179 \\
	2901 -6.8607 \\
	2902 2.6323 \\
	2903 4.2989 \\
	2904 -0.17326 \\
	2905 -8.3304 \\
	2906 5.1553 \\
	2907 12.6864 \\
	2908 0.97607 \\
	2909 -1.1151 \\
	2910 -4.0059 \\
	2911 -0.39649 \\
	2912 2.2086 \\
	2913 -5.0596 \\
	2914 -2.6687 \\
	2915 9.568 \\
	2916 -6.0591 \\
	2917 -5.208 \\
	2918 5.0639 \\
	2919 -5.0241 \\
	2920 8.8715 \\
	2921 0.21702 \\
	2922 -11.4149 \\
	2923 -7.5864 \\
	2924 -7.1806 \\
	2925 -6.0919 \\
	2926 -2.2528 \\
	2927 10.9873 \\
	2928 16.7834 \\
	2929 13.914 \\
	2930 -5.9564 \\
	2931 0.05558 \\
	2932 -5.5149 \\
	2933 -4.5235 \\
	2934 21.7995 \\
	2935 -0.68553 \\
	2936 -11.8647 \\
	2937 -11.9081 \\
	2938 10.2923 \\
	2939 0.94933 \\
	2940 -22.2523 \\
	2941 6.6987 \\
	2942 -1.7674 \\
	2943 0.84816 \\
	2944 1.8567 \\
	2945 0 \\
	2946 2.9379 \\
	2947 -9.4392 \\
	2948 0.1677 \\
	2949 -11.5773 \\
	2950 -8.2753 \\
	2951 -14.0671 \\
	2952 1.1575 \\
	2953 16.7508 \\
	2954 -4.2762 \\
	2955 -0.79417 \\
	2956 6.9579 \\
	2957 2.92 \\
	2958 11.9387 \\
	2959 17.1628 \\
	2960 -7.6363 \\
	2961 -13.914 \\
	2962 0.35191 \\
	2963 -9.4454 \\
	2964 -9.5276 \\
	2965 3.9852 \\
	2966 0.53757 \\
	2967 3.2005 \\
	2968 -10.0999 \\
	2969 -14.7449 \\
	2970 3.1326 \\
	2971 -10.5023 \\
	2972 -1.3769 \\
	2973 -1.361 \\
	2974 -0.26726 \\
	2975 9.1966 \\
	2976 -3.611 \\
	2977 10.1193 \\
	2978 -2.5024 \\
	2979 -4.7742 \\
	2980 -2.0779 \\
	2981 -3.9984 \\
	2982 9.2303 \\
	2983 6.7653 \\
	2984 10.2644 \\
	2985 6.0176 \\
	2986 14.4321 \\
	2987 -4.4713 \\
	2988 -8.3144 \\
	2989 11.3009 \\
	2990 -0.28347 \\
	2991 5.1253 \\
	2992 -6.3854 \\
	2993 -8.8544 \\
	2994 6.8021 \\
	2995 5.9501 \\
	2996 -3.8966 \\
	2997 9.0607 \\
	2998 6.0969 \\
	2999 -11.8714 \\
	3000 9.7866 \\
	3001 -0.43405 \\
	3002 -5.699 \\
	3003 -1.941 \\
	3004 2.8889 \\
	3005 4.8488 \\
	3006 -3.7985 \\
	3007 9.7861 \\
	3008 -8.6547 \\
	3009 10.1193 \\
	3010 13.0279 \\
	3011 -4.7227 \\
	3012 -5.7888 \\
	3013 7.6495 \\
	3014 5.5964 \\
	3015 -3.3571 \\
	3016 -8.2354 \\
	3017 4.3187 \\
	3018 4.6149 \\
	3019 -7.9291 \\
	3020 -10.7979 \\
	3021 -1.0917 \\
	3022 -0.81663 \\
	3023 8.129 \\
	3024 9.2352 \\
	3025 -3.7947 \\
	3026 8.455 \\
	3027 -7.4791 \\
	3028 -10.9334 \\
	3029 -1.0149 \\
	3030 8.369 \\
	3031 7.5655 \\
	3032 -8.3625 \\
	3033 5.5836 \\
	3034 -6.2877 \\
	3035 10.5717 \\
	3036 5.6953 \\
	3037 -12.5281 \\
	3038 13.7675 \\
	3039 -2.8862 \\
	3040 2.0269 \\
	3041 7.5895 \\
	3042 -6.7367 \\
	3043 -11.5893 \\
	3044 -0.84857 \\
	3045 1.2049 \\
	3046 -10.5995 \\
	3047 0.21232 \\
	3048 -5.0288 \\
	3049 0.74097 \\
	3050 -3.5898 \\
	3051 -4.9189 \\
	3052 3.0387 \\
	3053 7.2892 \\
	3054 20.0774 \\
	3055 -2.0534 \\
	3056 -4.6154 \\
	3057 -1.2649 \\
	3058 -7.1941 \\
	3059 -11.8062 \\
	3060 -3.7474 \\
	3061 9.8693 \\
	3062 -10.0947 \\
	3063 -5.4686 \\
	3064 2.4572 \\
	3065 2.0059 \\
	3066 5.7341 \\
	3067 2.4561 \\
	3068 1.0477 \\
	3069 -6.3184 \\
	3070 -3.9651 \\
	3071 2.9182 \\
	3072 4.4993 \\
	3073 -13.914 \\
	3074 -5.7151 \\
	3075 4.7395 \\
	3076 -12.7962 \\
	3077 6.1788 \\
	3078 -2.1624 \\
	3079 -9.3314 \\
	3080 1.6059 \\
	3081 -2.7468 \\
	3082 6.2635 \\
	3083 6.9978 \\
	3084 5.0155 \\
	3085 19.8021 \\
	3086 6.1783 \\
	3087 -12.2592 \\
	3088 2.6212 \\
	3089 5.0596 \\
	3090 11.1436 \\
	3091 7.8708 \\
	3092 -4.9272 \\
	3093 -0.71551 \\
	3094 -5.7765 \\
	3095 6.8574 \\
	3096 -0.12947 \\
	3097 -16.8407 \\
	3098 -3.7181 \\
	3099 -5.2892 \\
	3100 1.2683 \\
	3101 7.435 \\
	3102 -7.915 \\
	3103 -7.7902 \\
	3104 7.09 \\
	3105 -1.2649 \\
	3106 -8.1163 \\
	3107 10.6692 \\
	3108 -0.85368 \\
	3109 -7.4437 \\
	3110 9.7081 \\
	3111 3.2228 \\
	3112 -0.45942 \\
	3113 -9.9023 \\
	3114 -9.1186 \\
	3115 -1.3765 \\
	3116 -7.5025 \\
	3117 8.1158 \\
	3118 8.1357 \\
	3119 1.06 \\
	3120 -1.6545 \\
	3121 2.5298 \\
	3122 9.8921 \\
	3123 -7.0527 \\
	3124 -0.71451 \\
	3125 -0.5494 \\
	3126 -1.1756 \\
	3127 -7.7244 \\
	3128 -15.9663 \\
	3129 11.781 \\
	3130 4.4284 \\
	3131 8.7393 \\
	3132 15.4507 \\
	3133 2.5944 \\
	3134 -1.9328 \\
	3135 -9.4525 \\
	3136 1.8329 \\
	3137 5.0596 \\
	3138 7.8341 \\
	3139 -6.3798 \\
	3140 -4.2609 \\
	3141 2.7786 \\
	3142 8.8527 \\
	3143 5.7975 \\
	3144 -4.9415 \\
	3145 5.1232 \\
	3146 -9.9291 \\
	3147 -1.4331 \\
	3148 4.2012 \\
	3149 -0.12723 \\
	3150 10.9866 \\
	3151 0.18592 \\
	3152 -0.57543 \\
	3153 -2.5298 \\
	3154 -2.0732 \\
	3155 -13.6832 \\
	3156 -10.9467 \\
	3157 6.0652 \\
	3158 -10.3585 \\
	3159 -9.1913 \\
	3160 -4.2711 \\
	3161 -4.7791 \\
	3162 5.4468 \\
	3163 -6.141 \\
	3164 -11.8068 \\
	3165 7.0506 \\
	3166 -0.11839 \\
	3167 4.0363 \\
	3168 8.6234 \\
	3169 -12.6491 \\
	3170 -7.1414 \\
	3171 -7.705 \\
	3172 2.7411 \\
	3173 0.27521 \\
	3174 -13.2932 \\
	3175 1.817 \\
	3176 8.7381 \\
	3177 17.6452 \\
	3178 0.89932 \\
	3179 -4.5281 \\
	3180 2.5259 \\
	3181 -3.6675 \\
	3182 6.9141 \\
	3183 -1.232 \\
	3184 1.7886 \\
	3185 12.6491 \\
	3186 -2.2835 \\
	3187 2.9038 \\
	3188 8.9751 \\
	3189 -6.5892 \\
	3190 -3.6907 \\
	3191 -10.2043 \\
	3192 -5.7999 \\
	3193 14.8983 \\
	3194 2.1874 \\
	3195 1.5489 \\
	3196 -0.68005 \\
	3197 -10.8454 \\
	3198 -4.3524 \\
	3199 13.8505 \\
	3200 15.8085 \\
	3201 22.7684 \\
	3202 -4.3169 \\
	3203 0.63782 \\
	3204 4.5239 \\
	3205 -13.4509 \\
	3206 9.0069 \\
	3207 -2.9941 \\
	3208 8.117 \\
	3209 5.4301 \\
	3210 -5.3651 \\
	3211 4.5775 \\
	3212 1.7606 \\
	3213 -0.20151 \\
	3214 -9.8067 \\
	3215 4.1767 \\
	3216 4.7167 \\
	3217 2.5298 \\
	3218 -10.9018 \\
	3219 -22.916 \\
	3220 3.4777 \\
	3221 4.57 \\
	3222 0.090586 \\
	3223 2.8144 \\
	3224 2.0908 \\
	3225 7.3089 \\
	3226 10.4444 \\
	3227 2.8939 \\
	3228 -6.2057 \\
	3229 -7.508 \\
	3230 2.1798 \\
	3231 -0.33898 \\
	3232 -8.4802 \\
	3233 5.0596 \\
	3234 -1.1429 \\
	3235 1.4139 \\
	3236 6.5533 \\
	3237 -3.2999 \\
	3238 -2.0974 \\
	3239 -8.1757 \\
	3240 -5.8746 \\
	3241 7.219 \\
	3242 14.8027 \\
	3243 4.4635 \\
	3244 0.77639 \\
	3245 -4.6411 \\
	3246 -15.3712 \\
	3247 -3.1156 \\
	3248 10.9818 \\
	3249 -2.5298 \\
	3250 -11.1823 \\
	3251 11.1791 \\
	3252 -8.992 \\
	3253 -10.5877 \\
	3254 8.4035 \\
	3255 -1.3299 \\
	3256 -0.96392 \\
	3257 -12.3685 \\
	3258 -4.5532 \\
	3259 -7.3093 \\
	3260 -3.9899 \\
	3261 9.8208 \\
	3262 9.6903 \\
	3263 3.9035 \\
	3264 1.6274 \\
	3265 6.3246 \\
	3266 -4.5324 \\
	3267 -7.864 \\
	3268 8.1201 \\
	3269 3.6828 \\
	3270 0.026811 \\
	3271 5.4953 \\
	3272 -5.4852 \\
	3273 12.0616 \\
	3274 4.5826 \\
	3275 8.3102 \\
	3276 1.8842 \\
	3277 -10.7167 \\
	3278 7.9438 \\
	3279 -7.8462 \\
	3280 2.0262 \\
	3281 3.7947 \\
	3282 2.4399 \\
	3283 0.26402 \\
	3284 -8.1561 \\
	3285 4.2087 \\
	3286 -9.74 \\
	3287 0.22877 \\
	3288 -0.81614 \\
	3289 -10.0557 \\
	3290 9.3835 \\
	3291 10.4877 \\
	3292 7.8702 \\
	3293 7.724 \\
	3294 -2.1516 \\
	3295 -6.4222 \\
	3296 -3.7527 \\
	3297 -6.3246 \\
	3298 6.3536 \\
	3299 6.3033 \\
	3300 -7.3409 \\
	3301 -12.8441 \\
	3302 -11.9767 \\
	3303 4.6387 \\
	3304 12.4307 \\
	3305 3.1173 \\
	3306 -1.4326 \\
	3307 9.7879 \\
	3308 0.89847 \\
	3309 -0.23346 \\
	3310 11.3513 \\
	3311 8.2377 \\
	3312 -1.4505 \\
	3313 -6.3246 \\
	3314 11.7027 \\
	3315 -9.0025 \\
	3316 -17.466 \\
	3317 -2.6369 \\
	3318 -7.4424 \\
	3319 4.9959 \\
	3320 8.6472 \\
	3321 2.4663 \\
	3322 -9.1269 \\
	3323 -3.1078 \\
	3324 -5.1803 \\
	3325 -4.3633 \\
	3326 2.7375 \\
	3327 -14.3876 \\
	3328 -2.3485 \\
	3329 -1.2649 \\
	3330 5.2407 \\
	3331 1.4296 \\
	3332 -4.2677 \\
	3333 -5.8747 \\
	3334 -3.5575 \\
	3335 -6.0807 \\
	3336 2.0118 \\
	3337 8.2669 \\
	3338 1.1964 \\
	3339 -2.1118 \\
	3340 -2.9446 \\
	3341 -4.8628 \\
	3342 9.5134 \\
	3343 1.313 \\
	3344 -13.1076 \\
	3345 13.914 \\
	3346 14.2226 \\
	3347 8.0431 \\
	3348 13.7972 \\
	3349 17.2218 \\
	3350 5.3467 \\
	3351 -9.5961 \\
	3352 3.2671 \\
	3353 -4.4721 \\
	3354 -9.965 \\
	3355 -3.8558 \\
	3356 3.3463 \\
	3357 -1.8522 \\
	3358 -5.3586 \\
	3359 1.1182 \\
	3360 -0.39696 \\
	3361 3.7947 \\
	3362 -14.3837 \\
	3363 -1.9009 \\
	3364 8.9219 \\
	3365 1.773 \\
	3366 6.1584 \\
	3367 -6.8238 \\
	3368 2.1465 \\
	3369 -0.6774 \\
	3370 -8.5951 \\
	3371 0.42253 \\
	3372 12.6304 \\
	3373 2.55 \\
	3374 -7.8168 \\
	3375 3.0127 \\
	3376 0.58216 \\
	3377 1.2649 \\
	3378 -4.6668 \\
	3379 10.7508 \\
	3380 -2.9045 \\
	3381 -10.5903 \\
	3382 1.7693 \\
	3383 -25.3118 \\
	3384 4.2939 \\
	3385 4.4721 \\
	3386 -12.539 \\
	3387 2.4015 \\
	3388 0.91072 \\
	3389 6.6948 \\
	3390 -6.9231 \\
	3391 -3.1683 \\
	3392 2.0712 \\
	3393 -4.4409e-16 \\
	3394 6.2333 \\
	3395 -2.1735 \\
	3396 -19.562 \\
	3397 -2.928 \\
	3398 -2.3432 \\
	3399 -1.4672 \\
	3400 -1.2075 \\
	3401 -8.2669 \\
	3402 12.8021 \\
	3403 9.6502 \\
	3404 -8.9453 \\
	3405 -7.0555 \\
	3406 -6.8917 \\
	3407 5.3555 \\
	3408 8.0392 \\
	3409 -2.5298 \\
	3410 0.63634 \\
	3411 0.73362 \\
	3412 -2.2161 \\
	3413 1.4705 \\
	3414 4.1663 \\
	3415 6.2097 \\
	3416 8.8107 \\
	3417 3.7312 \\
	3418 -0.87736 \\
	3419 -6.3308 \\
	3420 -1.8544 \\
	3421 7.5501 \\
	3422 5.6362 \\
	3423 6.0707 \\
	3424 -0.15276 \\
	3425 2.5298 \\
	3426 8.4576 \\
	3427 11.1864 \\
	3428 3.3843 \\
	3429 -6.2333 \\
	3430 1.0849 \\
	3431 -14.2522 \\
	3432 -5.4363 \\
	3433 0.6774 \\
	3434 -4.141 \\
	3435 14.6043 \\
	3436 -10.2006 \\
	3437 -12.532 \\
	3438 5.9191 \\
	3439 -5.5535 \\
	3440 -2.9962 \\
	3441 7.5895 \\
	3442 0.51446 \\
	3443 -2.771 \\
	3444 11.3565 \\
	3445 0.10135 \\
	3446 -0.83258 \\
	3447 10.5576 \\
	3448 0.17471 \\
	3449 -8.7908 \\
	3450 -1.291 \\
	3451 10.5182 \\
	3452 -8.6075 \\
	3453 -5.6714 \\
	3454 1.2844 \\
	3455 -11.9802 \\
	3456 -0.94454 \\
	3457 -2.5298 \\
	3458 8.4708 \\
	3459 -8.3258 \\
	3460 1.2768 \\
	3461 3.744 \\
	3462 -5.2159 \\
	3463 0.32775 \\
	3464 -9.9701 \\
	3465 5.737 \\
	3466 -6.3152 \\
	3467 -10.7386 \\
	3468 8.6646 \\
	3469 0.1995 \\
	3470 -4.2387 \\
	3471 -7.9742 \\
	3472 -4.2139 \\
	3473 -5.0596 \\
	3474 17.3809 \\
	3475 5.9095 \\
	3476 -11.6636 \\
	3477 15.9787 \\
	3478 -2.1329 \\
	3479 -1.0319 \\
	3480 14.4776 \\
	3481 9.9658 \\
	3482 0.4407 \\
	3483 -7.1732 \\
	3484 1.6065 \\
	3485 -1.8654 \\
	3486 -2.2686 \\
	3487 4.2532 \\
	3488 1.6665 \\
	3489 -5.0596 \\
	3490 13.6499 \\
	3491 8.0975 \\
	3492 0.72612 \\
	3493 11.9588 \\
	3494 -9.2847 \\
	3495 2.477 \\
	3496 -0.96444 \\
	3497 -3.2072 \\
	3498 13.2221 \\
	3499 -5.5706 \\
	3500 -3.7193 \\
	3501 -5.0421 \\
	3502 -1.1475 \\
	3503 3.2225 \\
	3504 -11.2846 \\
	3505 -10.1193 \\
	3506 -4.6747 \\
	3507 1.0401 \\
	3508 -2.096 \\
	3509 -3.8535 \\
	3510 2.462 \\
	3511 -2.207 \\
	3512 3.9877 \\
	3513 15.3324 \\
	3514 8.7814 \\
	3515 1.5821 \\
	3516 -2.8186 \\
	3517 4.1782 \\
	3518 1.2282 \\
	3519 -14.2464 \\
	3520 -16.0332 \\
	3521 -2.5298 \\
	3522 -6.3046 \\
	3523 -17.3205 \\
	3524 -6.9459 \\
	3525 5.9437 \\
	3526 -0.55995 \\
	3527 5.39 \\
	3528 3.7953 \\
	3529 -2.9003 \\
	3530 -4.6193 \\
	3531 -9.988 \\
	3532 -0.60504 \\
	3533 -6.1291 \\
	3534 3.5652 \\
	3535 4.7116 \\
	3536 4.8209 \\
	3537 2.5298 \\
	3538 -8.4723 \\
	3539 7.4722 \\
	3540 0.52368 \\
	3541 7.5852 \\
	3542 11.6216 \\
	3543 0.12682 \\
	3544 1.9464 \\
	3545 -1.9423 \\
	3546 -9.7491 \\
	3547 -8.062 \\
	3548 6.0294 \\
	3549 3.5109 \\
	3550 4.553 \\
	3551 -8.9531 \\
	3552 -3.1 \\
	3553 7.5895 \\
	3554 -16.849 \\
	3555 -10.3868 \\
	3556 -2.4094 \\
	3557 -0.66704 \\
	3558 -1.9646 \\
	3559 2.9883 \\
	3560 0.15952 \\
	3561 -4.6892 \\
	3562 10.5471 \\
	3563 4.1929 \\
	3564 1.462 \\
	3565 -5.562 \\
	3566 -1.4527 \\
	3567 6.571 \\
	3568 -5.1127 \\
	3569 2.2204e-16 \\
	3570 0.85203 \\
	3571 -4.6291 \\
	3572 3.8134 \\
	3573 4.8469 \\
	3574 2.5619 \\
	3575 4.144 \\
	3576 -13.6876 \\
	3577 7.002 \\
	3578 18.1885 \\
	3579 3.3038 \\
	3580 14.5384 \\
	3581 -4.4689 \\
	3582 8.2014 \\
	3583 10.3196 \\
	3584 -15.3477 \\
	3585 11.3842 \\
	3586 -5.4584 \\
	3587 8.9653 \\
	3588 21.1563 \\
	3589 12.793 \\
	3590 -6.7309 \\
	3591 -2.7797 \\
	3592 22.2148 \\
	3593 -5.0596 \\
	3594 -11.6245 \\
	3595 6.9873 \\
	3596 8.1731 \\
	3597 -0.4 \\
	3598 3.6364 \\
	3599 -3.1502 \\
	3600 -13.764 \\
	3601 2.5298 \\
	3602 -7.6572 \\
	3603 2.8304 \\
	3604 9.7476 \\
	3605 -2.0775 \\
	3606 -1.7044 \\
	3607 -10.7849 \\
	3608 -2.6679 \\
	3609 1.2649 \\
	3610 -0.87976 \\
	3611 -4.1974 \\
	3612 4.5004 \\
	3613 -0.95589 \\
	3614 -10.1416 \\
	3615 -0.36185 \\
	3616 -9.6911 \\
	3617 3.7947 \\
	3618 3.492 \\
	3619 -5.5934 \\
	3620 5.3792 \\
	3621 2.4759 \\
	3622 7.2725 \\
	3623 7.2484 \\
	3624 -2.8716 \\
	3625 -5.0596 \\
	3626 -7.1705 \\
	3627 -9.0007 \\
	3628 9.6056 \\
	3629 5.2426 \\
	3630 -9.7318 \\
	3631 -4.0673 \\
	3632 -12.3285 \\
	3633 5.0596 \\
	3634 15.2555 \\
	3635 -6.6364 \\
	3636 -5.6234 \\
	3637 -0.54224 \\
	3638 -0.95005 \\
	3639 -1.2733 \\
	3640 -7.9816 \\
	3641 1.2649 \\
	3642 -1.1362 \\
	3643 -8.5341 \\
	3644 5.1726 \\
	3645 -1.3569 \\
	3646 3.1709 \\
	3647 -0.010063 \\
	3648 -0.66376 \\
	3649 1.2649 \\
	3650 -5.4107 \\
	3651 5.0956 \\
	3652 1.1402 \\
	3653 15.577 \\
	3654 8.9794 \\
	3655 -8.048 \\
	3656 -0.75155 \\
	3657 -3.2708 \\
	3658 0.16877 \\
	3659 7.5581 \\
	3660 -3.3337 \\
	3661 -19.1694 \\
	3662 -7.2037 \\
	3663 -3.2287 \\
	3664 3.3687 \\
	3665 12.6491 \\
	3666 -8.7523 \\
	3667 -6.0484 \\
	3668 1.5554 \\
	3669 -5.2308 \\
	3670 -9.3046 \\
	3671 -8.4566 \\
	3672 2.3687 \\
	3673 3.0538 \\
	3674 -7.1757 \\
	3675 -10.2559 \\
	3676 -1.5714 \\
	3677 6.5441 \\
	3678 5.8772 \\
	3679 -5.8909 \\
	3680 2.699 \\
	3681 3.7947 \\
	3682 -7.06 \\
	3683 8.8988 \\
	3684 -8.6212 \\
	3685 -14.0052 \\
	3686 9.5559 \\
	3687 8.9805 \\
	3688 4.1624 \\
	3689 -6.8485 \\
	3690 7.9767 \\
	3691 4.3339 \\
	3692 -19.2064 \\
	3693 1.0638 \\
	3694 1.8857 \\
	3695 -3.1379 \\
	3696 1.3382 \\
	3697 -5.0596 \\
	3698 0.63769 \\
	3699 -0.35649 \\
	3700 2.8511 \\
	3701 -3.9305 \\
	3702 1.5888 \\
	3703 19.3051 \\
	3704 -3.8922 \\
	3705 -0.52394 \\
	3706 10.3644 \\
	3707 5.9534 \\
	3708 7.8155 \\
	3709 3.972 \\
	3710 7.9918 \\
	3711 -4.5831 \\
	3712 -0.041994 \\
	3713 1.2649 \\
	3714 9.3077 \\
	3715 1.6726 \\
	3716 -8.913 \\
	3717 1.428 \\
	3718 -1.2465 \\
	3719 -11.526 \\
	3720 -3.2433 \\
	3721 10.2092 \\
	3722 2.2683 \\
	3723 8.283 \\
	3724 12.3457 \\
	3725 -14.0677 \\
	3726 -14.8653 \\
	3727 -7.316 \\
	3728 6.2503 \\
	3729 7.5895 \\
	3730 -14.2681 \\
	3731 -1.109 \\
	3732 10.3075 \\
	3733 5.5653 \\
	3734 -2.9424 \\
	3735 -8.7024 \\
	3736 2.5326 \\
	3737 -8.6374 \\
	3738 -6.7671 \\
	3739 7.4747 \\
	3740 -4.5676 \\
	3741 -5.5693 \\
	3742 -9.2331 \\
	3743 -1.7832 \\
	3744 7.1077 \\
	3745 1.2649 \\
	3746 6.1717 \\
	3747 -0.075181 \\
	3748 11.997 \\
	3749 11.7451 \\
	3750 -12.8025 \\
	3751 -0.83096 \\
	3752 7.6754 \\
	3753 -7.6794 \\
	3754 -4.1844 \\
	3755 4.7209 \\
	3756 -6.8188 \\
	3757 10.2729 \\
	3758 17.0452 \\
	3759 -6.0315 \\
	3760 -6.3024 \\
	3761 -7.5895 \\
	3762 -6.3573 \\
	3763 -7.2098 \\
	3764 4.3864 \\
	3765 4.0301 \\
	3766 -10.1354 \\
	3767 1.2548 \\
	3768 -5.6911 \\
	3769 -1.4819 \\
	3770 5.8056 \\
	3771 1.4218 \\
	3772 3.5972 \\
	3773 1.7746 \\
	3774 11.8457 \\
	3775 9.637 \\
	3776 -0.30561 \\
	3777 2.5298 \\
	3778 -1.388 \\
	3779 6.9869 \\
	3780 4.7981 \\
	3781 1.9072 \\
	3782 -9.9898 \\
	3783 -5.0215 \\
	3784 6.4986 \\
	3785 -2.6833 \\
	3786 2.2729 \\
	3787 5.9224 \\
	3788 -0.48195 \\
	3789 8.032 \\
	3790 2.2827 \\
	3791 -11.9383 \\
	3792 4.2134 \\
	3793 0 \\
	3794 1.3351 \\
	3795 1.6865 \\
	3796 -17.0435 \\
	3797 0.93532 \\
	3798 -3.7158 \\
	3799 -12.0645 \\
	3800 -1.8269 \\
	3801 -11.8446 \\
	3802 -3.9286 \\
	3803 -4.4989 \\
	3804 -16.1101 \\
	3805 1.5873 \\
	3806 7.2188 \\
	3807 11.69 \\
	3808 12.7562 \\
	3809 -7.5895 \\
	3810 -6.6181 \\
	3811 3.7315 \\
	3812 9.852 \\
	3813 -12.1163 \\
	3814 -14.1229 \\
	3815 3.5939 \\
	3816 -6.4808 \\
	3817 2.6833 \\
	3818 1.3627 \\
	3819 14.2736 \\
	3820 9.7909 \\
	3821 -9.2969 \\
	3822 4.1718 \\
	3823 -3.4065 \\
	3824 -4.8365 \\
	3825 -12.6491 \\
	3826 3.8091 \\
	3827 13.9413 \\
	3828 0.24863 \\
	3829 6.744 \\
	3830 -1.9463 \\
	3831 2.3249 \\
	3832 2.4027 \\
	3833 4.2551 \\
	3834 -1.9046 \\
	3835 3.4936 \\
	3836 14.0054 \\
	3837 -2.8523 \\
	3838 11.0419 \\
	3839 -0.35696 \\
	3840 -7.667 \\
	3841 -1.2649 \\
	3842 2.3499 \\
	3843 3.1012 \\
	3844 -11.801 \\
	3845 11.0113 \\
	3846 8.9101 \\
	3847 -0.47484 \\
	3848 -2.4141 \\
	3849 -3.1437 \\
	3850 15.5752 \\
	3851 6.0519 \\
	3852 -3.3175 \\
	3853 -4.1737 \\
	3854 -6.3352 \\
	3855 -7.6188 \\
	3856 -9.2343 \\
	3857 10.1193 \\
	3858 2.2473 \\
	3859 -16.054 \\
	3860 -3.7414 \\
	3861 6.3256 \\
	3862 7.1269 \\
	3863 -0.79391 \\
	3864 -1.9402 \\
	3865 0.52394 \\
	3866 2.4753 \\
	3867 -6.2207 \\
	3868 -14.8497 \\
	3869 -1.6853 \\
	3870 -15.1903 \\
	3871 -12.0862 \\
	3872 -3.1205 \\
	3873 -11.3842 \\
	3874 10.0448 \\
	3875 0.6596 \\
	3876 -10.3084 \\
	3877 -0.67497 \\
	3878 3.9184 \\
	3879 1.8918 \\
	3880 4.8089 \\
	3881 18.3226 \\
	3882 -0.65608 \\
	3883 9.6451 \\
	3884 10.0183 \\
	3885 -19.7325 \\
	3886 -1.0082 \\
	3887 10.2986 \\
	3888 2.3726 \\
	3889 -5.0596 \\
	3890 -0.4044 \\
	3891 4.09 \\
	3892 3.3984 \\
	3893 11.1662 \\
	3894 4.6906 \\
	3895 -4.6348 \\
	3896 -3.6644 \\
	3897 -3.0538 \\
	3898 -9.2977 \\
	3899 3.7867 \\
	3900 8.267 \\
	3901 2.8231 \\
	3902 5.9112 \\
	3903 -1.7608 \\
	3904 5.1684 \\
	3905 -8.8544 \\
	3906 -17.7695 \\
	3907 3.7913 \\
	3908 0.47859 \\
	3909 -5.5879 \\
	3910 9.7577 \\
	3911 8.6523 \\
	3912 13.5801 \\
	3913 -10.7967 \\
	3914 2.1227 \\
	3915 -7.727 \\
	3916 -18.1429 \\
	3917 6.1104 \\
	3918 -11.9127 \\
	3919 4.7486 \\
	3920 4.6839 \\
	3921 1.2649 \\
	3922 5.127 \\
	3923 -3.5997 \\
	3924 3.6861 \\
	3925 -10.2624 \\
	3926 -0.79614 \\
	3927 12.6858 \\
	3928 9.0763 \\
	3929 8.7908 \\
	3930 -3.6824 \\
	3931 -1.7926 \\
	3932 20.2801 \\
	3933 7.2259 \\
	3934 -6.8696 \\
	3935 6.5986 \\
	3936 -0.69345 \\
	3937 -1.2649 \\
	3938 5.2371 \\
	3939 6.4104 \\
	3940 -2.7264 \\
	3941 -8.3261 \\
	3942 -9.4679 \\
	3943 -6.0453 \\
	3944 11.5716 \\
	3945 -1.8524 \\
	3946 -3.5278 \\
	3947 1.8253 \\
	3948 -3.9641 \\
	3949 2.003 \\
	3950 -0.97594 \\
	3951 -0.91618 \\
	3952 8.7581 \\
	3953 1.2649 \\
	3954 -15.849 \\
	3955 3.0833 \\
	3956 3.6328 \\
	3957 -3.6516 \\
	3958 6.9779 \\
	3959 4.5118 \\
	3960 3.9268 \\
	3961 -3.7312 \\
	3962 6.0075 \\
	3963 3.0687 \\
	3964 -9.5315 \\
	3965 -2.6902 \\
	3966 5.2632 \\
	3967 -4.9374 \\
	3968 -14.2582 \\
	3969 15.1789 \\
	3970 -8.1575 \\
	3971 11.2629 \\
	3972 13.9205 \\
	3973 -6.1596 \\
	3974 -5.0248 \\
	3975 0.74466 \\
	3976 4.7848 \\
	3977 8.7009 \\
	3978 6.7189 \\
	3979 -3.5634 \\
	3980 4.7222 \\
	3981 1.1711 \\
	3982 -5.4586 \\
	3983 1.2322 \\
	3984 0.043923 \\
	3985 -2.5298 \\
	3986 -6.4712 \\
	3987 -5.3899 \\
	3988 -4.6451 \\
	3989 3.333 \\
	3990 1.1654 \\
	3991 -10.8562 \\
	3992 -6.811 \\
	3993 0.15346 \\
	3994 3.2293 \\
	3995 1.1428 \\
	3996 3.329 \\
	3997 -1.4572 \\
	3998 -5.4399 \\
	3999 -1.7405 \\
	4000 -2.9855 \\
	4001 0 \\
	4002 -6.3429 \\
	4003 -8.7756 \\
	4004 -2.379 \\
	4005 -4.7905 \\
	4006 -5.7533 \\
	4007 -6.1469 \\
	4008 -0.49044 \\
	4009 14.0675 \\
	4010 13.8967 \\
	4011 1.4824 \\
	4012 4.9552 \\
	4013 10.5201 \\
	4014 2.387 \\
	4015 0.10502 \\
	4016 -8.084 \\
	4017 -5.0596 \\
	4018 5.3798 \\
	4019 -13.3242 \\
	4020 -0.69857 \\
	4021 0.027656 \\
	4022 -15.2831 \\
	4023 17.3063 \\
	4024 10.3634 \\
	4025 -5.2131 \\
	4026 -9.1212 \\
	4027 -8.1332 \\
	4028 10.2854 \\
	4029 2.4152 \\
	4030 3.9177 \\
	4031 -5.7042 \\
	4032 4.0469 \\
	4033 -1.2649 \\
	4034 -7.3047 \\
	4035 -10.5068 \\
	4036 1.4793 \\
	4037 5.8323 \\
	4038 -2.232 \\
	4039 -13.1253 \\
	4040 -8.1441 \\
	4041 1.9423 \\
	4042 -7.2722 \\
	4043 -6.7465 \\
	4044 9.6836 \\
	4045 7.1842 \\
	4046 -8.9371 \\
	4047 -5.4388 \\
	4048 12.1071 \\
	4049 6.3246 \\
	4050 7.329 \\
	4051 6.1063 \\
	4052 1.1132 \\
	4053 5.5413 \\
	4054 0.90205 \\
	4055 0.50922 \\
	4056 -6.7975 \\
	4057 4.2551 \\
	4058 7.3292 \\
	4059 3.4946 \\
	4060 0.37805 \\
	4061 0.30108 \\
	4062 17.2266 \\
	4063 -10.1732 \\
	4064 -1.103 \\
	4065 11.3842 \\
	4066 -2.7961 \\
	4067 15.3382 \\
	4068 5.3673 \\
	4069 3.329 \\
	4070 2.4248 \\
	4071 -5.8913 \\
	4072 -3.581 \\
	4073 -7.002 \\
	4074 -4.4831 \\
	4075 -4.503 \\
	4076 -9.9806 \\
	4077 5.2479 \\
	4078 11.9281 \\
	4079 -16.7919 \\
	4080 -3.1569 \\
	4081 1.2649 \\
	4082 -4.2443 \\
	4083 5.9029 \\
	4084 -17.448 \\
	4085 -7.1131 \\
	4086 7.8787 \\
	4087 7.9541 \\
	4088 -5.866 \\
	4089 -11.8446 \\
	4090 5.1549 \\
	4091 -4.026 \\
	4092 1.3837 \\
	4093 4.9756 \\
	4094 7.4539 \\
	4095 7.5398 \\
	4096 -5.793 \\
	4097 -11.3842 \\
	4098 -2.3109 \\
	4099 -11.194 \\
	4100 -5.1011 \\
	4101 -0.010802 \\
	4102 -6.1515 \\
	4103 -8.6083 \\
	4104 2.0773 \\
	4105 4.3187 \\
	4106 -4.1835 \\
	4107 -6.2427 \\
	4108 -4.3002 \\
	4109 12.8114 \\
	4110 7.1599 \\
	4111 -1.0966 \\
	4112 6.3418 \\
	4113 -2.5298 \\
	4114 -10.3578 \\
	4115 -11.1229 \\
	4116 -6.722 \\
	4117 7.9616 \\
	4118 7.9602 \\
	4119 -1.0976 \\
	4120 -7.8532 \\
	4121 -15.2688 \\
	4122 -2.8084 \\
	4123 -0.35219 \\
	4124 2.9372 \\
	4125 7.596 \\
	4126 -4.9355 \\
	4127 2.4466 \\
	4128 10.2186 \\
	4129 6.3246 \\
	4130 1.4598 \\
	4131 1.5408 \\
	4132 1.7532 \\
	4133 9.1721 \\
	4134 1.1647 \\
	4135 -11.6186 \\
	4136 7.1189 \\
	4137 0.74097 \\
	4138 -4.3218 \\
	4139 12.9652 \\
	4140 13.4449 \\
	4141 -6.0528 \\
	4142 -2.4642 \\
	4143 5.77 \\
	4144 -4.2771 \\
	4145 10.1193 \\
	4146 3.7547 \\
	4147 7.5133 \\
	4148 -5.0015 \\
	4149 -9.5334 \\
	4150 19.2534 \\
	4151 -9.2132 \\
	4152 -1.0442 \\
	4153 2.6197 \\
	4154 -7.758 \\
	4155 1.833 \\
	4156 0.89374 \\
	4157 13.4735 \\
	4158 -5.5804 \\
	4159 -1.8805 \\
	4160 -0.367 \\
	4161 6.3246 \\
	4162 11.1431 \\
	4163 7.4548 \\
	4164 -14.9573 \\
	4165 9.9438 \\
	4166 4.2705 \\
	4167 -10.8568 \\
	4168 -0.27219 \\
	4169 -1.8524 \\
	4170 1.2958 \\
	4171 3.8696 \\
	4172 13.853 \\
	4173 0.36619 \\
	4174 9.2265 \\
	4175 17.9654 \\
	4176 -9.7573 \\
	4177 -3.7947 \\
	4178 1.6484 \\
	4179 9.4889 \\
	4180 9.3867 \\
	4181 -2.0523 \\
	4182 5.2103 \\
	4183 -0.53465 \\
	4184 -5.1209 \\
	4185 -3.7312 \\
	4186 -3.4706 \\
	4187 -0.85043 \\
	4188 1.1789 \\
	4189 0.51358 \\
	4190 -2.9911 \\
	4191 -0.25015 \\
	4192 0.43677 \\
	4193 3.7947 \\
	4194 6.275 \\
	4195 -5.1202 \\
	4196 -8.5071 \\
	4197 -6.5831 \\
	4198 -8.8837 \\
	4199 -8.5229 \\
	4200 -9.6995 \\
	4201 -10.7967 \\
	4202 -13.3437 \\
	4203 2.7385 \\
	4204 11.9458 \\
	4205 -2.3721 \\
	4206 3.6417 \\
	4207 2.4406 \\
	4208 1.9813 \\
	4209 6.3246 \\
	4210 -8.0271 \\
	4211 0.82557 \\
	4212 6.3212 \\
	4213 -8.8979 \\
	4214 -4.9728 \\
	4215 -1.986 \\
	4216 0.69542 \\
	4217 8.7908 \\
	4218 2.3834 \\
	4219 6.8914 \\
	4220 -2.8145 \\
	4221 -6.0972 \\
	4222 6.7137 \\
	4223 -13.4345 \\
	4224 -4.7895 \\
	4225 3.7947 \\
	4226 10.7259 \\
	4227 2.7108 \\
	4228 3.309 \\
	4229 -10.801 \\
	4230 -3.2317 \\
	4231 5.9888 \\
	4232 6.4243 \\
	4233 8.4839 \\
	4234 -5.4623 \\
	4235 11.8177 \\
	4236 0.63186 \\
	4237 -4.5528 \\
	4238 -7.0808 \\
	4239 -16.0056 \\
	4240 9.331 \\
	4241 -8.8544 \\
	4242 -15.9048 \\
	4243 3.8552 \\
	4244 -12.105 \\
	4245 -10.1089 \\
	4246 9.5824 \\
	4247 2.7396 \\
	4248 3.7353 \\
	4249 6.1711 \\
	4250 1.7282 \\
	4251 -1.9373 \\
	4252 0.6502 \\
	4253 9.3421 \\
	4254 -1.6271 \\
	4255 6.8862 \\
	4256 -1.9335 \\
	4257 -13.914 \\
	4258 -2.6833 \\
	4259 -7.2434 \\
	4260 -5.4617 \\
	4261 -13.5392 \\
	4262 3.1702 \\
	4263 12.0865 \\
	4264 -3.117 \\
	4265 6.695 \\
	4266 0.7561 \\
	4267 -15.6395 \\
	4268 -4.8929 \\
	4269 7.2996 \\
	4270 -1.9016 \\
	4271 6.435 \\
	4272 8.7966 \\
	4273 1.2649 \\
	4274 -4.5613 \\
	4275 -3.9482 \\
	4276 -0.26163 \\
	4277 -3.4982 \\
	4278 1.7154 \\
	4279 -18.2852 \\
	4280 3.1873 \\
	4281 11.5377 \\
	4282 -3.969 \\
	4283 15.4444 \\
	4284 -5.4319 \\
	4285 0.56019 \\
	4286 8.6243 \\
	4287 5.2143 \\
	4288 7.2574 \\
	4289 -2.5298 \\
	4290 2.3384 \\
	4291 9.0324 \\
	4292 -20.4695 \\
	4293 -4.267 \\
	4294 4.7649 \\
	4295 -0.74451 \\
	4296 11.9083 \\
	4297 6.3246 \\
	4298 3.0841 \\
	4299 9.3095 \\
	4300 3.184 \\
	4301 -6.568 \\
	4302 0.13045 \\
	4303 -4.5238 \\
	4304 5.6544 \\
	4305 8.8544 \\
	4306 -10.3103 \\
	4307 2.543 \\
	4308 9.6889 \\
	4309 1.5155 \\
	4310 -6.4607 \\
	4311 2.179 \\
	4312 3.6586 \\
	4313 -2.8367 \\
	4314 -0.095714 \\
	4315 -15.2913 \\
	4316 -4.0088 \\
	4317 9.1269 \\
	4318 7.932 \\
	4319 7.9809 \\
	4320 0.61131 \\
	4321 2.5298 \\
	4322 -3.0685 \\
	4323 -2.4838 \\
	4324 3.5195 \\
	4325 10.1575 \\
	4326 4.9347 \\
	4327 -10.9126 \\
	4328 -4.9148 \\
	4329 6.3246 \\
	4330 6.3964 \\
	4331 -9.2579 \\
	4332 -3.6921 \\
	4333 3.5142 \\
	4334 -5.0873 \\
	4335 9.7945 \\
	4336 9.3994 \\
	4337 8.8544 \\
	4338 8.5106 \\
	4339 -2.3702 \\
	4340 2.6355 \\
	4341 -9.9358 \\
	4342 -5.7687 \\
	4343 1.8887 \\
	4344 -0.96684 \\
	4345 7.8964 \\
	4346 -11.9146 \\
	4347 -6.6605 \\
	4348 -0.10867 \\
	4349 -8.603 \\
	4350 -5.5049 \\
	4351 -20.8411 \\
	4352 -5.9799 \\
	4353 5.0596 \\
	4354 3.7369 \\
	4355 0.78587 \\
	4356 -1.494 \\
	4357 -6.2413 \\
	4358 4.2869 \\
	4359 -7.2693 \\
	4360 -7.258 \\
	4361 5.8905 \\
	4362 -5.2137 \\
	4363 3.6853 \\
	4364 7.5438 \\
	4365 -3.0548 \\
	4366 -7.77 \\
	4367 -2.739 \\
	4368 4.7884 \\
	4369 3.7947 \\
	4370 9.4581 \\
	4371 -1.6015 \\
	4372 -2.6993 \\
	4373 6.4206 \\
	4374 -8.3778 \\
	4375 -3.8017 \\
	4376 -2.0155 \\
	4377 2.5298 \\
	4378 9.852 \\
	4379 1.3456 \\
	4380 5.5215 \\
	4381 -4.1628 \\
	4382 17.9166 \\
	4383 27.9225 \\
	4384 -9.9361 \\
	4385 2.5298 \\
	4386 9.9704 \\
	4387 -9.811 \\
	4388 -7.4331 \\
	4389 2.1396 \\
	4390 0.55736 \\
	4391 -1.6206 \\
	4392 -7.9191 \\
	4393 -8.4203 \\
	4394 6.633 \\
	4395 -4.5869 \\
	4396 -6.6508 \\
	4397 -7.8954 \\
	4398 -8.3085 \\
	4399 4.9758 \\
	4400 -4.3154 \\
	4401 1.2649 \\
	4402 -1.8098 \\
	4403 -4.1182 \\
	4404 -0.89494 \\
	4405 0.21087 \\
	4406 6.7058 \\
	4407 -13.4002 \\
	4408 -8.0355 \\
	4409 2.5298 \\
	4410 -3.1371 \\
	4411 -0.87805 \\
	4412 5.2388 \\
	4413 7.5235 \\
	4414 -4.1424 \\
	4415 0.99219 \\
	4416 5.2015 \\
	4417 2.5298 \\
	4418 -2.2008 \\
	4419 -2.4649 \\
	4420 6.2091 \\
	4421 4.9379 \\
	4422 15.3429 \\
	4423 12.6861 \\
	4424 -6.9917 \\
	4425 7.5259 \\
	4426 -2.7705 \\
	4427 -10.2085 \\
	4428 -2.7058 \\
	4429 6.8727 \\
	4430 17.2592 \\
	4431 -16.3525 \\
	4432 -8.647 \\
	4433 7.5895 \\
	4434 8.8408 \\
	4435 6.8619 \\
	4436 -6.0272 \\
	4437 6.4357 \\
	4438 -2.8176 \\
	4439 0.91505 \\
	4440 5.6099 \\
	4441 7.002 \\
	4442 3.7973 \\
	4443 -5.5911 \\
	4444 11.3988 \\
	4445 -4.2651 \\
	4446 -8.3894 \\
	4447 2.1199 \\
	4448 -0.96088 \\
	4449 -2.5298 \\
	4450 1.1715 \\
	4451 -4.1591 \\
	4452 -16.5757 \\
	4453 2.4345 \\
	4454 -6.8483 \\
	4455 -20.2249 \\
	4456 -9.9559 \\
	4457 -4.9961 \\
	4458 1.6328 \\
	4459 12.5325 \\
	4460 11.8631 \\
	4461 -6.3488 \\
	4462 4.9238 \\
	4463 -5.1185 \\
	4464 -7.2846 \\
	4465 10.1193 \\
	4466 -7.3776 \\
	4467 -2.7677 \\
	4468 -6.6106 \\
	4469 -6.2187 \\
	4470 9.068 \\
	4471 1.1301 \\
	4472 -1.8032 \\
	4473 -1.9423 \\
	4474 -2.2255 \\
	4475 0.73722 \\
	4476 1.5804 \\
	4477 1.2114 \\
	4478 0.95134 \\
	4479 -0.45338 \\
	4480 0.54355 \\
	4481 8.8544 \\
	4482 -0.23456 \\
	4483 -8.2448 \\
	4484 2.3598 \\
	4485 -2.1457 \\
	4486 6.2582 \\
	4487 2.1513 \\
	4488 -3.9898 \\
	4489 0.52394 \\
	4490 8.7514 \\
	4491 2.6882 \\
	4492 -11.5832 \\
	4493 9.2584 \\
	4494 -5.3875 \\
	4495 -15.4563 \\
	4496 -0.97738 \\
	4497 -12.6491 \\
	4498 -0.89617 \\
	4499 4.5691 \\
	4500 3.7671 \\
	4501 11.4656 \\
	4502 -9.0398 \\
	4503 -2.1717 \\
	4504 15.6794 \\
	4505 -0.74097 \\
	4506 -7.3462 \\
	4507 -11.9172 \\
	4508 -7.8128 \\
	4509 -1.8906 \\
	4510 1.9379 \\
	4511 1.8382 \\
	4512 -15.515 \\
	4513 -8.8544 \\
	4514 -2.2127 \\
	4515 -1.7192 \\
	4516 -2.7629 \\
	4517 2.3627 \\
	4518 8.3344 \\
	4519 -0.4885 \\
	4520 0.80058 \\
	4521 -3.0538 \\
	4522 4.3795 \\
	4523 3.7461 \\
	4524 3.2036 \\
	4525 -2.1929 \\
	4526 -13.2735 \\
	4527 0.74341 \\
	4528 -3.0502 \\
	4529 5.0596 \\
	4530 5.616 \\
	4531 9.4066 \\
	4532 13.6062 \\
	4533 -4.0931 \\
	4534 -1.6314 \\
	4535 -1.4071 \\
	4536 4.0102 \\
	4537 -4.3187 \\
	4538 11.3134 \\
	4539 16.6502 \\
	4540 -5.8375 \\
	4541 12.5339 \\
	4542 3.5505 \\
	4543 -10.5075 \\
	4544 -2.0174 \\
	4545 -1.2649 \\
	4546 8.1092 \\
	4547 4.4185 \\
	4548 10.9154 \\
	4549 5.4844 \\
	4550 -1.9145 \\
	4551 -2.6055 \\
	4552 -10.5412 \\
	4553 -5.5836 \\
	4554 -3.1935 \\
	4555 1.993 \\
	4556 -3.9359 \\
	4557 -7.224 \\
	4558 -4.0587 \\
	4559 11.4529 \\
	4560 19.9823 \\
	4561 5.0596 \\
	4562 4.4821 \\
	4563 -5.9103 \\
	4564 4.9805 \\
	4565 1.8992 \\
	4566 -14.8183 \\
	4567 0.42155 \\
	4568 5.3857 \\
	4569 3.5777 \\
	4570 -1.7962 \\
	4571 2.7847 \\
	4572 -6.278 \\
	4573 1.5841 \\
	4574 10.2028 \\
	4575 2.2191 \\
	4576 9.8364 \\
	4577 -11.3842 \\
	4578 -0.047124 \\
	4579 15.7445 \\
	4580 10.3748 \\
	4581 7.2546 \\
	4582 -1.1991 \\
	4583 12.58 \\
	4584 1.0459 \\
	4585 -2.0059 \\
	4586 5.9932 \\
	4587 -6.7528 \\
	4588 -6.5511 \\
	4589 -6.2559 \\
	4590 -11.2751 \\
	4591 -3.5209 \\
	4592 2.4915 \\
	4593 -10.1193 \\
	4594 -6.2771 \\
	4595 4.0699 \\
	4596 -1.6776 \\
	4597 -7.0488 \\
	4598 -0.0056359 \\
	4599 0.77109 \\
	4600 -12.0212 \\
	4601 -3.5777 \\
	4602 9.136 \\
	4603 -1.1686 \\
	4604 -9.9238 \\
	4605 -0.75319 \\
	4606 -3.4574 \\
	4607 -6.1394 \\
	4608 -3.9643 \\
	4609 -2.2204e-16 \\
	4610 1.0302 \\
	4611 -5.6264 \\
	4612 4.0934 \\
	4613 5.3389 \\
	4614 -5.6288 \\
	4615 -0.46939 \\
	4616 7.2041 \\
	4617 4.4721 \\
	4618 -7.5774 \\
	4619 -10.548 \\
	4620 7.2966 \\
	4621 -12.3435 \\
	4622 -7.2496 \\
	4623 6.9402 \\
	4624 -10.8821 \\
	4625 7.5895 \\
	4626 9.2068 \\
	4627 7.6459 \\
	4628 -5.5663 \\
	4629 0.79308 \\
	4630 7.0983 \\
	4631 -16.2743 \\
	4632 11.9638 \\
	4633 5.737 \\
	4634 -4.0976 \\
	4635 3.8201 \\
	4636 2.3499 \\
	4637 7.5912 \\
	4638 12.8988 \\
	4639 26.7954 \\
	4640 12.9858 \\
	4641 2.5298 \\
	4642 -16.3189 \\
	4643 -4.8354 \\
	4644 16.3448 \\
	4645 2.0335 \\
	4646 0.85323 \\
	4647 -13.7073 \\
	4648 -6.8936 \\
	4649 -4.4721 \\
	4650 -8.6859 \\
	4651 -1.9564 \\
	4652 -7.4493 \\
	4653 3.6162 \\
	4654 6.4562 \\
	4655 -3.1071 \\
	4656 -1.3139 \\
	4657 2.5298 \\
	4658 -7.5663 \\
	4659 -6.2555 \\
	4660 -3.4473 \\
	4661 -0.57605 \\
	4662 -1.1154 \\
	4663 -4.2783 \\
	4664 7.4399 \\
	4665 -3.2072 \\
	4666 -2.6362 \\
	4667 -7.5426 \\
	4668 -5.5982 \\
	4669 8.7255 \\
	4670 -7.0255 \\
	4671 -0.95897 \\
	4672 1.8305 \\
	4673 -3.7947 \\
	4674 -1.0877 \\
	4675 -0.64973 \\
	4676 11.9842 \\
	4677 -10.8374 \\
	4678 -16.4397 \\
	4679 -3.3744 \\
	4680 -14.3932 \\
	4681 -0.74097 \\
	4682 3.6066 \\
	4683 -2.5464 \\
	4684 -1.6897 \\
	4685 -2.4439 \\
	4686 0.42221 \\
	4687 -3.8856 \\
	4688 -2.2938 \\
	4689 -7.5895 \\
	4690 -10.4933 \\
	4691 -2.0613 \\
	4692 -4.9254 \\
	4693 -9.9868 \\
	4694 1.0011 \\
	4695 8.4461 \\
	4696 -2.8761 \\
	4697 3.3607 \\
	4698 21.34 \\
	4699 13.5123 \\
	4700 -3.6963 \\
	4701 -1.0673 \\
	4702 -4.0578 \\
	4703 -4.4413 \\
	4704 8.9295 \\
	4705 11.3842 \\
	4706 9.8228 \\
	4707 -4.912 \\
	4708 -1.1718 \\
	4709 1.9831 \\
	4710 -11.961 \\
	4711 -1.0756 \\
	4712 -0.57289 \\
	4713 -4.3187 \\
	4714 2.7301 \\
	4715 11.9913 \\
	4716 7.0255 \\
	4717 -6.7174 \\
	4718 -5.0838 \\
	4719 0.36635 \\
	4720 11.8007 \\
	4721 7.5895 \\
	4722 1.9592 \\
	4723 4.4795 \\
	4724 3.2911 \\
	4725 1.1324 \\
	4726 -3.1698 \\
	4727 7.1711 \\
	4728 -3.84 \\
	4729 -10.9502 \\
	4730 2.8397 \\
	4731 -4.6345 \\
	4732 1.3975 \\
	4733 2.6391 \\
	4734 -1.548 \\
	4735 11.9724 \\
	4736 1.15 \\
	4737 10.1193 \\
	4738 -2.4739 \\
	4739 9.0445 \\
	4740 9.8272 \\
	4741 -7.3681 \\
	4742 0.04543 \\
	4743 -7.326 \\
	4744 5.8913 \\
	4745 -6.261 \\
	4746 -2.0882 \\
	4747 12.7487 \\
	4748 10.8704 \\
	4749 3.8322 \\
	4750 -14.0479 \\
	4751 7.8818 \\
	4752 -7.7234 \\
	4753 -2.5298 \\
	4754 16.4692 \\
	4755 -7.5095 \\
	4756 7.2887 \\
	4757 8.0395 \\
	4758 -1.6078 \\
	4759 8.032 \\
	4760 15.2168 \\
	4761 11.8446 \\
	4762 -8.0933 \\
	4763 -8.6976 \\
	4764 6.6793 \\
	4765 3.8509 \\
	4766 5.3819 \\
	4767 -10.4129 \\
	4768 -21.1509 \\
	4769 7.5895 \\
	4770 -4.049 \\
	4771 -10.5173 \\
	4772 0.21626 \\
	4773 -4.6299 \\
	4774 4.1576 \\
	4775 3.1455 \\
	4776 3.3844 \\
	4777 6.261 \\
	4778 0.99217 \\
	4779 1.5498 \\
	4780 6.0534 \\
	4781 -6.8859 \\
	4782 -2.4828 \\
	4783 -0.3536 \\
	4784 -5.132 \\
	4785 12.6491 \\
	4786 -6.2131 \\
	4787 -10.8223 \\
	4788 -1.2781 \\
	4789 1.4287 \\
	4790 4.9356 \\
	4791 0.77411 \\
	4792 4.2832 \\
	4793 -4.2551 \\
	4794 -2.5674 \\
	4795 -11.0946 \\
	4796 -3.012 \\
	4797 -3.327 \\
	4798 -18.7165 \\
	4799 -6.8006 \\
	4800 -1.0568 \\
	4801 1.7764e-15 \\
	4802 -4.6702 \\
	4803 5.7018 \\
	4804 16.8003 \\
	4805 -0.35438 \\
	4806 -5.0913 \\
	4807 -8.2605 \\
	4808 8.741 \\
	4809 8.2669 \\
	4810 1.6286 \\
	4811 9.1849 \\
	4812 -1.354 \\
	4813 -5.5413 \\
	4814 3.1315 \\
	4815 9.7369 \\
	4816 -2.3404 \\
	4817 -2.5298 \\
	4818 -0.81639 \\
	4819 -2.6982 \\
	4820 -2.5724 \\
	4821 -8.8639 \\
	4822 -10.4609 \\
	4823 -15.5472 \\
	4824 0.50518 \\
	4825 13.8505 \\
	4826 7.4281 \\
	4827 -0.88282 \\
	4828 -0.63941 \\
	4829 -5.8323 \\
	4830 -4.9128 \\
	4831 7.7552 \\
	4832 2.7853 \\
	4833 20.2386 \\
	4834 10.8205 \\
	4835 -7.2437 \\
	4836 0.41793 \\
	4837 -1.9584 \\
	4838 3.4586 \\
	4839 -5.9944 \\
	4840 1.66 \\
	4841 -0.6774 \\
	4842 2.9229 \\
	4843 16.0723 \\
	4844 -7.4926 \\
	4845 7.1131 \\
	4846 2.8207 \\
	4847 -19.4534 \\
	4848 -2.432 \\
	4849 -5.0596 \\
	4850 -5.8371 \\
	4851 -2.4813 \\
	4852 -8.6661 \\
	4853 13.7066 \\
	4854 8.3075 \\
	4855 -14.4326 \\
	4856 0.62494 \\
	4857 1.3285 \\
	4858 0.73866 \\
	4859 -2.4741 \\
	4860 -8.7088 \\
	4861 -3.329 \\
	4862 0.65096 \\
	4863 0.65919 \\
	4864 -7.4482 \\
	4865 -2.5298 \\
	4866 -13.0295 \\
	4867 -7.3057 \\
	4868 -17.6545 \\
	4869 -4.8801 \\
	4870 6.1761 \\
	4871 10.7549 \\
	4872 -4.1749 \\
	4873 -13.5435 \\
	4874 7.8996 \\
	4875 -4.7176 \\
	4876 -12.6578 \\
	4877 6.3548 \\
	4878 -0.52697 \\
	4879 2.7252 \\
	4880 5.8735 \\
	4881 -2.5298 \\
	4882 9.6895 \\
	4883 9.6886 \\
	4884 13.5858 \\
	4885 9.4345 \\
	4886 1.7354 \\
	4887 4.1273 \\
	4888 -13.0366 \\
	4889 9.3148 \\
	4890 21.3343 \\
	4891 1.6803 \\
	4892 3.7108 \\
	4893 -3.5388 \\
	4894 3.2543 \\
	4895 1.8609 \\
	4896 -12.892 \\
	4897 -2.5298 \\
	4898 -5.6337 \\
	4899 -12.2637 \\
	4900 2.2534 \\
	4901 5.8381 \\
	4902 2.918 \\
	4903 15.5795 \\
	4904 10.9009 \\
	4905 -11.7547 \\
	4906 -1.1962 \\
	4907 7.3327 \\
	4908 -7.3494 \\
	4909 -2.9942 \\
	4910 -8.6896 \\
	4911 -8.4465 \\
	4912 1.7793 \\
	4913 -5.0596 \\
	4914 -0.19855 \\
	4915 -3.3822 \\
	4916 -4.3826 \\
	4917 2.2566 \\
	4918 9.1795 \\
	4919 12.291 \\
	4920 -1.5363 \\
	4921 -6.7849 \\
	4922 -3.6864 \\
	4923 3.9079 \\
	4924 -6.9955 \\
	4925 -7.4113 \\
	4926 1.1323 \\
	4927 -3.4748 \\
	4928 12.2179 \\
	4929 8.8544 \\
	4930 5.5539 \\
	4931 1.7369 \\
	4932 1.5271 \\
	4933 -6.4112 \\
	4934 -14.0689 \\
	4935 2.1724 \\
	4936 2.7128 \\
	4937 -1.6354 \\
	4938 -0.0021883 \\
	4939 12.4964 \\
	4940 -1.7806 \\
	4941 -18.5638 \\
	4942 12.7455 \\
	4943 6.3033 \\
	4944 10.3518 \\
	4945 11.3842 \\
	4946 -3.8681 \\
	4947 4.9865 \\
	4948 -6.688 \\
	4949 1.44 \\
	4950 3.8979 \\
	4951 -3.1491 \\
	4952 2.342 \\
	4953 -0.6774 \\
	4954 -6.5839 \\
	4955 0.56717 \\
	4956 6.3824 \\
	4957 3.3072 \\
	4958 5.566 \\
	4959 -9.4995 \\
	4960 3.67 \\
	4961 3.7947 \\
	4962 -7.8709 \\
	4963 -0.80076 \\
	4964 -16.051 \\
	4965 1.5686 \\
	4966 0.93689 \\
	4967 -0.96394 \\
	4968 16.8398 \\
	4969 -3.4242 \\
	4970 -18.1142 \\
	4971 -12.256 \\
	4972 -0.44624 \\
	4973 -5.3423 \\
	4974 -5.8939 \\
	4975 -7.7044 \\
	4976 -6.9388 \\
	4977 8.8544 \\
	4978 -4.7248 \\
	4979 -12.0302 \\
	4980 0.17222 \\
	4981 0.87284 \\
	4982 3.8307 \\
	4983 2.9885 \\
	4984 -0.031667 \\
	4985 8.2669 \\
	4986 -1.035 \\
	4987 0.24032 \\
	4988 10.5968 \\
	4989 -2.1694 \\
	4990 -0.72694 \\
	4991 4.7931 \\
	4992 7.6993 \\
	4993 -13.914 \\
	4994 6.5663 \\
	4995 -7.4827 \\
	4996 11.0349 \\
	4997 0.94654 \\
	4998 3.9215 \\
	4999 1.5321 \\
	5000 -21.5225 \\
	5001 5.4301 \\
	5002 -1.802 \\
	5003 -7.6709 \\
	5004 -2.9334 \\
	5005 1.8628 \\
	5006 3.8751 \\
	5007 5.352 \\
	5008 5.333 \\
	5009 -16.4438 \\
	5010 -4.5805 \\
	5011 3.1937 \\
	5012 -8.9313 \\
	5013 -3.1796 \\
	5014 -0.20119 \\
	5015 -9.2339 \\
	5016 -2.3801 \\
	5017 1.9423 \\
	5018 -4.5613 \\
	5019 -7.136 \\
	5020 -6.4597 \\
	5021 10.115 \\
	5022 -0.83853 \\
	5023 -1.2391 \\
	5024 12.368 \\
	5025 1.2649 \\
	5026 5.9283 \\
	5027 8.0789 \\
	5028 0.22729 \\
	5029 -0.42259 \\
	5030 10.1423 \\
	5031 10.2802 \\
	5032 4.8112 \\
	5033 7.219 \\
	5034 4.8636 \\
	5035 -2.05 \\
	5036 3.9785 \\
	5037 8.4735 \\
	5038 0.70867 \\
	5039 12.4127 \\
	5040 -2.484 \\
	5041 -13.914 \\
	5042 -6.6406 \\
	5043 -11.3793 \\
	5044 -3.0149 \\
	5045 0.1258 \\
	5046 3.9153 \\
	5047 2.9153 \\
	5048 2.9923 \\
	5049 -7.002 \\
	5050 8.2498 \\
	5051 9.2674 \\
	5052 -16.2357 \\
	5053 7.3767 \\
	5054 0.81122 \\
	5055 3.279 \\
	5056 -7.1415 \\
	5057 -7.5895 \\
	5058 -9.5544 \\
	5059 7.6578 \\
	5060 -14.6509 \\
	5061 -11.4411 \\
	5062 0.4361 \\
	5063 1.8537 \\
	5064 -2.9075 \\
	5065 -1.0479 \\
	5066 12.3286 \\
	5067 -8.9296 \\
	5068 -0.39108 \\
	5069 15.0185 \\
	5070 1.78 \\
	5071 -18.0268 \\
	5072 -4.5564 \\
	5073 6.3246 \\
	5074 1.6305 \\
	5075 16.4617 \\
	5076 -4.7821 \\
	5077 1.5118 \\
	5078 -8.2762 \\
	5079 -21.8135 \\
	5080 6.2446 \\
	5081 -10.9502 \\
	5082 2.9502 \\
	5083 10.861 \\
	5084 -7.5757 \\
	5085 7.3409 \\
	5086 2.2908 \\
	5087 0.27328 \\
	5088 7.3557 \\
	5089 -4.4409e-16 \\
	5090 7.2694 \\
	5091 9.245 \\
	5092 4.247 \\
	5093 -0.5569 \\
	5094 -3.7782 \\
	5095 6.5033 \\
	5096 1.3305 \\
	5097 6.1075 \\
	5098 2.2776 \\
	5099 -0.16471 \\
	5100 4.1166 \\
	5101 -7.953 \\
	5102 11.3388 \\
	5103 3.9795 \\
	5104 -4.334 \\
	5105 -1.2649 \\
	5106 -8.3053 \\
	5107 -6.4046 \\
	5108 -6.1696 \\
	5109 7.9564 \\
	5110 -5.5115 \\
	5111 -4.2523 \\
	5112 -2.3634 \\
	5113 3.3607 \\
	5114 20.8932 \\
	5115 -3.4284 \\
	5116 -4.284 \\
	5117 3.3023 \\
	5118 2.5882 \\
	5119 -3.9347 \\
	5120 -1.6376 \\
	5121 -3.7947 \\
	5122 -6.8569 \\
	5123 -5.6626 \\
	5124 -1.6532 \\
	5125 -5.0985 \\
	5126 -6.8453 \\
	5127 -2.2654 \\
	5128 -4.3296 \\
	5129 -4.4721 \\
	5130 7.0398 \\
	5131 12.1083 \\
	5132 12.3525 \\
	5133 -0.50236 \\
	5134 10.5375 \\
	5135 7.1712 \\
	5136 -3.6358 \\
	5137 6.3246 \\
	5138 -22.642 \\
	5139 0.34003 \\
	5140 8.0929 \\
	5141 -7.8368 \\
	5142 10.8299 \\
	5143 1.7205 \\
	5144 6.3712 \\
	5145 -0.37048 \\
	5146 0.73637 \\
	5147 1.9242 \\
	5148 2.0614 \\
	5149 0.59071 \\
	5150 -9.7371 \\
	5151 5.7814 \\
	5152 -9.4996 \\
	5153 -8.8544 \\
	5154 2.2396 \\
	5155 -0.73572 \\
	5156 -6.6419 \\
	5157 -1.226 \\
	5158 8.2271 \\
	5159 -17.9428 \\
	5160 -4.5581 \\
	5161 4.4721 \\
	5162 0.70801 \\
	5163 14.1011 \\
	5164 7.6432 \\
	5165 -1.0695 \\
	5166 -18.2119 \\
	5167 -13.2347 \\
	5168 9.7928 \\
	5169 3.7947 \\
	5170 -5.3442 \\
	5171 9.4563 \\
	5172 4.7586 \\
	5173 1.5122 \\
	5174 -1.8678 \\
	5175 -8.4723 \\
	5176 9.2665 \\
	5177 -2.1593 \\
	5178 1.7849 \\
	5179 3.8859 \\
	5180 -4.2791 \\
	5181 8.5706 \\
	5182 -0.95599 \\
	5183 1.9437 \\
	5184 4.6161 \\
	5185 5.0596 \\
	5186 -10.113 \\
	5187 3.644 \\
	5188 6.5243 \\
	5189 2.6921 \\
	5190 -3.1774 \\
	5191 -1.9742 \\
	5192 22.6396 \\
	5193 6.8485 \\
	5194 -3.4104 \\
	5195 0.71017 \\
	5196 5.416 \\
	5197 9.2173 \\
	5198 -3.865 \\
	5199 -5.1777 \\
	5200 0.94151 \\
	5201 -6.3246 \\
	5202 -0.86838 \\
	5203 7.8632 \\
	5204 -9.4111 \\
	5205 3.3542 \\
	5206 1.549 \\
	5207 -4.7245 \\
	5208 -0.9011 \\
	5209 -20.3285 \\
	5210 -3.6503 \\
	5211 -4.6838 \\
	5212 6.6473 \\
	5213 12.4089 \\
	5214 -2.1334 \\
	5215 19.7661 \\
	5216 2.5831 \\
	5217 -12.6491 \\
	5218 0.30487 \\
	5219 9.3598 \\
	5220 -6.4835 \\
	5221 -16.172 \\
	5222 -0.98305 \\
	5223 -0.40711 \\
	5224 6.9195 \\
	5225 3.2708 \\
	5226 0.36319 \\
	5227 -9.0289 \\
	5228 -10.3095 \\
	5229 8.5814 \\
	5230 3.7454 \\
	5231 12.0092 \\
	5232 9.8909 \\
	5233 -8.8544 \\
	5234 -1.3301 \\
	5235 1.0333 \\
	5236 -7.8343 \\
	5237 -2.5233 \\
	5238 -4.4333 \\
	5239 -6.7709 \\
	5240 5.3296 \\
	5241 -2.4399 \\
	5242 -10.7856 \\
	5243 6.2812 \\
	5244 2.008 \\
	5245 -12.4988 \\
	5246 8.4297 \\
	5247 2.4581 \\
	5248 -3.6025 \\
	5249 2.5298 \\
	5250 4.8972 \\
	5251 -5.5907 \\
	5252 -1.7774 \\
	5253 -0.31837 \\
	5254 -9.4839 \\
	5255 8.6112 \\
	5256 -12.9309 \\
	5257 -15.6393 \\
	5258 -3.8391 \\
	5259 -5.6476 \\
	5260 1.3711 \\
	5261 -4.2533 \\
	5262 2.1665 \\
	5263 -8.8161 \\
	5264 -5.145 \\
	5265 -2.2204e-16 \\
	5266 2.473 \\
	5267 3.8331 \\
	5268 -14.5078 \\
	5269 -4.4445 \\
	5270 -1.3545 \\
	5271 6.9723 \\
	5272 0.63267 \\
	5273 -2.4663 \\
	5274 17.8664 \\
	5275 -5.312 \\
	5276 -5.2357 \\
	5277 7.8714 \\
	5278 -5.3755 \\
	5279 5.2479 \\
	5280 7.2582 \\
	5281 -7.5895 \\
	5282 4.641 \\
	5283 12.8881 \\
	5284 4.7434 \\
	5285 -1.6875 \\
	5286 2.6743 \\
	5287 16.71 \\
	5288 10.6573 \\
	5289 0.46038 \\
	5290 -4.1929 \\
	5291 -7.2659 \\
	5292 -8.0975 \\
	5293 -3.1191 \\
	5294 5.4121 \\
	5295 0.98466 \\
	5296 -6.599 \\
	5297 -2.5298 \\
	5298 3.99 \\
	5299 -1.4452 \\
	5300 5.979 \\
	5301 -1.1391 \\
	5302 -6.65 \\
	5303 1.8219 \\
	5304 -1.7284 \\
	5305 10.0557 \\
	5306 0.6904 \\
	5307 13.5998 \\
	5308 19.6207 \\
	5309 -8.0884 \\
	5310 -3.7956 \\
	5311 -6.2337 \\
	5312 -4.36 \\
	5313 4.4409e-16 \\
	5314 -2.266 \\
	5315 9.8556 \\
	5316 -0.15552 \\
	5317 0.060043 \\
	5318 -4.6514 \\
	5319 1.2864 \\
	5320 -8.2594 \\
	5321 -14.0039 \\
	5322 -1.6547 \\
	5323 -5.801 \\
	5324 -5.16 \\
	5325 -5.1237 \\
	5326 16.1933 \\
	5327 24.3998 \\
	5328 0.22965 \\
	5329 -3.7947 \\
	5330 1.1201 \\
	5331 0.66751 \\
	5332 -5.8494 \\
	5333 -8.6044 \\
	5334 2.3507 \\
	5335 -5.1891 \\
	5336 -7.7761 \\
	5337 1.5718 \\
	5338 -5.2281 \\
	5339 9.3331 \\
	5340 12.1246 \\
	5341 4.3614 \\
	5342 -1.7878 \\
	5343 -5.173 \\
	5344 9.8948 \\
	5345 -2.5298 \\
	5346 5.5131 \\
	5347 15.2009 \\
	5348 2.165 \\
	5349 -6.3846 \\
	5350 -6.319 \\
	5351 5.9658 \\
	5352 -3.0856 \\
	5353 3.8846 \\
	5354 0.79851 \\
	5355 -7.5216 \\
	5356 2.4207 \\
	5357 7.1296 \\
	5358 -1.8399 \\
	5359 -20.5593 \\
	5360 -0.55448 \\
	5361 -1.2649 \\
	5362 -0.32378 \\
	5363 2.7178 \\
	5364 -13.9697 \\
	5365 2.2798 \\
	5366 -5.4902 \\
	5367 -1.0151 \\
	5368 12.8092 \\
	5369 -9.1613 \\
	5370 -10.1743 \\
	5371 10.9651 \\
	5372 6.3286 \\
	5373 1.2222 \\
	5374 3.64 \\
	5375 -4.7751 \\
	5376 8.957 \\
	5377 -1.2649 \\
	5378 -2.5725 \\
	5379 -0.058962 \\
	5380 0.45748 \\
	5381 -4.8025 \\
	5382 -1.8749 \\
	5383 -5.7047 \\
	5384 8.1266 \\
	5385 -0.83086 \\
	5386 -3.7671 \\
	5387 1.6784 \\
	5388 -0.091298 \\
	5389 11.4768 \\
	5390 -2.123 \\
	5391 8.1536 \\
	5392 4.8511 \\
	5393 -12.6491 \\
	5394 -2.209 \\
	5395 -6.7272 \\
	5396 -4.5855 \\
	5397 4.0243 \\
	5398 2.6063 \\
	5399 1.2039 \\
	5400 0.41605 \\
	5401 -6.8485 \\
	5402 -5.3677 \\
	5403 -1.0289 \\
	5404 -0.58631 \\
	5405 -3.927 \\
	5406 4.1115 \\
	5407 17.9294 \\
	5408 -2.3769 \\
	5409 -6.3246 \\
	5410 0.72114 \\
	5411 -3.0964 \\
	5412 -2.8183 \\
	5413 4.7126 \\
	5414 4.1003 \\
	5415 -19.5382 \\
	5416 -2.4464 \\
	5417 13.48 \\
	5418 -0.8166 \\
	5419 -4.1604 \\
	5420 -7.8153 \\
	5421 -11.2598 \\
	5422 -0.49941 \\
	5423 13.2274 \\
	5424 0.084384 \\
	5425 2.5298 \\
	5426 8.5264 \\
	5427 -6.3443 \\
	5428 1.9356 \\
	5429 13.7743 \\
	5430 2.3726 \\
	5431 -0.57097 \\
	5432 0.46554 \\
	5433 -3.2708 \\
	5434 10.545 \\
	5435 -5.5605 \\
	5436 -5.8668 \\
	5437 11.2995 \\
	5438 -3.6338 \\
	5439 0.47841 \\
	5440 0.13073 \\
	5441 2.5298 \\
	5442 1.1429 \\
	5443 1.4682 \\
	5444 -1.7414 \\
	5445 -11.7868 \\
	5446 -5.2937 \\
	5447 3.9317 \\
	5448 4.5815 \\
	5449 0.52394 \\
	5450 -10.9485 \\
	5451 -3.0098 \\
	5452 7.3126 \\
	5453 -14.5047 \\
	5454 7.0518 \\
	5455 13.214 \\
	5456 -1.8636 \\
	5457 1.2649 \\
	5458 -13.602 \\
	5459 8.4461 \\
	5460 8.7787 \\
	5461 -5.1649 \\
	5462 3.9911 \\
	5463 9.3468 \\
	5464 9.3994 \\
	5465 -1.4819 \\
	5466 6.2301 \\
	5467 0.60393 \\
	5468 3.4962 \\
	5469 3.6337 \\
	5470 -14.3582 \\
	5471 3.8347 \\
	5472 9.9832 \\
	5473 -2.5298 \\
	5474 -2.3849 \\
	5475 -1.9086 \\
	5476 2.5345 \\
	5477 6.5102 \\
	5478 4.2405 \\
	5479 -1.4232 \\
	5480 -13.0042 \\
	5481 -3.0538 \\
	5482 -7.0042 \\
	5483 -14.8733 \\
	5484 13.9814 \\
	5485 10.403 \\
	5486 5.746 \\
	5487 5.1255 \\
	5488 -4.7698 \\
	5489 -3.7947 \\
	5490 2.0353 \\
	5491 21.6638 \\
	5492 7.654 \\
	5493 -7.2672 \\
	5494 1.3026 \\
	5495 2.2758 \\
	5496 -9.6343 \\
	5497 -8.6374 \\
	5498 -4.9585 \\
	5499 -17.4501 \\
	5500 1.7846 \\
	5501 2.9978 \\
	5502 -3.5481 \\
	5503 -0.88779 \\
	5504 -8.1349 \\
	5505 4.4409e-16 \\
	5506 8.3148 \\
	5507 9.3537 \\
	5508 4.4551 \\
	5509 0.14309 \\
	5510 -9.5156 \\
	5511 6.2306 \\
	5512 3.8708 \\
	5513 -5.52 \\
	5514 -7.8358 \\
	5515 -3.8743 \\
	5516 -1.7176 \\
	5517 -7.4185 \\
	5518 -4.0038 \\
	5519 -0.84879 \\
	5520 1.1661 \\
	5521 4.4409e-16 \\
	5522 8.2474 \\
	5523 3.9046 \\
	5524 1.2975 \\
	5525 -1.7931 \\
	5526 -11.0171 \\
	5527 8.6316 \\
	5528 7.6596 \\
	5529 0.37048 \\
	5530 2.6537 \\
	5531 -15.0551 \\
	5532 -17.3608 \\
	5533 -3.5647 \\
	5534 1.8344 \\
	5535 12.2853 \\
	5536 7.9561 \\
	5537 -7.5895 \\
	5538 2.7425 \\
	5539 8.5066 \\
	5540 -3.9421 \\
	5541 -6.4676 \\
	5542 5.5424 \\
	5543 -3.8321 \\
	5544 -7.2084 \\
	5545 10.5797 \\
	5546 2.9366 \\
	5547 -5.3967 \\
	5548 -1.5755 \\
	5549 -2.1769 \\
	5550 -6.6477 \\
	5551 0.60029 \\
	5552 -3.0993 \\
	5553 5.0596 \\
	5554 6.7572 \\
	5555 -9.9839 \\
	5556 16.679 \\
	5557 -4.5314 \\
	5558 -9.0506 \\
	5559 13.834 \\
	5560 -0.65795 \\
	5561 2.1593 \\
	5562 -14.5652 \\
	5563 7.4854 \\
	5564 7.2239 \\
	5565 -9.6083 \\
	5566 13.4875 \\
	5567 -1.4835 \\
	5568 -4.6274 \\
	5569 -3.7947 \\
	5570 -2.7192 \\
	5571 5.4902 \\
	5572 -0.69529 \\
	5573 -7.0453 \\
	5574 1.8158 \\
	5575 5.1487 \\
	5576 23.832 \\
	5577 -1.4819 \\
	5578 6.602 \\
	5579 13.2616 \\
	5580 -5.7199 \\
	5581 4.719 \\
	5582 -10.6472 \\
	5583 -6.8174 \\
	5584 -2.636 \\
	5585 5.0596 \\
	5586 -2.8305 \\
	5587 -11.3029 \\
	5588 9.8243 \\
	5589 1.3842 \\
	5590 4.5901 \\
	5591 -0.35256 \\
	5592 -8.7868 \\
	5593 -4.9697 \\
	5594 -2.609 \\
	5595 15.1742 \\
	5596 -1.7347 \\
	5597 -17.1782 \\
	5598 7.6804 \\
	5599 -0.17205 \\
	5600 -14.8331 \\
	5601 -8.8544 \\
	5602 -10.9754 \\
	5603 -1.4163 \\
	5604 5.5493 \\
	5605 18.4295 \\
	5606 4.3655 \\
	5607 -9.628 \\
	5608 5.5003 \\
	5609 -8.6374 \\
	5610 2.3797 \\
	5611 0.59234 \\
	5612 -5.2037 \\
	5613 11.1637 \\
	5614 -2.7626 \\
	5615 1.37 \\
	5616 10.7865 \\
	5617 -4.4409e-16 \\
	5618 0.72088 \\
	5619 8.5312 \\
	5620 1.1259 \\
	5621 10 \\
	5622 1.4714 \\
	5623 -2.7576 \\
	5624 6.8762 \\
	5625 -22.8583 \\
	5626 -5.7474 \\
	5627 15.2065 \\
	5628 12.0329 \\
	5629 -6.2939 \\
	5630 -21.6922 \\
	5631 -1.97 \\
	5632 -5.5603 \\
	5633 1.2649 \\
	5634 4.3206 \\
	5635 0.6152 \\
	5636 6.7696 \\
	5637 -1.8801 \\
	5638 -7.0617 \\
	5639 -2.2419 \\
	5640 0.039673 \\
	5641 8.0498 \\
	5642 2.5959 \\
	5643 0.9555 \\
	5644 -1.7839 \\
	5645 0.63246 \\
	5646 14.2094 \\
	5647 12.0227 \\
	5648 -5.0219 \\
	5649 3.7947 \\
	5650 10.1355 \\
	5651 -3.2774 \\
	5652 6.7062 \\
	5653 -4.6371 \\
	5654 -2.6865 \\
	5655 0.40445 \\
	5656 -19.0666 \\
	5657 -3.9482 \\
	5658 7.5543 \\
	5659 12.67 \\
	5660 4.8286 \\
	5661 0.63246 \\
	5662 4.7407 \\
	5663 -7.3887 \\
	5664 -0.188 \\
	5665 3.7947 \\
	5666 -8.1394 \\
	5667 -6.0972 \\
	5668 -6.1634 \\
	5669 -5.1854 \\
	5670 2.9907 \\
	5671 -5.6307 \\
	5672 -2.5036 \\
	5673 -8.0498 \\
	5674 -9.4616 \\
	5675 0.044574 \\
	5676 -12.5722 \\
	5677 0.63246 \\
	5678 1.7191 \\
	5679 11.0711 \\
	5680 10.557 \\
	5681 -21.5035 \\
	5682 -10.9796 \\
	5683 7.7115 \\
	5684 13.3103 \\
	5685 -6.0062 \\
	5686 2.1244 \\
	5687 9.3841 \\
	5688 -8.4942 \\
	5689 1.4184 \\
	5690 -12.4322 \\
	5691 -7.5626 \\
	5692 5.3113 \\
	5693 0.63246 \\
	5694 10.4898 \\
	5695 7.6771 \\
	5696 -1.8483 \\
	5697 -5.0596 \\
	5698 5.2456 \\
	5699 3.3669 \\
	5700 -1.117 \\
	5701 10.6922 \\
	5702 -4.591 \\
	5703 -9.7551 \\
	5704 -8.4208 \\
	5705 4.6892 \\
	5706 -9.1653 \\
	5707 -13.7532 \\
	5708 -1.9815 \\
	5709 -23.0431 \\
	5710 -0.3204 \\
	5711 4.7837 \\
	5712 7.0861 \\
	5713 2.5298 \\
	5714 -22.4733 \\
	5715 1.7737 \\
	5716 7.1746 \\
	5717 0.48879 \\
	5718 -7.4138 \\
	5719 -2.4095 \\
	5720 10.3298 \\
	5721 0.15346 \\
	5722 -0.92875 \\
	5723 2.6582 \\
	5724 -0.31268 \\
	5725 -2.1403 \\
	5726 5.7278 \\
	5727 3.1718 \\
	5728 9.4312 \\
	5729 8.8818e-16 \\
	5730 -13.3344 \\
	5731 6.6528 \\
	5732 4.3639 \\
	5733 4.8835 \\
	5734 8.1728 \\
	5735 3.4523 \\
	5736 3.2388 \\
	5737 2.9003 \\
	5738 8.5003 \\
	5739 6.8675 \\
	5740 -2.9076 \\
	5741 -8.4525 \\
	5742 0.22063 \\
	5743 -0.22459 \\
	5744 -0.4261 \\
	5745 5.0596 \\
	5746 8.9117 \\
	5747 8.0111 \\
	5748 -6.9544 \\
	5749 -13.5347 \\
	5750 -12.2671 \\
	5751 4.0867 \\
	5752 10.4365 \\
	5753 -5.2131 \\
	5754 0.9098 \\
	5755 9.7212 \\
	5756 -4.5526 \\
	5757 -4.3114 \\
	5758 2.4475 \\
	5759 1.9543 \\
	5760 4.9696 \\
	5761 11.3842 \\
	5762 -0.74486 \\
	5763 8.9213 \\
	5764 -7.3828 \\
	5765 -9.6602 \\
	5766 1.5945 \\
	5767 4.3231 \\
	5768 10.2028 \\
	5769 -1.1115 \\
	5770 0.12211 \\
	5771 -3.655 \\
	5772 -0.19506 \\
	5773 1.6875 \\
	5774 7.5633 \\
	5775 14.2218 \\
	5776 -1.9992 \\
	5777 3.7947 \\
	5778 -2.12 \\
	5779 -12.3365 \\
	5780 -1.8328 \\
	5781 6.0422 \\
	5782 6.5255 \\
	5783 -3.718 \\
	5784 2.9019 \\
	5785 7.3089 \\
	5786 -5.7894 \\
	5787 -6.5437 \\
	5788 3.6226 \\
	5789 4.4445 \\
	5790 -0.54279 \\
	5791 4.1947 \\
	5792 4.2549 \\
	5793 8.8544 \\
	5794 10.0003 \\
	5795 -11.6411 \\
	5796 -8.5684 \\
	5797 6.2995 \\
	5798 9.3814 \\
	5799 7.5689 \\
	5800 -11.2512 \\
	5801 -6.478 \\
	5802 1.7074 \\
	5803 -9.949 \\
	5804 -5.9111 \\
	5805 0.31837 \\
	5806 4.863 \\
	5807 -0.22735 \\
	5808 -12.4686 \\
	5809 3.7947 \\
	5810 11.7019 \\
	5811 -7.7122 \\
	5812 9.8998 \\
	5813 4.908 \\
	5814 -23.4282 \\
	5815 -0.58462 \\
	5816 -3.6716 \\
	5817 -12.3685 \\
	5818 -6.8539 \\
	5819 -2.6207 \\
	5820 2.3443 \\
	5821 1.1391 \\
	5822 16.3776 \\
	5823 -10.5996 \\
	5824 -10.3033 \\
	5825 3.7947 \\
	5826 11.282 \\
	5827 -0.93083 \\
	5828 -7.4571 \\
	5829 -0.0089406 \\
	5830 -10.7159 \\
	5831 -9.0549 \\
	5832 8.0646 \\
	5833 1.6354 \\
	5834 -3.2177 \\
	5835 -0.97561 \\
	5836 -2.4573 \\
	5837 8.0334 \\
	5838 -0.178 \\
	5839 6.2708 \\
	5840 13.3428 \\
	5841 -1.2649 \\
	5842 -2.1738 \\
	5843 -17.1777 \\
	5844 8.2315 \\
	5845 13.3188 \\
	5846 -13.0233 \\
	5847 7.6893 \\
	5848 3.398 \\
	5849 -3.9482 \\
	5850 -7.6388 \\
	5851 6.7788 \\
	5852 15.1243 \\
	5853 -5.0407 \\
	5854 5.7536 \\
	5855 -3.524 \\
	5856 2.9342 \\
	5857 8.8544 \\
	5858 -11.2418 \\
	5859 5.3169 \\
	5860 6.8351 \\
	5861 0.22596 \\
	5862 2.9838 \\
	5863 -9.274 \\
	5864 -7.7273 \\
	5865 3.4242 \\
	5866 -9.5957 \\
	5867 -10.7492 \\
	5868 7.3444 \\
	5869 -2.4498 \\
	5870 -7.0719 \\
	5871 -5.2284 \\
	5872 -11.6153 \\
	5873 1.2649 \\
	5874 1.1175 \\
	5875 -10.5906 \\
	5876 -7.7276 \\
	5877 -5.9463 \\
	5878 1.5859 \\
	5879 4.5321 \\
	5880 1.0784 \\
	5881 1.4184 \\
	5882 -0.86604 \\
	5883 3.0299 \\
	5884 2.4411 \\
	5885 7.0466 \\
	5886 12.6423 \\
	5887 3.5295 \\
	5888 -1.4519 \\
	5889 -1.2649 \\
	5890 1.0986 \\
	5891 16.9714 \\
	5892 4.2542 \\
	5893 -5.0474 \\
	5894 6.1138 \\
	5895 -4.9699 \\
	5896 -0.51987 \\
	5897 3.2708 \\
	5898 13.2585 \\
	5899 11.7195 \\
	5900 -9.7371 \\
	5901 2.1284 \\
	5902 -1.9938 \\
	5903 3.1579 \\
	5904 1.2105 \\
	5905 -2.5298 \\
	5906 6.6516 \\
	5907 -0.98995 \\
	5908 8.2422 \\
	5909 -3.1224 \\
	5910 -7.9191 \\
	5911 5.9952 \\
	5912 2.0038 \\
	5913 -10.9502 \\
	5914 8.7242 \\
	5915 8.0217 \\
	5916 -17.7404 \\
	5917 8.202 \\
	5918 -2.7754 \\
	5919 -1.9536 \\
	5920 5.2558 \\
	5921 -11.3842 \\
	5922 7.3945 \\
	5923 0.52095 \\
	5924 -12.1461 \\
	5925 -7.3847 \\
	5926 5.1033 \\
	5927 -6.537 \\
	5928 -3.3762 \\
	5929 6.8485 \\
	5930 -4.6794 \\
	5931 5.0397 \\
	5932 -0.40953 \\
	5933 0.92537 \\
	5934 -15.8823 \\
	5935 -14.0289 \\
	5936 6.761 \\
	5937 -2.5298 \\
	5938 1.9907 \\
	5939 -3.8533 \\
	5940 3.4808 \\
	5941 -2.1543 \\
	5942 -2.1523 \\
	5943 8.0416 \\
	5944 -4.7299 \\
	5945 3.3607 \\
	5946 -7.9128 \\
	5947 -12.1318 \\
	5948 -2.47 \\
	5949 -8.7259 \\
	5950 3.0992 \\
	5951 15.3544 \\
	5952 9.8015 \\
	5953 11.3842 \\
	5954 -0.46317 \\
	5955 -6.8261 \\
	5956 1.3048 \\
	5957 -4.6318 \\
	5958 6.1513 \\
	5959 3.8905 \\
	5960 -2.8244 \\
	5961 7.4623 \\
	5962 -4.6765 \\
	5963 8.0263 \\
	5964 12.8766 \\
	5965 -5.5368 \\
	5966 7.3049 \\
	5967 1.8294 \\
	5968 5.1975 \\
	5969 5.0596 \\
	5970 -2.1103 \\
	5971 5.1154 \\
	5972 -8.4969 \\
	5973 -3.9285 \\
	5974 7.5244 \\
	5975 2.0043 \\
	5976 4.0922 \\
	5977 11.2571 \\
	5978 -2.7367 \\
	5979 -1.6251 \\
	5980 4.2529 \\
	5981 -7.3181 \\
	5982 1.1879 \\
	5983 -1.9186 \\
	5984 -3.342 \\
	5985 -1.2649 \\
	5986 7.2926 \\
	5987 8.2301 \\
	5988 -1.4298 \\
	5989 1.578 \\
	5990 -10.3703 \\
	5991 4.2463 \\
	5992 -0.50932 \\
	5993 -17.5816 \\
	5994 3.1726 \\
	5995 10.5629 \\
	5996 -9.0635 \\
	5997 -14.4848 \\
	5998 9.2348 \\
	5999 -3.7299 \\
	6000 -4.1014 \\
	6001 2.5298 \\
	6002 3.0373 \\
	6003 13.2852 \\
	6004 -7.9688 \\
	6005 4.4524 \\
	6006 8.8233 \\
	6007 6.6996 \\
	6008 12.3766 \\
	6009 -13.7869 \\
	6010 -19.9225 \\
	6011 -11.4704 \\
	6012 -3.6902 \\
	6013 -5.548 \\
	6014 -3.3304 \\
	6015 -7.9619 \\
	6016 -8.7935 \\
	6017 1.2649 \\
	6018 -3.1972 \\
	6019 -1.6645 \\
	6020 7.7061 \\
	6021 2.5288 \\
	6022 4.0235 \\
	6023 -2.4978 \\
	6024 -8.8679 \\
	6025 12.956 \\
	6026 18.6161 \\
	6027 18.9747 \\
	6028 1.618 \\
	6029 -7.8726 \\
	6030 -0.57073 \\
	6031 2.2251 \\
	6032 5.8189 \\
	6033 -5.0596 \\
	6034 3.6998 \\
	6035 -2.529 \\
	6036 2.4544 \\
	6037 9.5293 \\
	6038 -9.1348 \\
	6039 6.1324 \\
	6040 4.1257 \\
	6041 3.3607 \\
	6042 -2.8857 \\
	6043 -6.608 \\
	6044 3.2789 \\
	6045 -14.9143 \\
	6046 1.9954 \\
	6047 1.2649 \\
	6048 -17.5855 \\
	6049 -8.8544 \\
	6050 -3.6563 \\
	6051 5.0695 \\
	6052 -3.9944 \\
	6053 -2.3118 \\
	6054 4.259 \\
	6055 0.36634 \\
	6056 14.5311 \\
	6057 2.2229 \\
	6058 -17.6386 \\
	6059 -0.60058 \\
	6060 5.9702 \\
	6061 -5.3004 \\
	6062 -1.5419 \\
	6063 5.9605 \\
	6064 1.0708 \\
	6065 -10.1193 \\
	6066 -10.4041 \\
	6067 -0.44192 \\
	6068 1.8402 \\
	6069 -2.1569 \\
	6070 -3.8712 \\
	6071 3.5885 \\
	6072 5.9625 \\
	6073 -10.9502 \\
	6074 -3.9046 \\
	6075 2.9787 \\
	6076 -13.8137 \\
	6077 5.3189 \\
	6078 14.0918 \\
	6079 -1.8611 \\
	6080 0.0041022 \\
	6081 -17.7088 \\
	6082 -11.1161 \\
	6083 5.4053 \\
	6084 2.604 \\
	6085 -12.3047 \\
	6086 -3.9974 \\
	6087 7.1038 \\
	6088 -0.54243 \\
	6089 9.9023 \\
	6090 -4.0024 \\
	6091 -3.6568 \\
	6092 1.2809 \\
	6093 -4.5581 \\
	6094 14.8046 \\
	6095 -0.45613 \\
	6096 -4.6936 \\
	6097 6.3246 \\
	6098 3.5661 \\
	6099 1.1114 \\
	6100 -7.8409 \\
	6101 -13.7876 \\
	6102 -3.012 \\
	6103 2.4165 \\
	6104 -10.646 \\
	6105 4.0118 \\
	6106 5.5134 \\
	6107 4.3603 \\
	6108 10.1198 \\
	6109 3.0096 \\
	6110 3.4187 \\
	6111 -2.5311 \\
	6112 4.4084 \\
	6113 -5.0596 \\
	6114 -0.38199 \\
	6115 13.9528 \\
	6116 14.4837 \\
	6117 9.8648 \\
	6118 -7.5256 \\
	6119 5.2267 \\
	6120 -6.2408 \\
	6121 2.7468 \\
	6122 15.0543 \\
	6123 -9.6154 \\
	6124 5.7392 \\
	6125 -0.28456 \\
	6126 -6.5877 \\
	6127 -1.3805 \\
	6128 -5.5795 \\
	6129 3.7947 \\
	6130 -8.1141 \\
	6131 -15.8439 \\
	6132 -5.3329 \\
	6133 -6.5409 \\
	6134 -6.398 \\
	6135 -0.0020784 \\
	6136 5.3535 \\
	6137 11.1672 \\
	6138 1.5766 \\
	6139 -0.77334 \\
	6140 5.4721 \\
	6141 -0.69679 \\
	6142 -2.9177 \\
	6143 4.8019 \\
	6144 1.5338 \\
	6145 -7.5895 \\
	6146 5.7315 \\
	6147 1.8403 \\
	6148 0.71241 \\
	6149 2.24 \\
	6150 -3.6379 \\
	6151 -5.1251 \\
	6152 -8.132 \\
	6153 4.6892 \\
	6154 5.1504 \\
	6155 1.2154 \\
	6156 4.605 \\
	6157 4.0424 \\
	6158 -5.8042 \\
	6159 2.8449 \\
	6160 15.83 \\
	6161 -7.5895 \\
	6162 -7.554 \\
	6163 2.9327 \\
	6164 -2.5226 \\
	6165 4.2824 \\
	6166 5.0607 \\
	6167 -4.4085 \\
	6168 -2.8629 \\
	6169 -0.46038 \\
	6170 1.6878 \\
	6171 9.2519 \\
	6172 -5.5353 \\
	6173 -17.5713 \\
	6174 -4.2994 \\
	6175 -3.2495 \\
	6176 -7.4873 \\
	6177 2.5298 \\
	6178 -8.5656 \\
	6179 -2.6026 \\
	6180 12.3297 \\
	6181 -9.6124 \\
	6182 -8.6929 \\
	6183 -0.58403 \\
	6184 11.1765 \\
	6185 2.9003 \\
	6186 -14.9226 \\
	6187 -5.117 \\
	6188 -2.4408 \\
	6189 6.7806 \\
	6190 -3.9775 \\
	6191 8.9185 \\
	6192 7.3723 \\
	6193 -15.1789 \\
	6194 3.638 \\
	6195 5.4191 \\
	6196 11.1309 \\
	6197 -4.4995 \\
	6198 2.7136 \\
	6199 10.5517 \\
	6200 -8.257 \\
	6201 15.6393 \\
	6202 6.8109 \\
	6203 2.2391 \\
	6204 4.055 \\
	6205 -10.9605 \\
	6206 -3.6969 \\
	6207 6.231 \\
	6208 0.38402 \\
	6209 5.0596 \\
	6210 -2.4229 \\
	6211 -7.5604 \\
	6212 6.359 \\
	6213 8.6066 \\
	6214 8.6148 \\
	6215 -0.37714 \\
	6216 -0.4309 \\
	6217 11.3206 \\
	6218 -6.0542 \\
	6219 -0.064393 \\
	6220 4.5976 \\
	6221 5.1057 \\
	6222 -2.5421 \\
	6223 -8.0477 \\
	6224 -3.5722 \\
	6225 -20.2386 \\
	6226 0.24506 \\
	6227 7.7562 \\
	6228 6.5857 \\
	6229 3.35 \\
	6230 -12.3042 \\
	6231 1.0626 \\
	6232 3.783 \\
	6233 2.9003 \\
	6234 -3.6749 \\
	6235 -0.18504 \\
	6236 4.4333 \\
	6237 8.4073 \\
	6238 7.9353 \\
	6239 -7.9364 \\
	6240 1.9072 \\
	6241 -2.5298 \\
	6242 -4.9776 \\
	6243 -3.802 \\
	6244 -9.0051 \\
	6245 0.86162 \\
	6246 5.648 \\
	6247 5.7915 \\
	6248 -3.6744 \\
	6249 -1.2013 \\
	6250 0.2342 \\
	6251 0.13092 \\
	6252 8.0571 \\
	6253 -0.13593 \\
	6254 -2.1209 \\
	6255 2.0556 \\
	6256 1.8301 \\
	6257 5.0596 \\
	6258 -9.4353 \\
	6259 -11.1387 \\
	6260 6.5851 \\
	6261 -15.348 \\
	6262 -12.9981 \\
	6263 0.24442 \\
	6264 -14.4918 \\
	6265 4.6892 \\
	6266 13.8706 \\
	6267 -0.31554 \\
	6268 -1.0868 \\
	6269 14.451 \\
	6270 9.8629 \\
	6271 -7.9719 \\
	6272 -1.7577 \\
	6273 1.2649 \\
	6274 2.1706 \\
	6275 10.0535 \\
	6276 1.5052 \\
	6277 -2.1716 \\
	6278 8.8001 \\
	6279 9.1261 \\
	6280 0.90366 \\
	6281 -4.1017 \\
	6282 -0.38134 \\
	6283 -7.1477 \\
	6284 0.31804 \\
	6285 10.1331 \\
	6286 8.9546 \\
	6287 -5.9386 \\
	6288 -7.6894 \\
	6289 -5.0596 \\
	6290 -9.5769 \\
	6291 3.472 \\
	6292 -4.0335 \\
	6293 -2.3077 \\
	6294 4.8467 \\
	6295 4.1302 \\
	6296 9.417 \\
	6297 -2.0958 \\
	6298 -10.9298 \\
	6299 -1.5843 \\
	6300 4.4541 \\
	6301 -4.2952 \\
	6302 -8.0281 \\
	6303 -4.3801 \\
	6304 6.7664 \\
	6305 1.2649 \\
	6306 -6.4324 \\
	6307 -7.2337 \\
	6308 -4.8262 \\
	6309 0.16569 \\
	6310 -10.7514 \\
	6311 5.4619 \\
	6312 5.3274 \\
	6313 6.6315 \\
	6314 9.6973 \\
	6315 -6.4008 \\
	6316 4.3702 \\
	6317 0.38304 \\
	6318 12.2936 \\
	6319 8.8403 \\
	6320 -7.9935 \\
	6321 -5.0596 \\
	6322 6.1801 \\
	6323 14.7404 \\
	6324 -9.2362 \\
	6325 -3.2759 \\
	6326 0.47385 \\
	6327 -6.0691 \\
	6328 -2.5129 \\
	6329 12.2151 \\
	6330 7.1767 \\
	6331 -21.0783 \\
	6332 -4.7667 \\
	6333 -13.8103 \\
	6334 -4.3744 \\
	6335 14.1275 \\
	6336 -2.1229 \\
	6337 1.2649 \\
	6338 -2.7267 \\
	6339 -7.2524 \\
	6340 0.4036 \\
	6341 3.6547 \\
	6342 2.4566 \\
	6343 -3.6213 \\
	6344 1.944 \\
	6345 -5.0596 \\
	6346 -11.3825 \\
	6347 6.8758 \\
	6348 -0.98417 \\
	6349 1.3077 \\
	6350 -0.89665 \\
	6351 -11.4732 \\
	6352 -7.7434 \\
	6353 -10.1193 \\
	6354 -1.0642 \\
	6355 1.142 \\
	6356 -0.57503 \\
	6357 -6.53 \\
	6358 -11.5988 \\
	6359 13.701 \\
	6360 11.2899 \\
	6361 -3.0538 \\
	6362 6.4732 \\
	6363 -14.5227 \\
	6364 -4.4039 \\
	6365 7.6536 \\
	6366 -10.6747 \\
	6367 8.9543 \\
	6368 13.6131 \\
	6369 16.4438 \\
	6370 7.9641 \\
	6371 1.0971 \\
	6372 0.71033 \\
	6373 -18.1826 \\
	6374 -2.4626 \\
	6375 1.316 \\
	6376 4.219 \\
	6377 -5.0596 \\
	6378 -11.0761 \\
	6379 4.7138 \\
	6380 -5.8026 \\
	6381 -1.8316 \\
	6382 -1.5926 \\
	6383 7.3603 \\
	6384 7.4979 \\
	6385 0 \\
	6386 6.678 \\
	6387 -3.1899 \\
	6388 6.7632 \\
	6389 13.4684 \\
	6390 -2.338 \\
	6391 -5.2882 \\
	6392 10.2562 \\
	6393 0.52394 \\
	6394 4.2661 \\
	6395 16.1961 \\
	6396 -9.5542 \\
	6397 -4.5998 \\
	6398 -2.3829 \\
	6399 -5.8894 \\
	6400 2.724 \\
};

\end{axis}
\end{tikzpicture}}
\end{center}

\only<4>{More about quantization in lectures 5 and 8.}

\end{frame}

%
\begin{frame}{Example: noise and interference}

Noise and interfering signals are typically modeled as \textbf{random processes}

\begin{enumerate}
	\item What's the effect of the noise on the output? 
	\item How can we design the system to minimize the noise at the output?	
\end{enumerate}

\begin{figure}
	\centering
	\resizebox{\linewidth}{!}{\def\layersep{1.5cm}
\def\outsep{0.7cm}
\def\dy{1}

\begin{tikzpicture}[->, >=stealth, shorten >= 0pt, draw=black!50, node distance=\layersep, font=\sffamily]
    \tikzstyle{node}=[circle,fill=black,minimum size=2pt,inner sep=0pt]
    \tikzstyle{block}=[draw=black,rectangle,fill=none,minimum size=1.5cm, inner sep=0pt]
    \tikzstyle{summer}=[draw=black,circle,fill=none,minimum size=1cm, inner sep=0pt]
    \tikzstyle{annot} = []

	\node[node] (sc) at (0, -\dy cm) {};
	\node[summer] (add) at (1*\layersep, -\dy cm) {\Large $+$};
	\node[node, below=\dy cm of add] (nc) {};
    \node[block] (DSP) at ($(add.east) + (\layersep, 0)$) {System};
	\coordinate (yc) at ($(DSP.center) + (\layersep, 0)$) {};
	
	%\coordinate (mid1) at ($(ADC.east)!0.5!(DSP.west)$) {};
	%\coordinate (mid2) at ($(DSP.east)!0.5!(DAC.west)$) {};
		
    \path (sc) edge (add);
    \path (add) edge (DSP);
    \path (nc) edge (add);
    \path (DSP) edge (yc);
    
    \node[left = 0mm of sc, text width = 1cm, align=center] {Signal};
    \node[right = 0mm of yc, text width = 1cm, align=center] {Output}; 
    \node[below = 0mm of nc, text width = 2cm, align=center] {Noise or \\ interference};
    

\end{tikzpicture}}
\end{figure} 

\end{frame}

\section{Random processes}
\begin{frame}{Random processes}

\begin{block}{Definition}
	A random process (or \textit{stochastic process}) is an indexed set of random variables $x_n$, which are distributed according to some probability distribution $p_{x_n}(x)$
\end{block}	

\begin{block}{Examples}
\begin{columns}
	\begin{column}{0.5\textwidth}
		Consecutive coin tosses
	\end{column}
	\begin{column}{0.5\textwidth}  %%<--- here
		Random bit stream
	\end{column}
\end{columns}

\begin{columns}
	\begin{column}{0.5\textwidth}
		\begin{tikzpicture}[draw=black!50, node distance=1cm]
		\tikzstyle{block}=[draw=none,rectangle,fill=none,minimum size=1.5cm, inner sep=0pt]
		\node[block] (C1) {\resizebox{1.5cm}{!}{\includegraphics{figs/US_One_Cent_Obv.png}}};
		\node[block,right of=C1] (C2) {\resizebox{1.5cm}{!}{\includegraphics{figs/US_One_Cent_Rev.png}}};
		\node[block,right of=C2] (C3) {\resizebox{1.5cm}{!}{\includegraphics{figs/US_One_Cent_Rev.png}}};
		\node[block,right of=C3] (C4) {\resizebox{1.5cm}{!}{\includegraphics{figs/US_One_Cent_Obv.png}}};
		\node[block,right of=C4] (C5) {\Large$~~~\cdots$};
		\end{tikzpicture}
	\end{column}
	\begin{column}{0.5\textwidth}  %%<--- here
		\resizebox{0.8\linewidth}{!}{\begin{tikzpicture} 
\begin{axis}[
axis lines*=middle,
enlargelimits = true,
clip=false,
%scale only axis,
width=\textwidth,
height=0.7\textwidth,
ymin=0,
ymax=1.1,
xmin=-5,
xmax=5,
axis line style={->,>=stealth},
xlabel={$n$},
ylabel={bits},
yticklabel style = {yshift=0.25cm},
xticklabel style = {yshift=-0.1cm},
every axis x label/.style={
    at={(ticklabel* cs:1)},
    anchor=north,
},
every axis y label/.style={
    at={(ticklabel* cs:1)},
    xshift=0.1cm,
    anchor=south,
},
%xtick=\empty,
ytick={1},
xtick=\empty,
%xtick={-3.14, -1, 1, 3.14},
%xticklabels={$-\pi$, $-\omega_c$, $\omega_c$, $\pi$},
%xmajorgrids,
%ymajorgrids,
every outer y axis line/.append style={white!15!black},
every y tick label/.append style={font=\color{white!15!black}},
legend style={draw=white!15!black,fill=white,legend cell align=left}]
\pgfmathsetseed{99}
\addplot[ycomb, mark=*, fill=white, mark options={scale=1, fill=white}, line width=1.5pt, domain=-5:5, samples=11] {0.5*(sign(rand)+1)};
\end{axis}
\end{tikzpicture}
}
	\end{column}
\end{columns}
\end{block}
\end{frame}

\begin{frame}{Random processes}
A random process can be viewed as a function $X(n, \chi)$ of two variables, time $n$ and the outcome of the underlying random experiment $\chi$. 
\begin{itemize}
	\pause\item For fixed $n$, $X(n, \chi)$ is a random variable
	
	In the example of the fair coin tossing, 
	\begin{equation*}
	X(n=1, \chi) = \begin{cases}
	\mathrm{H},~\text{with probability}~0.5 \\
	\mathrm{T},~\text{with probability}~0.5
	\end{cases}
	\end{equation*}
	\pause\item For fixed $\chi$, $X(n, \chi)$ is a deterministic function of $n$ called a \textbf{sample function} or \textbf{sample sequence}
	
	\vspace{3mm}
	\centering
	\begin{tikzpicture}[draw=black!50, node distance=1cm]
		\tikzstyle{block}=[draw=none,rectangle,fill=none,minimum size=1.5cm, inner sep=0pt]
		\node[block] (C1) {\resizebox{1.5cm}{!}{\includegraphics{figs/US_One_Cent_Obv.png}}};
		\node[block,right of=C1] (C2) {\resizebox{1.5cm}{!}{\includegraphics{figs/US_One_Cent_Rev.png}}};
		\node[block,right of=C2] (C3) {\resizebox{1.5cm}{!}{\includegraphics{figs/US_One_Cent_Rev.png}}};
		\node[block,right of=C3] (C4) {\resizebox{1.5cm}{!}{\includegraphics{figs/US_One_Cent_Obv.png}}};
		\node[block,right of=C4] (C5) {\Large$~~~\cdots$};
		\node[below of=C1, scale=0.7] (res1) {$X(1, \chi) = \mathrm{H}$};
		\node[below of=C4, scale=0.7] (res4) {$X(4, \chi) = \mathrm{H}$};
		\node[scale=0.7] at ($(res1.east)!0.5!(res4.west)$) {$\cdots$};
	\end{tikzpicture}
\end{itemize}
\end{frame}

\begin{frame}{Ensemble of sample sequences}
The ensemble of sample sequences is a collection af all possible sequences generated by a random process.

\centering
\resizebox{0.9\linewidth}{!}{\def\W{6cm}
\def\H{2cm}
\begin{tikzpicture}[draw=black!50, node distance=0.3cm]
\pgfmathsetseed{99}
\tikzstyle{block}=[draw=none,rectangle,fill=none,inner sep=0pt]

\node[block] (s1) {\resizebox{0.1\linewidth}{!}{\begin{tikzpicture}
\begin{axis}[
	axis lines*=middle,
	enlargelimits = false,
	clip=false,
	scale only axis,
	hide y axis,
	width=0.5\textwidth,
	height=0.15\textwidth,
	ymin=-1.3,
	ymax=1.3,
	xmin=-11,
	xmax=11,
	axis line style={->,>=stealth},
	xlabel={\small $n$},
	every axis x label/.style={
		at={(ticklabel* cs:1)},
		xshift=0.2cm,
		anchor=north,
	},
	%xtick=\empty,
	ytick=\empty,
	xtick=\empty,
	%xtick={-3.14, -1, 1, 3.14},
	%xticklabels={$-\pi$, $-\omega_c$, $\omega_c$, $\pi$},
	%xmajorgrids,
	%ymajorgrids,
	every outer y axis line/.append style={white!15!black},
	every y tick label/.append style={font=\color{white!15!black}},
	legend style={draw=white!15!black,fill=white,legend cell align=left}]
	\addplot[ycomb, mark=*, fill=white, mark options={scale=0.75, fill=white}, line width=1pt, domain=-10:10, samples=21] {rand};
\end{axis}
\end{tikzpicture}
}};
\node[left=0cm of s1,scale=0.15] {$X(n, \chi_1)$};

\onslide<2->{
	\node[block, below of=s1] (s2) {\resizebox{0.1\linewidth}{!}{\begin{tikzpicture}
\begin{axis}[
	axis lines*=middle,
	enlargelimits = false,
	clip=false,
	scale only axis,
	hide y axis,
	width=0.5\textwidth,
	height=0.15\textwidth,
	ymin=-1.3,
	ymax=1.3,
	xmin=-11,
	xmax=11,
	axis line style={->,>=stealth},
	xlabel={\small $n$},
	every axis x label/.style={
		at={(ticklabel* cs:1)},
		xshift=0.2cm,
		anchor=north,
	},
	%xtick=\empty,
	ytick=\empty,
	xtick=\empty,
	%xtick={-3.14, -1, 1, 3.14},
	%xticklabels={$-\pi$, $-\omega_c$, $\omega_c$, $\pi$},
	%xmajorgrids,
	%ymajorgrids,
	every outer y axis line/.append style={white!15!black},
	every y tick label/.append style={font=\color{white!15!black}},
	legend style={draw=white!15!black,fill=white,legend cell align=left}]
	\addplot[ycomb, mark=*, fill=white, mark options={scale=0.75, fill=white}, line width=1pt, domain=-10:10, samples=21] {rand};
\end{axis}
\end{tikzpicture}
}};
	\node[left=0cm of s2,scale=0.15] {$X(n, \chi_2)$};
}

\onslide<3->{
	\node[block, below of=s2] (s3) {\resizebox{0.1\linewidth}{!}{\begin{tikzpicture}
\begin{axis}[
	axis lines*=middle,
	enlargelimits = false,
	clip=false,
	scale only axis,
	hide y axis,
	width=0.5\textwidth,
	height=0.15\textwidth,
	ymin=-1.3,
	ymax=1.3,
	xmin=-11,
	xmax=11,
	axis line style={->,>=stealth},
	xlabel={\small $n$},
	every axis x label/.style={
		at={(ticklabel* cs:1)},
		xshift=0.2cm,
		anchor=north,
	},
	%xtick=\empty,
	ytick=\empty,
	xtick=\empty,
	%xtick={-3.14, -1, 1, 3.14},
	%xticklabels={$-\pi$, $-\omega_c$, $\omega_c$, $\pi$},
	%xmajorgrids,
	%ymajorgrids,
	every outer y axis line/.append style={white!15!black},
	every y tick label/.append style={font=\color{white!15!black}},
	legend style={draw=white!15!black,fill=white,legend cell align=left}]
	\addplot[ycomb, mark=*, fill=white, mark options={scale=0.75, fill=white}, line width=1pt, domain=-10:10, samples=21] {rand};
\end{axis}
\end{tikzpicture}
}};
	\node[left=0cm of s3,scale=0.15] {$X(n, \chi_3)$};
	\node[below=0cm of s3, scale=0.2] (dots) {\Large $\vdots$};
}

\onslide<4->{
\draw[red, very thin] ($(s1.north) + (-0.02,0)$) rectangle ($(dots.south) + (0.03,0)$);
\node [scale=0.15] (dist) at ($(s1.north) + (0.3cm, 0.05cm)$) {$x_n\sim p_{x_n}(x_n)$};
\draw [-{Latex[length=0.2mm,width=0.2mm]}, very thin, red,scale=0.1] (s1.north) to[out=90, in=180] (dist.west) (dist);
}

\onslide<5->{
\draw[red, very thin] ($(s1.north) + (-0.42,0)$) rectangle ($(dots.south) + (-0.38,0)$);
\node [scale=0.15] (dist2) at ($(s1.north) + (-0.12cm, 0.1cm)$) {$x_m\sim p_{x_m}(x_m)$};
\draw [-{Latex[length=0.2mm,width=0.2mm]}, very thin, red,scale=0.1] ($(s1.north) + (-4,0)$) to[out=90, in=180] (dist2.west);
}
\end{tikzpicture}

}

\end{frame}

\begin{frame}{Averages of a random variable}

\textbf{Mean or expected value}
\begin{equation*}
	\mu_{x_n} = \E(x_n) = \int_{-\infty}^{\infty}xp_{x_n}(x)dx
\end{equation*}

\textbf{Second moment or average power}
\begin{equation*} 
	\E(|x_n|^2) = \int_{-\infty}^{\infty}|x|^2p_{x_n}(x)dx
\end{equation*}

\textbf{Variance}
\begin{align*}
\sigma^2_{x_n} &= \E(|x_n-\mu_{x_n}|^2) = \int_{-\infty}^{\infty}|x|^2p_{x_n}(x)dx \\
&= \E(|x_n|^2) - \mu_{x_n}^2
\end{align*}

The integrals should be replaced by sums when the random variable is discrete.

\end{frame}

\begin{frame}{Joint averages of random variables}
Expected value of a function of two random variables
\begin{equation*}
\E(g(x_n,y_m)) = \int_{-\infty}^{\infty}\int_{-\infty}^{\infty}g(x, y)p_{x_n, y_n}(x, y)dxdy
\end{equation*}

Two random variables are \textbf{uncorrelated} if
\begin{equation*}
\E(x_ny_m) = \E(x_n)\E(y_m)
\end{equation*}

Two random variables are \textbf{statistically independent} if
\begin{equation*}
p_{x_n, y_n}(x,y) = p_{x_n}(x)p_{y_n}(y)
\end{equation*}

Independent random variables are also uncorrelated, but not all uncorrelated random variables are independent.

\textbf{Exception:} uncorrelated Gaussian random variables are always independent.
\end{frame}

\begin{frame}{Correlation functions}

\textbf{Autocorrelation}

\begin{equation*}
\phi_{xx}[n, m] = \E(x_nx_m^*) 
\end{equation*}

\textbf{Autocovariance}

\begin{equation*}
\gamma_{xx}[n, m] = \E((x_n-\mu_{x_n})(x_m - \mu_{x_m})^*) 
\end{equation*}

\textbf{Cross-correlation}

\begin{equation*}
\phi_{xy}[n, m] = \E(x_ny_m^*) 
\end{equation*}

\textbf{Cross-covariance}

\begin{equation*}
\gamma_{xy}[n, m] = \E((x_n-\mu_{x_n})(y_m - \mu_{y_m})^*) 
\end{equation*}
\end{frame}

\begin{frame}{Example: Bernoulli random process}
	
	\begin{itemize}
		\item A Bernoulli random process is a sequence of binary random variables $\{x_n \sim \mathcal{B}(\rho)\}$. Canonically, 
		\begin{columns}
			\begin{column}{0.5\linewidth}
				\begin{equation*}
				x_n = \begin{cases}
				1,~\text{with probability}~\rho, \\
				0,~\text{with probability}~1-\rho,
				\end{cases}
				\end{equation*}
			\end{column}
			\begin{column}{0.5\linewidth}
				\begin{equation*}
				p_{x_n}(x) = \begin{cases}
				\rho, &x = 1 \\
				1-\rho, &x = 0 \\
				0, &\text{otherwise}
				\end{cases}
				\end{equation*}
			\end{column}
		\end{columns}
		\item A Bernoulli process is \textbf{independent and identically distributed (IID)}. That is, each $x_n$ is picked independently from the same distribution $\mathcal{B}(\rho)$.
	\end{itemize}
	From this we can conclude:
	\begin{align*} 
	\mu = 1\cdot\rho + 0\cdot(1-\rho) = \rho \\
	\E(x_n^2) = 1^2\cdot\rho + 0^2\cdot(1-\rho) = \rho \\
	\sigma^2 = \E(x_n^2) - \mu^2 = \rho(1 - \rho)
	\end{align*}
	\vspace{-0.5cm}
	\begin{equation*}\tag{From IID. assumption}
	\phi_{xx}[m]=\E(x_{n+m}x_n) = \rho\delta[m]
	\end{equation*}
	
\end{frame}

\begin{frame}{Example: Uniform random process}
	
	\begin{itemize}
		\item A uniform random process is a sequence of uniform random variables $\{x_n \sim \mathcal{U}[a, b]\}$.
		\begin{equation*}
		p_{x_n}(x) = \begin{cases}
		\displaystyle\frac{1}{b-a} & a\leq x\leq b \\
		0, &\text{otherwise}
		\end{cases}
		\end{equation*}
	\end{itemize}
	From this we can conclude:
	\begin{align*} 
	\mu = \int_{a}^{b} \frac{x}{b-a}dx = \frac{b+a}{2} \\
	\E(x_n^2) = \int_{a}^{b} \frac{x^2}{b-a}dx = \frac{b^2+ab+a^2}{3} \\
	\sigma^2 = \E(x_n^2) - \mu^2 = \frac{(b-a)^2}{12}
	\end{align*}
	\vspace{-0.5cm}
	\begin{equation*}\tag{Assuming IID}
	\phi_{xx}[m]=\E(x_{n+m}x_n) = \E(x_n^2)\delta[m]
	\end{equation*}
\end{frame}

\begin{frame}{Example: Gaussian random process}
	
	\begin{itemize}
		\item A Gaussian random process is a sequence of Gaussian random variables $\{x_n \sim \mathcal{N}(\mu, \sigma^2)\}$.
		\begin{equation*}
		p_{x_n}(x) = \frac{1}{\sqrt{2\pi\sigma^2}}\exp\bigg(-\frac{(x-\mu)^2}{2\sigma^2}\bigg)
		\end{equation*}

	\end{itemize}
	From this we can conclude:
	\begin{align*} 
	\E(x_n^2) = \sigma^2 + \mu^2 \\
	\end{align*}
	\vspace{-0.5cm}
	\begin{equation*}\tag{Assuming IID}
	\phi_{xx}[m]=\E(x_{n+m}x_n) = \E(x_n^2)\delta[m]
	\end{equation*}
	
\end{frame}

%%%%%%%%%%%%%%%
\subsection{Stationary random processes}
\begin{frame}{Stationary random processes}
\begin{itemize}
	\item Stationarity refers to \textbf{time invariance} of some, or all, of the statistics of a random process, such as mean, autocorrelation, joint distributions, etc
	\item A random process is \textbf{strict-sense stationary (SSS)}, if its finite-order distributions do not change over time. For the first-order distributions, this means $p_{x_m}(x_m) = p_{x_n}(x_n),~\forall n, m$.
\end{itemize}
 
\centering
\resizebox{0.7\linewidth}{!}{\input{figs/ensemble_stationary.tex}}
\end{frame}

\begin{frame}{Stationary random processes}

All statistics of a SSS random process are time invariant. 

As a result, the mean, average power, and variance are constant with $n$:
\begin{align*}
\mu &= \E(x_n) \\
\sigma^2 &= \E(|x_n|^2) - \mu^2
\end{align*}

And the autocorrelation only depend on the time difference $m$:
\begin{align*}
\phi_{xx}[m] &= \phi_{xx}[n+m, n] = \E(x_{n+m}x_n^*) \tag{autocorrelation}
\end{align*}

\textbf{Question:} What is an example of SSS random process?

\end{frame}

\begin{frame}{Wide-sense stationary (WSS) random processes}

\begin{itemize}
	\item Strict sense stationarity is a strong condition that is hard to verify in practice.
	\item A weaker (and more useful) condition is \textbf{wide-sense stationarity}
	\item A random process is \textbf{wide-sense stationary (WSS)} if its mean and autocorrelation function are \textbf{time invariant}. 
	\begin{align*}
	\mu &= \E(x_n) \\
	\phi_{xx}[m] &= \phi_{xx}[n+m, n] = \E(x_{n+m}x_n^*)
	\end{align*}
	The mean is constant, and the autocorrelation function only depends on the time difference $m$.
	\item SSS implies WSS, but WSS does not mean SSS. \textbf{Exception:} WSS Gaussian random processes are also SSS.	
\end{itemize}
\end{frame}

%
\begin{frame}{Autocorrelation function of WSS processes}

The autocorrelation function $\phi_{xx}[m]$ of a WSS process $x[n]$ has the following properties

\begin{enumerate}
	\item $\phi_{xx}[m]$ is \textbf{real and even}, i.e., $\phi_{xx}[m] = \phi_{xx}[-m]$
	\item The DTFT of $\phi_{xx}[m]$ must be \textbf{non-negative at all frequencies}
	\begin{equation}
	\mathcal{F}\{\phi_{xx}[m]\} \geq 0, \forall~\omega\in[0, 2\pi]
	\end{equation}
	$\mathcal{F}\{\cdot\}$ denotes the DTFT.	
\end{enumerate}

	Properties 1 and 2 are \textbf{necessary and sufficient} for a function to be an autocorrelation function of a WSS process.
\end{frame}

%
\begin{frame}
More properties
\begin{enumerate}\setcounter{enumi}{2}
	\item $|\phi_{xx}[m]| \leq \phi_{xx}[0] = \E(|x[n]|^2) = \text{average power of}~x[n]$
	
\textit{Proof:}
\begin{align*}
\phi_{xx}^2[m] &= [\E(x[m+n]x[n])]^2 \\
&\leq \E(|x[m+n]|^2)\E(|x[n]|^2) \tag{by Schwarz inequality} \\
&=\phi_{xx}^2[m] \tag{by stationarity}
\end{align*}
	\item If $\phi_{xx}^2[T] = \phi_{xx}^2[0]$ for some $T\neq 0$, then $\phi_{xx}^2[m]$ is periodic with period $T$.
\end{enumerate}
\end{frame}

%
\begin{frame}{Which functions can be $\phi_{xx}[m]$ of a WSS process?}
\begin{enumerate}
	\begin{columns}
		\begin{column}{0.33\linewidth}
			\item 
			\begin{center}
				\resizebox{\linewidth}{!}{\input{figs/right_sided_exp_curve.tex}}
			\end{center}	
		\end{column}
		\begin{column}{0.33\linewidth}
			\item 
			\begin{center}
			\resizebox{\linewidth}{!}{\input{figs/two_sided_exp_curve.tex}}
			\end{center}
		\end{column}
		\begin{column}{0.33\linewidth}
			\item 
			\begin{center}
			\resizebox{\linewidth}{!}{\begin{tikzpicture} 
\begin{axis}[
axis lines*=middle,
enlargelimits = true,
ymin=-0.5,
ymax=1.2,
xmin=-5,
xmax=5,
axis line style={->,>=stealth},
xlabel={\Huge $n$},
ylabel={\Huge $\mathrm{sinc}[n]$},
yticklabel style = {yshift=0.2cm},
xticklabel style = {yshift=-0.1cm},
every axis x label/.style={
    at={(ticklabel* cs:1)},
    anchor=north,
},
every axis y label/.style={
    at={(ticklabel* cs:1)},
    anchor=south,
},
%xtick=\empty,
ytick={1},
xtick=\empty,
%xtick={-3.14, -1, 1, 3.14},
%xticklabels={$-\pi$, $-\omega_c$, $\omega_c$, $\pi$},
%xmajorgrids,
%ymajorgrids,
every outer y axis line/.append style={white!15!black},
every y tick label/.append style={font=\color{white!15!black}},
legend style={draw=white!15!black,fill=white,legend cell align=left}]

\addplot[ycomb, mark=*, fill=white, mark options={scale=1.5, fill=white}, line width=1.5pt, domain=-5:5, samples=11] {sin(2*deg(x))/(2*pi*x) + (x == 0)};
\addplot[smooth, black!20, line width=1pt, domain=-5:5, samples=11] {sin(2*deg(x))/(2*pi*x) + (x == 0)};
\end{axis}
\end{tikzpicture}
}
			\end{center}
		\end{column}
	
	\end{columns}
	\begin{columns}
	\begin{column}{0.33\linewidth}
		\item 
		\begin{center}
			\resizebox{\linewidth}{!}{\begin{tikzpicture} 
\begin{axis}[
axis lines*=middle,
enlargelimits = true,
ymin=-0.5,
ymax=1.2,
xmin=-5,
xmax=5,
axis line style={->,>=stealth},
xlabel={\Huge $n$},
%ylabel={\Huge $\mathrm{sinc}[n]$},
yticklabel style = {yshift=0.2cm},
xticklabel style = {yshift=-0.1cm},
every axis x label/.style={
	at={(ticklabel* cs:1)},
	anchor=north,
},
every axis y label/.style={
	at={(ticklabel* cs:1)},
	anchor=south,
},
%xtick=\empty,
ytick=\empty,
xtick=\empty,
%xtick={-3.14, -1, 1, 3.14},
%xticklabels={$-\pi$, $-\omega_c$, $\omega_c$, $\pi$},
%xmajorgrids,
%ymajorgrids,
every outer y axis line/.append style={white!15!black},
every y tick label/.append style={font=\color{white!15!black}},
legend style={draw=white!15!black,fill=white,legend cell align=left}]

\addplot[ycomb, mark=*, fill=white, mark options={scale=1.5, fill=white}, line width=1.5pt] coordinates {(-5, 0.1) (-4, 0.4) (-3, 0.8) (-2, 1.2) (-1, 1.05) (0, 0.9) (5, 0.1) (4, 0.4) (3, 0.8) (2, 1.2) (1, 1.05)};

\addplot[smooth, black!20, line width=1pt] coordinates {(-5, 0.1) (-4, 0.4) (-3, 0.8) (-2, 1.2) (-1, 1.05) (0, 0.9) (1, 1.1)  (2, 1.2) (3, 0.8) (4, 0.4) (5, 0.1)};
\end{axis}
\end{tikzpicture}
}
		\end{center}		
	\end{column}
	\begin{column}{0.33\linewidth}
		\item 
		\begin{center}
		\resizebox{\linewidth}{!}{\input{figs/saw.tex}}
		\end{center}
	\end{column}
		\begin{column}{0.33\linewidth}
		\item 
		\begin{center}
		\resizebox{\linewidth}{!}{\input{figs/constant.tex}}
		\end{center}
	\end{column}
\end{columns}
\end{enumerate}
\end{frame}

\subsection{Ergodic random processes}
%
\begin{frame}{Time averages}

\begin{itemize}
	\item So far we have focused on probability averages $\E(\cdot)$
	\item We can also define time averages $\langle\cdot\rangle$
\end{itemize}

\begin{equation*}
\langle x_n \rangle = \lim_{L\to\infty}\frac{1}{2L + 1}\sum_{n=-L}^L x_n
\end{equation*}

\begin{equation*}
\langle x_{n+m}x_n^* \rangle = \lim_{L\to\infty}\frac{1}{2L + 1}\sum_{n=-L}^L x_{n+m}x_n^*
\end{equation*}

\centering
\resizebox{0.8\linewidth}{!}{\begin{tikzpicture}
\begin{axis}[
	axis lines*=middle,
	enlargelimits = false,
	clip=false,
	scale only axis,
	hide y axis,
	width=0.5\textwidth,
	height=0.15\textwidth,
	ymin=-1.3,
	ymax=1.3,
	xmin=-11,
	xmax=11,
	axis line style={->,>=stealth},
	xlabel={\small $n$},
	every axis x label/.style={
		at={(ticklabel* cs:1)},
		xshift=0.2cm,
		anchor=north,
	},
	%xtick=\empty,
	ytick=\empty,
	xtick=\empty,
	%xtick={-3.14, -1, 1, 3.14},
	%xticklabels={$-\pi$, $-\omega_c$, $\omega_c$, $\pi$},
	%xmajorgrids,
	%ymajorgrids,
	every outer y axis line/.append style={white!15!black},
	every y tick label/.append style={font=\color{white!15!black}},
	legend style={draw=white!15!black,fill=white,legend cell align=left}]
	\addplot[ycomb, mark=*, fill=white, mark options={scale=0.75, fill=white}, line width=1pt, domain=-10:10, samples=21] {rand};
\end{axis}
\end{tikzpicture}
}

\end{frame}

\begin{frame}{Ergodic random processes}

A random process is \textbf{ergodic} if its time averages are equal to its probability averages:

\begin{equation*}
\langle x_n \rangle = \lim_{L\to\infty}\frac{1}{2L + 1}\sum_{n=-L}^L x_n = \E(x_n) = \mu
\end{equation*}

\begin{equation*}
\langle x_{n+m}x_n^* \rangle = \lim_{L\to\infty}\frac{1}{2L + 1}\sum_{n=-L}^L x_{n+m}x_n^* = \E(x_{n+m}x_n^*) = \phi_{xx}[m]
\end{equation*}

\begin{itemize}
	\item In practice, we don't have an ensemble of sample functions that we can use to estimate the mean and autocorrelation function.
	\item We generally have only one sample function.
	\item With the \textbf{ergodic assumption}, we can estimate probability averages from a single sample function
\end{itemize}

\end{frame}

%
\subsection{LTI systems with random input}
\begin{frame}{LTI system with a random input}
\begin{center}
\resizebox{\linewidth}{!}{\def\layersep{1.5cm}
\def\outsep{0.7cm}
\def\dy{1.25}

\begin{tikzpicture}[draw=black!50, node distance=\layersep, font=\sffamily]
    \tikzstyle{node}=[circle,fill=black,minimum size=2pt,inner sep=0pt]
    \tikzstyle{block}=[draw=black,rectangle,fill=none,minimum width=3cm, minimum height=2cm, inner sep=0pt]
    \tikzstyle{annot} = []

	\node[node] (xc) at (0, -\dy cm) {};
    \node[block, text width = 3cm, align= center] (DSP) at (2*\layersep, -\dy cm) {LTI System \\ $h[n]\leftrightarrow H(e^{j\omega})$};
	\coordinate (yc) at (4*\layersep, -\dy cm) {};
		
    \path[->, >=stealth, shorten >= 0pt] (xc) edge (DSP);
    \path[->, >=stealth, shorten >= 0pt] (DSP) edge (yc);
    
	\node[block, draw=none] (tx_signal) at ($(xc.center)+(-1, 0.75)$) {\resizebox{7cm}{!}{\begin{tikzpicture}
\begin{axis}[
	axis lines*=middle,
	enlargelimits = false,
	clip=false,
	scale only axis,
	hide y axis,
	width=0.5\textwidth,
	height=0.15\textwidth,
	ymin=-1.3,
	ymax=1.3,
	xmin=-11,
	xmax=11,
	axis line style={->,>=stealth},
	xlabel={\small $n$},
	every axis x label/.style={
		at={(ticklabel* cs:1)},
		xshift=0.2cm,
		anchor=north,
	},
	%xtick=\empty,
	ytick=\empty,
	xtick=\empty,
	%xtick={-3.14, -1, 1, 3.14},
	%xticklabels={$-\pi$, $-\omega_c$, $\omega_c$, $\pi$},
	%xmajorgrids,
	%ymajorgrids,
	every outer y axis line/.append style={white!15!black},
	every y tick label/.append style={font=\color{white!15!black}},
	legend style={draw=white!15!black,fill=white,legend cell align=left}]
	\addplot[ycomb, mark=*, fill=white, mark options={scale=0.75, fill=white}, line width=1pt, domain=-10:10, samples=21] {rand};
\end{axis}
\end{tikzpicture}
}};
    \node[below = 0.5mm of xc] {$x[n]$};	
	\node[block, draw=none] (tx_signal) at ($(yc.center)+(2.5, 0.75)$) {\resizebox{7cm}{!}{\begin{tikzpicture}
\begin{axis}[
	axis lines*=middle,
	enlargelimits = false,
	clip=false,
	scale only axis,
	hide y axis,
	width=0.5\textwidth,
	height=0.15\textwidth,
	ymin=-1.3,
	ymax=1.3,
	xmin=-11,
	xmax=11,
	axis line style={->,>=stealth},
	xlabel={\small $n$},
	every axis x label/.style={
		at={(ticklabel* cs:1)},
		xshift=0.2cm,
		anchor=north,
	},
	%xtick=\empty,
	ytick=\empty,
	xtick=\empty,
	%xtick={-3.14, -1, 1, 3.14},
	%xticklabels={$-\pi$, $-\omega_c$, $\omega_c$, $\pi$},
	%xmajorgrids,
	%ymajorgrids,
	every outer y axis line/.append style={white!15!black},
	every y tick label/.append style={font=\color{white!15!black}},
	legend style={draw=white!15!black,fill=white,legend cell align=left}]
	\addplot[ycomb, mark=*, fill=white, mark options={scale=0.75, fill=white}, line width=1pt, domain=-10:10, samples=21] {rand};
\end{axis}
\end{tikzpicture}
}};
    \node[below = 0.5mm of yc] {$y[n]$};
\end{tikzpicture}}
\end{center}

As usual, we can apply the convolution sum

\begin{equation*}
y[n] = \sum_{n=-\infty}^{\infty} x[m-n]h[m] 
\end{equation*}

\begin{itemize}
	\item $x[n]$ is just a sample function of the random process 
	\item We generally care about the effect of the system on the statistics (e.g., mean and autocorrelation function) of the random process, rather than the system output to any particular sample function
\end{itemize}

\end{frame}

%
\begin{frame}{LTI system with a random input}

\begin{block}{Mean or expected value}
	\begin{align*}
	\E(y[n]) &= \E\bigg(\sum_{n=-\infty}^{\infty} x[m-n]h[m]\bigg) \\
	&= \sum_{n=-\infty}^{\infty} \E(x[m-n])h[m] \tag{Expectation is a linear operator and $h[n]$ is not random} \\
	&= \mu_x\sum_{n=-\infty}^{\infty} h[m] \tag{Assuming $x[n]$ is WSS} \\
	&= \mu_xH(e^{j0})
	\end{align*}
\end{block}

The mean is scaled by the gain of the LTI system at zero frequency.
\end{frame}

\begin{frame}{LTI system with a random input}

\begin{block}{Autocorrelation function}
	\begin{align*}
	\phi_{yy}[m] &= \E(y[n+m]y^*[n]) \tag{by definition}  \\
	&= \E\bigg\lbrace\bigg(\sum_{r=-\infty}^{\infty} x[n+m-r]h[r]\bigg)\cdot\bigg(\sum_{k=-\infty}^{\infty} x^*[n-r]h^*[k]\bigg)\bigg\rbrace  \\
	&= \sum_{r=-\infty}^{\infty} h[r] \sum_{k=-\infty}^{\infty} h^*[k]\E(x[n+m-r]x^*[n-k]) \\
	&= \sum_{l=-\infty}^{\infty} \bigg(\sum_{k=-\infty}^{\infty} h[l+k]h^*[k]\bigg)\phi_{xx}[m-l] \tag{change variables $l = r-k$}
	\end{align*}
\end{block}
\end{frame}

\begin{frame}
Let's define the \textbf{autocorrelation function of deterministic signals}

\begin{equation*}
c_{hh}[l] \equiv \displaystyle\sum_{k=-\infty}^{\infty} h[l+k]h^*[k]
\end{equation*} 

Note that the autocorrelation function of deterministic signals and convolution are closely related:
\begin{equation*}
c_{hh}[l] = h[l]\ast h^*[-l]
\end{equation*} 

Now we can rewrite the autocorrelation function of the output of a LTI system to a random process more compactly:
\begin{align*}
\phi_{yy}[m] &= \sum_{l=-\infty}^{\infty} c_{hh}[l]\phi_{xx}[m-l] \\
&= c_{hh}[m]\ast \phi_{xx}[m]
\end{align*} 

The autocorrelation function of the input random process is \textit{filtered} by the deterministic autocorrelation function of the impulse response.

\end{frame}

\begin{frame}{In the frequency domain}

From the previous derivation, we can write in the time domain:
\begin{align*}
\phi_{yy}[m] &= c_{hh}[m]\ast \phi_{xx}[m] \\
&= h[l]\ast h^*[-l]\ast \phi_{xx}[m]
\end{align*}

In the frequency domain:
\begin{align*}
\mathcal{F}\{\phi_{yy}[m]\} &= H(e^{j\omega})\cdot  H^*(e^{j\omega})\cdot \mathcal{F}\{\phi_{xx}[m]\} \\
&= |H(e^{j\omega})|^2\cdot\mathcal{F}\{\phi_{xx}[m]\}
\end{align*}

The DTFT of the autocorrelation function of a random process is called \textbf{power spectrum density (PSD)}. The PSD has units of W/Hz or dBm/Hz.
~\\
~\\
\textbf{Notation:} $\Phi_{xx}(e^{j\omega}) \equiv \mathcal{F}\{\phi_{xx}[m]\}$

\end{frame}

\begin{frame}{Properties of the power spectrum density}

\begin{enumerate} 
	\item The PSD is \textbf{real-valued}
	\begin{equation*}
		\Phi_{xx}(e^{j\omega}) = \Phi^*_{xx}(e^{j\omega}),
	\end{equation*}
	since the autocorrelation function has\textbf{ even symmetry}: $\phi_{xx}[m] = \phi_{xx}[-m]$.

	\item The PSD is \textbf{even symmetry} 
	\begin{equation*}
		\Phi_{xx}(e^{j\omega}) = \Phi_{xx}(e^{-j\omega}),
	\end{equation*}
	since the autocorrelation function is always real. 

	\item The PSD is \textbf{non-negativity} 
	\begin{equation*}
		\Phi_{xx}(e^{j\omega}) \geq 0,
	\end{equation*}
	This is the same condition required by an autocorrelation function of a WSS random process.
	
	\item The area under $\Phi_{xx}(e^{j\omega})$ is the average power
	\begin{equation}
	\int_{-\pi}^{\pi} \Phi_{xx}(e^{j\omega}) = \phi_{xx}[0] = E(|x[n]|^2)
	\end{equation}
	
\end{enumerate} 

\end{frame}


\begin{frame}{White noise}

White noise is a particularly important class of random process that have constant power spectrum density over all frequencies.

\begin{equation*}
\phi_{xx}[m] = \sigma_x^2\delta[m] \Longleftrightarrow \Phi_{xx}(e^{j\omega}) = \sigma_x^2, |\omega|\leq\pi.
\end{equation*}

\begin{center}
	\resizebox{0.7\linewidth}{!}{\begin{tikzpicture} 
\begin{axis}[
axis lines*=middle,
enlargelimits = true,
xmax=4,
xmin=-4,
ymin=0,
ymax=1.2,
width=\textwidth,
height=0.5\textwidth,
axis line style={->,>=stealth},
xlabel={$\omega$},
ylabel={$\Phi_{xx}(e^{j\omega})$},
yticklabel style = {yshift=0.3cm},
every axis x label/.style={
    at={(ticklabel* cs:1)},
    anchor=north,
},
every axis y label/.style={
    at={(ticklabel* cs:1)},
    anchor=south,
    yshift=0.1cm,
},
xtick=\empty,
ytick={1},
xtick={-3.14, 3.14},
xticklabels={$-\pi$, $\pi$},
yticklabels={$\sigma_x^2$},
%xmajorgrids,
%ymajorgrids,
axis on top,
every outer y axis line/.append style={white!15!black},
every y tick label/.append style={font=\color{white!15!black}},
legend style={draw=white!15!black,fill=white,legend cell align=left}]
\addplot[domain=-3.14:3.14, samples=2,line width=1.5pt,fill=black!20] coordinates {(-3.14, 0) (-3.14, 1) (3.14, 1) (3.14, 0)};
\end{axis}
\end{tikzpicture}
}
\end{center}

\end{frame}

\begin{frame}{White noise}

If the input noise is white, 
\begin{equation*} \tag{output autocorrelation function}
\phi_{yy}[m] = c_{hh}[m]\ast \phi_{xx}[m] = \sigma_x^2c_{hh}[m]
\end{equation*}

\begin{align*} 
\Phi_{yy}(e^{j\omega}) &= |H(e^{j\omega})|^2\Phi_{xx}(e^{j\omega}) \\
&= \sigma_x^2|H(e^{j\omega})|^2 \tag{output PSD}
\end{align*}

\begin{itemize}
	\item Note that the output noise PSD is not white. Hence, we say that the filter $H(e^{j\omega})$ \textbf{colored} the noise or \textbf{shaped} the noise.
	\item It is typically easier to analyze systems with white noise. As a result, it is common to employ a \textbf{noise whitening filters} to make the noise white. 
\end{itemize}


\end{frame}

%
\begin{frame}{White noise into moving average filter}
\textbf{Moving average filter}: $H(z) = \frac{1}{4}(1 + z^{-1} + z^{-2} + z^{-3})$
\begin{center}
	\resizebox{0.7\linewidth}{!}{\begin{tikzpicture}
\onslide<1-|handout:1>{
\begin{axis}[
	name=plot1,
	axis lines*=middle,
	enlargelimits = false,
	clip=false,
	scale only axis,
	width=0.7\textwidth,
	height=0.15\textwidth,
	ymin=0,
	ymax=0.5,
	xmin=-5,
	xmax=5,
	axis line style={->,>=stealth},
	xlabel={\small $m$},
	ylabel={\small $h[m]$},
	every axis x label/.style={
		at={(ticklabel* cs:1)},
		%xshift=0.2cm,
		anchor=north,
	},
	every axis y label/.style={
		at={(ticklabel* cs:0.8)},
		anchor=south,
		xshift=0.4cm,
	},
	%xtick=\empty,
	ytick={0.25},
	yticklabel={\small 0.25},
	xtick={0, 3},
	xticklabels={$0$, 3},
	every outer y axis line/.append style={white!15!black},
	every y tick label/.append style={font=\color{white!15!black}},
	legend style={draw=white!15!black,fill=white,legend cell align=left}]
	\addplot[ycomb, mark=*, fill=white, mark options={scale=0.75, fill=white}, line width=1pt, domain=0:3, samples=4] {0.25};
\end{axis}
}
\onslide<2-|handout:1>{
\begin{axis}[
	name=plot2,
	at=(plot1.below south east), anchor=above north east,
	axis lines*=middle,
	enlargelimits = false,
	clip=false,
	scale only axis,
	width=0.7\textwidth,
	height=0.14\textwidth,
	ymin=0,
	ymax=0.5,
	xmin=-5,
	xmax=5,
	axis line style={->,>=stealth},
	xlabel={\small $m$},
	ylabel={\small $c_{hh}[m] = h[m]\ast h^*[-m]$},
	yticklabel style = {yshift=0.1cm},
	every axis x label/.style={
		at={(ticklabel* cs:1)},
		%xshift=0.2cm,
		anchor=north,
	},
	every axis y label/.style={
		at={(ticklabel* cs:0.8)},
		anchor=south,
		xshift=1.9cm,
	},
	ytick={0.25},
	yticklabel={\small 0.25},
	xtick={-3, 3},
	xticklabels={\small $-3$, \small 3},
	every outer y axis line/.append style={white!15!black},
	every y tick label/.append style={font=\color{white!15!black}},
	legend style={draw=white!15!black,fill=white,legend cell align=left}]
	\addplot[ycomb, mark=*, fill=white, mark options={scale=0.75, fill=white}, line width=1pt] coordinates {(-3, 0.0625)    (-2, 0.1250)    (-1, 0.1875)    (0, 0.2500)    (1, 0.1875)    (2, 0.1250)    (3, 0.0625)};
\end{axis}
}
\onslide<3|handout:1>{
\begin{axis}[
	name=plot3,
	at=(plot2.below south east), anchor=above north east,
	axis lines*=middle,
	enlargelimits = false,
	clip=false,
	scale only axis,
	width=0.7\textwidth,
	height=0.14\textwidth,
	ymin=0,
	ymax=0.5,
	xmin=-5,
	xmax=5,
	axis line style={->,>=stealth},
	xlabel={\small $m$},
	yticklabel style = {yshift=0.1cm},
	ylabel={\small $\phi_{yy}[m] = \sigma_x^2c_{hh}[m]$},
	every axis x label/.style={
		at={(ticklabel* cs:1)},
		%xshift=0.2cm,
		anchor=north,
	},
	every axis y label/.style={
		at={(ticklabel* cs:0.8)},
		anchor=south,
		xshift=1.6cm,
	},
	ytick={0.25},
	yticklabel={\small $0.25\sigma_x^2$},
	xtick={-3, 3},
	xticklabels={\small $-3$, 3},
	every outer y axis line/.append style={white!15!black},
	every y tick label/.append style={font=\color{white!15!black}},
	legend style={draw=white!15!black,fill=white,legend cell align=left}]
	\addplot[ycomb, mark=*, fill=white, mark options={scale=0.75, fill=white}, line width=1pt] coordinates {(-3, 0.0625)    (-2, 0.1250)    (-1, 0.1875)    (0, 0.2500)    (1, 0.1875)    (2, 0.1250)    (3, 0.0625)};
\end{axis}
}
\end{tikzpicture}
}
\end{center}
\end{frame}

%
\begin{frame}{White noise into moving average filter}

Frequency response of the filter
\begin{align*}
H(e^{j\omega}) &= \frac{1}{4}(1 + e^{-j\omega} + z^{-j2\omega} + z^{-j3\omega}) \\
&= \frac{\sin(2\omega)}{4\sin(\omega/2)}e^{-j2\omega}
\end{align*} 

Output signal power spectrum when the input is white noise:

\begin{columns}
	\begin{column}{0.5\linewidth}
		\begin{align*}
			\Phi_{yy}(e^{j\omega}) &= \sigma_x^2|H(e^{j\omega})|^2 \\
			&= \sigma^2_x\bigg(\frac{\sin(2\omega)}{4\sin(\omega/2)}\bigg)^2
		\end{align*} 
	\end{column}
	\begin{column}{0.5\linewidth}
		\begin{center}
			\resizebox{0.9\linewidth}{!}{\input{figs/psd_moving_average4_to_white_noise.tex}}
		\end{center}
	\end{column}
\end{columns}
\end{frame}

\begin{frame}{Simulation example}
	
	
\end{frame}

\end{document}
