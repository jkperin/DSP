\documentclass[10pt]{beamer}
\usefonttheme{professionalfonts}
%\usetheme{CambridgeUS}
%
% Choose how your presentation looks.
%
% For more themes, color themes and font themes, see:
% http://deic.uab.es/~iblanes/beamer_gallery/index_by_theme.html
%
\mode<presentation>
{
  \usetheme{default}      % or try Darmstadt, Madrid, Warsaw, ...
  \usecolortheme{beaver} % or try albatross, beaver, crane, ...
  \usefonttheme{default}  % or try serif, structurebold, ...
  \setbeamertemplate{navigation symbols}{}
  \setbeamertemplate{caption}[numbered]
} 

\usepackage[english]{babel}
\usepackage[utf8x]{inputenc}
\usepackage{tikz}
\usepackage{pgfplots}
\usepackage{array}  % for table column M
\usepackage{makecell} % to break line within a cell
\usepackage{verbatim}
\usepackage{graphicx}
\usepackage{epstopdf}
\usepackage{amsfonts}
\usepackage{xcolor}
\usepackage{ifthen}
%\usepackage{mathtools}
\usepackage[makeroom]{cancel}
\usetikzlibrary{spy}
%\captionsetup{compatibility=false}
%\usepackage{dsfont}
\usepackage[absolute,overlay]{textpos}
\usetikzlibrary{calc, angles,quotes}
\usetikzlibrary{pgfplots.fillbetween, backgrounds}
\usetikzlibrary{positioning}
\usetikzlibrary{arrows}
\usetikzlibrary{pgfplots.groupplots}
\usetikzlibrary{arrows.meta}
\usetikzlibrary{plotmarks}
\usetikzlibrary{decorations.markings}
\usepgfplotslibrary{groupplots}
\pgfplotsset{compat=newest} 
%\pgfplotsset{plot coordinates/math parser=false}

\usepackage{hyperref}
\hypersetup{
    colorlinks=true,
    linkcolor=blue,
    filecolor=magenta,      
    urlcolor=cyan,
}

%
%\def\EXTERNALIZE{1}
%%% Externalizing
\usepgfplotslibrary{external} 
\ifdefined\EXTERNALIZE
	\makeatletter	
	\newcommand*{\overlaynumber}{\number\beamer@slideinframe}
	\tikzset{
		beamer externalizing/.style={%
			execute at end picture={%
				\tikzifexternalizing{%
					\ifbeamer@anotherslide
					\pgfexternalstorecommand{\string\global\string\beamer@anotherslidetrue}%
					\fi
				}{}%
			}%
		},
		external/optimize=false
	}
	\let\orig@tikzsetnextfilename=\tikzsetnextfilename
	\renewcommand\tikzsetnextfilename[1]{\orig@tikzsetnextfilename{#1-\overlaynumber}}
	\makeatother
	
	\tikzset{every picture/.style={beamer externalizing}}
	
	\tikzexternalize[prefix=external/]
\fi

%%% Page numbering
\usepackage{etoolbox} % necessary for excluding beamer-only frames from page numbering

\makeatletter
\pretocmd{\beamer@@@@frame}{\alt<#1>{}{\beamer@noframenumberingtrue}}{}{}
\makeatother

\addtobeamertemplate{navigation symbols}{}{%
	\usebeamerfont{footline}%
	\usebeamercolor[fg]{footline}%
	\hspace{1em}%
	\scriptsize\insertframenumber/\inserttotalframenumber
}
%%%

% For circular convolution
\newcommand\encircle[1]{%
	\tikz[baseline=(X.base)] 
	\node (X) [anchor=south, draw, shape=circle, inner sep=0mm, align=center] {\scriptsize\strut #1};}

\definecolor{matlabcomment}{RGB}{34,139,34}

\pgfmathdeclarefunction{gauss}{1}{%
	\pgfmathparse{1/(sqrt(2*pi))*exp(-((#1)^2)/2)}%
}

\pgfmathdeclarefunction{sign}{1}{%
	\pgfmathparse{1*(#1 > 0) - 1*(#1 < 0)}%
}

\pgfmathdeclarefunction{laplacian}{2}{%
	\pgfmathparse{1/(#2*2)*exp(-(abs(x-#1))/(#2))}%
}

\pgfmathdeclarefunction{pretty_func}{1}{%
	\pgfmathparse{cos(deg(#1/2)) - sin(deg(#1)) + cos(deg(#1/2)-45) - sin(deg(#1/4)-154)}%
}

\pgfplotsset{
	dirac/.style={
		mark=triangle*,
		mark options={scale=2},
		ycomb,
		scatter,
		visualization depends on={y/abs(y)-1 \as \sign},
		scatter/@pre marker code/.code={\scope[rotate=90*\sign,yshift=-2pt]}
	}
}

\def\thickness{very thick}

\tikzset{
amark/.style 2 args={
	decoration={             
		markings, 
		mark=at position {0.5} with { 
			\arrow{stealth},
			\node[#2] {#1};
		}
	}, \thickness,
	postaction={decorate}
},
earlymark/.style 2 args={
	decoration={             
		markings, 
		mark=at position {0.25} with { 
			\arrow{stealth},
			\node[#2] {#1};
		}
	}, \thickness,
	postaction={decorate}
},
latemark/.style 2 args={
	decoration={             
		markings, 
		mark=at position {0.8} with { 
			\arrow{stealth},
			\node[#2] {#1};
		}
	}, \thickness,
	postaction={decorate}
},
zpath/.style={
	decoration={             
		markings, 
		mark=at position {0.5} with { 
			\arrow{stealth},
			\node[#1] {$z^{-1}$};
		}
	}, \thickness,
	postaction={decorate}
},
terminal/.style 2 args={draw,circle,inner sep=2pt,label={#1:#2}},
}


\tikzset{
	invisible/.style={opacity=0},
	visible on/.style={alt={#1{}{invisible}}},
	alt/.code args={<#1>#2#3}{%
		\alt<#1>{\pgfkeysalso{#2}}{\pgfkeysalso{#3}} % \pgfkeysalso doesn't change the path
	},
}

\newcommand\PlotSampledSpectrum[4]{%
	\def\fs{#2}%
	\def\fmax{#3}%
	\def\ros{#4}%
	\input{#1}%
}

\pgfmathdeclarefunction{invgauss}{2}{%
	\pgfmathparse{sqrt(-2*ln(#1))*cos(deg(2*pi*#2))}%
}

\pgfmathdeclarefunction{triang}{2}{%
	\pgfmathparse{(1/#2)*(#1 + #2)*and(#1 <= 0, #1 >= -#2) + (1/#2)*(-#1 + #2)*and(#1 > 0, #1 <= #2)}%
}

\pgfmathdeclarefunction{rect}{2}{%
	\pgfmathparse{and(#1 >= 0, #1 < #2)}%
}

\pgfmathdeclarefunction{dsinc}{2}{%
	\pgfmathparse{(and(#1 != 0, 1)*(sin(deg(#1*#2/2))/sin(deg(#1/2))) + and(#1 == 0, 1) * (#2)}%
}

\tikzset{
	declare function={
		sinc(\x) = (and(\x!=0, 1) * (sin(deg(pi*\x))/(pi*\x)) +
		(and(\x==0, 1) * 1);
	}
}

\DeclareMathOperator{\E}{\mathbb{E}} % expectation

\newcolumntype{M}[1]{>{\centering\arraybackslash}m{#1}}

\definecolor{blue2}{RGB}{51, 105, 232}  
\definecolor{red2}{RGB}{213, 15, 37}  
\definecolor{green2}{RGB}{0, 153, 37}  
\definecolor{green3}{rgb}{0.1922, 0.6392, 0.3294}% 
\definecolor{yellow2}{RGB}{238, 178, 17} 
\definecolor{gray2}{RGB}{102, 102, 102}
\definecolor{orange2}{RGB}{230, 85, 13}

% Qualitative pallete set1 from www.ColorBrewer.org
\definecolor{Qred}{RGB}{228,26,28}
\definecolor{Qblue}{RGB}{55,126,184}
\definecolor{Qgreen}{RGB}{77,175,74}
\definecolor{Qpurple}{RGB}{152,78,163}
\definecolor{Qorange}{RGB}{255,127,0}
\definecolor{Qyellow}{RGB}{255,255,51}
\definecolor{Qbrown}{RGB}{166,86,40}
\definecolor{Qpink}{RGB}{247,129,191}
\definecolor{Qgray}{RGB}{153,153,153}

\newcommand\SimpleSys[4]{%
	\def\xin{#2}%
	\def\Hz{#3}%
	\def\yout{#4}
	\input{#1}%
}

\newcommand\PlotExp[5]{%
	\def\A{#2}%
	\def\Legend{#3}%
	\def\ymin{#4}
	\def\ymax{#5}
	\input{#1}%
}

\newcommand\PlotSinc[5]{%
	\def\xmin{#2}%
	\def\xmax{#3}%
	\def\samples{#4}
	\def\w{#5}
	\input{#1}%
}

% Gaussian characteristic function
\newcommand\PlotGaussianCF[4]{%
	\def\sig{#2}%
	\def\ws{#3}%
	\def\cap{#4}
	\input{#1}%
}
 % some definitions

%% 
\title[EE 264]{Power Spectrum Density Estimation}
\author{Jose Krause Perin}
\institute{Stanford University}
\date{August 15, 2017}

\begin{document}

\begin{frame}
  \titlepage
\end{frame}

%
\begin{frame}{Announcements}
\begin{itemize}
	\item Homework \# 6 due Thursday by \textbf{noon}
	\item Select time and day you'd like to take the final exam: \href{https://docs.google.com/spreadsheets/d/1fps6RCAxMloZ0jjxLB7DXeVGF_Vi5WAW8vqnk6T3erU}{sign-up sheet}
	\item Review session for the final will be on last lecture on Thursday
\end{itemize}
\end{frame}

%
\begin{frame}{Last lecture}
\begin{itemize}
	\item Leakage and resolution are important considerations in spectrum analysis
	\item By properly choosing windows we can minimize these issues
	\item Kaiser window is a nearly optimal choice. Must choose correct $\beta$ and window length $L$
	\item $\beta$ controls the ratio between the amplitudes of the main-lobe and the largest side-lobe i.e., $\beta$ controls the amount of leakage.
	\item The larger the main-lobe width, the smaller the resolution
	\item By increasing the window length we reduce the main-lobe width and consequently improve the resolution
	\item Time-dependent Fourier transform or short-time Fourier transform allows us to keep track of frequency variation in time
	\item Spectrogram is a commonly used way to display the TDFT
	\item In the spectrogram the TDFT is sampled both in time and in frequency
	\item The window length determines the resolution of the spectrogram
\end{itemize}
\end{frame}

%
\begin{frame}{Today's lecture}
	How to estimate the power spectrum density (PSD) of WSS signals.
	\tableofcontents
\end{frame}

\begin{frame}{Stationary random signals}
	\begin{itemize}
		\item Stationarity refers to time-invariance of some or all statistics of a random process
		\item We will limit our studies to wide-sense stationary (WSS) random processes
		\item The mean of a WSS process is constant, while the autocorrelation function depends only on the time deference $m$, and not on the absolute time $n$.
		\begin{align}
			\E(x[n]) &=\mu \tag{constant mean} \\
			\E(x[m+n]x^*[n]) &= \phi_{xx}[m] \tag{autocorrelation function depends only on time difference $m$}
		\end{align}
		\item The power spectrum density (PSD) of a WSS process is the Fourier transform of the autocorrelation function:
		\begin{equation*}
			P(\omega) \equiv \Phi_{xx}(e^{j\omega}) = \mathcal{F}\{\phi_{xx}[m]\}
		\end{equation*}
		
		The PSD is purely real, even symmetric, and non-negative. 
		\item Average power
		\begin{equation}
			\E(|x[n]|^2) = \phi_{xx}[0] = \frac{1}{2\pi}\int_{-\pi}^{\pi} \Phi_{xx}(e^{j\omega}) d\omega \tag{average power}
		\end{equation}
	\end{itemize}
\end{frame}

\begin{frame}{Conventions}
A zero-mean random process has PSD $P(\omega)$ and average power $\sigma^2$.

\begin{equation*}
	\sigma^2 = \frac{1}{2\pi} \int_{-\pi}^{\pi} P(\omega)d\omega = \int_{-0.5}^{0.5} P(f)df
\end{equation*}

\begin{columns}
	\begin{column}{0.5\textwidth}
		\textbf{Two-sided PSD}
	\end{column}

	\begin{column}{0.5\textwidth}
		\textbf{One-sided PSD}
	\end{column}
\end{columns}
\begin{center}
	\resizebox{\textwidth}{!}{\begin{tikzpicture}
\begin{axis}[
name=plot1,
axis y line=middle, axis x line=middle,
enlargelimits = true, clip=true,
axis on top,
xmin=-3.14, xmax=3.14,
ymin=0, ymax=2.5,
width=0.5\textwidth, height=0.25\textwidth,
axis line style={->,>=stealth},
xlabel={$\omega$},
%ylabel={$P(\omega)$},
scale only axis,
every axis x label/.style={
at={(ticklabel* cs:1)},
anchor=north,
},
every axis y label/.style={
at={(ticklabel* cs:1)},
anchor=south,
},
xtick={-3.14, 0, 3.14}, xticklabels={$-\pi$, $0$, $\pi$},
ytick=1, yticklabels={\large $P(\omega)$},
yticklabel style={yshift=0.25cm},
every outer y axis line/.append style={white!15!black},
every y tick label/.append style={font=\color{white!15!black}},
legend style={draw=white!15!black,fill=white,legend cell align=left}]
\addplot [black, fill=black!20] coordinates {(-3.14, 0) (-3.14, 1) (3.14, 1) (3.14, 0)};
\node at (axis cs: 0.75, 0.5) {\large $2\pi\sigma^2$};
\end{axis}


\begin{axis}[
name=plot2,
at= (plot1.east), anchor=west, xshift=3cm,
axis y line=middle, axis x line=middle,
enlargelimits = true, clip=true,
axis on top,
width=0.5\textwidth, height=0.25\textwidth,
xmin=0, xmax=6.28,
ymin=0, ymax=2.5,
scale only axis,
axis line style={->,>=stealth},
xlabel={$\omega$},
%ylabel={$P(\omega)$},
every axis x label/.style={
	at={(ticklabel* cs:1)},
	anchor=north,
},
every axis y label/.style={
	at={(ticklabel* cs:1)},
	anchor=south,
},
xtick={0, 3.14}, xticklabels={$0$, $\pi$},
ytick=2, yticklabels={\large $2P(\omega)$},
every outer y axis line/.append style={white!15!black},
every y tick label/.append style={font=\color{white!15!black}},
legend style={draw=white!15!black,fill=white,legend cell align=left}]
\addplot [black, fill=black!20] coordinates {(0, 0) (0, 2) (3.14, 2) (3.14, 0)};
\node at (axis cs: pi/2, 1) {\large $2\pi\sigma^2$};
\end{axis}

\begin{axis}[
name=plot1a,
at=(plot1.below south east), anchor=above north east,
axis y line=middle, axis x line=bottom,
axis on top,
enlargelimits = true, clip=true,
width=0.5\textwidth, height=0.25\textwidth,
scale only axis,
axis line style={->,>=stealth},
axis y line=middle, axis x line=middle,
enlargelimits = true, clip=true,
xmin=-0.5, xmax=0.5,
ymin=0, ymax=2.5,
axis line style={->,>=stealth},
xlabel={$f$},
%ylabel={$P(f)$},
every axis x label/.style={
	at={(ticklabel* cs:1)},
	anchor=north,
},
every axis y label/.style={
	at={(ticklabel* cs:1)},
	anchor=south,
},
xtick={-0.5, 0, 0.5},
ytick=1, yticklabels={\large $P(f) = P(\omega/(2\pi))$},
yticklabel style={yshift=0.3cm},
every outer y axis line/.append style={white!15!black},
every y tick label/.append style={font=\color{white!15!black}},
legend style={draw=white!15!black,fill=white,legend cell align=left}]
\addplot [black, fill=black!20] coordinates {(-0.5, 0) (-0.5, 1) (0.5, 1) (0.5, 0)};
\node at (axis cs: 0.1, 0.5) {\large $\sigma^2$};
\end{axis}

\begin{axis}[
name=plot2a,
at= (plot1a.east), anchor=west, xshift=3cm,
axis y line=middle, axis x line=middle,
enlargelimits = true, clip=true,
axis on top,
width=0.5\textwidth, height=0.25\textwidth,
scale only axis,
xmin=0, xmax=1,
ymin=0, ymax=2.5,
axis line style={->,>=stealth},
xlabel={$f$},
every axis x label/.style={
	at={(ticklabel* cs:1)},
	anchor=north,
},
every axis y label/.style={
	at={(ticklabel* cs:1)},
	anchor=south,
},
xtick={-0.5, 0, 0.5},
ytick=\empty,
xticklabel style={xshift=0.1cm},
ytick=2, yticklabels={\large $2P(f)$},
yticklabel style={yshift=0.25cm},
every outer y axis line/.append style={white!15!black},
every y tick label/.append style={font=\color{white!15!black}},
legend style={draw=white!15!black,fill=white,legend cell align=left}]
\addplot [black, fill=black!20] coordinates {(0, 0) (0, 2) (0.5, 2) (0.5, 0)};
\node at (axis cs: 0.25, 1) {\large $\sigma^2$};
\end{axis}
\end{tikzpicture}}
\end{center}
Generally, one-sided PSD is used when dealing with real signals, while two-sided PSD is used when dealing with complex signals.
\end{frame}

%
\begin{frame}{Matlab conventions}
The Matlab functions for PSD estimation follow these conventions:
\begin{itemize}
	\item Matlab returns \textbf{one-sided PSD} if signal is real and \textbf{two-sided PSD} if the signal is complex.
	\item You can force Matlab to return two-sided PSD by specifying frequency vector $\omega$ with negative frequencies
	\item When working with normalized frequency $f = \omega/(2\pi)$, Matlab returns 
	\begin{equation*}
		\texttt{Pf} = \begin{cases}
		2P(f), &\text{if one-sided PSD} \\
		P(f), &\text{if two-sided PSD} \\
		\end{cases}
	\end{equation*}
	\item When working with normalized frequency $\omega$, Matlab returns 	
	\begin{equation*}
	\texttt{Pw\_norm} = \displaystyle\begin{cases}
		\displaystyle\frac{1}{\pi}P(\omega), &\text{if one-sided PSD} \\
		\displaystyle\frac{1}{2\pi}P(\omega), &\text{if two-sided PSD} \\
	\end{cases}
	\end{equation*}
\end{itemize}
Therefore, the area under the PSD in Matlab is equal to the average power, even when working with $\omega$.
\end{frame}

%
\section{Periodogram}
\begin{frame}{Outline}
	\tableofcontents[currentsection]
\end{frame}
\begin{frame}{The periodogram}
	The \textbf{periodogram} of $x[n]$ is given by
	\begin{equation*}
		P(\omega) \equiv \frac{1}{LU}|\mathcal{F}\{w[n]x[n]\}|^2, \quad\text{where}\quad U = \frac{1}{L}\sum_{n = 0}^{L-1}|w[n]|^2
	\end{equation*}
	
	The periodogram is the magnitude squared of the DTFT of $w[n]x[n]$, normalized by the window length $L$ and the window power $U$.
	
	\vspace{0.25cm}
	The sampled periodogram can be computed efficiently using the DFT:
	\begin{equation*}
		P[k] = P\Big(\frac{2\pi k}{N}\Big) = \frac{1}{LU}|\mathrm{DFT}\{w[n]x[n]\}|^2
	\end{equation*}
	For $N \geq L$, the $N$-point DFT of $w[n]x[n]$ is simply the DTFT $\mathcal{F}\{w[n]x[n]\}$ sampled with period $2\pi/N$.
\end{frame}

%
\begin{frame}{Periodogram and the deterministic autocorrelation function}
Define $v[n] = w[n]x[n] \Longleftrightarrow V(e^{j\omega}) = \mathcal{F}\{w[n]x[n]\}$
\begin{align*}
	P(\omega) &= \frac{1}{LU}|\mathcal{F}\{w[n]x[n]\}|^2 =  \frac{1}{LU}\Big(V(e^{j\omega})V^*(e^{j\omega})\Big) \\
	&=\frac{1}{LU}\mathcal{F}\{v[n]\ast v^*[-n]\} \tag{$|V(e^{j\omega})|^2 =\mathcal{F}\{v[n]\ast v^*[-n]\}$}  \\
	&=\frac{1}{LU}\mathcal{F}\{c_{vv}[m]\} \tag{by definition $c_{vv}[m] = v[m]\ast v^*[-m]$}
\end{align*}

The periodogram is the DTFT of the \textbf{deterministic autocorrelation function} $c_{vv}[m]$, normalized by the window length $L$ and the window power $U$.

\vspace{0.5cm}
Writing $c_{vv}[m]$ in terms of $x[n]$ and $w[n]$:
\begin{equation*}
	c_{vv}[m] = (x[m]w[m])\ast (x^*[-m]w^*[-m]) = \sum_{n = 0}^{L-1} x[n]x^*[m-n]w[n]w^*[m-n]
\end{equation*}

\end{frame}

\begin{frame}{The periodogram as a random variable}
The periodogram is a function of the random variable $x[n]$, therefore it is itself a random variable

\textbf{The mean of the periodogram}
\begin{align*}
	\E(P(\omega)) = \E\bigg(\frac{1}{LU}\mathcal{F}\{c_{vv}[m]\}\bigg) = \E\bigg(\frac{1}{LU}\sum_{m=L-1}^{L-1}c_{vv}[m]e^{-j\omega m}\bigg) \\
	& = \frac{1}{LU}\sum_{m=L-1}^{L-1}\E(c_{vv}[m])e^{-j\omega m} \\
	& = \frac{1}{LU}\sum_{m=L-1}^{L-1}c_{ww}[m]\phi_{xx}[m]e^{-j\omega m} \tag{the window is deterministic} \\
	& = \frac{1}{2\pi LU}C_{ww}(e^{j\omega})\ast \Phi_{xx}(e^{j\omega}) \tag{product in time $\Longleftrightarrow$ convolution in frequency} \\
\end{align*}

\vspace{-0.25cm}
\textbf{Conclusion:} \textit{in average}, the periodogram is the PSD of $x[n]$, $\Phi_{xx}(e^{j\omega})$, \textit{smeared} by the deterministic PSD of the window $C_{ww}(e^{j\omega}) = |W(e^{j\omega})|^2$
\end{frame}

\begin{frame}{The periodogram as a random variable}

\textbf{The variance of the periodogram}

It can be shown that
\begin{align*}
	\mathrm{var}(P(\omega)) \approx \Phi_{xx}^2(e^{j\omega})
\end{align*}

\vspace{-0.25cm}
\textbf{Conclusion:} the periodogram has \underline{large variance}, making it useless by itself for estimating the power spectrum of a random signal.

\vspace{0.25cm}
To mitigate this problem we can average several periodograms. 

\begin{equation*}
\bar{P}(\omega) = \frac{1}{N_{avg}}\sum_{i = 0}^{N_{avg}-1} P_i(\omega)
\end{equation*}
where $P_i(\omega) = \Phi_{xx}(e^{j\omega}) + N_i(\omega)$, and $N_i(\omega)$ is the error of the $i$th periodogram. If the error $N_i(\omega)$ has variance $\sigma_N^2$. Then,

\begin{equation*}
	\mathrm{var}(\bar{P}(\omega)) = \frac{1}{N_{avg}}\sigma_N^2
\end{equation*}
The variance is reduced by the number of averages $N_{avg}$

\end{frame}

\begin{frame}{Periodogram in Matlab}
Periodogram definition:
\begin{align*}
P[k] &\equiv \frac{1}{LU}|V(e^{j2\pi/Nk})|^2, \quad\text{where}\quad U = \frac{1}{L}\sum_{n = 0}^{L-1}|w[n]|^2 \\
&= \texttt{1/(L*U)*abs(fft(x.*window, N)).\textasciicircum 2} \tag{Matlab code}
\end{align*}
This returns the two-sided periodogram at frequencies $\omega_k = 2\pi/Nk, k=0, \ldots, L-1$.

Matlab's periodogram function:
\begin{align*}
&\texttt{>> dw = 2*pi/N} \\
&\texttt{>> Pw = 2*pi*periodogram(x, window, -pi:dw:pi-dw)}
\end{align*}

\textbf{Careful with conventions}: this particular call of the \texttt{periodogram} function returns a \textbf{two-sided periodogram} for frequencies in the interval $[-\pi, \pi)$. The multiplicative factor \texttt{2*pi} corrects for the PSD normalization done by Matlab.
\end{frame}

\begin{frame}{Example}

\begin{center}
	\resizebox{0.7\textwidth}{!}{\SimpleSys{figs/simple_in_dsp_out.tex}{$x[n]$}{$H(z)$}{$y[n]$}}
\end{center}

Let's assume that $H(z) = 1 - z^{-2}$. Hence
\begin{equation*}
	H(e^{j\omega}) = 1 - e^{-j2\omega} = 2j\sin(\omega)e^{-j\omega}
\end{equation*}
And for the power spectrum density:
\begin{equation*}
	\Phi_{yy}(e^{j\omega}) = |H(e^{j\omega})|^2\Phi_{xx}(e^{j\omega}) = 4\sin^2(\omega)\Phi_{xx}(e^{j\omega})
\end{equation*}
$\Phi_{yy}(e^{j\omega})$ is the \textbf{two-sided PSD} of $y[n]$, as usual.
\vspace{0.25cm}

If $x[n]$ is a white noise with average power $\sigma_x^2$
\begin{equation*}
\Phi_{yy}(e^{j\omega}) = 4\sin^2(\omega)\sigma_x^2
\end{equation*}

See Matlab script: Canvas/Files/Matlab/\texttt{periodogram\_example.m}
\end{frame}

\section{Welch's method}
\begin{frame}{Outline}
\tableofcontents[currentsection]
\end{frame}
\begin{frame}{Welch's method of estimating the PSD}
	Welch's PSD estimate is obtained by averaging several periodograms
	
	\begin{enumerate}
	\item Divide $x[n]$ into $K$ overlapping segments
		\begin{equation*}
			x_r[n] = x[rR + n], \quad n = 0, \ldots, L-1
		\end{equation*}
		Usually $R = L/2$. i.e., the subsequences overlap by $50\%$.
	\item Compute the periodogram for each sequence $x_r[n]$
		\begin{align*}
			P_r(\omega) &= \frac{1}{LU}|\mathcal{F}\{x_r[n]w[n]\}|^2, \quad\text{where}\quad U = \frac{1}{L}\sum_{n = 0}^{L-1}|w[n]|^2 \\
			P_r[k] &= \frac{1}{LU}|\mathrm{DFT}\{x_r[n]w[n]\}|^2
		\end{align*}
	\item Average the periodograms $P_r[k]$:
		\begin{align*}
			\bar{P}[k] = \frac{1}{K}\sum_{r = 0}^{K-1}P_r[k]
		\end{align*}
	\end{enumerate}
\end{frame}

%
\begin{frame}{Welch's method example}
Same scenario as before: white noise $x[n]$ filtered by $H(z) = 1 - z^{-2}$

These plots assume $R = 128$, $L = N = 256$.
\begin{center}
	\resizebox{0.9\textwidth}{!}{\begin{tikzpicture}
\begin{axis}[
	name=plot1,
	axis y line=middle, axis x line=bottom,
	enlargelimits = upper, clip=true,
	scale only axis,
	width=\textwidth,
	height=0.2\textwidth,
	ymin=-6,	ymax=6,
	xmin=0, xmax=512,
	axis line style={->,>=stealth},
	%y axis line style={draw=none},
	x axis line style={shorten >= -0.25cm}, 
	xlabel={\large $n$},
	ylabel={\large $y[n] = x[n] \ast h[n]$},
	every axis x label/.style={
		at={(ticklabel* cs:1)},
		xshift=0.25cm,
		anchor=north,
	},
	every axis y label/.style={
		at={(ticklabel* cs:0.8)},
		anchor=south,
		xshift=1.75cm,
	},
	ytick=\empty,
	xtick={0, 128,...,512},
	every outer y axis line/.append style={white!15!black},
	every y tick label/.append style={font=\color{white!15!black}},
	legend style={draw=white!15!black,fill=white,legend cell align=left}]
	\addplot[black, line width=1pt] table[x index=0,y index=1, each nth point={2}] {figs/data/welch_filter_output.dat};
	
	\only<2-|handout:2->{
		\addplot[blue2, line width=1pt] table[x index=0,y index=1, each nth point={2}, restrict x to domain=0:255] {figs/data/welch_filter_output.dat};
	}
	
	\only<3-|handout:3->{
		\addplot[red2, line width=1pt] table[x index=0,y index=1, each nth point={2}, restrict x to domain=128:383] {figs/data/welch_filter_output.dat};
	}

	\only<4-|handout:4->{
	\addplot[green2, line width=1pt] table[x index=0,y index=1, each nth point={2}, restrict x to domain=256:512] {figs/data/welch_filter_output.dat};
	}
	
\end{axis}

\begin{axis}[
	name=plot2,
	at=(plot1.below south east), anchor=above north east,
	axis y line=middle, axis x line=middle,
	enlargelimits = false, clip=true,
	scale only axis,
	width=\textwidth,
	height=0.2\textwidth,
	ymin=0,	ymax=14,
	xmin={0}, xmax={1},
	axis line style={->,>=stealth},
	x axis line style={shorten >= -0.25cm}, 
	xlabel={\large $\omega$},
	ylabel={\large PSD},
	every axis x label/.style={
		at={(ticklabel* cs:1)},
		xshift=0.25cm,
		anchor=north,
	},
	every axis y label/.style={
		at={(ticklabel* cs:0.8)},
		anchor=south,
		xshift=0.6cm,
	},
	xtick={-1, -0.5, 0.5, 1},
	xticklabels={$-\pi$, $-\frac{\pi}{2}$, $\frac{\pi}{2}$, $\pi$},
	ytick=\empty,
	every outer y axis line/.append style={white!15!black},
	every y tick label/.append style={font=\color{white!15!black}},
	legend style={draw=white!15!black,fill=white,legend cell align=left}]
	\only<2-|handout:2->{
		\addplot[smooth, black!50, line width=2pt, domain=0:1] {4*(sin(deg(pi*x)))^2}; \addlegendentry{$\Phi_{yy}(\omega)$};
		
		\addplot[blue2, line width=1pt] table[x index=0,y index=3] {figs/data/welch_partial_periodograms.dat}; \addlegendentry{$P_0(\omega)$};
	}

	\only<3-|handout:3->{	
		\addplot[red2, line width=1pt] table[x index=0,y index=2] {figs/data/welch_partial_periodograms.dat}; \addlegendentry{$P_1(\omega)$};
	}

	\only<4-|handout:4->{	
	\addplot[green2, line width=1pt] table[x index=0,y index=1] {figs/data/welch_partial_periodograms.dat}; \addlegendentry{$P_2(\omega)$};
	}
\end{axis}

\begin{axis}[
name=plot3,
at=(plot2.below south east), anchor=above north east,
	axis y line=middle, axis x line=middle,
enlargelimits = false, clip=true,
scale only axis,
width=\textwidth,
height=0.2\textwidth,
ymin=0,	ymax=14,
xmin={0}, xmax={1},
axis line style={->,>=stealth},
x axis line style={shorten >= -0.25cm}, 
xlabel={\large $\omega$},
ylabel={\large PSD},
every axis x label/.style={
	at={(ticklabel* cs:1)},
	xshift=0.25cm,
	anchor=north,
},
every axis y label/.style={
	at={(ticklabel* cs:0.8)},
	anchor=south,
	xshift=0.6cm,
},
xtick={-1, -0.5, 0.5, 1},
xticklabels={$-\pi$, $-\frac{\pi}{2}$, $\frac{\pi}{2}$, $\pi$},
ytick=\empty,
every outer y axis line/.append style={white!15!black},
every y tick label/.append style={font=\color{white!15!black}},
legend style={draw=white!15!black,fill=white,legend cell align=left}]

\only<3-|handout:3->{
	\addplot[smooth, black!50, line width=2pt, domain=0:1, forget plot] {4*(sin(deg(pi*x)))^2};
	
	\addplot[black, line width=1pt] table[x index=0,y index=4] {figs/data/welch_psd.dat}; \addlegendentry{$\bar{P}(\omega) = \frac{P_0(\omega) + P_1(\omega) + P_2(\omega)}{3}$};
}

\end{axis}
\end{tikzpicture}	
}
\end{center}
\end{frame}

%
\begin{frame}{Welch's method and spectrogram}
Recall that the spectrogram is defined by

\begin{align*}
	X[r, k] &= \sum_{m = 0}^{L-1}x[rR+m]w[m]e^{-j(2\pi/N)km} \tag{spectrogram}
\end{align*}
For each $r$, the spectrogram is the DFT of $x[rR+m]w[m]$. Hence, we can calculate Welch's PSD estimate from the spectrogram

\begin{equation*}
	\bar{P}[k] = \frac{1}{LUK} \sum_{r = 0}^{K-1} |X[r, k]|^2 \tag{Welch's PSD estimate}
\end{equation*}

This is equivalent to averaging the squared magnitude of the spectrogram over time $r$.

\end{frame}

%
\begin{frame}{Welch's method in Matlab}
\begin{itemize}
	\item Using the \texttt{pwelch} function:
	\begin{align*}
	& \texttt{>> dw = 2*pi/Nfft} \\
	& \texttt{>> Pw = 2*pi*pwelch(x, window, L-R, -pi:dw:pi-dw)} \tag{two-sided PSD estimate}
	\end{align*}
	The same \underline{convention considerations} for the periodogram apply to the \texttt{pwelch} function.
	
	\item Using the spectrogram
	\begin{align*}
		& \texttt{>> dw = 2*pi/Nfft} \\
		& \texttt{>> S = spectrogram(x, window, L-R, -pi:dw:pi-dw)}\\
		& \texttt{>> Pw = mean(1/(L*U)*abs(S).\textasciicircum 2, 2)} \tag{two-sided PSD estimate}
	\end{align*}
	\texttt{L} is the window length and \texttt{U} is the window power.
\end{itemize}

\vspace{0.25cm}
See Matlab script: Canvas/Files/Matlab/\texttt{welch\_example.m}
\end{frame}

\section{Blackman-Tukey method}
\begin{frame}{Outline}
\tableofcontents[currentsection]
\end{frame}

%
\begin{frame}{Blackman-Tukey PSD estimate}
	The Blackman-Tukey method first estimates the autocorrelation function $\hat{\phi}_{xx}[m]$. The PSD follows by computing the DFT of $\hat{\phi}_{xx}[m]$.
	\begin{enumerate}
		\item Estimate the autocorrelation function from $Q$ samples of $x[n]$
		\begin{equation*}
			\hat{\phi}_{xx}[m] = \frac{1}{Q-|m|}\sum_{n = 0}^{Q-1-|m|} x[m+n]x[n], \quad |m|\leq Q-1
		\end{equation*}
		\item Multiply autocorrelation by window $w[m]$
		\begin{equation*}
			s[m] = \begin{cases}
			\hat{\phi}_{xx}[m]w[m], & -(M-1)\leq m \leq (M-1) \\
			0, & \text{otherwise}
			\end{cases}
		\end{equation*}
		window is centered around $m = 0$ and it has length $L = 2M-1$.
		\item Form DFT-even sequence
		\begin{equation*}
			s_e[m] = s[((m))_N]
		\end{equation*}	
		This guarantees that the PSD is linear phase.
		\item Compute DFT
		\begin{equation*}
			S[k] = \bigg|\sum_{m = 0}^{N-1}s[m]e^{-j(2\pi/N)k}\bigg|, \quad k = 0, \ldots, N-1
		\end{equation*}
	\end{enumerate}
\end{frame}

%
\begin{frame}{The mean of Blackman-Tukey PSD estimate}

The estimator of the autocorrelation function from $Q$ samples of $x[n]$
\begin{equation*}
\hat{\phi}_{xx}[m] = \frac{1}{Q-|m|}\sum_{n = 0}^{Q-1-|m|} x[m+n]x[n], \quad |m|\leq Q-1
\end{equation*}
has mean
\begin{align*}
\E(\hat{\phi}_{xx}[m]) &= \frac{1}{Q-|m|}\sum_{n=0}^{Q-1-|m|} \E(x[m+n]x[n]) = \frac{1}{Q-|m|}\sum_{n=0}^{Q-1-|m|}\phi_{xx}[m] \\
&=\phi_{xx}[m], \quad |m|\leq L-1
\end{align*}
Therefore, this estimator is \textbf{unbiased} i.e., its mean is equal to the ideal value.
\end{frame}

%
\begin{frame}{The mean of Blackman-Tukey PSD estimate}
The mean of the PSD estimate is simply
\begin{align*}
	\E(P(\omega)) &= \sum_{m = 0}^{N-1}\E(\hat{\phi}_{xx}[m]w[m])e^{-j\omega m} \tag{since $s[m] = \hat{\phi}_{xx}[m]w[m]$} \\
	& = \sum_{m = 0}^{N-1}\E(\hat{\phi}_{xx}[m])w[m]e^{-j\omega m} \tag{window is deterministic}\\
	& = \sum_{m = 0}^{N-1}\phi_{xx}[m]w[m]e^{-j\omega m} \tag{since $\hat{\phi}_{xx}[m]$ is unbiased} \\
	&= \frac{1}{2\pi}P(\omega)\ast W(e^{j\omega}) \tag{product in time $\implies$ convolution in frequency}
\end{align*}

\textbf{Conclusions:} 
\begin{itemize} 
	\item\textit{In average}, the Blackman-Tukey PSD estimate is equal to $P(\omega)$ \textit{smeared} by the window $W(e^{j\omega})$.
	\item For the previous methods: $\E(P(\omega)) = \frac{1}{2\pi}P(\omega)\ast |W(e^{j\omega})|^2$
	\item If $W(e^{j\omega})$ takes on negative values, the PSD estimate could be negative. Thus, non-negative windows are preferred e.g, Bartlett
\end{itemize} 
\end{frame}

%
\begin{frame}{Blackman-Tukey PSD estimate in Matlab}

\begin{align*}
& \texttt{>> phi\_hat = xcorr(x, x, M-1, `unbiased')} \tag{unbiased autocorrelation estimate}\\
& \texttt{>> s = phi\_hat.*w} \tag{windowing} \\
& \texttt{>> s\_e = ifftshift(s)} \tag{form DFT-even sequence} \\
& \texttt{>> Pw = abs(fftshift(fft(s\_e)))} \tag{two-sided PSD in interval $[-\pi, \pi)$}
\end{align*}

As you'll see in HW6, \texttt{xcorr} can be computed using FFT/IFFT with complexity $\mathcal{O}(N\log_2N)$.

\end{frame}

\begin{frame}{Example}
	See Matlab script: Canvas/Files/Matlab/\texttt{blackman\_tukey\_example.m}. This script requires the function Canvas/Files/Matlab/\texttt{blackman\_tukey\_psd.m}.
	
	\vspace{0.25cm}
	\textbf{Conclusions:} 
	\begin{itemize}
		\item Increasing the number of samples $Q$ improves accuracy
		\item Reducing the window length $L = 2M-1$ improves accuracy at the expense of poorer resolution of the PSD
		\item The FFT length $N$ determines the number of samples of the estimated PSD
	\end{itemize}
\end{frame}


%
\begin{frame}{Summary}
	\begin{itemize}
		\item The PSD is the DTFT of the autocorrelation function
		\item The PSD may be two-sided or one-sided. Careful with conventions!
		\item The periodogram method estimates the PSD directly from the magnitude squared of the DFT of the windowed signal
		\item The periodogram is an biased estimator of the PSD, and it has large variance. Hence, the periodogram must be averaged to produce useful estimates
		\item The Welch method averages several periodograms
		\item The Welch method breaks the data into overalapping segments, each of length $L$. Usually, the segments overlap by $L/2$.
		\item Differently from the periodogram and Welch method, the Blackman-Tukey method estimates the PSD by computing the DFT of the estimated autocorrelation function
		\item Although the estimator of the autocorrelation function may be unbiased, the PSD estimate is biased. Windows with non-negative frequency response are typically preferred e.g., Bartlett
		\item Increasing the sequence length $Q$ improves accuracy. Reducing the window length $L$ improves accuracy at the expense of poorer frequency resolution.
	\end{itemize}
\end{frame}

\end{document}
