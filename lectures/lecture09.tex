\documentclass[10pt, handout]{beamer}
\usefonttheme{professionalfonts}
%\usetheme{CambridgeUS}
%
% Choose how your presentation looks.
%
% For more themes, color themes and font themes, see:
% http://deic.uab.es/~iblanes/beamer_gallery/index_by_theme.html
%
\mode<presentation>
{
  \usetheme{default}      % or try Darmstadt, Madrid, Warsaw, ...
  \usecolortheme{beaver} % or try albatross, beaver, crane, ...
  \usefonttheme{default}  % or try serif, structurebold, ...
  \setbeamertemplate{navigation symbols}{}
  \setbeamertemplate{caption}[numbered]
} 

\usepackage[english]{babel}
\usepackage[utf8x]{inputenc}
\usepackage{tikz}
\usepackage{pgfplots}
\usepackage{array}  % for table column M
\usepackage{makecell} % to break line within a cell
\usepackage{verbatim}
\usepackage{graphicx}
\usepackage{epstopdf}
\usepackage{amsfonts}
\usepackage{xcolor}
\usepackage{ifthen}
%\usepackage{mathtools}
\usepackage[makeroom]{cancel}
%\captionsetup{compatibility=false}
%\usepackage{dsfont}
\usepackage[absolute,overlay]{textpos}
\usetikzlibrary{calc, angles,quotes}
\usetikzlibrary{pgfplots.fillbetween, backgrounds}
\usetikzlibrary{positioning}
\usetikzlibrary{arrows}
\usetikzlibrary{pgfplots.groupplots}
\usetikzlibrary{arrows.meta}
\usetikzlibrary{plotmarks}
\usetikzlibrary{decorations.markings}

\usepgfplotslibrary{groupplots}
\pgfplotsset{compat=newest} 
%\pgfplotsset{plot coordinates/math parser=false}

\usepackage{hyperref}
\hypersetup{
    colorlinks=true,
    linkcolor=blue,
    filecolor=magenta,      
    urlcolor=cyan,
}

%%% Externalizing
\usepgfplotslibrary{external} 
\ifdefined\EXTERNALIZE
	\makeatletter	
	\newcommand*{\overlaynumber}{\number\beamer@slideinframe}
	\tikzset{
		beamer externalizing/.style={%
			execute at end picture={%
				\tikzifexternalizing{%
					\ifbeamer@anotherslide
					\pgfexternalstorecommand{\string\global\string\beamer@anotherslidetrue}%
					\fi
				}{}%
			}%
		},
		external/optimize=false
	}
	\let\orig@tikzsetnextfilename=\tikzsetnextfilename
	\renewcommand\tikzsetnextfilename[1]{\orig@tikzsetnextfilename{#1-\overlaynumber}}
	\makeatother
	
	\tikzset{every picture/.style={beamer externalizing}}
	
	\tikzexternalize[prefix=external/]
\fi

%%% Page numbering
\usepackage{etoolbox} % necessary for excluding beamer-only frames from page numbering

\makeatletter
\pretocmd{\beamer@@@@frame}{\alt<#1>{}{\beamer@noframenumberingtrue}}{}{}
\makeatother

\addtobeamertemplate{navigation symbols}{}{%
	\usebeamerfont{footline}%
	\usebeamercolor[fg]{footline}%
	\hspace{1em}%
	\scriptsize\insertframenumber/\inserttotalframenumber
}
%%%

% For circular convolution
\newcommand\encircle[1]{%
	\tikz[baseline=(X.base)] 
	\node (X) [anchor=south, draw, shape=circle, inner sep=0mm, align=center] {\scriptsize\strut #1};}

\definecolor{matlabcomment}{RGB}{34,139,34}

\pgfmathdeclarefunction{gauss}{1}{%
	\pgfmathparse{1/(sqrt(2*pi))*exp(-((#1)^2)/2)}%
}

\pgfmathdeclarefunction{sign}{1}{%
	\pgfmathparse{1*(#1 > 0) - 1*(#1 < 0)}%
}

\pgfmathdeclarefunction{laplacian}{2}{%
	\pgfmathparse{1/(#2*2)*exp(-(abs(x-#1))/(#2))}%
}

\pgfmathdeclarefunction{pretty_func}{1}{%
	\pgfmathparse{cos(deg(#1/2)) - sin(deg(#1)) + cos(deg(#1/2)-45) - sin(deg(#1/4)-154)}%
}

\pgfplotsset{
	dirac/.style={
		mark=triangle*,
		mark options={scale=2},
		ycomb,
		scatter,
		visualization depends on={y/abs(y)-1 \as \sign},
		scatter/@pre marker code/.code={\scope[rotate=90*\sign,yshift=-2pt]}
	}
}

\def\thickness{very thick}

\tikzset{
amark/.style 2 args={
	decoration={             
		markings, 
		mark=at position {0.5} with { 
			\arrow{stealth},
			\node[#2] {#1};
		}
	}, \thickness,
	postaction={decorate}
},
earlymark/.style 2 args={
	decoration={             
		markings, 
		mark=at position {0.25} with { 
			\arrow{stealth},
			\node[#2] {#1};
		}
	}, \thickness,
	postaction={decorate}
},
latemark/.style 2 args={
	decoration={             
		markings, 
		mark=at position {0.8} with { 
			\arrow{stealth},
			\node[#2] {#1};
		}
	}, \thickness,
	postaction={decorate}
},
zpath/.style={
	decoration={             
		markings, 
		mark=at position {0.5} with { 
			\arrow{stealth},
			\node[#1] {$z^{-1}$};
		}
	}, \thickness,
	postaction={decorate}
},
terminal/.style 2 args={draw,circle,inner sep=2pt,label={#1:#2}},
}


\tikzset{
	invisible/.style={opacity=0},
	visible on/.style={alt={#1{}{invisible}}},
	alt/.code args={<#1>#2#3}{%
		\alt<#1>{\pgfkeysalso{#2}}{\pgfkeysalso{#3}} % \pgfkeysalso doesn't change the path
	},
}

\newcommand\PlotSampledSpectrum[4]{%
	\def\fs{#2}%
	\def\fmax{#3}%
	\def\ros{#4}%
	\input{#1}%
}

\pgfmathdeclarefunction{invgauss}{2}{%
	\pgfmathparse{sqrt(-2*ln(#1))*cos(deg(2*pi*#2))}%
}

\pgfmathdeclarefunction{triang}{2}{%
	\pgfmathparse{(1/#2)*(#1 + #2)*and(#1 <= 0, #1 >= -#2) + (1/#2)*(-#1 + #2)*and(#1 > 0, #1 <= #2)}%
}

\pgfmathdeclarefunction{rect}{2}{%
	\pgfmathparse{and(#1 >= 0, #1 < #2)}%
}

\pgfmathdeclarefunction{dsinc}{2}{%
	\pgfmathparse{(and(#1 != 0, 1)*(sin(deg(#1*#2/2))/sin(deg(#1/2))) + and(#1 == 0, 1) * (#2)}%
}

\tikzset{
	declare function={
		sinc(\x) = (and(\x!=0, 1) * (sin(deg(pi*\x))/(pi*\x)) +
		(and(\x==0, 1) * 1);
	}
}

\DeclareMathOperator{\E}{\mathbb{E}} % expectation

\newcolumntype{M}[1]{>{\centering\arraybackslash}m{#1}}

\definecolor{blue2}{RGB}{51, 105, 232}  
\definecolor{red2}{RGB}{213, 15, 37}  
\definecolor{green2}{RGB}{0, 153, 37}  
\definecolor{green3}{rgb}{0.1922, 0.6392, 0.3294}% 
\definecolor{yellow2}{RGB}{238, 178, 17} 
\definecolor{gray2}{RGB}{102, 102, 102}
\definecolor{orange2}{RGB}{230, 85, 13}

% Qualitative pallete set1 from www.ColorBrewer.org
\definecolor{Qred}{RGB}{228,26,28}
\definecolor{Qblue}{RGB}{55,126,184}
\definecolor{Qgreen}{RGB}{77,175,74}
\definecolor{Qpurple}{RGB}{152,78,163}
\definecolor{Qorange}{RGB}{255,127,0}
\definecolor{Qyellow}{RGB}{255,255,51}
\definecolor{Qbrown}{RGB}{166,86,40}
\definecolor{Qpink}{RGB}{247,129,191}
\definecolor{Qgray}{RGB}{153,153,153}

\newcommand\SimpleSys[4]{%
	\def\xin{#2}%
	\def\Hz{#3}%
	\def\yout{#4}
	\input{#1}%
}

\newcommand\PlotExp[5]{%
	\def\A{#2}%
	\def\Legend{#3}%
	\def\ymin{#4}
	\def\ymax{#5}
	\input{#1}%
}

\newcommand\PlotSinc[5]{%
	\def\xmin{#2}%
	\def\xmax{#3}%
	\def\samples{#4}
	\def\w{#5}
	\input{#1}%
}

% Gaussian characteristic function
\newcommand\PlotGaussianCF[4]{%
	\def\sig{#2}%
	\def\ws{#3}%
	\def\cap{#4}
	\input{#1}%
}


\makeatletter
\def\env@dcases{%
	\let\@ifnextchar\new@ifnextchar
	\left\lbrace
	\def\arraystretch{2}%
	\array{@{}l@{\quad}l@{}}}
\makeatother

%% 
\title[EE 264]{Filter Design (Part I)}
\author{Jose Krause Perin}
\institute{Stanford University}
\date{July 27, 2017}

\begin{document}

\begin{frame}
  \titlepage
\end{frame}

%
%\begin{frame}{Announcements}
%	\begin{itemize}
%		\item Homework \#4 due on Sunday, July 30. Start early!
%		\item Midterm review session will be on Friday at 1:30pm at \textbf{Gates B03}
%		\item Please fill out the mid-quarter teaching evaluation survey and get 2\% extra credit: \url{https://tinyurl.com/y8cyfddy}. \textbf{Deadline:} Today
%		\item We'll release practice midterms today. Their solutions will be released on Friday after the review session.
%	\end{itemize}
%\end{frame}

%
\begin{frame}{Last lecture}
\begin{itemize}
	\item Two's complement is a fixed-point representation that allows fractions to be represented as integers
	\item There's an inherent trade-off between roundoff noise and overflow/clipping
	\item FIR systems remain stable after coefficient quantization
	\item Linear phase FIR systems remain linear phase after coefficient quantization, since the impulse response remains symmetric
	\item Coefficient quantization may lead to instability in IIR systems, as poles may move outside the unit circle
	\item Similarly to quantization noise, roundoff noise is modeled by an additive white noise that is independent of the input signal (the linear noise model).
	\item Roundoff noise is minimized by performing quantization only after accumulation, but this requires $(2B+1)$-bit adders
	\item In FIR structures the equivalent roundoff noise at the output is white
	\item IIR structures lead to roundoff noise shaping
	\item Least noisy IIR structure depends on the system
	\item Cascade and parallel forms are used to mitigate total roundoff noise
\end{itemize}
\end{frame}

%
\section{Outline}

\begin{frame}{Practice and theory}
\begin{block}{In practice}
	\vspace{-0.5cm}
	\begin{center}
		\resizebox{\linewidth}{!}{\def\layersep{1.5cm}
\def\outsep{0.7cm}
\def\dy{1.25}

\begin{tikzpicture}[->, >=stealth, shorten >= 0pt, draw=black!50, node distance=\layersep, font=\sffamily]
    \tikzstyle{node}=[circle,fill=black,minimum size=2pt,inner sep=0pt]
    \tikzstyle{block}=[draw=black,rectangle,fill=none,minimum size=1.5cm, inner sep=0pt]
    \tikzstyle{annot} = []

	\node[node] (xc) at (0, -\dy cm) {};
    \node[block] (ADC) at (1*\layersep, -\dy cm) {ADC};
    \node[block, text width = 2.5cm, align= center] (DSP) at (3*\layersep, -\dy cm) {Digital Signal Processor};
    \node[block] (DAC) at (5*\layersep, -\dy cm) {DAC};
	\coordinate (yc) at (6*\layersep, -\dy cm) {};
	
	\coordinate (mid1) at ($(ADC.east)!0.5!(DSP.west)$) {};
	\coordinate (mid2) at ($(DSP.east)!0.5!(DAC.west)$) {};
		
    \path (xc) edge (ADC);
    \path (ADC) edge (DSP);
    \path (DSP) edge (DAC);
    \path (DAC) edge (yc);
    
    \node[above = 0.5mm of mid1] {$x[n]$};
    \node[above = 0.5mm of mid2] {$y[n]$};
    \node[above = 0mm of xc, text width = 1cm, align=center] {$x_c(t)$};
    \node[above = 0mm of yc, text width = 1cm, align=center] {$y_c(t)$}; 
    

\end{tikzpicture}}
	\end{center}
\end{block}

\begin{block}{DSP theory}
	\vspace{-0.5cm}
	\begin{center}
		\def\Heff{1}
		\resizebox{\linewidth}{!}{\def\layersep{2cm}
\def\outsep{0.7cm}
\def\dy{1.25}

\begin{tikzpicture}[->, >=stealth, shorten >= 0pt, draw=black!50, node distance=\layersep, font=\sffamily]
    \tikzstyle{node}=[circle,fill=black,minimum size=2pt,inner sep=0pt]
    \tikzstyle{block}=[draw=black,rectangle,fill=none,minimum size=1.5cm, inner sep=0pt]
    \tikzstyle{annot} = []

	\node[node] (xc) at (0, -\dy cm) {};
    \node[block] (ADC) at (1*\layersep, -\dy cm) {C-to-D};
    \node[block, text width = 2cm, align= center] (DSP) at (3*\layersep, -\dy cm) {LTI \\ System};
    \node[block] (DAC) at (5*\layersep, -\dy cm) {D-to-C};
	\coordinate (yc) at (6*\layersep, -\dy cm) {};
	
	\coordinate (mid1) at ($(ADC.east)!0.5!(DSP.west)$) {};
	\coordinate (mid2) at ($(DSP.east)!0.5!(DAC.west)$) {};
		
    \path (xc) edge (ADC);
    \path (ADC) edge (DSP);
    \path (DSP) edge (DAC);
    \path (DAC) edge (yc);
    
    \node[above = 0.5mm of mid1] {$x[n]$};
    \node[below = 0.5mm of mid1] {$X(e^{j\omega})$};
    \node[above = 0.5mm of mid2] {$y[n]$};
    \node[below = 0.5mm of mid2] {$Y(e^{j\omega})$};
    \node[above = 0mm of xc, text width = 1cm, align=center] {$x_c(t)$};
    \node[below = 0mm of xc, text width = 1cm, align=center] {$X_c(j\Omega)$};
    \node[above = 0mm of yc, text width = 1cm, align=center] {$y_r(t)$}; 
    \node[below = 0mm of yc, text width = 1cm, align=center] {$Y_r(j\Omega)$};
    \node at ($(DSP.south)-(0, 0.25cm)$) {$h[n] \leftrightarrow H(e^{j\omega})$};
\end{tikzpicture}}
	\end{center}
\end{block}

\pause
\textbf{Question:} how to design a \textbf{rational} $H(z)$ from a desired $H(e^{j\omega})$?
\end{frame}

\begin{frame}{Digital filter design}
	We'll cover two different design problems
	\begin{enumerate}
		\item Digital filter design \underline{from analog filter}
		
		Given a continuous-time LTI filter defined by $h(t) \Longleftrightarrow H(s)$, how to obtain the corresponding discrete-time filter $h[n] \Longleftrightarrow H(z)$ such that
		\begin{equation*}
			H(e^{j\Omega T}) = H(j\Omega), |\Omega| < \Omega_s/2
		\end{equation*}
		
		\textbf{Design techniques:}
		\begin{itemize}
			\item Impulse invariance
			\item Bilinear transformation
		\end{itemize}
		Design by impulse invariance can result in either FIR or IIR filters, but bilinear transformation generally results in IIR filters. 
	\end{enumerate}
\end{frame}

%
\begin{frame}{Digital filter design}
We'll cover two different design problems
\begin{enumerate}\setcounter{enumi}{1}
	\item Digital filter design \underline{from specifications}
	
	\begin{center}
		\resizebox{0.6\linewidth}{!}{\begin{tikzpicture}
\begin{axis}[
axis lines*=middle,
enlargelimits = upper, clip=true,
scale only axis,
axis on top=false,
axis line style={->,>=stealth},
width=0.6\textwidth,
height=0.4\textwidth,
xlabel={$\omega$},
ylabel={$|H(e^{j\omega})|$},
every axis x label/.style={
	at={(ticklabel* cs:1)},
	xshift=-0.2cm,
	anchor=north,
},
every axis y label/.style={
	at={(ticklabel* cs:1)},
	xshift=0.4cm,
	%yshift=0.35cm,
	anchor=south,
},
every outer x axis line/.append style={white!15!black},
every x tick label/.append style={font=\color{white!15!black}},
xmin=0, xmax=1,
ymin=0, ymax=1.25,
ytick={0.2, 0.9, 1.1}, yticklabels={$\delta_2$, $1-\delta_1$, $1+\delta_1$},
xtick={0, 0.25, 0.6, 1},
xticklabels ={$0$, $\omega_p$, $\omega_s$, $\pi$},
every outer y axis line/.append style={white!15!black},
every y tick label/.append style={font=\color{white!15!black}},
legend style={draw=white!15!black,fill=white,legend cell align=left}]
\addplot[solid, line width=1pt, domain=0:0.25, samples=2] {1.1}; 
\addplot[solid, line width=1pt, domain=0:0.25, samples=2] {0.9};
\addplot[solid, line width=1pt, domain=0.6:1, samples=2] {0.2};

\addplot[blue2, line width=2pt, domain=0:0.25, samples=2] {1};
\addplot[blue2, line width=2pt, domain=0.6:1, samples=2] {0};
\addplot[dashed, line width=1pt] coordinates {(0.25, 0) (0.25, 0.9)};
\addplot[dashed, line width=1pt] coordinates {(0.6, 0) (0.6, 0.9)};

\fill [black!20] (axis cs: 0, 1.1) rectangle (0.25, 1.2);
\fill [black!20] (axis cs: 0, 0.9) rectangle (0.25, 0.8);
\fill [black!20] (axis cs: 0.6, 0.2) rectangle (1, 0.3);

\node[align=center, text width = 2cm, scale=0.8] at ($(axis cs: 0, 0.5)!0.5!(axis cs: 0.25, 0.5)$) {Passband};
\node[align=center, text width = 2cm, scale=0.8] at ($(axis cs: 0.25, 0.5)!0.5!(axis cs: 0.6, 0.5)$) {Transition};
\node[align=center, text width = 2cm, scale=0.8] at ($(axis cs: 0.6, 0.5)!0.5!(axis cs: 1, 0.5)$) {Stopband};

\end{axis}
\end{tikzpicture}}
	\end{center}
	
	\textbf{Design techniques:}
	\begin{itemize}
		\item Window method
		\item Parks-McClellan algorithm (Remez exchange algorithm)
		\item Least squares (convex optimization)	
	\end{itemize}
	These techniques are \textit{limited} to FIR filters.
\end{enumerate}
\end{frame}

\begin{frame}{Outline}
\begin{itemize}
	\item Digital filter design from analog filter
	\begin{itemize}
		\item Impulse invariance
		\item Bilinear transformation
	\end{itemize}
	\item FIR filter design from specifications
	\begin{itemize}
		\item Window method
		\item Parks-McClellan algorithm (Remez exchange algorithm)
		\item Least squares (convex optimization)	
	\end{itemize}
\end{itemize}	
\end{frame}

\section{Design from Analog Filter}
\begin{frame}{Digital processing of analog signals}
\begin{center}
	\def\Heff{1}
	\resizebox{\linewidth}{!}{\def\layersep{2cm}
\def\outsep{0.7cm}
\def\dy{1.25}

\begin{tikzpicture}[->, >=stealth, shorten >= 0pt, draw=black!50, node distance=\layersep, font=\sffamily]
    \tikzstyle{node}=[circle,fill=black,minimum size=2pt,inner sep=0pt]
    \tikzstyle{block}=[draw=black,rectangle,fill=none,minimum size=1.5cm, inner sep=0pt]
    \tikzstyle{annot} = []

	\node[node] (xc) at (0, -\dy cm) {};
    \node[block] (ADC) at (1*\layersep, -\dy cm) {C-to-D};
    \node[block, text width = 2cm, align= center] (DSP) at (3*\layersep, -\dy cm) {LTI \\ System};
    \node[block] (DAC) at (5*\layersep, -\dy cm) {D-to-C};
	\coordinate (yc) at (6*\layersep, -\dy cm) {};
	
	\coordinate (mid1) at ($(ADC.east)!0.5!(DSP.west)$) {};
	\coordinate (mid2) at ($(DSP.east)!0.5!(DAC.west)$) {};
		
    \path (xc) edge (ADC);
    \path (ADC) edge (DSP);
    \path (DSP) edge (DAC);
    \path (DAC) edge (yc);
    
    \node[above = 0.5mm of mid1] {$x[n]$};
    \node[below = 0.5mm of mid1] {$X(e^{j\omega})$};
    \node[above = 0.5mm of mid2] {$y[n]$};
    \node[below = 0.5mm of mid2] {$Y(e^{j\omega})$};
    \node[above = 0mm of xc, text width = 1cm, align=center] {$x_c(t)$};
    \node[below = 0mm of xc, text width = 1cm, align=center] {$X_c(j\Omega)$};
    \node[above = 0mm of yc, text width = 1cm, align=center] {$y_r(t)$}; 
    \node[below = 0mm of yc, text width = 1cm, align=center] {$Y_r(j\Omega)$};
    \node at ($(DSP.south)-(0, 0.25cm)$) {$h[n] \leftrightarrow H(e^{j\omega})$};
\end{tikzpicture}}
\end{center}

As long as there is no aliasing and that the reconstruction filter is the ideal lowpass filter these equalities hold:

\begin{equation}
H_{eq}(j\Omega) = \begin{cases}
H(e^{j\Omega T}), & |\Omega| < \pi/T \\
0, & |\Omega| > \pi/T
\end{cases} \tag{from DSP to analog}
\end{equation}

\begin{equation}
H(e^{j\omega}) = H_{eq}(j\omega/T), \quad|\omega| < \pi  \tag{from analog to DSP}
\end{equation}

In practice, these are good approximations.
\end{frame}

\subsection{Impulse Invariance}
\begin{frame}{Impulse invariance}

\textbf{Question:} How to design $h[n] \longleftrightarrow H(e^{j\omega})$ if we know $h_{eq}(t) \longleftrightarrow H_{eq}(j\Omega)$?

\begin{center}
	\def\Heff{1}
	\resizebox{\linewidth}{!}{\def\layersep{2cm}
\def\outsep{0.7cm}
\def\dy{1.25}

\begin{tikzpicture}[->, >=stealth, shorten >= 0pt, draw=black!50, node distance=\layersep, font=\sffamily]
    \tikzstyle{node}=[circle,fill=black,minimum size=2pt,inner sep=0pt]
    \tikzstyle{block}=[draw=black,rectangle,fill=none,minimum size=1.5cm, inner sep=0pt]
    \tikzstyle{annot} = []

	\node[node] (xc) at (0, -\dy cm) {};
    \node[block] (ADC) at (1*\layersep, -\dy cm) {C-to-D};
    \node[block, text width = 2cm, align= center] (DSP) at (3*\layersep, -\dy cm) {LTI \\ System};
    \node[block] (DAC) at (5*\layersep, -\dy cm) {D-to-C};
	\coordinate (yc) at (6*\layersep, -\dy cm) {};
	
	\coordinate (mid1) at ($(ADC.east)!0.5!(DSP.west)$) {};
	\coordinate (mid2) at ($(DSP.east)!0.5!(DAC.west)$) {};
		
    \path (xc) edge (ADC);
    \path (ADC) edge (DSP);
    \path (DSP) edge (DAC);
    \path (DAC) edge (yc);
    
    \node[above = 0.5mm of mid1] {$x[n]$};
    \node[below = 0.5mm of mid1] {$X(e^{j\omega})$};
    \node[above = 0.5mm of mid2] {$y[n]$};
    \node[below = 0.5mm of mid2] {$Y(e^{j\omega})$};
    \node[above = 0mm of xc, text width = 1cm, align=center] {$x_c(t)$};
    \node[below = 0mm of xc, text width = 1cm, align=center] {$X_c(j\Omega)$};
    \node[above = 0mm of yc, text width = 1cm, align=center] {$y_r(t)$}; 
    \node[below = 0mm of yc, text width = 1cm, align=center] {$Y_r(j\Omega)$};
    \node at ($(DSP.south)-(0, 0.25cm)$) {$h[n] \leftrightarrow H(e^{j\omega})$};
\end{tikzpicture}}
\end{center}

Design $h[n]$ by sampling $h_{eq}(t)$ with period $T$.
\begin{equation}
	h[n] = Th_c(nT) \tag{impulse invariance}
\end{equation}

The scaling factor $T$ compensates for the $1/T$ attenuation due sampling.

The resulting $h[n]$ depends on the sampling period $T$.

\end{frame}

\begin{frame}{Impulse invariance example: Butterworth filter}
\begin{block}{Butterworth filter}
	Butterworth filters are \textbf{maximally flat} in the passband and are monotonic overall. The downside of Butterworth filter is their relatively slow rolloff.
	
	In Matlab, see function \texttt{butter}.
\end{block}

For this example, consider the following 6th-order continuous-time Butterworth filter:

\begin{align*}
&H_{eq}(s)  \\
&= \frac{0.12093}{(s^2 + 0.364s + 0.4945)(s^2 + 0.9945s + 0.4945)(s^2 + 1.3385 + 0.4945)}
\end{align*}

\end{frame}

%
\begin{frame}{Impulse invariance example: Butterworth filter}
To design an \textbf{ FIR filter} through impulse invariance we must
\begin{enumerate}
	\item Obtain the continuous-time impulse response $h(t)$ (\texttt{impulse} in Matlab)
	\item Sample $h(t)$ with period $T$ and record only $N$ first samples 
	\begin{equation*}
		h[n] = \begin{cases}
		Th_{eq}(nT), & n = 0, \ldots, N-1 \\
		0, & \text{otherwise}
		\end{cases}, \tag{for causal $h_{eq}(t)$}
	\end{equation*}
	$h[n]$ is the FIR filter coefficients. $N$ is typically chosen to satisfy some energy criterion. For instance, samples must contain $95\%$ of the signal energy.
\end{enumerate} 
\begin{center}
	\resizebox{0.7\linewidth}{!}{\begin{tikzpicture}
\begin{axis}[
	name=plot1,
	axis lines*=middle,
	enlargelimits = false, clip=true,
	scale only axis,
	width=0.7\textwidth,
	height=0.3\textwidth,
	ymin=-0.1,	ymax=0.28,
	xmin=0, xmax=24.5,
	axis line style={->,>=stealth, shorten >= -0.5cm},
	xlabel={\small $t$},
	ylabel={\small $h_{eq}(t)$},
	every axis x label/.style={
		at={(ticklabel* cs:1)},
		xshift=0.5cm,
		anchor=north,
	},
	every axis y label/.style={
		at={(ticklabel* cs:1)},
		anchor=south,
		xshift=0.5cm,
	},
	xtick={0, 1, 5, 24},
	xticklabels={0, $T$, $5T$, $24T$},
	ytick=\empty,
	every outer y axis line/.append style={white!15!black},
	every y tick label/.append style={font=\color{white!15!black}},
	legend style={draw=white!15!black,fill=white,legend cell align=left}]
	
	\addplot [smooth, color=black, solid, line width=1.5pt] table[x index=0,y index=1] {figs/data/imp_invar_samples.dat};
	
	\only<2|handout:1>{
	\addplot [ycomb, color=red2, mark=*, fill=white, mark options={scale=1, fill=white}, line width=1.5pt, each nth point={4}] table[x index=0,y index=1] {figs/data/imp_invar_samples.dat};
}
\end{axis}
\end{tikzpicture}}
\end{center}

\end{frame}

%
\begin{frame}{Impulse invariance example: Butterworth filter}

\begin{columns}
	\begin{column}{0.5\textwidth}
		\textbf{Magnitude}
		\begin{center}
			\resizebox{\linewidth}{!}{\input{figs/imp_invar_FIR_butter_mag.tex}}
		\end{center}
	\end{column}
	\begin{column}{0.5\textwidth}
		\textbf{Phase}
		\begin{center}
			\resizebox{\linewidth}{!}{\input{figs/imp_invar_FIR_butter_phase.tex}}
		\end{center}
	\end{column}
\end{columns}

\vspace{0.2cm}
\textbf{Questions:}
\begin{enumerate}
	\item What would happen if we take fewer samples (smaller $N$)?
	\item What would happen if we decrease the sampling period e.g., $T_2 = 0.5T$?
\end{enumerate}

\end{frame}

%
\begin{frame}{Impulse invariance example: Butterworth filter}

\begin{itemize}
	\item Impulse invariance with FIR system is straightforward, and there are the additional implementation advantages discussed in lectures 7 and 8
	\item \textbf{Problem:} it may require prohibitively many samples to achieve good accuracy
\end{itemize}

To design an \textbf{IIR filter} through impulse invariance we must

\begin{enumerate}
	\item Invert the $s$-transform. Use \textbf{partial fraction expansion} to obtain $h_{eq}(t)$ analytically. Function \texttt{residue} in Matlab	
	\item Sample $h_{eq}(t)$: $h[n] = Th_{eq}(nT)$
	\item Calculate the $z$-transform $H(z)$ of $h[n]$
\end{enumerate}
\end{frame}


%
\begin{frame}{Impulse invariance example: Butterworth filter}

For the 6th-order Butterworth example:

\begin{columns}
	\begin{column}{0.5\textwidth}
		\begin{align*}
		H(z) =& \frac{0.2871 -0.4466z^{-1}}{1 -1.2971z^{-1} + 0.6949z^{-2}} \\ &+\frac{-2.1428 +1.1455z^{-1}}{1 -1.0691z^{-1} + 0.3699z^{-2}}  \\
		&+\frac{1.8557 - 0.6303z^{-1}}{1 -0.9972^{-1} + 0.2570z^{-2}} 
		\end{align*}
	\end{column}
	
	\begin{column}{0.5\textwidth}
		\begin{center}
			\resizebox{\linewidth}{!}{\input{figs/imp_invar_butter_iir_pole_zero.tex}}
		\end{center}
	\end{column}
\end{columns}
\end{frame}

%
\begin{frame}{Impulse invariance example: Butterworth filter}

\begin{columns}
	\begin{column}{0.5\textwidth}
		\textbf{Magnitude}
		\begin{center}
			\resizebox{\linewidth}{!}{\begin{tikzpicture}
\begin{axis}[
name=plot2a,
%at=(plot2.below south east), anchor=above north east,
xshift=1cm,
axis lines*=middle,
enlargelimits = upper, clip=true,
scale only axis,
axis line style={->,>=stealth},
xlabel={$\omega$},
ylabel={dB},
every axis x label/.style={
	at={(ticklabel* cs:1)},
	xshift=-0.2cm,
	anchor=north,
},
every axis y label/.style={
	at={(ticklabel* cs:1)},
	xshift=0.4cm,
	%yshift=0.35cm,
	anchor=south,
},
every outer x axis line/.append style={white!15!black},
every x tick label/.append style={font=\color{white!15!black}},
xmin=0, xmax=1,
ymin=-60, ymax=0,
xtick={0, 0.25, 0.5, 0.75, 1},
xticklabels ={$0$, $\pi/4$, $\pi/2$, $3\pi/4$, $\pi$},
ymajorgrids,
every outer y axis line/.append style={white!15!black},
every y tick label/.append style={font=\color{white!15!black}},
legend style={draw=white!15!black,fill=white,legend cell align=left, at={(axis cs: 1.05, -5)}}]

\addplot [smooth, color=black, solid, line width=2pt] table[x index=0,y index=1] {figs/data/imp_invar_FIR_butter_fir_mag_Hc.dat}; \addlegendentry{$H_{eq}(j\Omega)$};

\addplot [smooth, color=red2, solid, line width=2pt]
table[x index=0,y index=1] {figs/data/imp_invar_IIR_butter_mag_Hiir.dat}; \addlegendentry{$H(e^{j\omega})$ IIR};

\end{axis}
\end{tikzpicture}}
		\end{center}
	\end{column}
	\begin{column}{0.5\textwidth}
		\textbf{Phase}
		\begin{center}
			\resizebox{\linewidth}{!}{\input{figs/imp_invar_IIR_butter_phase.tex}}
		\end{center}
	\end{column}
\end{columns}

\begin{itemize}
	\item IIR systems achieve better accuracy while requiring fewer operations (coefficients) than FIR systems.
	\item Similarly to FIR systems, if we change the sampling frequency the behavior of the filter changes.
\end{itemize}
\end{frame}

%
\subsection{Bilinear Transform}
\begin{frame}<beamer:1|handout:0>{Outline}
\tableofcontents[currentsubsection]
\end{frame}

%
\begin{frame}{Bilinear transformation}
Another away to answer the question: How to design $h[n] \longleftrightarrow H(e^{j\omega})$ if we know $h_{eq}(t) \longleftrightarrow H_{eq}(j\Omega)$?

\begin{block}{Bilinear transformation}
	The bilinear transformation \underline{maps} the left-hand side of the $s$-plane into the unit circle in the $z$-plane.
	
	\begin{equation*}
		s = \frac{2}{T}\bigg(\frac{1 - z^{-1}}{1 + z^{-1}}\bigg) \tag{Bilinear transformation}
	\end{equation*}
	
	\begin{center}
		\resizebox{0.9\linewidth}{!}{\begin{tikzpicture}
\begin{axis}[
name=plot1,
axis equal,
axis lines*=middle,
enlargelimits = false, clip=true,
axis on top=true,
axis line style={->,>=stealth},
xlabel={$\sigma$},
ylabel={$\Omega$},
every axis x label/.style={
	at={(ticklabel* cs:1)},
	xshift=-0.2cm,
	anchor=north,
},
every axis y label/.style={
	at={(ticklabel* cs:1)},
	xshift=0.4cm,
	%yshift=0.35cm,
	anchor=south,
},
every outer x axis line/.append style={white!15!black},
every x tick label/.append style={font=\color{white!15!black}},
xmin=-2, xmax=2,
ymin=-2, ymax=2,
xtick=\empty,
ytick=\empty,
every outer y axis line/.append style={white!15!black},
every y tick label/.append style={font=\color{white!15!black}},
legend style={draw=white!15!black,fill=white,legend cell align=left}]

\fill[black!20] (axis cs:-3, -2) rectangle (0, 1.8);

\end{axis}

\begin{axis}[
name=plot2,
axis equal,
at=(plot1.east), anchor=west, xshift=1cm,
axis lines*=middle,
enlargelimits = false, clip=true,
axis on top=true,
axis line style={->,>=stealth},
xlabel={$\mathrm{Re}\{z\}$},
ylabel={$\mathrm{Im}\{z\}$},
every axis x label/.style={
	at={(ticklabel* cs:1)},
	xshift=-0.2cm,
	anchor=north,
},
every axis y label/.style={
	at={(ticklabel* cs:1)},
	xshift=0.4cm,
	%yshift=0.35cm,
	anchor=south,
},
every outer x axis line/.append style={white!15!black},
every x tick label/.append style={font=\color{white!15!black}},
xmin=-2, xmax=2,
ymin=-2, ymax=2,
xtick=1,
ytick=\empty,
xticklabel style={xshift=0.1cm},
every outer y axis line/.append style={white!15!black},
every y tick label/.append style={font=\color{white!15!black}},
legend style={draw=white!15!black,fill=white,legend cell align=left}]

\draw[black, fill=black!20, line width=1pt] (axis cs:0,0) circle [radius=1];

\end{axis}

\node[below] at (plot1.south) {$s$-plane};
\node[below] at (plot2.south) {$z$-plane};
\end{tikzpicture}}
	\end{center}	
\end{block}
\end{frame}

\begin{frame}{Bilinear transformation}

To design a digital filter from an analog filter using the bilinear transformation we just need to make a change of variables:

\begin{equation*}
H(z) = H_{eq}(s)|_{s = \frac{2}{T}\frac{1 - z^{-1}}{1 + z^{-1}}}
\end{equation*}

The resulting $H(z)$ will generally be IIR.

Designing digital filters using the bilinear transformation is easier and more systematic than the impulse invariance method.

In Matlab: \texttt{[bz, az] = bilinear(bs, as, 1/T)}

\end{frame}

\begin{frame}{Frequency warping}
	Evaluating $z$ on the unit circle is equivalent to evaluating $s$ on the imaginary axis $\Omega$.
	\begin{equation*}
j\Omega = \frac{2}{T}\bigg(\frac{1 - e^{-j\omega}}{1 + e^{-j\omega}}\bigg) = j\frac{2}{T}\tan\omega/2
\end{equation*}

This results in the following relation
\begin{equation*}
	\omega =2\arctan(\Omega T/2) \tag{frequency warping}
\end{equation*}

\textbf{Problem:} with the bilinear transformation we no longer have the linear relation $\omega = \Omega T$.

This is known as \textbf{frequency warping}.

\begin{center}
	\resizebox{0.4\linewidth}{!}{\begin{tikzpicture} 
\begin{axis}[
axis lines*=middle,
enlargelimits = true, clip=true,
xmax=3.14, xmin=-3.14,
ymin=-4, ymax=4,
y axis line style={->,>=stealth},
x axis line style={->,>=stealth, shorten >= -0.5cm},
xlabel={$\Omega T$},
ylabel={$\omega = 2\arctan(\Omega T/2)$},
every axis x label/.style={
    at={(ticklabel* cs:1)},
    anchor=north,
    xshift=0.5cm
},
every axis y label/.style={
    at={(ticklabel* cs:1)},
    anchor=south,
},
ytick={-3.14, 3.14}, yticklabels={-$\pi$, $\pi$},
xtick={-3.14, -1.57, 1.57, 3.14}, xticklabels={$-\pi$, $-\frac{\pi}{2}$, $\frac{\pi}{2}$, $\pi$},
%xmajorgrids,
ymajorgrids,
every outer y axis line/.append style={white!15!black},
every y tick label/.append style={font=\color{white!15!black}},
legend style={draw=white!15!black,fill=white,legend cell align=left}]

\addplot[smooth, black!20, solid, line width=1pt, domain=-20:20, samples=2] {x};
\addplot[smooth, black, solid, line width=1.5pt, domain=-20:20, samples=51] {rad(2*atan(x/2))};


\end{axis}
\end{tikzpicture}
}
\end{center}
\end{frame}

%
\begin{frame}{Bilinear transformation example: Butterworth filter}
Revisiting the Butterworth filter example, to obtain $H(z)$ we simply make:
\begin{equation*}
	H(z) = H_{eq}(s)|_{s = \frac{2}{T}\frac{1 - z^{-1}}{1 + z^{-1}}}
\end{equation*}

\textbf{Pole-zero diagram}
\begin{center}
	\resizebox{0.7\linewidth}{!}{\begin{tikzpicture}
\begin{axis}[
axis equal,
axis lines*=middle,
enlargelimits = false, clip=true,
xmin=-1.40,
xmax=1.39,
ymin=-1.10,
ymax=1.10,
axis line style={->,>=stealth},
xlabel={$\mathrm{Re}\{z\}$},
ylabel={$\mathrm{Im}\{z\}$},
every axis x label/.style={
at={(ticklabel* cs:1)},
anchor=north,
},
every axis y label/.style={
at={(ticklabel* cs:1)},
anchor=south,
},
xmajorgrids,
ymajorgrids,
every outer y axis line/.append style={white!15!black},
every y tick label/.append style={font=\color{white!15!black}},
legend style={draw=white!15!black,fill=white,legend cell align=left}]
\draw (axis cs:0,0) circle [black!50, dashed, line width=2pt, radius=1];
\addplot [line width=1pt,mark=x, only marks, mark size = 3pt]
table[row sep=crcr]{
	0.27199 0.73096 \\
	0.27199 -0.73096 \\
	0.20326 0.40067 \\
	0.20326 -0.40067 \\
	0.17843 0.1524 \\
	0.17843 -0.1524 \\
};

\addplot [line width=1pt,mark=*, only marks, mark size = 3pt, mark options={fill=white}]
table[row sep=crcr]{
 -1 0 \\
};

% Annotations
\node[anchor=south west] at (axis cs: -1, 0) {\scriptsize $\times 6$};
\end{axis}
\end{tikzpicture}}
\end{center}

\end{frame}


%
\begin{frame}{Bilinear transformation example: Butterworth filter}

\vspace{4mm}
\begin{columns}
	\begin{column}{0.5\textwidth}
		\textbf{Magnitude}
		\begin{center}
			\resizebox{\linewidth}{!}{\input{figs/bilinear_butter_mag.tex}}
		\end{center}
	\end{column}
	\begin{column}{0.5\textwidth}
		\textbf{Phase}
		\begin{center}
			\resizebox{\linewidth}{!}{\begin{tikzpicture}
\begin{axis}[
name=plot2a,
%at=(plot2.below south east), anchor=above north east,
xshift=1cm,
axis lines*=middle,
enlargelimits = upper, clip=true,
scale only axis,
axis line style={->,>=stealth},
xlabel={$\omega$},
ylabel={rad},
every axis x label/.style={
	at={(ticklabel* cs:1)},
	xshift=-0.2cm,
	anchor=north,
},
every axis y label/.style={
	at={(ticklabel* cs:1)},
	xshift=0.4cm,
	%yshift=0.35cm,
	anchor=south,
},
every outer x axis line/.append style={white!15!black},
every x tick label/.append style={font=\color{white!15!black}},
xmin=0, xmax=1,
ymin=-10, ymax=0,
xtick={0, 0.25, 0.5, 0.75, 1},
xticklabels ={$0$, $\pi/4$, $\pi/2$, $3\pi/4$, $\pi$},
ymajorgrids,
every outer y axis line/.append style={white!15!black},
every y tick label/.append style={font=\color{white!15!black}},
legend style={draw=white!15!black,fill=white,legend cell align=left, at={(axis cs: 1.05, -2)}}]

% T = 0.5
\addplot [smooth, color=black, solid, line width=2pt] table[x index=0,y index=1] {figs/data/bilinear_butter_phase_Hc.dat}; \addlegendentry{$H_{eq}(j\Omega)$};

\addplot [smooth, color=red2, solid, line width=2pt]
table[x index=0,y index=1] {figs/data/bilinear_butter_phase_Hbi.dat}; \addlegendentry{$H(e^{j\omega})$ bilinear $T = 0.5$};

% T = 2
\addplot [smooth, color=black, solid, line width=2pt, forget plot] table[x index=0,y index=1] {figs/data/bilinear_butter_phase_T2_Hc.dat};

\addplot [smooth, color=blue2, solid, line width=2pt]
table[x index=0,y index=1] {figs/data/bilinear_butter_phase_T2_Hbi.dat}; \addlegendentry{$H(e^{j\omega})$ bilinear $T = 2$};

\end{axis}
\end{tikzpicture}}
		\end{center}
	\end{column}
\end{columns}

\end{frame}

%
\begin{frame}{Frequency pre-warping}
Frequency warping distorts the frequency response at high frequencies ($\omega\to\pi$).

\textbf{Frequency pre-warping} mitigates this problem by scaling $s$ so that $H(e^{j\Omega_p T}) = H_{eq}(j\Omega_p)$ (no distortion) at some specified frequency $\Omega_p$.

\begin{equation*}
H(z) = H_{eq}(s)|_{s = \frac{\Omega_p}{\tan(\Omega_pT/2)}\frac{1 - z^{-1}}{1 + z^{-1}}} \tag{bilinear transformation with frequency pre-warping}
\end{equation*}
~\\

$\Omega_p$ is chosen so that $H(e^{j\omega})$ will be faithful to a particular characteristic of $H_{eq}(j\Omega)$ e.g., $\Omega_p$ is made equal to the $3$-dB bandwidth.


~\\
~\\
In Matlab: \texttt{[bz, az] = bilinear(bs, as, 1/T, wp/(2*pi))}
\end{frame}

%
\begin{frame}{Bilinear transformation example: Butterworth filter}

Example of frequency prewarping with $\Omega_p = 0.6\pi$, when $T =2$, and $\Omega_p = 0.2\pi$, when $T =0.5$.

\vspace{0.5cm}
\begin{columns}
	\begin{column}{0.5\textwidth}
		\textbf{Magnitude}
		\begin{center}
			\resizebox{\linewidth}{!}{\begin{tikzpicture}
\begin{axis}[
name=plot2a,
%at=(plot2.below south east), anchor=above north east,
xshift=1cm,
axis lines*=middle,
enlargelimits = upper, clip=true,
scale only axis,
axis line style={->,>=stealth},
xlabel={$\omega$},
ylabel={dB},
every axis x label/.style={
	at={(ticklabel* cs:1)},
	xshift=-0.2cm,
	anchor=north,
},
every axis y label/.style={
	at={(ticklabel* cs:1)},
	xshift=0.4cm,
	%yshift=0.35cm,
	anchor=south,
},
every outer x axis line/.append style={white!15!black},
every x tick label/.append style={font=\color{white!15!black}},
xmin=0, xmax=1,
ymin=-60, ymax=0,
xtick={0, 0.25, 0.5, 0.75, 1},
xticklabels ={$0$, $\pi/4$, $\pi/2$, $3\pi/4$, $\pi$},
ymajorgrids,
every outer y axis line/.append style={white!15!black},
every y tick label/.append style={font=\color{white!15!black}},
legend style={draw=white!15!black,fill=white,legend cell align=left, at={(axis cs: 1.05, -30)}}]

% T = 0.5
\addplot [smooth, color=black, solid, line width=2pt] table[x index=0,y index=1] {figs/data/bilinear_butter_mag_Hc.dat}; \addlegendentry{$H_{eq}(j\Omega)$};

\addplot [smooth, color=red2, solid, line width=2pt]
table[x index=0,y index=1] {figs/data/bilinear_butter_mag_prewarp_Hbi.dat}; \addlegendentry{$H(e^{j\omega})$ bilinear $T = 0.5, \Omega_p=0.2\pi$};

\draw[purple, dashed, line width=2pt] (axis cs: 0.6, -15) circle[purple, radius=0.5cm] {}; \node[anchor=south east] at (axis cs: 0.9, -15) {$\Omega_p = 0.6\pi$};

% T = 2
\addplot [smooth, color=black, solid, line width=2pt, forget plot] table[x index=0,y index=1] {figs/data/bilinear_butter_mag_T2_Hc.dat};

\addplot [smooth, color=blue2, solid, line width=2pt]
table[x index=0,y index=1] {figs/data/bilinear_butter_mag_T2_prewarp_Hbi.dat}; \addlegendentry{$H(e^{j\omega})$ bilinear $T = 2, \Omega_p = 0.6\pi$};

\end{axis}
\end{tikzpicture}}
		\end{center}
	\end{column}
	\begin{column}{0.5\textwidth}
		\textbf{Phase}
		\begin{center}
			\resizebox{\linewidth}{!}{\input{figs/bilinear_prewarp_butter_phase.tex}}
		\end{center}
	\end{column}
\end{columns}

\end{frame}

\begin{frame}{Famous filters}
	\begin{itemize}
		\item \textbf{Butterworth:}
		It's monotonic in the passband and the stopband.
		
		Matlab function \texttt{butter}
		\item \textbf{Chebyshev type I:}
		It has equiripple frequency response in the passband and varies monotonically in the stopband.
		
		Matlab function \texttt{cheby1}
		\item \textbf{Chebyshev type II:}
		It has equiripple frequency response in the stopband and varies monotonically in the passband.
		
		Matlab function \texttt{cheby2}
		\item \textbf{Elliptic:}
		It has equiripple frequency response in both the passband and the stopband.
		
		Matlab function \texttt{ellipt}
		
		\item \textbf{Bessel:}
		It has maximally linear phase response (constant group delay).
		
		Matlab function \texttt{besself} (only continuous time)
	\end{itemize}

	In general (and in Matlab) these filters are first designed in continuous-time $H(s)$, and then they are converted to discrete-time $H(z)$ using the bilinear transformation with frequency pre-warping.
\end{frame}

\begin{frame}{Summary on designing digital filters from analog filters}
\textbf{Impulse invariance}
\begin{itemize}
	\item Matches impulse response of discrete-time system to the impulse response of the continuous-time system
	\item In FIR implementations the impulse response is truncated up to a specified number of samples
	\item In IIR implementations the discrete-time system is obtained analytically.
\end{itemize}

\textbf{Bilinear transformation}
	\begin{itemize}
		\item The bilinear transformation maps the left-hand side of the $s$-plane into the unit circle in the $z$-plane.
		\item This mapping is non-linear, therefore there will be a distortion between $\Omega$ and $\omega$, which is known as frequency warping.
		\item Frequency warping is more severe at high frequencies ($\omega\to\pi$) because of the $\arctan(\cdot)$ shape. 
		\item Frequency pre-warping is a form of mitigating frequency warping. 
		\item Oversampling would also mitigate frequency warping, since the spectrum would be more compressed at lower frequencies, where the $\arctan(\cdot)$ is more linear.
	\end{itemize}
\end{frame}

\end{document}
