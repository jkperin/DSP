\begin{frame}{Last lecture}
\begin{itemize}
	\item Quantization is unavoidable in DSP systems
	\item Although quantization is a nonlinear operation on a signal, it is a linear operation on the signal PDF (area sampling)
	\item The probabilistic interpretation of quantization allows us to model the quantization error as an uniformly distributed random process (linear noise model)
	\item Using this linear noise model, we simply replace quantizers by noise sources of average power $\sigma_e^2 = \Delta^2/12$
	\item Quantization noise is assumed white (samples are uncorrelated)
	\item Every extra bit of resolution in a quantizer improves the SNR by 6.02 dB
	\item The signal amplitude must be matched to the dynamic range of the quantizer, otherwise there'll be excessive clipping or some bits won't be used
	\item Noise shaping is a strategy that minimizes quantization noise in A-to-D and D-to-A converters. The goal is to shape the quantization noise PSD, so that most of the noise power falls outside the signal band
	\item Noise shaping requires oversampling to minimize noise aliasing
\end{itemize}
\end{frame}